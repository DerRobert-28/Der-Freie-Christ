\documentclass[10pt,a5paper]{article}

\usepackage[ngerman]{babel}
\usepackage{enumitem}
\usepackage[utf8]{inputenc}
\usepackage[T1]{fontenc}
\usepackage[left=10mm,right=10mm,top=20mm,bottom=20mm]{geometry}
\usepackage{graphicx}
\usepackage[colorlinks=true,urlcolor=blue,linkcolor=black]{hyperref}

\newcommand{\Allmaechtige}[0]{\textbf{ALLM\"ACHTIGE}}
\newcommand{\Allmaechtigen}[0]{\textbf{ALLM\"ACHTIGEN}}
\newcommand{\Allmaechtiger}[0]{\textbf{ALLM\"ACHTIGER}}
\newcommand{\Christi}[0]{\textbf{CHRISTI}}
\newcommand{\Christus}[0]{\textbf{CHRISTUS}}
\newcommand{\Deinem}[0]{\textbf{DEINEM}}
\newcommand{\Deinen}[0]{\textbf{DEINEN}}
\newcommand{\Deiner}[0]{\textbf{DEINER}}
\newcommand{\Deines}[0]{\textbf{DEINES}}
\newcommand{\Deine}[0]{\textbf{DEINE}}
\newcommand{\Dein}[0]{\textbf{DEIN}}
\newcommand{\Deren}[0]{\textbf{DEREN}}
\newcommand{\Dich}[0]{\textbf{DICH}}
\newcommand{\Dir}[0]{\textbf{DIR}}
\newcommand{\Du}[0]{\textbf{DU}}
\newcommand{\Elohim}[0]{\textbf{ELOHIM}}
\newcommand{\Erloesers}[0]{\textbf{ERL\"OSERS}}
\newcommand{\Erloeser}[0]{\textbf{ERL\"OSER}}
\newcommand{\Er}[0]{\textbf{ER}}
\newcommand{\Geiste}[0]{\textbf{GEISTE}}
\newcommand{\Geist}[0]{\textbf{GEIST}}
\newcommand{\Gottes}[0]{\textbf{GOTTES}}
\newcommand{\Gott}[0]{\textbf{GOTT}}
\newcommand{\Heiligen}[0]{\textbf{HEILIGEN}}
\newcommand{\Heiliger}[0]{\textbf{HEILIGER}}
\newcommand{\Heilige}[0]{\textbf{HEILIGE}}
\newcommand{\Heilig}[0]{\textbf{HEILIG}}
\newcommand{\Herrn}[0]{\textbf{HERRN}}
\newcommand{\Herr}[0]{\textbf{HERR}}
\newcommand{\Himmel}[0]{\textbf{HIMMEL}}
\newcommand{\Himmels}[0]{\textbf{HIMMELS}}
\newcommand{\Ihm}[0]{\textbf{IHM}}
\newcommand{\Ihnen}[0]{\textbf{IHNEN}}
\newcommand{\Ihn}[0]{\textbf{IHN}}
\newcommand{\Ihr}[0]{\textbf{IHR}}
\newcommand{\Jahwe}[0]{\textbf{JAHWE}}
\newcommand{\Jesus}[0]{\textbf{JESUS}}
\newcommand{\Jesu}[0]{\textbf{JESU}}
\newcommand{\Messias}[0]{\textbf{MESSIAS}}
\newcommand{\Saviour}[0]{\textbf{SAVIOUR}}
\newcommand{\Schoepfer}[0]{\textbf{SCH\"OPFER}}
\newcommand{\Schoepfers}[0]{\textbf{SCH\"OPFERS}}
\newcommand{\Seinem}[0]{\textbf{SEINEM}}
\newcommand{\Seinen}[0]{\textbf{SEINEN}}
\newcommand{\Seiner}[0]{\textbf{SEINER}}
\newcommand{\Seines}[0]{\textbf{SEINES}}
\newcommand{\Seine}[0]{\textbf{SEINE}}
\newcommand{\Sein}[0]{\textbf{SEIN}}
\newcommand{\Selbst}[0]{\textbf{SELBST}}
\newcommand{\Sich}[0]{\textbf{SICH}}
\newcommand{\Sie}[0]{\textbf{SIE}}
\newcommand{\Sohnes}[0]{\textbf{SOHNES}}
\newcommand{\Sohne}[0]{\textbf{SOHNE}}
\newcommand{\Sohn}[0]{\textbf{SOHN}}
\newcommand{\Spirit}[0]{\textbf{SPIRIT}}
\newcommand{\Vater}[0]{\textbf{VATER}}
\newcommand{\Vaters}[0]{\textbf{VATERS}}
\newcommand{\You}[0]{\textbf{YOU}}

\newcommand{\q}[1]{\char"22{#1}\char"22 }
\newcommand{\qq}[1]{\textit{\q{#1}}}

\title{\textbf{Der Freie Christ}}
\author{Robert Lang-Kirchh\"ofer}
\date{\textit{Letzte \"Anderung: 14. Oktober 2023}}

\begin{document}
	\setlength{\parindent}{0mm}
	\maketitle
	\begin{figure}[h]
		\centering
		\includegraphics[width=1\textwidth,keepaspectratio]{"FreeChristian.jpeg"}
	\end{figure}

	\newpage
	\pagecolor{white}
	\tableofcontents
	
	\newpage
	\section{Vorwort}

	\subsection{Verwendungshinweise}
		Ich werde als pers\"onliche Anrede das informelle \q{du},
		und auch das generische Maskulinum verwenden.
		Das soll einerseits eine angenehme,
		lockere Atmosph\"are schaffen,
		und andererseits den Lesefluss erleichtern.
		Selbstverst\"andlich gilt dir,
		mein lieber Leser,
		unabh\"angig von deinem tats\"achlichen Geschlecht,
		mein vollster Respekt.
		Da ich selbst auch kein professioneller Author bin,
		ist mein Schreibstil auch nicht perfekt,
		sondern teilweise etwas lockerer.
		Solltest du die englische Version lesen,
		wirst du vielleicht auch feststellen,
		dass mein Englisch nicht perfekt ist,
		da dies nicht meine Muttersprache ist.
		Desweiteren ist dieses e"~Book nicht unbedingt so geschrieben,
		dass ein Kapitel auf dem anderen basiert.
		Das hei{\ss}t,
		du brauchst es weder komplett,
		noch \q{von A-Z} durchlesen,
		sondern kannst so viel,
		oder auch so wenig wie du m\"ochtest lesen,
		und auch in beliebiger Reihenfolge.

	\subsection{Danksagung}
		Als n\"achstes m\"ochte ich dir,
		lieber Leser,
		meinen herzlichsten Dank aussprechen,
		dass du dich daf\"ur entschieden hast,
		hier reinzuschnuppern.
		Ich hoffe nat\"urlich,
		dass du dieses e"~Book bis zum Schluss durchlie{\ss}t,
		und seine Weiterentwicklung beobachtest.
		Ich kann nicht mit 100\%iger Sicherheit sagen,
		dass es je \q{fertig} sein wird,
		weil ich m\"oglichweise immer wieder neue Gedanken,
		oder neues Material finden werde,
		das ich hier aufnehmen werde.
		Es handelt sich hierbei n\"amlich um ein christliches Schriftst\"uck.
		Ich will dir hiermit moralische Werte \"ubermitteln,
		insbesondere wie sie,
		nat\"urlich nach bestem Wissen und Gewissen,
		von {\Gott},
		dem {\Herrn},
		und {\Seinem} {\Sohn} {\Jesus} {\Christus} gew\"unscht sind.
		Ich m\"ochte selbstverst\"andlich keine neue \q{Religion},
		oder das Christentum neu erfinden,
		sondern lediglich ein paar neue Perspektiven aufzeigen.
		Wie du in diesem Vorwort schon erkennen kannst,
		sind Worte die sich direkt auf {\Gott},
		{\Jesus} oder auch den {\Heiligen} {\Geist} beziehen,
		in Majuskeln,
		also komplett in Gro{\ss}buchstaben,
		und zus\"atzlich in Fettschrift geschrieben.
		Wenn dir etwas am Herzen liegt,
		oder dir allgemein etwas hierzu einf\"allt,
		bist du herzlich eingeladen,
		in meiner \href{https://github.com/DerRobert-28/Der-Freie-Christ/discussions}{GitHub-Diskussion} mitzuwirken.
	
	\subsection{Ein paar Worte zu mir}
		Ich selbst wurde,
		soweit ich mich richtig erinnere,
		mit etwa ein/zwei Jahren katholisch getauft,
		bin aber Ende Juli 2023 aus der Kirche ausgetreten.
		Die Gr\"unde hierf\"ur sind pers\"onlicher Art,
		und sind hier nicht von Bedeutung.
		Das hat jedoch nichts mit meinem Glauben zu tun.
		Ich selbst glaube,
		dass {\Gott} existiert,
		und dass {\Jesus} der {\Erloeser} ist.
		Das hei{\ss}t aber nicht,
		dass ich sowas wie der \q{perfekte Christ} bin,
		falls es sowas unter uns Menschen heutzutage \"uberhaupt gibt.
		Mehr zu mir kannst du im Kapitel \q{\hyperref[MeinLebenMitGott]{Mein Leben mit {\Gott}}} lesen.
	
	\newpage
	\section{Was macht einen \q{Freien Christen} aus?}
	
	\subsection{Keine Kirchengemeinde}
		Ein Freier Christ ist nicht an eine Kirchengemeinde gebunden.
		Das hei{\ss}t,
		man braucht nicht getauft sein.
		Man kann auch aus der Kirche ausgetreten sein.
		Das spielt alles keine Rolle.
		Ich halte es sogar f\"ur viel sinnvoller,
		gar nicht erst einer Kirchengemeinde anzugeh\"oren,
		insbesondere der katholischen,
		um nicht von ihr manipuliert zu werden.
		Wichtig ist nur,
		dass man {\Gott},
		den {\Herrn},
		und {\Jesus} {\Christus},
		seinen eingeborenen Sohn,
		in sein Leben l\"asst,
		und sich zu {\Ihnen} bekennt.
	
	\subsection{Der wahre Bund}
		F\"ur dich,
		als Freien Christ,
		ist der einzig wahre,
		bestehende Bund zwischen {\Gott},
		{\Jesus} {\Christus} und dir.
		Wenn du dich zu {\Ihnen} bekennst,
		pflegst du diese Beziehung von Herzen.
		Weltliche B\"unde \textit{(Beziehungen)} sind verg\"anglich,
		dennoch ist es nicht weniger wichtig,
		auch diese herzlich zu pflegen.
	
	\subsection{Die Bibel als \q{Werkzeug}}
		Wenn es der Beziehung zwischen {\Gott},
		{\Jesus} und dir dient,
		hast du,
		als Freier Christ,
		die Freiheit,
		Bibelstellen besser,
		also moderner oder verst\"andlicher,
		auszulegen,
		und entsprechend umzuformulieren.
		Das ist jedoch \textbf{kein} Freibrief daf\"ur,
		das Wort {\Gottes} nach Gutdünken umzuschreiben,
		und damit beispielsweise {\Seinen} Willen zu beugen,
		so wie es,
		meinen Informationen und Recherchen nach,
		die katholische Kirche in der Vergangenheit \q{gerne} gemacht hat.
	
	\subsection{Glaube und Wissenschaft}
		Glaube und Wissenschaft schlie{\ss}en sich,
		meiner Meinung nach,
		nicht gegenseitig aus.
		{\Gott} kann man weder beweisen,
		noch widerlegen.
		Du hast jederzeit die freie Wahl,
		ob du mit {\Gott} leben willst,
		oder nicht.
		Daf\"ur hast du als Mensch einen freien Willen bekommen.
	
	\subsection{Was wei{\ss} ich \"uber {\Gott}?}
		Nun ... was \q{wei{\ss}} man schon wirklich \"uber {\Gott}?
		Klar,
		ich kann {\Gott} und {\Jesus} durch {\Sein} Heiliges Wort,
		die Bibel,
		besser kennenlernen.
		Ich w\"urde mir jedoch niemals anma{\ss}en,
		zu behaupten,
		dass ich {\Gott} \q{kenne}.
		Schon gar nicht absolut.
		Nur {\Gott} kennt {\Sich} selbst ganz und gar.
		Und in diesem e"~Book m\"ochte ich dir einfach meine Erfahrung mitgeben.
		Und was ich \"uber {\Gott} sage,
		ist nur nach bestem Wissen und Gewissen,
		und nur zu {\Seinen} Gunsten.

	\newpage
	\section{Glaubst du (nicht) an {\Gott}?}
		Bei diesem e"~Book handelt es sich zwar um ein christliches Schriftst\"uck,
		das hei{\ss}t aber nicht,
		dass nur Christen dieses e"~Book lesen d\"urfen.
		Ganz im Gegenteil!
		Auch wenn du an etwas anderes oder auch nicht glaubst,
		bist du herzlich eingeladen,
		hier reinzust\"obern.
		Gott liebt uns alle gleich.
		Doch ... woran glaubst du eigentlich?
	
	\subsection{\q{Ich bin Jude und glaube nicht an {\Jesus} als {\Erloeser}.}}
		Dann bist du dennoch herzlich eingeladen,
		weiterzulesen.
		Wir glauben ja effektiv an den gleichen {\Gott},
		der in der Heiligen Schrift auch {\Elohim} oder {\Jahwe} genannt wird.
		Meines Wissens nach sind gro{\ss}e Teile deines heilgen Buches,
		z.B. die Torah,
		deckungsgleich mit dem Alten Testament der Bibel.
		Als Beispiel nenne ich die Entstehungsgeschichte,
		oder den Auszug aus \"Agypten.		
		Da ich ja bereits einger\"aumt habe,
		dass ich mir nicht anma{\ss}e,
		{\Gott} zu kennen,
		w\"urde ich auch nicht behaupten,
		dass du,
		nur weil du nicht an {\Jesus} glaubst,
		ein schlechterer Mensch bist,
		und deswegen nicht in den {\Himmel} kommst.
		Wenn du {\Jesus} nicht als deinen {\Erloeser} anerkennst,
		dann akzeptiere ich das.
		Ich bitte dich,
		es ebenfalls zu akzeptieren,
		dass f\"ur mich {\Jesus} der {\Erloeser} ist.
		Es geht mir auch gar nicht darum,
		dir {\Jesus} \q{auszuzwingen},
		sondern im Gro{\ss}en und Ganzen um moralische Werte,
		und darum,
		meine Erfahrungen mit dir zu teilen.
		
	\subsection{\q{Ich bin Muslim und glaube an Allah.}}
		Dann bist auch du herzlich eingeladen,
		weiterzulesen.
		Ich habe bereits einger\"aumt,
		dass ich mir nicht anma{\ss}e,
		{\Gott} zu kennen.
		Von daher wei{\ss} ich auch nicht,
		ob (der) {\Gott},
		an den ich glaube,
		und dein Gott,
		Allah,
		effektiv der gleiche Gott sind,
		oder zwei verschiedene,
		und die Gleichstellung vielleicht sogar Blasphemie,
		also Gottesl\"asterung,
		ist.
		Ich werde dir deinen Glauben nicht absprechen.
		Du darfst zu Allah beten,
		und ihn als deinen alleinigen Gott ansehen.
		Wenn du mich als Ungl\"aubigen betrachtest,
		nehme ich das als deine Meinung an.
		Ich akzeptiere,
		dass Allah f\"ur dich dein alleiniger Gott ist.
		Ich bitte dich,
		zu akzeptieren,
		dass {\Gott} f\"ur mich mein alleiniger Gott ist.
		Ich hoffe dennoch darauf,
		dass wir uns auch au{\ss}erhalb unseres Glaubens,
		einfach aus menschlicher Sicht,
		gegenseitig respektieren k\"onnen.
		Es geht mir hier schlie{\ss}lich gar nicht darum,
		dir einen anderen Gott \q{aufzuzwingen},
		sondern im Gro{\ss}en und Ganzen um moralische Werte,
		und darum,
		meine Erfahrungen mit dir zu teilen.
		
	\subsection{\q{Ich geh\"ore einer anderen Religion an.}}
		Ich kann leider nicht auf alle Glaubensrichtungen,
		Religionen und Sekten dieser Welt eingehen.
		Das w\"urde den Rahmen dieses e"~Books sprengen.
		Von daher verzeih mir,
		wenn ich deinen Glauben nicht explizit aufgef\"uhrt habe.
		Wenn du die beiden vorherigen Abschnitte liest,
		und du beispielsweise zu dem Schluss kommst,
		dass ich aufgrund deines Glaubens ein Ungl\"aubiger bin,
		dann nehme ich das als deine Meinung an.
		Ich akzeptiere,
		dass die Gottheit,
		oder die Gottheiten,
		und selbst wenn es der Teufel ist,
		diejenigen sind,
		die f\"ur dich anbetungsw\"urdig sind.
		Ich bitte dich ebenfalls,
		zu akzeptieren,
		dass {\Gott} f\"ur mich mein alleiniger Gott ist.
		Ich hoffe dennoch darauf,
		dass wir uns auch au{\ss}erhalb unseres Glaubens,
		einfach aus menschlicher Sicht,
		gegenseitig respektieren k\"onnen.
		Es geht mir hier,
		wie oben bereits erw\"ahnt,
		schlie{\ss}lich gar nicht darum,
		dir \q{meinen} {\Gott} aufzuzwingen,
		sondern im Gro{\ss}en und Ganzen um moralische Werte,
		und darum,
		meine Erfahrungen mit dir zu teilen.
	
	\subsection{\q{Ich glaube an keinen Gott, oder bin Agnostiker.}}
		Auch du bist herzlich zum Weiterlesen eingeladen.
		Denn selbst du,
		mein Freund,
		wenn du nicht (mehr) an {\Gott} glaubst,
		oder dir die Existenz von etwas g\"ottlichem gleichg\"ultig oder unbekannt ist,
		hast dennoch in irgendeiner Form moralische Werte.
		M\"ochtest du einfach so verletzt oder gar get\"otet werden?
		M\"ochtest du bestohlen oder betrogen werden?
		M\"ochtest du,
		dass dir dein Partner fremdgeht,
		wenn du nicht gerade sowas wie eine offene Beziehung f\"uhrst?
		M\"ochtest du belogen werden?
		M\"ochtest du nicht auch einen gesunden K\"orper haben,
		und eine hohe Lebensqualit\"at genie{\ss}en?
		M\"ochtest du nicht ganz tief in dir einfach nur geliebt werden,
		so wie du bist?
		Deswegen akzeptiere ich,
		wenn du in deinem Leben keinen Gott brauchst.
		Ich bitte dich dennoch,
		zu akzeptieren,
		dass {\Gott} f\"ur mich existiert und der alleinige Gott ist.
		Ich hoffe darauf,
		dass wir uns auch einfach aus menschlicher Sicht,
		gegenseitig respektieren k\"onnen.
		Es geht mir hier schlie{\ss}lich gar nicht darum,
		dir einen Gott oder einen Glauben \q{aufzuzwingen},
		sondern im Gro{\ss}en und Ganzen um moralische Werte,
		und darum,
		meine Erfahrungen mit dir zu teilen.

	\newpage
	\section{Wer und wie ist {\Gott}?}
		Dieses Kapitel dient auch mir selbst,
		so dass ich etwas Bibelarbeit machen kann.
		Es geht darum,
		wie {\Gott} in der Bibel genannt wird,
		also seine Namen und generellen Bezeichnungen,
		und wie {\Er} {\Sich} manifestiert.
		Dazu geh\"oren nat\"urlich auch {\Jesus} und der {\Heilige} {\Geist}.
		Ich werde im folgenden eine Liste auff\"uhren,
		die meist gehalten ist im Stil \q{{\Er} ist ...},
		oder \q{{\Er} hei{\ss}t ...},
		und dann wer oder wie {\Gott} ist.
		Und wenn ich eine Quelle habe,
		gebe ich in Klammern die erste Bibelstelle an,
		wo das auftaucht.
		Als \"Ubersetzung werde ich die Lutherbibel von 2017 verwenden.
		Viel Spa{\ss} beim gemeinsamen Entdecken von {\Gott}.
	
	\subsection{{\Gott} ist ...}
		\begin{itemize}[nosep]
			\item {\Er} ist {\Gott}. \textit{(1. Mose 1, 1)}
			\item {\Er} ist der {\Geist} ({\Gottes}). \textit{(1. Mose 1, 2)}
		\end{itemize}
		
	\newpage
	\section{Die Zehn Gebote}
		Die traditionellen 10 Gebote werden \"ublicherweise aus der Sicht {\Gottes} \"uberliefert,
		also in der Form \q{Du sollst (nicht) ...}.
		Im folgenden sind die 10 Gebote aus der Sicht,
		wenn man selbst zu {\Gott} sprechten w\"urde,
		und {\Ihm} die Gebote als Versprechen geben w\"urde.
		Auch sind sie etwas besser ausgearbeitet,
		da man manche Gebote bei genauerer Betrachtung auch zusammenfassen k\"onnte.
		Das bedeutet selbstverst\"andlich nicht,
		dass ich die traditionellen,
		von {\Gott} gegebenen Gebote ablehne.
		Ich m\"ochte nur eine andere Betrachtungsweise zeigen.
		Desweiteren werde ich sie \q{Angebote} nennen,
		um den guten Willen beider Seiten unterstreichen.
	
	\subsection{Das Oberste Angebot}
		Das Oberste Angebot lautet:
		Ich will {\Gott}, den {\Herrn}, von ganzem Herzen lieben und {\Ihn} ehren. Und ich will meinen N\"achsten lieben, wie auch mich selbst.
		
	\subsubsection{Das Erste Angebot}
		{\Du} bist der {\Herr},
		mein {\Gott},
		mein {\Erloeser}.
		Ich will keine anderen G\"otter neben {\Dir} haben,
		und sie nicht anbeten oder verehren.
		Und ich will mir kein G\"otzenbild schaffen.
		
	\subsubsection{Das Zweite Angebot}
		{\Du} bist der {\Herr},
		mein {\Gott}.
		Ich will {\Deinen} Namen nicht missbrauchen.
		Ich will {\Dir} nicht l\"astern.
		Und ich will mich ehrlich zu {\Dir} bekennen.
		\\
		\textit{Kurzer Hinweis:
		Hiermit sollen auch umgangssprachliche Phrasen,
		wie beispsielsweise \q{Oh (mein) G...},
		oder \q{Um G...es Willen},
		die man schnell sagt,
		ohne aber wirklich {\Gott} selbst zu meinen,
		oder zu ihm zu beten,
		oder \"ahnliches.}
			
	\subsubsection{Das Dritte Angebot}
		{\Du} bist der {\Herr},
		mein {\Gott}.
		Ich will {\Dich} nicht auf die Probe stellen.
		Ich will {\Dich} nicht versuchen.
		Ich will auch in der Not zu {\Dir} stehen.
		\\
		\textit{Kurzer Hinweis:
		Hiermit sollen auch Situationen abgedeckt werden,
		in denen man leichtfertig solche Dinge sagt wie beispielsweise,
		wie {\Gott} dieses oder jenes Leid zulassen kann.}
		
	\subsubsection{Das Vierte Angebot}
		{\Du} bist der {\Herr},
		mein {\Gott}.
		Ich will {\Dir} den Sabbat heiligen.
		Ich will am Sabbat des Fleischlichen,
		und Suchterzeugenden enthaltsam bleiben.
		
	\subsubsection{Das F\"unfte Angebot}
		Ich will meinen Vater und meine Mutter,
		die mir mein Leben geschenkt,
		mich gro{\ss}gezogen und ern\"ahrt haben,
		ehren.
		Und ich will \"Altere Menschen ehren.
			
	\subsubsection{Das Sechste Angebot} \label{DasSechsteAngebot}
		Ich will nicht t\"oten oder morden.
		Ich will meine Beziehungen pflegen.
		Ich will das Leben und Wohlergehen allen Lebens respektieren,
		und nach M\"oglichkeit auch sch\"utzen.
		\\
		\textit{Kurzer Hinweis:
		Das T\"oten ist hier nicht nur w\"ortlich,
		also physisch gemeint,
		sondern auch symbolisch,
		indem man beispielsweise aus Zorn irgendetwas zu jemandem sagt,
		was ihn verletzt,
		und damit der Beziehung schadet.
		Lies gerne dazu die Bibelstelle \href{https://www.die-bibel.de/bibeln/online-bibeln/lesen/LU17/MAT.5/Matthäus-5}{Matthäus 5, 21-26}.}
		
	\subsubsection{Das Siebte Angebot}
		Ich will nicht die Ehe brechen.
		Ich will nicht die Frau meines N\"achsten begehren.
		Ich will nicht den Mann meiner N\"achsten begehren.
		
	\subsubsection{Das Achte Angebot}
		Ich will nicht stehlen oder betr\"ugen.
		Ich will nicht rauben oder entf\"uhren.
		Ich will nicht begehren meines N\"achsten Haus.
		Ich will nicht begehren meines N\"achsten Hab und Gut.
		Ich will dem Hab und Gut meines N\"achsten keinen Schaden zuf\"ugen.
		
	\subsubsection{Das Neunte Angebot} \label{DasNeunteAngebot}
		Ich will nicht falsch Zeugnis geben wider meinem N\"achsten.
		Ich will nicht l\"ugen oder betr\"ugen.
		Ich will nicht schw\"oren.
		Ich will gegen\"uber meinem N\"achsten ehrlich und gerecht handeln.
		\\
		\textit{Kurzer Hinweis:
		Das Wort \q{schw\"oren} ist im Englischen zweideutig.
		Da es n\"amlich mit \q{(to) swear} \"ubersetzt wird,
		kann man darunter \q{(einen Eid) schw\"oren},
		oder \q{fluchen} verstehen.
		Somit k\"onnte man es auch \"ubersetzen mit
		\q{Ich will nicht (be)schw\"oren oder (ver)fluchen.}}
		
	\subsubsection{Das Zehnte Angebot} \label{DasZehnteAngebot}
		Mein K\"orper und mein Leben sind ein Geschenk von {\Dir},
		und somit heilig.
		Ich will sie ehren und pflegen.
	
	\subsection{Gegen\"uberstellung}
		Im Judentum sind die 10 Gebote,
		die in der Torah sinngem\"a{\ss} \q{10 Worte, die {\Gott} gesprochen hat} hei{\ss}en,
		so \"uberliefert,
		dass man von beiden Gebotstafeln je zwei gegen\"uberstellen kann,
		und im weitesten eine Verbindung aufbauen kann.
		Beispielsweise lautet das erste Gebot dort:
		\q{Du wirst {\Gott} als {\Herrn} und Befreier aus Ägypten anerkennen.}
		Und das sechste,
		als parallele Verbindung,
		lautet:
		\q{Du wirst nicht morden.}
		Damit ist nat\"urlich nicht nur gemeint,
		dass man seinem Mitmenschen keinen k\"orperlichen Schaden zuf\"ugt,
		sondern auch keinen seelischen,
		beispielsweise durch Kr\"ankungen,
		wie bereits beim \q{\hyperref[DasSechsteAngebot]{Sechsten Angebot}} erkl\"art.
		Die Parallele besteht hier,
		dass man beim ersten Gebot {\Gott} komplett akzeptiert,
		und beim sechsten Gebot seinen Mitmenschen komplett akzeptiert.
		Es geht also in beiden F\"allen um die bedingungslose Liebe,
		einmal gegen\"uber {\Gott},
		einmal gegen\"uber seinem N\"achsten.
		Die \textbf{10 Angebote} kann man ebenso gegen\"uberstellen,
		und ich werde dir die Verbindungen zeigen:
		\\
		\begin{itemize}
			\item	\textbf{Das 1. und 6. Angebot:}
			\\		Es soll auf der einen Seite {\Gott},
					und auf der anderen Seite dein N\"achster,
					voll und ganz angenommen werden.
					Es geht also um die bedingungslose Liebe gegen\"uber {\Gott} und deinem Mitmenschen.
					Es ist zus\"atzlich gemeint,
					dass hier durch andere G\"otter,
					egal ob neben oder anstelle von {\Gott},
					der Beziehung zu {\Gott} ein Schaden zugef\"ugt wird.
					Und zus\"atzlich,
					dass z.B. durch T\"oten deinem N\"achsten ein Schaden zugef\"ugt wird.
			\item	\textbf{Das 2. und 7. Angebot:}
			\\		Hier besteht die Verbindung etwas tiefer.
					Selbstverst\"andflich auf der einen Seite gegen\"uber zu {\Gott}.
					Und auf der anderen Seite gegen\"uber deinem Ehepartner,
					oder auch gegen\"uber einem anderen verheirateten Mitmenschen.
					In unserer modernen Zeit,
					wo man nicht immer sofort oder fr\"uh heiratet,
					k\"onnte man den Ehebruch auch noch auf jegliche Liebesbeziehung ausweiten,
					im Sinne von \q{Fremdgehen},
					oder generell eine Liebesbeziehung sch\"adigen.
					Letztendlich geht es hier um tiefen und intimen Respekt.
					Es ist gemeint,
					wenn du auf der einen Seite {\Gottes} Namen missbrauchst,
					dass das {\Ihm} gegen\"uber sehr respektlos ist,
					und das sehr sch\"adlich f\"ur eine tiefe Beziehung zu {\Gott} ist.
					Und,
					auf der anderen Seite,
					wenn du ehebrichst,
					fremdgehst,
					oder \"ahnliches,
					dann ist das sehr respektlos gegen\"uber deinem Mitmenschen.
					Es ist ja auch gleichzeitig ein Vertrauensbruch.
					Wenn du bspw. verheiratet bist,
					und dann mit einer beziehungsfremden Person intim wirst,
					woher will dein Ehepartner wissen,
					dass du das nie wieder tun wirst?
					Oder anders herum:
					Wie w\"urdest du dich f\"uhlen?
					W\"urdest du das wollen?
					\textit{Anmerkung:
					Hier geht es nicht um das Thema \q{lockere/offene Beziehung}.
					Ich m\"ochte das ganze nicht unn\"otig verkomplizieren.}
			\item	\textbf{Das 3. und 8. Angebot:}
			\\		Die Verbindung in der Versuchung {\Gottes},
					und im Diebstahl besteht darin,
					dass man auf der einen Seite,
					im symbolischen Sinne,
					einen Teil von {\Gottes} Allmacht \q{stehlen} will.
					Ich dir ein Extrembeispiel zeigen:
					Angenommen,
					du springst aus dem Fenster,
					und sagst dir sowas wie:
					\q{Wenn es {\Gott} gibt, wird {\Er} mich auffangen.}.
					Damit w\"urdest du also einen Teil {\Seiner} Allmacht \q{stehlen} (wollen).
					Und auf der anderen Seite,
					gegen\"uber deinem Mitmenschen,
					ist es ja klar,
					dass gemeint ist,
					dass du deinem Mitmenschen nichts unberechtigt entwendest,
					also richtiger Diebstahl,
					oder auch Dienstleistungen unberechtigt in Anspruch nimmst:
					\q{Dienstleistungs-Diebstahl},
					was man \"ublicherweise mit \q{Betrug} bezeichnet.
			\item	\textbf{Das 4. und 9. Angebot:}
			\\		Hier ist die Parallele,
					dass es gewisse Dinge gibt,
					die heilig sind,
					und von daher auch ehrenhaft behandelt werden sollen.
					Auf der einen Seite ist das der Sabbat,
					also der \textbf{siebte} Tag,
					da {\Gott} nach der Sch\"opfung am \textbf{siebten} Tag geruht hat.
					Daher kommt das ja auch in unserer Gesellschaft,
					dass der Sonntag,
					der \textbf{siebte} Tag der Woche,
					in den meisten Branchen ein arbeitsfreier Tag ist.
					Auf der anderen Seite wiederum gilt auch hier,
					dass das,
					was du mitteilst,
					heilig bzw. ehrenhaft sein soll.
					Deswegen sollst du die Wahrheit sprechen,
					und gerecht gegen\"uber deinem N\"achsten sein.
					Gleichzeitig,
					und darauf bin ich im \q{\hyperref[DasNeunteAngebot]{Neunten Angebot}} bereits eingegangen,
					kann ja das \q{schw\"oren} auch mit dem englischen \q{(to) swear} zusammenh\"angen.
					Und Beschw\"oren,
					Fluchen,
					oder Verfluchen
					sind alles andere als ehrenhaft und heilig.
			\item	\textbf{Das 5. und 10. Angebot:}
			\\		Hier ist die Verbindung,
					meiner Meinung nach,
					mehr als offensichtlich.
					Deine \textit{biologischen} Eltern haben dir das Leben geschenkt,
					und in der Regel haben sie dich auch gro{\ss}gezogen und ern\"ahrt.
					Freilich kann es auch sein,
					dass du beispielsweise Adoptiveltern hast,
					oder es ganz andere Umst\"ande bei dir gibt.
					Aber zweifelsohne wurdest du von zwei Menschen gezeugt und geboren.
					Und diese beiden haben dir dein Leben,
					deinen K\"orper geschenkt.
					Und selbst wenn,
					angenommen du w\"urdest im Extremfall wirklich deine leiblichen Eltern nicht kennen,
					hast du einen K\"orper.
					Und du hast nur dieses eine irdische Leben.
					Von daher ist es wichtig,
					dass du auch dir selbst gut tust.
					Und auch,
					wenn die Rede vom K\"orper ist,
					sind auch deine Gedanken,
					und deine Psyche gemeint,
					dein Charakter und deine Pr\"agungen,
					die sich in deinem Gehirn befinden,
					was effektiv wiederum ein Teil deines K\"orpers ist.
					Achte stets auf das,
					was von au{\ss}en kommt,
					sowohl die k\"orperliche,
					als auch die geistige und geistliche Nahrung.
					Es ist weniger wichtig,
					wie lange du lebst,
					sondern dass es dir,
					in der Zeit,
					in der du am Leben bist,
					auch so gut wie m\"oglich geht.
		\end{itemize}
	
	\subsection{Zusammenfassung}
		Gek\"urzt kann man die 10 \textit{(An-)}Gebote wie folgt zusammenfassen:
		\\
		\begin{enumerate}[nosep]
			\item Ich will keine anderen G\"otter neben {\Dir} haben.
			\item Ich will {\Deinen} Namen nicht missbrauchen.
			\item Ich will {\Dich} nicht versuchen.
			\item Ich will {\Dir} den Sabbat heiligen.
			\item Ich will Vater und Mutter ehren.
			\item Ich will nicht t\"oten \textit{(oder morden)}.
			\item Ich will nicht ehebrechen.
			\item Ich will nicht stehlen \textit{(oder betr\"ugen)}.
			\item Ich will nicht falsch Zeugnis geben \textit{(wider meinem N\"achsten)}.
			\item Ich will mein Leben, {\Dein} Geschenk, ehren.
		\end{enumerate}
	
	\newpage
	\section{Tugenden und (Tod-)S\"unden}
		Im folgenden will ich hier auf die Sieben Tugenden,
		und die Sieben Tods\"unden eingehen.
		Ich habe darauf geachtet,
		nach M\"oglichkeit Alternativbezeichnungen zu w\"ahlen.
		Es kann aber durchaus sein,
		dass sie offiziell anders hei{\ss}en.
	
	\subsection{Die Sieben Tugenden}
	
	\subsubsection{Die Erste Tugend}
		\textbf{Die erste Tugend ist Bescheidenheit,
		Demut,
		Devotion und Dienstwille.}
		Sei also im Allgemeinen bereit deinem N\"achsten zu dienen,
		ohne eine Gegenleistung zu erwarten.
		Ich wei{\ss},
		dass es in unserer Gesellschaft schwierig ist,
		das immer umzusetzen,
		da man ja bspw. eine Arbeitsstelle ben\"otigt,
		um sich von seinem Gehalt oder Lohn sein Leben zu finanzieren.
		Aber bspw. unter Freunden muss man nicht immer auf eine Gegenleistung hoffen.

	\subsubsection{Die Zweite Tugend}
		\textbf{Die zweite Tugend ist Hochachtung,
		Liebe,
		Mildt\"atigkeit,
		N\"achstenliebe,
		Teuerung,
		Wohlt\"atigkeit
		und Wohlwollen.}
		Sch\"atze deine Mitmenschen,
		liebe und respektiere sie.
		Jeder hat bspw. auch schwierige Zeiten,
		also steh ihnen Liebe und Wohlwollen beiseite.
	
	\subsubsection{Die Dritte Tugend}
		\textbf{Die dritte Tugend ist Abstinenz,
		Enthaltsamkeit,
		Keuschheit,
		M\"a{\ss}igung und Triebverzicht.}
		Besonders wenn du den Sabbat halten willst,
		h\"altst du dich erst recht fern von bspw. Alkohol,
		oder anderen berauschenden Substanzen.
		Aber auch au{\ss}erhalb des Sabbats,
		ist es nicht sch\"oner seinen Trieb mit seinem Partner auszuleben,
		den man liebt,
		anstelle t\"aglich wechselnde Partner zu haben.
		Ist letzteres nicht zu stressig?

	\subsubsection{Die Vierte Tugend}
		\textbf{Die vierte Tugend ist Geduld,
		Gelassenheit,
		Hoffnung,
		Langmut und Standhaftigkeit.}
		Und ja,
		genau hier darf auch ich mich an die eigene Nase fassen.
		Dort darf ich selbst an mir arbeiten,
		denn wenn etwas nicht gelingen will,
		und wieder nicht,
		und immer wieder nicht,
		verliere ich schon schnell die Geduld.
		Also,
		lass uns gemeinsam daran arbeiten!

	\subsubsection{Die F\"unfte Tugend}
		\textbf{Die f\"unfte Tugend ist Ma{\ss},
		M\"a{\ss}igkeit und M\"a{\ss}igung.}
		Von allem,
		was man haben kann oder will,
		muss es ja nicht immer zu viel sein.
		Wenn du Lust auf Schokolade hast,
		reicht nicht ein St\"uck,
		anstelle einer ganzen Tafel?
		Wenn du Lust auf Wein hast,
		reicht nicht ein Glas,
		anstelle einer ganzen Flasche?
		Wenn du deine Mahlzeit zu dir nimmst,
		reicht es nicht zu essen,
		bis du satt bist,
		anstelle alles reinzustopfen,
		hauptsache,
		du hast aufgegessen?		

	\subsubsection{Die Sechste Tugend}
		\textbf{Die sechste Tugend ist Benevolenz,
		Dankbarkeit,
		Empathie,
		Gunst,
		Offenheit,
		Solidarrit\"at,
		Sympathie und Wohlwollen.}
		Auch in schwierigen Zeiten,
		und besonders auch danach,
		darfst du dankbar sein.
		Du hast sie schlie{\ss}lich \"uberstanden.
		Du hast ein ganzes Leben voller M\"oglichkeiten geschenkt bekommen,
		wof\"ur du dankbar sein darfst.
		Und gib diese Dankbarkeit deinen Mitmenschen weiter.
		Das macht dich sympathisch,
		und verhilft euch zu gegenseitiger Offenheit und Wohlwollen.
	
	\subsubsection{Die Siebte Tugend}
		\textbf{Die siebte Tugend ist Eifer,
		Flei{\ss},
		Kampfeseifer und Zielstrebigkeit.}
		Mit \q{Kampf} ist nat\"urlich kein echter Kampf,
		oder ein Streit gemeint.
		Es ist eher im symbolischen Sinne gemeint.
		Arbeite flei{\ss}ig an einer Sache,
		und bleib dran,
		auch wenn es manchmal schwerf\"allt.
		Du darfst auch Pausen machen,
		um dich zu erholen und neue Erergie zu sch\"opfen.

	\subsection{Die Sieben Tods\"unden}
	
	\subsubsection{Die Erste Tods\"unde}
		\textbf{Die erste Tods\"unde ist Eitelkeit,
		Hochmut,
		Stolz und \"Ubermut.}
		Das hei{\ss}t nicht,
		dass du auf nichts mehr stolz sein darfst.
		Es geht vielmehr darum,
		dass manche Menschen auf die falschen Dinge \q{stolz} sind,
		f\"ur die sie nichts getan haben,
		z.B. ihre Nationalit\"at,
		und dann sogar mit etwas angeben,
		um sich damit zu schm\"ucken.
		Lass dich f\"ur eine gute Leistung loben,
		aber h\"ang es nicht an die gro{\ss}e Glocke!

	\subsubsection{Die Zweite Tods\"unde}
		\textbf{Die zweite Tods\"unde ist Geiz,
		Habgier und Habsucht.}
		Es ist schon in Ordnung,
		Geld zu haben.
		In unserer modernen Gesellschaft kommt man ohne nicht mehr aus.
		Aber sei doch mal ganz ehrlich zu dir:
		Brauchst du wirklich Millionen und Aber-Millionen auf deinem Konto?
		Oder wieviel reicht dir wirklich f\"ur einen vern\"unfitgen Lebensstandard,
		mit dem dennoch mehr als nur \"uberleben kann?
		Und wenn du etwas \"ubrig hast,
		sei bereit,
		es mit deinem N\"achsten zu teilen!
	
	\subsubsection{Die Dritte Tods\"unde}
		\textbf{Die dritte Tods\"unde ist Ausschweifung,
		Genussucht,
		Begehren,
		Unkeuschheit und Wollust.}
		Brauchst du das wirklich f\"ur dich,
		bspw. jedes Wochenende Saufen bis der Arzt kommt?
		Brauchst du jeden Tag jemand anderes im Bett?
		W\"are es nicht viel sch\"oner,
		sich in einem sicheren Hafen zu wissen?
	
	\subsubsection{Die Vierte Tods\"unde}
		\textbf{Die vierte Tods\"unde ist Ungeduld,
		J\"azorn,
		Rachsucht,
		Wut und Zorn.}
		Wenn dich jemand oder etwas verletzt hast,
		ist es in Ordnung die Emotionen zuzulassen.
		F\"uhle sie,
		aber lass sie nicht an deinem N\"achsten aus!
		Gib bescheid und zieh dich lieber zur\"uck.
		Man kann sich darin \"uben,
		eine gewisse emotionale Intelligenz zu entwickeln.
		Aber wenn du w\"utend bist,
		ist und war es bereits in dir,
		und deine Mitmenschen k\"onnen nichts daf\"ur!

	\subsubsection{Die F\"unfte Tods\"unde}
		\textbf{Die f\"unfte Tods\"unde ist Gefr\"a{\ss}igkeit,
		Ma{\ss}losigkeit,
		Selbstucht,
		Unm\"a{\ss}igkeit und V\"ollerei.}
		Reicht es nicht,
		zu essen bist du satt bist?
		Muss es gar ein riesiger,
		\"uberf\"ullter Teller sein?
		Muss denn tats\"achlich Masse \"uber Klasse gehen?
		Und wenn du schon etwas \"ubrig hast,
		kannst du gut und gerne davon abgeben!
	
	\subsubsection{Die Sechste Tods\"unde}
		\textbf{Die sechste Tods\"unde ist Eifersucht,
		Missgunst und Neid.}
		Dein N\"achster hat es also nicht verdient,
		auch etwas zu haben?
		Warum denn nicht?
		Bist du selbst nicht in der Lage,
		es dir selbst zu erarbeiten?
		Oder was steckt hinter deinem Neid?

	\subsubsection{Die Siebte Tods\"unde}
		\textbf{Die siebte Tods\"unde ist Faulheit,
		Feigheit,
		Ignoranz,
		Tr\"agheit und \"Uberdruss.}
		Verwechsle das nicht mit dem Pausieren!
		Wenn du ersch\"opft bist,
		von deiner Arbeit,
		dann erhol dich,
		bis du wieder Energie zum weiterarbeiten hast.
		Aber ist es nicht beispielsweise ungerecht,
		daheim herumzusitzen,
		sich vom Staat bezahlen zu lassen,
		und von anderen aushalten zu lassen?

	\subsection{Gegen\"uberstellung}
		Die Tugenden und die Todsünden kann man also wie folgt gegen\"uberstellen:
		\begin{enumerate}
			\item \textbf{Bescheidenheit $\Longleftrightarrow$ Eitelkeit}
			\item \textbf{Wohlwollen $\Longleftrightarrow$ Habsucht}
			\item \textbf{Keuschheit $\Longleftrightarrow$ Wollust}
			\item \textbf{Gelassenheit $\Longleftrightarrow$ Wut}
			\item \textbf{M\"a{\ss}igung $\Longleftrightarrow$ Gefr\"a{\ss}igkeit}
			\item \textbf{Dankbarkeit $\Longleftrightarrow$ Missgunst}
			\item \textbf{Flei{\ss} $\Longleftrightarrow$ Faulheit}
		\end{enumerate}

	\newpage
	\section{Komme ich in den {\Himmel}?}
	
	\subsection{Ganz allgemein gesagt}
		Ob wir in den {\Himmel} kommen,
		das liegt wohl letztenendes komplett bei {\Gott} {\Selbst}.
		Ich will mir hier nicht anma{\ss}en,
		zu behaupten,
		inwiefern wir f\"ur irgendwelche S\"unden bestraft werden.
		Ich glaube eher,
		dass wir uns m\"oglicherweise schon f\"ur unsere Taten rechtfertigen d\"urfen.
		Ich sage nicht,
		dass wir tats\"achlich \q{bestraft} werden,
		und ich sage auch nur \q{m\"oglicherweise}.
		Ich hoffe schon darauf,
		dass Gott g\"utig und gn\"adig ist,
		und uns unsere S\"unden vergibt.
		Uns es geht auch nicht darum,
		ob du,
		lieber Leser,
		speziell jetzt an {\Jesus} glaubst,
		oder nicht.
		Aber wir selbst k\"onnen uns von unseren S\"unden nicht befreien,
		denn was wir getan haben,
		haben wir getan.
		Das l\"asst sich nicht einfach ungeschehen machen.
		
	\subsection{Wir sind alle S\"under}
		Ich m\"ochte hier nochmal auf {\Gottes} \textit{(An-)}Gebote,
		und auf die sieben Tugenden und Tods\"uden eingehen.
		Sei gerne 'mal ganz ehrlich zu dir selbst.
		\\
		\begin{itemize}[nosep]
			\item	Hast du,
					im weitesten Sinne,
					nicht schonmal einen anderen \q{Gott},
					auch im symbolischen Sinne,
					angebetet oder ihn gelobt?
					Dazu geh\"ort auch das \q{verg\"ottern} von Personen und Dingen.
			\item	Hast du dir nicht schonmal vorgestellt,
					wie {\Gott} aussehen k\"onnte?
					Das nennt man auch,
					sich ein G\"otzen- bzw. Gottesbild machen.
			\item	Hast du nicht schonmal eine dieser typischen,
					umgangssprachlichen Phrasen,
					wie \qq{Oh (mein) G...} verwendet?
					Und zwar au{\ss}erhalb eines Gebetes,
					ohne {\Gott} zu meinen?
					Nur,
					weil man das \q{halt so sagt}?
					Man kann ja auch schlie{\ss}lich sowas wie
					\qq{Ach, du meine G\"ute.}
					sagen.
			\item	Hast du nicht schonmal vor Verzweiflung,
					oder aus anderen Gr\"unden,
					sowas \"ahnliches wie
					\qq{Wie kann {\Gott} solches Leid zulassen?}
					gedacht oder sogar gesagt?
			\item	Sind dir Feiertage wirklich heilig,
					oder \q{feierst} du sie nur wegen dem Kommerz?
					Hauptsache an Ostern ein sch\"ones Osternest gesucht,
					und hauptsache an Weihnachten ein sch\"ones Geschenk bekommen?
			\item	Warst du schonmal respektlos gegen\"uber deinen Eltern,
					entweder direkt oder indirekt?
					Hast du auf deine Eltern geschimpft?
					K\"onnte es nicht sein,
					dass sie dennoch ihr Bestes geben,
					und es auch das ein oder andere Mal an dir lag,
					oder an ganz anderen Umst\"anden,
					auf die deine Eltern keinen Einfluss haben?
			\item	K\"ummerst du dich genug um deine Beziehungen?
					Und damit meine ich nicht nur bspw. die Beziehung zum (Ehe-)Partner.
					Ich meine jegliche zwischenmenschliche Beziehung:
					Deine Eltern,
					deine Kinder (wenn du welche hast),
					deine Freunde,
					deine Kollegen,
					und und und.
			\item	Hast du schon get\"otet oder gar gemordet?
					Bist du eventuell Soldat oder \"ahnliches?
					Wenn ja,
					wer bist du,
					dass du dir herausnimmst,
					\"uber Leben oder eher Ableben deines \q{Feindes} zu entscheiden?
			\item	Hast du schonmal jemanden beleidigt oder gekr\"ankt?
					Hast du bspw. im Stra{\ss}enverkehr jemandem \qq{Idiot} hinterhergerufen,
					weil er dir die Vorfahrt genommen hat?
			\item	Hast du schon ein Insekt,
					z.B. eine Fliege,
					oder anderes kleines Getier get\"otet,
					nur weil es dir l\"astig war?
			\item	Bist du schonmal fremdgegangen?
					Wenn nicht,
					hast du zumindest schonmal einem anderen Mann,
					oder einer anderen Frau hinterhergeschaut,
					weil er oder sie dir gefallen hat?
			\item	Und jetzt 'mal ganz konkret:
					Wie stehst du zu Pornographie,
					besonders wenn du dich in einer Beziehung befindest?
					Was w\"urdest du von deinem Partner halten,
					wenn er solche Inhalte konsumieren w\"urde?
			\item	Hast du jemandem schonmal etwas weggenommen?
					Auch hier an dich, falls du Vater,
					Mutter oder Lehrer bist:
					Hast du deinem Sohn,
					deiner Tochter,
					oder deinem Sch\"uler,
					schonmal vor\"ubergehend etwas wegenommen,
					weil er (oder sie) \q{unartig} war?
					Auch Kinder haben das Recht auf Eigentum.
					Wer also bist du,
					ihm oder ihr etwas wegzunehmen?
			\item	Warst du schonmal neidisch,
					weil jemand etwas hatte,
					was du auch gerne gehabt h\"attest?
					Warum g\"onnst du ihm das nicht?
			\item	Hast du schon gelogen?
					Auch wenn du dich f\"ur einen (realtiv) ehrlichen Menschen h\"altst,
					was sagst du,
					wenn dich jemand fragt:
					\qq{Wie geht es dir?}
					Sagst du einfach nur \qq{Gut.} aus H\"oflichkeit,
					um dein Gegen\"uber nicht zu belasten,
					und in Wirklichkeit geht es dir total schlecht?
			\item	Schimpfst du,
					wnn du die Geduld verlierst?
					Oder verwendest du generell viele Schimpfw\"orter,
					egal in welchem Zusammenhang?
			\item	Brauchst du immer das neueste,
					technische Ger\"at,
					sei es ein Smartphone,
					eine Smartwatch,
					oder was auch immer,
					nur um deinem N\"achsten zu imponieren?
					Brauchst du das wirklich?
			\item	Ist bei dir alles nur ein ewiges Geben-und-Nehmen?
					Oder kannst du auch etwas tun,
					ohne gleich eine Gegenleistung zu verlangen?
			\item	Respektierst du deine Mitmenschen?
					Oder f\"allst du dir oft vorschnell Vorurteile?
					Du musst nicht jeden sympathisch finden,
					aber jeder Mensch hat seine eigene Lebensgeschichte.
					Urteile \"uber niemanden,
					in dessen \q{Schuhe} du nicht mindestens einen Tag lang gelaufen bist!
			\item	Wie oft hast du gerne 'mal das ein oder andere Glas Bier oder Wein zuviel getrunken?
					Warum hast du das n\"otig?
					Damit es dir am n\"achsten Tag besch... \textit{(sehr schlecht)} geht?
			\item	Brauchst du jeden Tag jemanden anderes im Bett?
			\item	Bist du sehr geduldig,
					oder rastest du leicht aus?
			\item	Isst du beim Essen immer ganz auf,
					weil man es als Kind beigebracht hat,
					auch wenn du schon satt bist?
					Oder kannst du dich m\"a{\ss}igen,
					und von vornherein weniger auf deinen Teller tun?
			\item	Bist du dankbar f\"ur deinen K\"orper?
					Akzeptierst du ihn grunds\"atzlich?
					Oder schadest du ihm bspw. mit Rauchen oder anderem?
			\item	Bist du insgesamt dankbar f\"ur dein Leben,
					oder hast du an allem etwas auszusetzen?
					Denk daran,
					was du alles geschafft und \"uberstanden hast,
					um bis hierhin zu kommen!
			\item	H\"altst du bis zum Ende durch?
					F\"uhrst du deine Projekte bis zum Ende,
					oder gibtst du mittendrin,
					auf halber Strecke auf?
			\item	Bist du stolz darauf,
					einer bestimmten Nation anzugeh\"ren,
					ein Mann oder eine Frau zu sein,
					ein Kind,
					ein Erwachsener,
					oder sonstiges,
					wof\"ur du nichts kannst?
					Es ist in Ordnung auf eigene Leistungen stolz zu sein.
					Aber was hast du daf\"ur getan,
					um z.B. als Mann geboren zu sein?
			\item	Beh\"altst du alles f\"ur dich und gibst nichts ab?
					Brauchst du alles f\"ur dich allein?
					Insbesondere beim Thema Geld?
					Hast du nicht vielleicht den ein oder anderen \q{Groschen} \"ubrig,
					um deinen Mitmenschen zu helfen?
			\item	Bist du st\"andig w\"utend auf deine Mitmenschen,
					ohne Grenzen?
					Denkst du dir stets,
					dass du ihnen nicht vergeben kannst,
					egal,
					was vorgefallen ist?
					\\
		\end{itemize}
		Mir w\"urde sicher noch einiges mehr einfallen,
		aber das w\"urde den Rahmen sprengen.
		Und ja,
		ich darf mich auch an die eigene Nase fassen!
		
	\subsection{Was kann ich tun?}
		Wie bereits erw\"ahnt,
		was du getan hast,
		hast du getan.
		Du kannst nichts tun,
		um es ungeschehen zu machen.
		Sowohl deine guten Taten,
		als auch deine schlechten.
		Deine Erfolge,
		und deine \q{Fehler}.
		Du kannst von dir aus nichts tun,
		um {\Gott} zu \q{gefallen}.
		Du kannst nur {\Sein} Geschenk annehmen,
		{\Seine} Gnade.
		Wenn du dich f\"ur ein Leben mit {\Gott} entscheidest,
		und {\Jesus} als deinen {\Erloeser} annimmst,
		dann nimmst du {\Gottes} Geschenk an.
		Lies gerne dazu folgende Bibelstellen durch:
		\href{https://www.die-bibel.de/bibeln/online-bibeln/lesen/LU17/JHN.14/Johannes-14}{Johannes 14, 6},
		\href{https://www.die-bibel.de/bibeln/online-bibeln/lesen/LU17/ACT.4/Apostelgeschichte-4}{Apostelgeschichte 4, 12},
		\href{https://www.die-bibel.de/bibeln/online-bibeln/lesen/LU17/ROM.3/Römer-3}{R\"omer 3,23-24},
		\href{https://www.die-bibel.de/bibeln/online-bibeln/lesen/LU17/GAL.2/Galater-2}{Galater 2, 16} und \href{https://www.die-bibel.de/bibeln/online-bibeln/lesen/LU17/EPH.2/Epheser-2}{Epheser 2, 8-9}.
		\\
		\textit{Und jetzt kommt etwas GANZ wichtiges:}
		Du \textbf{darfst} dennoch {\Gottes} \textit{(An-)}Gebote halten,
		aber achte darauf,
		welches Motiv dahinter steckt!
		Wenn du sie nur einh\"altst,
		in der Hoffnung,
		dass du in den {\Himmel} kommst,
		hast du das falsch verstanden.
		Doch der Fehler liegt nicht bei dir.
		Wir leben (leider) in einer Welt des st\"andigen Gebens und Nehmens.
		Einfaches Beispiel:
		Du gehst jeden Tag in die Arbeit,
		und am Monatsende bekommst du dein Geld daf\"ur.
		Und dann bezahlst du deine Miete,
		damit du weiterhin in deiner Wohnung leben darfst.
		Du bezahlst Strom und Gas,
		damit du Licht hast,
		und im Winter nicht frierst.
		Und du kaufst Lebensmittel,
		damit du nicht verhungerst.
		Also immer Leistung und Gegenleistung.
		Unsere Gesellschaft funktioniert halt nunmal so.
		Und das \"ubertragen wir f\"alschlicherweise auf {\Gott}.
		Aber {\Gott} \textbf{schenkt} dir {\Seine} Gnade und G\"ute,
		weil {\Er} dich liebt.
		Und zwar so wie du bist.
		Du musst nichts,
		absolut nichts daf\"ur tun.
		Aber wir alle denken
		- und ich bin da keine Ausnahme
		- wir sind es aus irgendeinem Grund nicht wert,
		von {\Gott} geliebt zu werden,
		weil wir ja dauernd s\"undigen.
		Hiermal eine kleine (Not-)L\"uge,
		da mal jemanden aus Wut beschimpft.
		Und dann halten wir uns f\"ur ungeliebt.
		Aber es gibt eine gute Nachricht:
		{\Jesus} ist f\"ur dich,
		und deine S\"unden am Kreuz gestorben.
		Das einzige,
		was du \q{tun} brauchst,
		ist,
		{\Ihn} in dein Leben zu lassen,
		{\Ihm} dein Herz zu \"offnen.
		Denn {\Er} liebt dich bedingungslos.
		Und genau so solltest du auch handeln.
		Wenn du {\Gottes} Gebote einhalten willst,
		nicht \q{damit} du in den {\Himmel} kommst.
		Nein,
		einfach aus Dankbarkeit,
		weil {\Gott},
		weil {\Jesus} dieses riesige Opfer gebracht hat,
		das niemand je wieder gutmachen k\"onnte.
		Halte nicht seine Gebote ein,
		und erwarte oder verlange dann in den {\Himmel} zu kommen!
		Achte auch auf folgendes:
		Nicht,
		\q{wenn} du {\Gott} liebst, h\"altst du seine Gebote,
		sondern \q{weil} du {\Ihn} liebst.
		Du darfst dich aber auch nicht darauf ausruhen,
		indem du dir sinngem\"a{\ss} so etwas sagst wie:
		\q{Naja,
		{\Jesus} ist ja sowieso auch f\"ur mich gestorben,
		jetzt kann ich s\"undigen ohne Ende.}
		Das w\"are ebenfalls egoistisch,
		in Bezug auf die Beziehung zu {\Gott}.
		Es mag vielleicht schwierig sein,
		das alles zu unterscheiden.
		Aber achte zumindest darauf,
		wenn du bewusst {\Gottes} Gebote einhalten willst,
		warum du es tust.
		Tust du es gerne,
		aus Dankbarkeit zu {\Jesus}?
		Oder erhoffst du dir \q{ein St\"uckchen} {\Himmel}?
		
	\subsection{Zusammenfassung}
		Von daher kann ich nur sagen,
		hoffe ich einfach auf {\Gottes} Gnade,
		und auf {\Seine} G\"ute.
		Ich hoffe,
		dass wenn wir es ernst mit {\Ihm} meinen,
		und wir {\Seinen} Weg gehen,
		{\Seinen} Willen erf\"ullen,
		dann wird sich {\Gott} auch gn\"adig zeigen.
		Und mit \q{{\Seinen} Willen} meine ich nicht,
		dass wir willenlose Marionetten sind.
		Schlie{\ss}lich schenkte uns {\Gott} einen eigenen Willen,
		von {\Sich} aus,
		in {\Seiner} G\"ute.
		Das hei{\ss}t,
		wir k\"onnen selbst entscheiden,
		ob wir mit {\Ihm} leben oder nicht.
		Das hei{\ss}t lediglich,
		dass wir {\Seine} Gebote befolgen,
		und ehrlich und liebevoll miteinander umgehen.

	\newpage
	\section{Verschiedene S\"unden genauer betrachtet}
		In diesem Kapitel m\"ochte ich auf ein paar S\"unden genauer eingehen.
		Manche davon macht man vielleicht bewusst,
		manche unbewusst,
		manche sind schwer zu vermeiden,
		und so weiter.
		Ich m\"ochte dir einfach nur zeigen,
		dass das Leben nicht immer Schwarz-Wei{\ss} ist.
		Es gibt oftmals nicht einfach nur \q{Gut} und \q{B\"ose}.
		Manchmal gibt es Grauzonen.
		Und ich bin mir sicher,
		dass sieht {\Gott} genauso.
		Deswegen gab {\Er} uns ja den freien Willen,
		damit wir selbst entscheiden k\"onnen.
		Damit wir selbst die Erfahrungen machen,
		und daraus lernen k\"onnen.
	
	\subsection{Vom T\"oten}
		Hier m\"ochte ich vom tats\"achlichen,
		physischen T\"oten eingehen,
		und auch vom Verletzen,
		also nicht im \"ubertragenen Sinne,
		wie es {\Jesus} beschrieben hat
		\textit{(siehe \href{https://www.die-bibel.de/bibeln/online-bibeln/lesen/LU17/MAT.5/Matthäus-5}{Matthäus 5, 21-26})}.
		Das Thema kann sehr sensibel sein,
		und auch hier gibt es kein einfaches Schwarz-Wei{\ss}-Schema,
		im Sinne von \q{T\"oten ist immer falsch}.
		Ich m\"ochte hier lediglich das Bewusstsein st\"arken,
		und meine eigene Meinung dazu \"au{\ss}ern.
	
	\subsubsection{T\"oten von Mitmenschen - im Allgemeinen}
		Doch, fangen wir klein an.
		Es kann eventuell sein,
		dass du mir so etwas \"ahnliches sagen m\"ochtest,
		wie \q{Ich t\"ote nicht.
		Und ich habe noch nie get\"otet!
		Ich bin doch kein M\"order!}
		Und auf den Gro{\ss}teil der Menschen mag das bestimmt zutreffen,
		dass sie nicht einfach hinausgehen,
		und wahllos jeden umbringen,
		der ihnen entgegenkommt.
		Und auch ich bin grunds\"atzlich gegen T\"oten.
		Ich m\"ochte selbst leben d\"urfen,
		und will von daher auch nicht get\"otet werden.
		Und auch ich sage,
		dass jeder meiner Mitmenschen das Recht hat,
		egal ob juristisch,
		moralisch,
		oder wie auch immer,
		leben zu d\"urfen.
		Auch wenn man nicht jeden mag oder sympathisch findet,
		es sollte zumindest soviel N\"achstenliebe und Respekt da sein,
		dass der andere Leben darf.
		Und ich finde,
		kein Mensch hat - zun\"achst - das Recht,
		\"uber das Leben,
		oder Nicht-Leben eines anderen Menschen zu entscheiden.
		\\
		Und auch keine Authorit\"at.
		Deswegen halte ich auch Kriege f\"ur falsch.
		Das gr\"o{\ss}ere Problem ist auch noch,
		dass diejenigen,
		die den Krieg wollen,
		selbst nicht aktiv daran teilnehmen,
		indem sie selbst an die Front gehen.
		Sprich,
		ich meine entsprechene Politiker,
		Pr\"asidenten,
		fr\"uher die K\"onige,
		und wie sie nicht alle hei{\ss}en.
		Und auch Soldaten,
		die aktiv an der Front k\"ampfen,
		und andere Menschen t\"oten,
		sind in meinen Augen M\"order.
		Es gibt durchaus Stimmen,
		die behaupten,
		Soldaten h\"atten ja gar keine Wahl,
		sie befolgen einfach nur Befehle.
		\\
		Doch!
		Wenn mir jemand den Befehl erteilt,
		einen anderen Menschen zu t\"oten,
		kann ich jederzeit \q{Nein!} sagen.
		Oder ich kann von vornherein sagen,
		dass ich nicht zum Milit\"ar gehe,
		oder zumindest keine Waffe in die Hand nehme.
		\\
		Ich halte auch die Todesstrafe f\"ur falsch.
		Wenn ich einen M\"order damit bestrafe,
		dass ich ihn umbringe,
		bin ich dann besser als er?
		Klar,
		man sollte einen M\"order schon zeigen,
		dass seine Taten Konsequenzen haben.
		Aber wer bin ich,
		dass ich dar\"uber bestimmen w\"urde,
		ob auch er leben oder sterben soll?
		Ich finde nur einer hat dieses Recht,
		und das ist {\Gott} allein.
		Und wenn es soweit sein sollte,
		wird {\Er} uns schon zu {\Sich} holen.
		
	\subsubsection{T\"oten von Mitmenschen - im Speziellen}
		In diesem Abschnitt wird es etwas schwieriger.
		Denn hier habe ich selbst Schwierigkeiten,
		mir eine konkrete Meinung zu bilden.
		Denn es gibt wenige,
		aber wirklich nur wenige F\"alle,
		in denen es m\"oglicherweise sein kann,
		dass beispielsweise ein anderer Mensch \"uber Leben oder Tod eines anderen Menschen entscheiden \q{muss}.
		Ich sagte zwar im vorherigen Abschnitt,
		dass nur {\Gott} das Recht hat,
		\"uber Leben und Tod zu entscheiden.
		Das hei{\ss}t,
		ich w\"urde mir jetzt m\"oglicherweise selbst widersprechen.
		Ein Szenario,
		was ich mir vorstellen k\"onnte,
		was in etwa in diese Richtung ginge,
		w\"are,
		wenn jemand im Krankenhaus im Koma liegt,
		und das nicht erst seit einer Woche,
		sondern vielleicht sehr viele Monate,
		vielleicht sogar schon ein paar Jahre.
		Und die Person wird nur mithilfe von Maschinen am Leben erhalten.
		Und die \"Arzte sehen kaum noch Chancen,
		dass die Person je wieder aufwacht.
		Und,
		bei allem Respekt gegen\"uber Leben und Mitmenschen,
		aber im Krankenhaus zu liegen kostet ja auch Geld.
		Und in diesem Fall w\"urden vermutlich die Angeh\"origen die Kosten \"ubernehmen,
		wenn sie nicht eine gute Versicherung haben.
		Oder es k\"onnte auch sein,
		dass die Versicherung irgendwann gar nicht mehr bezahlt,
		und die Kosten sind f\"ur die Angeh\"origen auf Dauer schwer zu tragen.
		Was ist,
		wenn man diese Person einfach nicht mehr aus dem Koma holen kann?
		Soll man einfach die Maschinen abstellen,
		und die Person stirbt?
		Was,
		wenn trotz geringer Chancen die Person dennoch wieder gekommen w\"are?
		Viele Fragen kann man hier stellen,
		und viele M\"oglichkeiten mit vielen Variationen kann es geben.
		Ehrlich gesagt,
		w\"unsche ich mir,
		dass es in meinem Leben niemals der Fall sein wird,
		dass ich vor die Wahl gestellt werde,
		ob jemand anders leben oder sterben soll.
		\\
		Ein andere Szenario ist das sogenannte Trolley-Problem,
		ein Gedankenexperiment,
		vereinfacht erkl\"art,
		bei dem ein Zug auf eine Gruppe von bspw. f\"unf Menschen zurast,
		welche das aber nicht mitbekommen.
		Und auf einem anderen Gleis steht auch ein Mensch,
		der das ganze ebenfalls nicht mitbekommt.
		Und durch Stellen der Weichen,
		hat man die Gelegenheit,
		dass der Zug das andere Gleis nimmt,
		und stattdessen der einzelne Mensch stirbt.
		Die Entscheidung ist dann,
		l\"asst man f\"unf Menschen sterben,
		oder einen einzigen,
		weil das dann vier weniger w\"aren?
		\\
		Das waren jetzt ein paar Spezialf\"alle vom T\"oten von Mitmenschen.
		
	\subsubsection{Andere Formen des T\"otens}
		
	
	\newpage
	\section{Gebete und Loblieder}
		Ich m\"ochte dir in diesem Kapitel ein paar sch\"one Gebete,
		oft in der Form \q{Ich spreche zu {\Gott}},
		und Loblieder anbieten.
	
	\subsection{Das {\Vater}-Unser}
		Wie auch bspw. bei den 10 Geboten lehne ich das {\Vater}-Unser,
		so wie es in der Bibel steht,
		auf keinen Fall ab.
		Auch hier m\"ochte ich dir eine pers\"onlichere Form zeigen,
		die auch weniger gebietend ist.
		Ich finde n\"amlich,
		dass dort zu viele Imperativformen enthalten sind.
		Anstatt \q{geheiligt werde {\Dein} Name} zu beten,
		ist es besser zu sagen \q{geheiligt ist {\Dein} Name},
		denn {\Gottes} Name \textbf{ist} heilig.
		Oder anstelle von \q{{\Dein} Wille geschehe},
		besser \q{{\Dein} Wille wird geschehen.},
		da {\Gott} existiert,
		und meiner Meinung nach,
		im Zweifel genau das geschieht,
		was er will,
		auch wenn wir es vielleicht nicht immer gleich erkennen,
		und schon gar nicht immer gleich verstehen.
	
	\subsubsection{Wenn du alleine betest}
		\begin{itemize}[nosep]
			\item	Mein {\Vater},
					\\
					(der) {\Du} bist im {\Himmel},
					\\
					geheiligt ist {\Dein} Name.
			\item	{\Dein} Reich wird kommen.
			\item	{\Dein} Wille wird geschehen,
					\\
					wie im {\Himmel},
					so auf Erden.
			\item	Mein t\"aglich' Brot gibst {\Du} mir heute.
			\item	Bitte,
					vergib mir meine Schuld,
			\item[]	und auch ich vergebe meinen Schuldigern.
			\item	{\Du} f\"uhrst mich nicht in Versuchung,
					\\
					sondern erl\"ost mich von dem B\"osen.
			\item	Denn {\Dein} ist das Reich,
					\\
					und die Kraft,
					\\
					und die Herrlichkeit,
					\\
					in Ewigkeit.
			\item	Amen!
		\end{itemize}
			
	\subsubsection{Wenn ihr in der Gruppe betet}
		\begin{itemize}[nosep]
			\item	Unser {\Vater},
					\\
					(der) {\Du} bist im {\Himmel},
					\\
					geheiligt ist {\Dein} Name.
			\item	{\Dein} Reich wird kommen.
			\item	{\Dein} Wille wird geschehen,
					\\
					wie im {\Himmel},
					so auf Erden.
			\item	Unser t\"aglich' Brot gibst {\Du} uns heute.
			\item	Bitte,
					vergib uns unsere Schuld,
					\\
					und auch wir vergeben unseren Schuldigern.
			\item	{\Du} f\"uhrst uns nicht in Versuchung,
					\\
					sondern erl\"ost uns von dem B\"osen.
			\item	Denn {\Dein} ist das Reich,
					\\
					und die Kraft,
					\\
					und die Herrlichkeit,
					\\
					in Ewigkeit.
			\item	Amen!
		\end{itemize}

	\subsection{{\Jesus}, komm in mein Leben}
		\begin{itemize}[nosep]
			\item	{\Jesus},
					ich m\"ochte,
					dass {\Du} in mein Leben kommst.
			\item	{\Jesus},
					ich \"offne {\Dir} meine \textit{(Herzens-)}T\"ur.
					\\
					Ich \"offne {\Dir} mein Herz.
			\item	{\Jesus},
					ich m\"ochte,
					\\
					dass mein Leben von {\Dir} gef\"uhrt wird,
					\\
					dass {\Du} die Leitung \"uber mein Leben \"ubernimmst.
			\item	{\Jesus},
					ich will mit {\Dir} leben,
					\\
					und ich glaube an {\Dich}.
			\item	Ich glaube an die Auferstehung;
					\\
					und ich glaube daran,
					\\
					dass {\Du} der Weg,
					die Wahrheit und das Leben bist.
			\item	{\Jesus},
					ich \"ubergebe {\Dir} mein Leben.
			\item	Amen!
		\end{itemize}

	\subsection{\"Ubergabegebet}
		\begin{itemize}[nosep]
			\item	{\Jesus},
					\\
					ich m\"ochte einfach nur verstehen,
					\\
					wer {\Du} bist.
			\item	Und {\Jesus},
					\\
					ich m\"ochte {\Dir} jetzt mein Leben geben.
			\item	Vielleicht verstehe ich nicht,
					\\
					was es bedeutet;
					\\
					und vielleicht verstehe ich nicht,
					\\
					warum ich das jetzt gerade brauche.
			\item	Vielleicht verstehe ich nicht,
					\\
					wer {\Du} bist,
					\\
					was meine Aufgabe ist,
					\\
					was mein Sinn auf dieser Welt ist.
			\item	Und {\Jesus},
					\\
					vielleicht wenn ich auch auf {\Dich} sauer.
					\\
					Vielleicht bin ich verletzt.
					\\
					Vielleicht gibt es Dinge,
					\\
					die für mich noch irgendwie im Wege stehen,
					\\
					um mich {\Dir} v\"ollig hinzugeben.
			\item	Aber {\Jesus},
					\\
					ich bete hiermit,
					\\
					dass {\Du} all das wegnimmst,
					\\
					all die Zweifel,
					\\
					und alles,
					\\
					was mich noch von {\Dir} trennt.
			\item	Und {\Jesus},
					\\
					ich bete,
					\\
					dass ich jetzt mein Leben in {\Deine} H\"ande behutsam legen kann,
					\\
					und {\Du} daraus machst,
					\\
					was f\"ur {\Deinen} Plan,
					\\
					und f\"ur {\Dein} Reich am besten ist.
			\item	Und ich bete,
					\\
					dass {\Du} mein Leben wieder in Ordnung bringst,
					\\
					und mir wieder Lebensfreude gibst,
					\\
					und Freude die von {\Dir} kommt,
					\\
					lebendiges Wasser in mir,
					\\
					das von {\Dir} kommt.
			\item	Und {\Jesus},
					\\
					\textit{ich bete,}
					\\
					dass {\Du} mich von Ketten befreist,
					\\
					aus denen ich jetzt gerade nicht 'rauskomme,
					oder wo ich noch drin stecke.
			\item	{\Jesus},
					\\
					lass mich {\Deine} Freiheit spüren,
					\\
					und schenk mir deinen Frieden.
			\item	Und nimm {\Du} mein Leben in {\Deine} Hand,
					\\
					und lass mich {\Dein} Kind werden;
					\\
					und lass mich {\Dich} im Himmel sehen.
			\item	Amen!
		\end{itemize}

	\subsection{Das Glaubensbekenntnis}
		Das Glaubensbekenntnis ist hier fast 1:1 \"ubernommen.
		Es fehlt lediglich der Teil mit der Kirche,
		da ein Freier Christ nicht an eine Kirchengemeinde gebunden ist.
		\\
		\begin{itemize}[nosep]
			\item	Ich glaube an {\Gott},
					\\
					den {\Vater},
					den {\Allmaechtigen},
					\\
					den {\Schoepfer} des {\Himmels} und der Erde.
			\item	Ich glaube an {\Jesus} {\Christus},
					\\
					{\Seinen} eingebohrenen {\Sohn},
					meinen {\Herrn},
					\\
					empfangen durch den {\Heiligen} {\Geist},
					\\
					geboren von der Jungfrau Maria,
					\\
					gelitten unter Pontius Pilatus,
					\\
					gekreuzigt,
					gestorben und begraben,
					\\
					hinabgestiegen in das Reich des Todes,
					\\
					am Dritten Tage auferstanden von den Toten,
					\\
					aufgefahren in den {\Himmel};
					\\
					{\Er} sitzt zur Rechten {\Gottes},
					\\
					des {\Allmaechtigen} {\Vaters};
					\\
					von dort wird {\Er} kommen,
					\\
					zu richten die Lebenden und die Toten.
			\item	Ich glaube an den {\Heiligen} {\Geist},
					\\
					an die Vergebung der S\"unden,
					\\
					an die Auferstehung der Toten,
					\\
					und das Ewige Leben.
			\item	Amen!
		\end{itemize}
		
	\subsection{Wie ein Fest nach langer Trauer}
		\subsubsection{Infos}
		\begin{itemize}[nosep]
			\item Liederbuch: Von {\Jesus} singen 2
			\item ISBN: 9783775123099
			\item Komponist: J\"urgen Werth
		\end{itemize}
		
		\subsubsection{Text}
		\begin{itemize}
			\item	\textbf{Strophe 1:}
			\\		Wie ein Fest nach langer Trauer,
			\\		wie ein Feuer in der Nacht.
			\\		Ein off'nes Tor in einer Mauer,
			\\		f\"ur die Sonne auf gemacht.
			\\		Wie ein Brief nach langem Schweigen,
			\\		wie ein unverhoffter Gru{\ss}.
			\\		Wie ein Blatt an toten Zweigen.
			\\		Ein \q{Ich mag dich trotzdem.}-Kuss.
			\item	\textbf{Strophe 2:}
			\\		Wie ein Regen in der W\"uste,
			\\		frischer Tau auf d\"urrem Land.
			\\		Heimatkl\"ange für vermisste.
			\\		Alte Feinde Hand in Hand.
			\\		Wie ein Schl\"ussel im Gef\"angnis,
			\\		wie in Seenot \q{Land in Sicht!}.
			\\		Wie ein Weg aus der Bedr\"angnis.
			\\		Wie ein strahlendes Gesicht.
			\item	\textbf{Strophe 3:}
			\\		Wie ein Wort von toten Lippen,
			\\		wie ein Blick der Hoffnung weckt.
			\\		Wie ein Licht auf steilen Klippen,
			\\		wie ein Erdteil neu entdeckt.
			\\		Wie der Fr\"uhling,
					wie der Morgen,
			\\		Wie ein Lied,
					wie ein Gedicht.
			\\		Wie das Leben,
					wie die Liebe,
			\\		Wie {\Gott} {\Selbst},
					das wahre Licht!
			\item	\textbf{Refrain \textit{(je 2x)}:}
			\\		So ist Vers\"ohnung,
			\\		so muss der wahre Friede sein.
			\\		So ist Vers\"ohnung,
			\\		so ist vergeben und verzeih'n.
		\end{itemize}

	\newpage
	\section{Mein Leben mit {\Gott}} \label{MeinLebenMitGott}
		Hierbei handelt es sich um eine Art unregelm\"a{\ss}iges \q{Tagebuch} im weitesten Sinn,
		wie ich meine Reise mit und zu {\Gott} erlebe,
		und was ich sonst noch dabei lernen darf.
	
	\subsection{Mittwoch, der 27. September 2023}
		Ich bin seit etwa Mitte 2023 auf einer Art Reise,
		bei der ich mich entschieden habe,
		{\Gott} und {\Jesus} in mein Leben zu lassen.
		Ich habe selbst noch viele Fehler,
		und obgleich der von {\Gott} gegebenen \textit{(An-)}Gebote,
		s\"undige ich noch viel zu oft.
		Wie im Vorwort erw\"ahnt,
		bin ich weit davon entfernt,
		so etwas wie der \q{perfekte Christ} zu sein.
		Viele der allt\"aglichen Gewohnheiten,
		Pr\"agungen und sonstiges haben so eine starke Sogwirkung,
		dass ich auch nicht immer an {\Gott} denke,
		nicht so oft bete,
		oder in der Bibel lese,
		wie ich gerne w\"urde.
		Und wenn ich dann \q{wieder} an {\Gott} denke,
		habe ich oft ein schlechtes Gewissen,
		weil ich {\Ihn} dann gef\"uhlt \q{vergessen} habe.
		Also kurzum:
		Ich darf noch sehr, sehr, sehr, ..., sehr viel lernen!
		
	\subsection{Freitag, der 29. September 2023}
		Heute habe ich mir ein Video angesehen,
		das mir sehr zu denken gegeben hat.
		Ich wei{\ss} nicht,
		ob es sich dabei um Gottesl\"asterung handelt.
		Trotzdem will ich mit dir teilen,
		was ich darin gesehen habe.
		Es war im Prinzip ein kurzer Trickfilm,
		in dem eine Muslimin,
		ein Atheist und ein Christ in den {\Himmel} gekommen sind.
		Da man hier nicht wirklich {\Gott} selbst gesehen hat,
		sondern lediglich eine Karikatur,
		werde ich hier die normale Gro{\ss}schreibung verwenden.
		Es ging also darum,
		von welchen Aussagen sich Gott beleidigt f\"uhlt.
		Und im Endeffekt hat er dem Atheisten seinen Frieden geschenkt,
		und ihn tats\"achlich in dem Himmel geschickt,
		weil dieser ja nie an Gott geglaubt hat,
		und ihm weder das eine,
		noch das andere nachgesagt hat.
		Und von der Muslimin und vom Christen war er entt\"auscht,
		weil sie im Prinzip in so \q{b\"ose} dargestellt haben,
		als ob er alle Menschen,
		die s\"undigen und nicht an ihn glauben,
		einfach in die H\"olle werfen w\"urde.
		Das hat ihn sehr verletzt,
		weil er sich effektiv wie ein grausames Monster gef\"uhlt hat.
		Das ist sozusagen die Kurzversion.
		Und das hat mir zu denken gegeben.
		Ich kann nat\"urlich nur spekulieren.
		Aber vielleicht ist es ja so,
		dass uns {\Gott} nirgendwo \q{hinschickt}.
		Wenn wir uns f\"ur {\Ihn} entscheiden,
		so l\"adt {\Er} uns auch nach dem Tode zu {\Sich} in den {\Himmel} ein.
		Und wenn wir uns beispielsweise f\"ur den Teufel entscheiden,
		dann kann es schon sein,
		dass wir in die H\"olle kommen.
		Aber nicht weil uns {\Gott} dort hinschickt,
		sondern weil der Teufel uns mitnimmt.
		Wie gesagt ... ich wei{\ss} es nicht.
		Das Video hat mich nur zum Nachdenken gebracht.
		Denn es ist sicher oft so,
		dass man vielleicht \"uber {\Gott} dieses und jenes sagt,
		aber es effektiv nicht wei{\ss},
		was die Wahrheit ist.
		Aber man kann ja zumindest erstmal nachdenken,
		wenn man \"uber {\Gott} etwas sagt,
		ob man selber wollen w\"urde,
		dass jemand anderes \"Ahnliches \"uber einen sagt.
		Falls du dir \"uber das Video selbst ein Urteil bilden m\"ochtest,
		hier der Link: \url{https://www.youtube.com/watch?v=ttevamkS6gw}.

	\subsection{Dienstag, der 3. Oktober 2023}
		Am vergangenen Wochenende,
		bis einschlie{\ss}lich gestern,
		war ich mit meiner Frau und meinen Eltern in Hamburg.
		Aus irgendeinem Grund ging es mir ab Sonntag im Laufe das Tages nicht so gut.
		Ich war u.a. \"uberm\"udet und gequ\"alt von Kopfschmerzen.
		Und ich vermute,
		auch mein Bewegungsmangel hat sich hier stark gezeigt,
		da ich jede Treppe als Qual empfunden habe.
		Wenigstens konnte ich morgens in meiner Bibel-App lesen,
		was auch schonmal viel Wert war.
		Zwischendrin kam mir ein Verdacht,
		woher m\"oglicherweise meine Kopfschmerzen kamen.
		Ich kann es aber nicht beweisen,
		es bleibt also bei einer Vermutung.
		Jedenfalls,
		da ich ja diesmal den Sabbat durchziehen wollte,
		habe ich sowohl u.a. auf Kaffee und potenziell zuckerhaltiges verzichtet.
		Meine Frau meinte zwar \"ofters,
		ich sollte einen Kaffee trinken,
		aber auf die Versuchung wollte ich gar nicht erst eingehen.
		Und ich habe ja 'mal geh\"ort,
		dass Zucker auch s\"uchtig machen kann.
		Und deswegen geh\"ort dies auch zu den Dingen,
		auf die ich am Sabbat verzichten will.
		Ich habe aber auch geh\"ort,
		dass bei manchen S\"uchten,
		z.B. bei Zucker,
		Kopfschmerzen eine Entzugserscheinung sein kann.
		Und wenn das stimmt,
		dann s\"undige ich au{\ss}erhalb des Sabbats ganz sch\"on viel,
		was das betrifft.
		Siehe auch: \q{\hyperref[DasZehnteAngebot]{Das Zehnte Angebot}}.
		
	\subsection{Samstag, der 7. Oktober 2023}
		Ich habe langsam das Gef\"uhl,
		es wird ernst.
		Nat\"urlich im positiven Sinne.
		Ich habe mich deswegen heute spontan dazu entschieden,
		die im Vorwort erw\"ahnte GitHub-Diskussion f\"ur dich vorzubereiten,
		und dieses Werk als monatliche Ausgabe zu releasen.
		Die erste ist die Oktoberausgabe,
		die ich vorhin ver\"offentlicht habe.
		Die Novemberausgabe sollte dann p\"unktlich zum Monatswechsel,
		bzw. zum 1. November erscheinen.
	
	\subsection{Sonntag, der 8. Oktober 2023}
		Mich hat ja jetzt leider nach dem Wochenende in Hamburg eine Erk\"altung erwischt.
		Ich m\"ochte aber dennoch mit dir heute einfach eine sch\"one Bibelstelle teilen.
		Es handelt sich um R\"omer 10,
		Verse 5 bis 11,
		mit der \"Uberschrift \q{Die Erl\"osung steht f\"ur alle bereit}
		aus der \q{Neues Leben Bibel}.
		Viel Freude beim Lesen.
		\begin{enumerate}[nosep,start=5]
			\item	Denn Mose schrieb,
					dass man alle Gebote des Gesetzes erf\"ullen muss,
					um durch das Gesetz vor {\Gott} gerecht zu werden.
			\item	Wer aber durch den Glauben vor {\Gott} bestehen will,
					dem sollt ihr sagen:
					\q{Du musst nicht in den {\Himmel} hinaufsteigen.},
					um {\Christus} zu finden und ihn herabzuholen.
			\item	Und:
					\q{Du musst nicht in die Tiefe hinabsteigen.},
					um {\Christus} wieder von den Toten heraufzuholen.
			\item	Denn in der Schrift hei{\ss}t es:
					\q{Die Botschaft ist dir ganz nahe;
					sie ist auf deinen Lippen und in deinem Herzen.}
					Es ist die Botschaft von der Erl\"osung durch den Glauben an {\Christus},
					die wir verk\"unden.
			\item	Wenn du mit deinem Mund bekennst,
					dass {\Jesus} der {\Herr} ist,
					und wenn du in deinem Herzen glaubst,
					dass {\Gott} {\Ihn} von den Toten auferweckt hat,
					wirst du gerettet werden.
			\item	Denn durch den Glauben in deinem Herzen wirst du vor {\Gott} gerecht,
					und durch das Bekenntnis deines Mundes wirst du gerettet.
			\item	So hei{\ss}t es in der Schrift:
					\q{Wer an {\Ihn} glaubt,
					wird nicht umkommen.}
		\end{enumerate}
				
	\newpage
	\section{Friede sei mit dir!}
		Zum Schluss m\"ochte ich noch einen wundersch\"onen Refrain mit dir teilen.
		Er ist aus dem Lied \q{Oceans} von der Band Hillsong United.
		Da der Text meines Wissens nach,
		zum Stand Oktober 2023,
		urheberrechtlich gesch\"utzt ist,
		m\"ochte ich mich dennoch auf das Zitatrecht berufen,
		und den Text unver\"andert widergeben:
		\\	
		\begin{itemize}[nosep]
			\item[]	{\Spirit} lead me,
					where my trust is without borders.
			\item[] Let me walk upon the waters,
			\item[] wherever {\You} would call me
			\item[]	Take me deeper than my feet could ever wander.
			\item[]	And my faith will be made stronger
			\item[]	in the presence of my {\Saviour}.
			\\
		\end{itemize}
		Und hier die \"Ubersetung,
		falls du nicht so gut Englisch kannst:
		\\
		\begin{itemize}[nosep]
			\item[]	({\Heiliger}) {\Geist},
					f\"uhre mich dorthin,
					wo mein Vertrauen grenzenlos ist.
			\item[]	Lass mich \"uber die Gew\"asser gehen,
			\item[] wohin auch immer {\Du} mich rufst.
			\item[]	Nimm mich tiefer mit,
					denn meine F\"u{\ss}e je wandern k\"onnten.
			\item[]	Und mein Glaube wird st\"arker gemacht (werden)
			\item[]	in der Anwesenheit meines {\Erloesers}.
			\\
		\end{itemize}
		\begin{figure}[h]
			\centering
			\includegraphics[width=0.75\textwidth,keepaspectratio]{"FreeChristian.jpeg"}
		\end{figure}

\end{document}
