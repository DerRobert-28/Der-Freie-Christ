\documentclass[12pt,a5paper]{article}

\usepackage[ngerman]{babel}
\usepackage{enumitem}
\usepackage[utf8]{inputenc}
\usepackage[T1]{fontenc}
\usepackage[left=20mm,right=20mm,top=20mm,bottom=20mm]{geometry}
\usepackage{graphicx}
\usepackage[colorlinks=true,urlcolor=blue,linkcolor=black]{hyperref}

\newcommand{\Allmaechtige}[0]{\textbf{ALLM\"ACHTIGE}}
\newcommand{\Allmaechtigen}[0]{\textbf{ALLM\"ACHTIGEN}}
\newcommand{\Allmaechtiger}[0]{\textbf{ALLM\"ACHTIGER}}
\newcommand{\Christi}[0]{\textbf{CHRISTI}}
\newcommand{\Christus}[0]{\textbf{CHRISTUS}}
\newcommand{\Deinem}[0]{\textbf{DEINEM}}
\newcommand{\Deinen}[0]{\textbf{DEINEN}}
\newcommand{\Deiner}[0]{\textbf{DEINER}}
\newcommand{\Deines}[0]{\textbf{DEINES}}
\newcommand{\Deine}[0]{\textbf{DEINE}}
\newcommand{\Dein}[0]{\textbf{DEIN}}
\newcommand{\Deren}[0]{\textbf{DEREN}}
\newcommand{\Dich}[0]{\textbf{DICH}}
\newcommand{\Dir}[0]{\textbf{DIR}}
\newcommand{\Du}[0]{\textbf{DU}}
\newcommand{\Elohim}[0]{\textbf{ELOHIM}}
\newcommand{\Erloesers}[0]{\textbf{ERL\"OSERS}}
\newcommand{\Erloeser}[0]{\textbf{ERL\"OSER}}
\newcommand{\Er}[0]{\textbf{ER}}
\newcommand{\Geiste}[0]{\textbf{GEISTE}}
\newcommand{\Geist}[0]{\textbf{GEIST}}
\newcommand{\Gottes}[0]{\textbf{GOTTES}}
\newcommand{\Gott}[0]{\textbf{GOTT}}
\newcommand{\Heiligen}[0]{\textbf{HEILIGEN}}
\newcommand{\Heiliger}[0]{\textbf{HEILIGER}}
\newcommand{\Heilige}[0]{\textbf{HEILIGE}}
\newcommand{\Heilig}[0]{\textbf{HEILIG}}
\newcommand{\Herrn}[0]{\textbf{HERRN}}
\newcommand{\Herr}[0]{\textbf{HERR}}
\newcommand{\Himmel}[0]{\textbf{HIMMEL}}
\newcommand{\Himmels}[0]{\textbf{HIMMELS}}
\newcommand{\Ich}[0]{\textbf{ICH}}
\newcommand{\Ihm}[0]{\textbf{IHM}}
\newcommand{\Ihnen}[0]{\textbf{IHNEN}}
\newcommand{\Ihn}[0]{\textbf{IHN}}
\newcommand{\Ihr}[0]{\textbf{IHR}}
\newcommand{\Jahwe}[0]{\textbf{JAHWE}}
\newcommand{\Jesus}[0]{\textbf{JESUS}}
\newcommand{\Jesu}[0]{\textbf{JESU}}
\newcommand{\Mein}[0]{\textbf{MEIN}}
\newcommand{\Meine}[0]{\textbf{MEINE}}
\newcommand{\Meines}[0]{\textbf{MEINES}}
\newcommand{\Meiner}[0]{\textbf{MEINER}}
\newcommand{\Meinem}[0]{\textbf{MEINEM}}
\newcommand{\Meinen}[0]{\textbf{MEINEN}}
\newcommand{\Messias}[0]{\textbf{MESSIAS}}
\newcommand{\Mir}[0]{\textbf{MIR}}
\newcommand{\Mich}[0]{\textbf{MICH}}
\newcommand{\Saviour}[0]{\textbf{SAVIOUR}}
\newcommand{\Schoepfer}[0]{\textbf{SCH\"OPFER}}
\newcommand{\Schoepfers}[0]{\textbf{SCH\"OPFERS}}
\newcommand{\Seinem}[0]{\textbf{SEINEM}}
\newcommand{\Seinen}[0]{\textbf{SEINEN}}
\newcommand{\Seiner}[0]{\textbf{SEINER}}
\newcommand{\Seines}[0]{\textbf{SEINES}}
\newcommand{\Seine}[0]{\textbf{SEINE}}
\newcommand{\Sein}[0]{\textbf{SEIN}}
\newcommand{\Selbst}[0]{\textbf{SELBST}}
\newcommand{\Sich}[0]{\textbf{SICH}}
\newcommand{\Sie}[0]{\textbf{SIE}}
\newcommand{\Sohnes}[0]{\textbf{SOHNES}}
\newcommand{\Sohne}[0]{\textbf{SOHNE}}
\newcommand{\Sohn}[0]{\textbf{SOHN}}
\newcommand{\Spirit}[0]{\textbf{SPIRIT}}
\newcommand{\Uns}[0]{\textbf{UNS}}
\newcommand{\Unser}[0]{\textbf{UNSER}}
\newcommand{\Unsere}[0]{\textbf{UNSERE}}
\newcommand{\Unserem}[0]{\textbf{UNSEREM}}
\newcommand{\Unseren}[0]{\textbf{UNSEREN}}
\newcommand{\Unserer}[0]{\textbf{UNSERER}}
\newcommand{\Unseres}[0]{\textbf{UNSERES}}
\newcommand{\Vater}[0]{\textbf{VATER}}
\newcommand{\Vaters}[0]{\textbf{VATERS}}
\newcommand{\You}[0]{\textbf{YOU}}

\newcommand{\q}[1]{\char"22{#1}\char"22 }
\newcommand{\qq}[1]{\textit{\q{#1}}}

\title{\textbf{Der Freie Christ}}
\author{Robert Lang-Kirchh\"ofer}
\date{\textit{Letzte \"Anderung: 7. Januar 2024}}

\begin{document}
	\setlength{\parindent}{0mm}
	\maketitle
	\begin{figure}[h]
		\centering
		\includegraphics[width=1\textwidth,keepaspectratio]{"FreeChristian.jpeg"}
	\end{figure}

	\newpage
	\pagecolor{white}
	\tableofcontents
	
	\newpage
	\section{Vorwort}
	
	\subsection{Ein paar Worte zu mir}
		Ich selbst wurde,
		soweit ich mich richtig erinnere,
		mit etwa ein/zwei Jahren katholisch getauft,
		bin aber Ende Juli 2023 aus der Kirche ausgetreten.
		Die Gr\"unde hierf\"ur sind pers\"onlicher Art,
		und sind hier nicht von Bedeutung.
		\\
		\\
		Das hat jedoch nichts mit meinem Glauben zu tun.
		\\
		\\
		Ich selbst glaube,
		und bin sogar davon \"uberzeugt,
		dass {\Gott} existiert,
		und dass {\Jesus} der {\Erloeser} ist.
		\\
		\\
		Das hei{\ss}t aber nicht,
		dass ich sowas wie der \q{perfekte Christ} bin,
		falls es sowas unter uns Menschen heutzutage \"uberhaupt gibt.
		\\
		\\
		Mehr zu mir kannst du im Kapitel \q{\hyperref[MeinLebenMitGott]{Mein Leben mit {\Gott}}} lesen.

	\newpage
	\subsection{Verwendungshinweise}
		Ich werde als pers\"onliche Anrede das informelle \q{du},
		und auch das generische Maskulinum verwenden.
		Das soll einerseits eine angenehme,
		lockere Atmosph\"are schaffen,
		und andererseits den Lesefluss erleichtern.
		Selbstverst\"andlich gilt dir,
		mein lieber Leser,
		unabh\"angig von deinem tats\"achlichen Geschlecht,
		mein vollster Respekt.
		\\
		\\
		Da ich selbst auch kein professioneller Author bin,
		ist mein Schreibstil auch nicht perfekt,
		sondern teilweise etwas lockerer.
		Solltest du die englische Version lesen,
		wirst du vielleicht auch feststellen,
		dass mein Englisch nicht perfekt ist,
		da dies nicht meine Muttersprache ist.
		\\
		\\
		Desweiteren ist dieses Buch nicht unbedingt so geschrieben,
		dass ein Kapitel auf dem anderen basiert.
		Das hei{\ss}t,
		du brauchst es weder komplett,
		noch \q{von A-Z} durchlesen,
		sondern kannst so viel,
		oder auch so wenig wie du m\"ochtest lesen,
		und auch in beliebiger Reihenfolge.
		\\
		\\
		Solltest du minderj\"ahrig sein,
		also unter 18 oder 21,
		oder was auch immer das Gesetz \q{deines} Landes als vollj\"ahrig bestimmt,
		oder f\"ur bestimmte Themen noch jung und unerfahren sein,
		kannst du sie gerne \"uberspringen.
		Falls sie dich dennoch interessieren,
		empfehle ich dir erfahrene oder vollj\"ahrige Personen hinzuzuziehen,
		wie z.B. deine Eltern oder andere Erziehungsberechtigte.

	\newpage
	\subsection{Danksagung}
		Als n\"achstes m\"ochte ich dir,
		lieber Leser,
		meinen herzlichsten Dank aussprechen,
		dass du dich daf\"ur entschieden hast,
		hier reinzuschnuppern.
		Ich hoffe nat\"urlich,
		dass du dieses Buch bis zum Schluss durchlie{\ss}t,
		und seine Weiterentwicklung beobachtest.
		\\
		\\
		Ich kann nicht mit 100\%iger Sicherheit sagen,
		dass es je \q{fertig} sein wird,
		weil ich m\"oglichweise immer wieder neue Gedanken,
		oder neues Material finden werde,
		das ich hier aufnehmen werde.
		Es handelt sich hierbei n\"amlich um ein christliches Schriftst\"uck.
		Ich will dir hiermit moralische Werte \"ubermitteln,
		insbesondere wie sie,
		nat\"urlich nach bestem Wissen und Gewissen,
		von {\Gott},
		dem {\Herrn},
		und {\Seinem} {\Sohn} {\Jesus} {\Christus} gew\"unscht sind.
		\\
		\\
		Ich m\"ochte selbstverst\"andlich keine neue \q{Religion},
		oder das Christentum neu erfinden,
		sondern lediglich ein paar neue Perspektiven aufzeigen.
		\\
		\\
		Wie du in diesem Vorwort schon erkennen kannst,
		sind Worte die sich direkt auf {\Gott},
		{\Jesus} oder auch den {\Heiligen} {\Geist} beziehen,
		in Majuskeln,
		also komplett in Gro{\ss}buchstaben,
		und zus\"atzlich in Fettschrift geschrieben.
		\\
		\\
		Wenn dir etwas am Herzen liegt,
		oder dir allgemein etwas hierzu einf\"allt,
		bist du herzlich eingeladen,
		in meiner \href{https://github.com/DerRobert-28/Der-Freie-Christ/discussions}{GitHub-Diskussion} mitzuwirken.
	
	\newpage
	\section{Was macht einen \q{Freien Christen} aus?}
	
	\subsection{Keine Kirchengemeinde}
		Ein Freier Christ ist nicht an eine Kirchengemeinde gebunden.
		Das hei{\ss}t,
		man braucht nicht getauft sein.
		Man kann auch aus der Kirche ausgetreten sein.
		Das spielt alles keine Rolle.
		Ich halte es sogar f\"ur viel sinnvoller,
		gar nicht erst einer Kirchengemeinde anzugeh\"oren,
		insbesondere der katholischen,
		um nicht von ihr manipuliert zu werden.
		Wichtig ist nur,
		dass man {\Gott},
		den {\Herrn},
		und {\Jesus} {\Christus},
		seinen eingeborenen Sohn,
		in sein Leben l\"asst,
		und sich zu {\Ihnen} bekennt.
	
	\subsection{Der wahre Bund}
		F\"ur einen Freien Christ ist der einzig wahre,
		bestehende Bund zwischen {\Gott},
		{\Jesus} {\Christus} und dir.
		Wenn du dich zu {\Ihnen} bekennst,
		pflegst du diese Beziehung von Herzen.
		Zwar sind weltliche B\"unde \textit{(Beziehungen)} verg\"anglich,
		doch auch d\"urfen diese herzlich gepflegt werden.
	
	\subsection{Die Bibel als \q{Werkzeug}}
		Wenn es der Beziehung zwischen {\Gott},
		{\Jesus} und dir dient,
		hast du,
		als Freier Christ,
		die Freiheit,
		Bibelstellen besser,
		also moderner oder verst\"andlicher,
		auszulegen,
		und entsprechend umzuformulieren.
		Das ist jedoch \textbf{\underline{kein}} Freibrief daf\"ur,
		das Wort {\Gottes} nach Gutd\"unken umzuschreiben,
		und damit beispielsweise {\Seinen} Willen zu beugen,
		so wie es,
		meinen Informationen und Recherchen nach,
		die katholische Kirche in der Vergangenheit \q{gerne} gemacht hat.
	
	\subsection{Glaube und Wissenschaft}
		Glaube und Wissenschaft schlie{\ss}en sich,
		meiner Meinung nach,
		nicht gegenseitig aus.
		Ganz im Gegenteil:
		Je mehr man \"uber unser Universum erf\"ahrt,
		ist es dann nicht erstaunlich,
		welch gro{\ss}artiges Werk {\Gott} da vollbracht hat?
		\\
		\\
		Und {\Gott} {\Selbst} kann man weder beweisen,
		noch widerlegen.
		\\
		\\
		Du hast jederzeit die freie Wahl,
		ob du mit {\Gott} leben willst,
		oder nicht.
		Daf\"ur hast du als Mensch einen freien Willen bekommen.
	
	\subsection{Was wei{\ss} ich \"uber {\Gott}?}
		Nun ... was \q{wei{\ss}} man schon wirklich \"uber {\Gott}?
		Klar,
		ich kann {\Gott} und {\Jesus} durch {\Sein} Heiliges Wort,
		die Bibel,
		besser kennenlernen.
		Ich w\"urde mir jedoch niemals anma{\ss}en,
		zu behaupten,
		dass ich {\Gott} \q{kenne}.
		Schon gar nicht absolut.
		\\
		\\
		Nur {\Gott} kennt {\Sich} selbst ganz und gar.
		\\
		\\
		Und in diesem Buch m\"ochte ich dir einfach meine Erfahrung mitgeben.
		Und was ich \"uber {\Gott} sage,
		ist nur nach bestem Wissen und Gewissen,
		und nur zu {\Seinen} Gunsten.

	\newpage
	\section{Glaubst du (nicht) an {\Gott}?}
		Bei diesem Buch handelt es sich zwar um ein christliches Schriftst\"uck,
		das hei{\ss}t aber nicht,
		dass nur Christen dieses Buch lesen d\"urfen.
		\\
		\\
		Ganz im Gegenteil!
		Auch wenn du an etwas anderes oder auch nicht glaubst,
		bist du herzlich eingeladen,
		hier reinzust\"obern.
		{\Gott} liebt uns alle gleich.
		\\
		\\
		Doch ... woran glaubst DU eigentlich?
	
	\subsection{\q{Als Jude glaube ich nicht an {\Jesus} als {\Erloeser}.}}
		Dann bist du dennoch herzlich eingeladen,
		weiterzulesen.
		Wir glauben ja effektiv an den gleichen {\Gott},
		der in der Heiligen Schrift auch {\Elohim} oder {\Jahwe} genannt wird.
		\\
		\\
		Meines Wissens nach sind gro{\ss}e Teile deines heiligen Buches,
		z.B. die Torah,
		deckungsgleich mit dem Alten Testament der Bibel.
		Als Beispiel nenne ich die Entstehungsgeschichte,
		oder den Auszug aus \"Agypten.		
		\\
		\\
		Da ich ja bereits einger\"aumt habe,
		dass ich mir nicht anma{\ss}e,
		{\Gott} zu kennen,
		w\"urde ich auch nicht behaupten,
		dass du,
		nur weil du nicht an {\Jesus} glaubst,
		ein schlechterer Mensch bist,
		und deswegen nicht in den {\Himmel} kommst.
		Wenn du {\Jesus} nicht als deinen {\Erloeser} anerkennst,
		dann akzeptiere ich das.
		Ich bitte dich,
		es ebenfalls zu akzeptieren,
		dass f\"ur mich {\Jesus} der {\Erloeser} ist.
		\\
		\\
		Es geht mir auch gar nicht darum,
		dir {\Jesus} \q{aufzuzwingen},
		sondern im Gro{\ss}en und Ganzen um moralische Werte,
		und darum,
		meine Erfahrungen mit dir zu teilen.
		
	\subsection{\q{Als Muslim glaube ich an Allah.}}
		Dann bist auch du herzlich eingeladen,
		weiterzulesen.
		Ich habe bereits einger\"aumt,
		dass ich mir nicht anma{\ss}e,
		{\Gott} zu kennen.
		Von daher wei{\ss} ich auch nicht,
		ob (der) {\Gott},
		an den ich glaube,
		und dein Gott,
		Allah,
		effektiv der gleiche Gott sind,
		oder zwei verschiedene,
		und die Gleichstellung vielleicht sogar Blasphemie,
		also Gottesl\"asterung,
		ist.
		\\
		\\
		Ich werde dir deinen Glauben nicht absprechen.
		Du darfst zu Allah beten,
		und ihn als deinen alleinigen Gott ansehen.
		Wenn du mich als Ungl\"aubigen betrachtest,
		nehme ich das als deine Meinung an.
		\\
		\\
		Ich akzeptiere,
		dass Allah f\"ur dich dein alleiniger Gott ist.
		Ich bitte dich,
		zu akzeptieren,
		dass {\Gott} f\"ur mich mein alleiniger Gott ist.
		Ich hoffe dennoch darauf,
		dass wir uns auch au{\ss}erhalb unseres Glaubens,
		einfach aus menschlicher Sicht,
		gegenseitig respektieren k\"onnen.
		\\
		\\
		Es geht mir hier schlie{\ss}lich gar nicht darum,
		dir einen anderen Gott \q{aufzuzwingen},
		sondern im Gro{\ss}en und Ganzen um moralische Werte,
		und darum,
		meine Erfahrungen mit dir zu teilen.
			
	\subsection{\q{Ich glaube an keinen Gott, oder bin Agnostiker.}}
		Auch du bist herzlich zum Weiterlesen eingeladen.
		Denn selbst du,
		mein Freund,
		wenn du nicht (mehr) an {\Gott} glaubst,
		oder dir die Existenz von etwas g\"ottlichem gleichg\"ultig oder unbekannt ist,
		hast dennoch in irgendeiner Form moralische Werte.
		\\
		\\
		M\"ochtest du einfach so verletzt oder gar get\"otet werden?
		M\"ochtest du bestohlen oder betrogen werden?
		M\"ochtest du,
		dass dir dein Partner fremdgeht,
		wenn du nicht gerade sowas wie eine offene Beziehung f\"uhrst?
		M\"ochtest du belogen werden?
		M\"ochtest du nicht auch einen gesunden K\"orper haben,
		und eine hohe Lebensqualit\"at genie{\ss}en?
		M\"ochtest du nicht ganz tief in dir einfach nur geliebt werden,
		so wie du bist?
		\\
		\\
		Deswegen akzeptiere ich,
		wenn du in deinem Leben keinen Gott brauchst.
		Ich bitte dich dennoch,
		zu akzeptieren,
		dass {\Gott} f\"ur mich existiert und der alleinige Gott ist.
		Ich hoffe darauf,
		dass wir uns auch einfach aus menschlicher Sicht,
		gegenseitig respektieren k\"onnen.
		\\
		\\
		Es geht mir hier schlie{\ss}lich gar nicht darum,
		dir einen Gott oder einen Glauben \q{aufzuzwingen},
		sondern im Gro{\ss}en und Ganzen um moralische Werte,
		und darum,
		meine Erfahrungen mit dir zu teilen.
		
	\subsection{\q{Ich geh\"ore einer anderen Religion an.}}
		Ich kann leider nicht auf alle Glaubensrichtungen,
		Religionen und Sekten dieser Welt eingehen.
		Das w\"urde den Rahmen dieses Buchs sprengen.
		Von daher verzeih mir,
		wenn ich deinen Glauben nicht explizit aufgef\"uhrt habe.
		\\
		\\
		Wenn du die vorherigen Abschnitte gelsen hast,
		und du beispielsweise zu dem Schluss kommst,
		dass ich aufgrund deines Glaubens ein Ungl\"aubiger bin,
		dann nehme ich das als deine Meinung an.
		Ich akzeptiere,
		dass die Gottheit,
		oder die Gottheiten,
		und selbst wenn es der Teufel selbst ist,
		diejenigen sind,
		die f\"ur dich anbetungsw\"urdig sind.
		Ich bitte dich ebenfalls,
		zu akzeptieren,
		dass {\Gott} f\"ur mich mein alleiniger Gott ist.
		Ich hoffe dennoch darauf,
		dass wir uns auch au{\ss}erhalb unseres Glaubens,
		einfach aus menschlicher Sicht,
		gegenseitig respektieren k\"onnen.
		\\
		\\
		Es geht mir hier,
		wie oben bereits erw\"ahnt,
		schlie{\ss}lich gar nicht darum,
		dir \q{meinen} {\Gott} aufzuzwingen,
		sondern im Gro{\ss}en und Ganzen um moralische Werte,
		und darum,
		meine Erfahrungen mit dir zu teilen.


	\subsection{\q{Ich will (wieder) an {\Gott} glauben.}}
		Bist du von deinem Glauben abgefallen, ihn verloren?
		Egal ob eine der oben erw\"ahnten Glaubensrichtungen,
		vielleicht hilft dir dieses Buch sogar ein wenig,
		(wieder) zu {\Gott} zu finden,
		und vielleicht sogar {\Jesus} als deinen {\Erloeser} anzunehmen.
		Vielleicht hilft es dir aber auch nur,
		deine Moralvorstellungen zu \"ubderdenken,
		dich also selbst zu reflektieren.
		Selbst dann w\"urde mich das schon sehr freuen.
		
	\subsection{Zusammenfassung}
		Also ganz egal,
		woran du glaubst,
		glauben willst,
		oder auch nicht,
		du bist auf jeden Fall herzlich zum Lesen eingeladen.
		Ich zwinge dir nichts auf.
		\\
		\\
		Dein (Nicht-)Glaube ist genau so richtig,
		wie meiner.
		Mir ist nur wichtig,
		dass du dich hier wohlf\"uhlst.
		\\
		\\
		Deswegen kannst du,
		wie bereits erw\"ahnt,
		dieses Buch in (fast) beliebiger Reihenfolge lesen,
		oder einzelne Themen sogar \"uberspringen,
		wenn oder sobald du dich unwohl f\"uhlst.
		\\
		\\
		Du bist \underline{wertvoll} als \underline{Mensch},
		und du bist \underline{liebenswert},
		und nicht ob oder woran du glaubst!

	\newpage
	\section{Wer und wie ist {\Gott}?}
		Dieses Kapitel dient auch mir selbst,
		so dass ich etwas Bibelarbeit machen kann.
		Es geht darum,
		wie {\Gott} in der Bibel genannt wird,
		also seine Namen und generellen Bezeichnungen,
		und wie {\Er} {\Sich} manifestiert.
		Dazu geh\"oren nat\"urlich auch {\Jesus} und der {\Heilige} {\Geist}.
		\\
		\\
		Ich werde im folgenden eine Liste auff\"uhren,
		die meist gehalten ist im Stil \q{{\Er} ist ...},
		oder \q{{\Er} hei{\ss}t ...},
		und dann wer oder wie {\Gott} ist.
		\\
		\\
		Desweiteren m\"ochte ich Bibelstellen mit dir teilen,
		die ich besonders erw\"ahnenswert finde,
		weil sie besonders sch\"on sind,
		weil sie arg bzw. heftig sind,
		oder weil sie sonstwie interessant sind.
		Und wenn ich eine Quelle habe,
		gebe ich in Klammern die erste Bibelstelle an,
		wo das auftaucht.
		\\
		\\
		Als \"Ubersetzung werde ich haupts\"achlich die \q{Lutherbibel}
		\textit{(von 2017)}
		verwenden,
		teilweise auch die \q{Neues Leben}-Bibel
		\textit{(von 2022)}.
		\\
		\\
		Viel Spa{\ss} beim gemeinsamen Entdecken von {\Gott}.
	
	\newpage
	\subsection{{\Gott} ist ...}
		\begin{itemize}[nosep]
			\item {\Er} ist {\Gott}. \textit{(1. Mose 1, 1)}
			\item {\Er} ist der {\Geist} ({\Gottes}). \textit{(1. Mose 1, 2)}
			\item {\Er} ist der {\Herr}. \textit{(1. Mose 2, 4)}
			\item {\Er} ist der {\Gott} deines Vaters. \textit{(2. Mose 3, 6)}
			\item {\Er} ist der {\Gott} Abrahams. \textit{(2. Mose 3, 6)}
			\item {\Er} ist der {\Gott} Isaaks. \textit{(2. Mose 3, 6)}
			\item {\Er} ist der {\Gott} Jakobs. \textit{(2. Mose 3, 6)}
			\item {\Er} ist der {\Gott} Israels. \textit{(Jeremia 33, 4)}
			\item {\Er} ist der {\Herr} Zebaoth. \textit{(Jeremia 33, 11)}
			\item {\Er} ist der {\Allmaechtige}. \textit{(Jeremia 33, 11)}
			\item {\Er} ist der unsere Gerechtigkeit. \textit{(Jeremia 33, 16)}
			\item {\Er} ist der Allerh\"ochste. \textit{(Psalm 9, 3)}
			\item {\Er} ist ein (ge)rechter Ritter. \textit{(Psalm 9, 5)}
			\item {\Er} ist des Armen Schutz. \textit{(Psalm 9, 10)}
			\item {\Er} ist der Schutz in Zeiten der Not. \textit{(Psalm 9, 10)}
			\item {\Er} ist (euer) König. \textit{(Psalm 98, 6)}
			\item {\Er} ist heilig. \textit{(Psalm 99, 3)}
			\item {\Er} ist freundlich/gut. \textit{(Psalm 100, 5)}
			\item {\Er} ist/hei{\ss}t {\Jesus} {\Christus}. \textit{(Matth\"aus 1, 1)}
			\item {\Er} ist wie eine Taube. \textit{(Matth\"aus 3, 16)}
			\item {\Er} ist der Gro{\ss}e K\"onig. \textit{(Matth\"aus 5, 35)}
			\item {\Er} ist ein Meister \textit{(ein Lehrer)}. \textit{(Lukas 20, 21)}
			\item {\Er} ist (ein) {\Gott} der Lebenden. \textit{(Lukas 20, 38)}		
			\item {\Er} ist {\Gottes} Lamm. \textit{(Johannes 1, 29)}			
			\item {\Er} ist ein Rabbi. \textit{(Johannes 3, 2)}
			\item {\Er} ist ein Lehrer. \textit{(Johannes 3, 2)}
			\item {\Er} ist der Menschensohn. \textit{(Johannes 3, 13)}
			\item {\Er} ist das Brot des Lebens. \textit{(Johannes 6, 35)}
			\item {\Er} ist das lebendige Brot, das vom Himmel gekommen ist. \textit{(Johannes 6, 51)}
			\item {\Er} ist der {\Vater} (des Lichts). \textit{(Jakobus 1, 17)}
		\end{itemize}

	\newpage
	\subsection{Besondere Bibelstellen}
		\begin{itemize}[nosep]
			\item	Ihr werdet aber verraten werden von Eltern und Geschwistern,
					Verwandten und Freunden;
					und sie werden einige von euch zu Tode bringen.
					Und ihr werdet gehasst sein von jedermann um {\Meines} Namens willen.
					\textit{(Lukas 20, 16-17)}
		\end{itemize}
		
	\newpage
	\section{Die Zehn Gebote}
		Die traditionellen 10 Gebote werden \"ublicherweise aus der Sicht {\Gottes} \"uberliefert,
		also in der Form \q{Du sollst (nicht) ...}.
		\\
		\\
		Im folgenden sind die 10 Gebote aus der Sicht,
		wenn man selbst zu {\Gott} sprechten w\"urde,
		und {\Ihm} die Gebote als Versprechen geben w\"urde.
		Auch sind sie etwas besser ausgearbeitet,
		da man manche Gebote bei genauerer Betrachtung auch zusammenfassen k\"onnte.
		Das bedeutet selbstverst\"andlich nicht,
		dass ich die traditionellen,
		von {\Gott} gegebenen Gebote ablehne.
		Ich m\"ochte nur eine andere Betrachtungsweise zeigen.
		\\
		\\
		Desweiteren werde ich sie \q{Angebote} nennen,
		um zu unterstreichen,
		dass du freiwillig ein bestimmtes verhalten an den Tag legst,
		wenn du dich zu {\Gott} und {\Jesus} bekennst.
	
	\subsection{Das Oberste Angebot}
		Das Oberste Angebot lautet:
		Ich will {\Gott}, den {\Herrn}, von ganzem Herzen lieben und {\Ihn} ehren. Und ich will meinen N\"achsten lieben, wie auch mich selbst.

	\subsubsection{Das Erste Angebot}
		{\Du} bist der {\Herr},
		mein {\Gott},
		mein {\Erloeser}.
		Ich will keine anderen G\"otter neben {\Dir} haben,
		und sie nicht anbeten oder verehren.
		Und ich will mir kein G\"otzenbild
		\textit{(Gottesbild/G\"otterbild)}
		schaffen.
		
	\subsubsection{Das Zweite Angebot}
		{\Du} bist der {\Herr},
		mein {\Gott}.
		Ich will {\Deinen} Namen nicht missbrauchen.
		Ich will {\Dir} nicht l\"astern.
		Und ich will mich ehrlich zu {\Dir} bekennen.
		\\
		\\
		\textit{Hinweis:
		Hiermit sollen auch umgangssprachliche Phrasen,
		wie beispsielsweise \q{Oh (mein) G...},
		oder \q{Um G...es Willen},
		die man schnell sagt,
		ohne aber wirklich {\Gott} selbst zu meinen,
		oder zu ihm zu beten,
		oder \"ahnliches.}
			
	\subsubsection{Das Dritte Angebot}
		{\Du} bist der {\Herr},
		mein {\Gott}.
		Ich will {\Dich} nicht auf die Probe stellen.
		Ich will {\Dich} nicht versuchen.
		Ich will auch in der Not zu {\Dir} stehen.
		\\
		\\
		\textit{Hinweis:
		Hiermit sollen auch Situationen abgedeckt werden,
		in denen man leichtfertig solche Dinge sagt wie beispielsweise,
		wie {\Gott} dieses oder jenes Leid zulassen kann.}
		
	\subsubsection{Das Vierte Angebot}
		{\Du} bist der {\Herr},
		mein {\Gott}.
		Ich will {\Dir} den Sabbat heiligen.
		Ich will am Sabbat des Fleischlichen,
		und Suchterzeugenden enthaltsam bleiben.
		
	\subsubsection{Das F\"unfte Angebot}
		Ich will meinen Vater und meine Mutter,
		die mir mein Leben geschenkt,
		mich gro{\ss}gezogen und ern\"ahrt haben,
		ehren.
		Und ich will \"Altere Menschen ehren.
			
	\subsubsection{Das Sechste Angebot} \label{DasSechsteAngebot}
		Ich will nicht t\"oten oder morden.
		Ich will meine Beziehungen pflegen.
		Ich will das Leben und Wohlergehen allen Lebens respektieren,
		und nach M\"oglichkeit auch sch\"utzen.
		\\
		\\
		\textit{Hinweis:
		Das T\"oten ist hier nicht nur w\"ortlich,
		also physisch gemeint,
		sondern auch symbolisch,
		indem man beispielsweise aus Zorn irgendetwas zu jemandem sagt,
		was ihn verletzt,
		und damit der Beziehung schadet.
		Lies gerne dazu die Bibelstelle \href{https://www.die-bibel.de/bibeln/online-bibeln/lesen/LU17/MAT.5/Matthäus-5}{Matthäus 5, 21-26}.}
		
	\subsubsection{Das Siebte Angebot}
		Ich will nicht die Ehe brechen.
		Ich will nicht die Frau meines N\"achsten begehren.
		Ich will nicht den Mann meiner N\"achsten begehren.
		
	\subsubsection{Das Achte Angebot}
		Ich will nicht stehlen oder betr\"ugen.
		Ich will nicht rauben oder entf\"uhren.
		Ich will nicht begehren meines N\"achsten Haus.
		Ich will nicht begehren meines N\"achsten Hab und Gut.
		Ich will dem Hab und Gut meines N\"achsten keinen Schaden zuf\"ugen.
		
	\subsubsection{Das Neunte Angebot} \label{DasNeunteAngebot}
		Ich will nicht falsch Zeugnis geben wider meinem N\"achsten.
		Ich will nicht l\"ugen oder betr\"ugen.
		Ich will nicht schw\"oren.
		Ich will gegen\"uber meinem N\"achsten ehrlich und gerecht handeln.
		\\
		\\
		\textit{Hinweis:
		Das Wort \q{schw\"oren} ist im Englischen zweideutig.
		Da es n\"amlich mit \q{(to) swear} \"ubersetzt wird,
		kann man darunter \q{(einen Eid) schw\"oren},
		oder \q{fluchen} verstehen.
		Somit k\"onnte man es auch \"ubersetzen mit
		\q{Ich will nicht (be)schw\"oren oder (ver)fluchen.}}
		
	\subsubsection{Das Zehnte Angebot} \label{DasZehnteAngebot}
		Mein K\"orper und mein Leben sind ein Geschenk von {\Dir},
		und somit heilig.
		Ich will sie ehren und pflegen.
	
	\newpage
	\subsection{Gegen\"uberstellung}
		Im Judentum sind die 10 Gebote,
		die in der Torah sinngem\"a{\ss} \q{10 Worte, die {\Gott} gesprochen hat} hei{\ss}en,
		so \"uberliefert,
		dass man von beiden Gebotstafeln je zwei gegen\"uberstellen kann,
		und im weitesten eine Verbindung aufbauen kann.
		\\
		\\
		Beispielsweise lautet das erste Gebot dort:
		\q{Du wirst {\Gott} als {\Herrn} und Befreier aus Ägypten anerkennen.}
		Und das sechste,
		als parallele Verbindung,
		lautet:
		\q{Du wirst nicht morden.}
		Damit ist nat\"urlich nicht nur gemeint,
		dass man seinem Mitmenschen keinen k\"orperlichen Schaden zuf\"ugt,
		sondern auch keinen seelischen,
		beispielsweise durch Kr\"ankungen,
		wie bereits beim \q{\hyperref[DasSechsteAngebot]{Sechsten Angebot}} erkl\"art.
		\\
		\\
		Die Parallele besteht hier,
		dass man beim ersten Gebot {\Gott} komplett akzeptiert,
		und beim sechsten Gebot seinen Mitmenschen komplett akzeptiert.
		Es geht also in beiden F\"allen um die bedingungslose Liebe,
		einmal gegen\"uber {\Gott},
		einmal gegen\"uber seinem N\"achsten.
		\\
		\\
		Die \textbf{10 Angebote} kann man ebenso gegen\"uberstellen,
		und ich werde dir die Verbindungen zeigen:
		\\
		\begin{itemize}
			\item	\textbf{Das 1. und 6. Angebot:}
					\\
					Es soll auf der einen Seite {\Gott},
					und auf der anderen Seite dein N\"achster,
					voll und ganz angenommen werden.
					Es geht also um die bedingungslose Liebe gegen\"uber {\Gott} und deinem Mitmenschen.
					\\
					Es ist zus\"atzlich gemeint,
					dass hier durch andere G\"otter,
					egal ob neben oder anstelle von {\Gott},
					der Beziehung zu {\Gott} ein Schaden zugef\"ugt wird.
					Und zus\"atzlich,
					dass z.B. durch T\"oten deinem N\"achsten ein Schaden zugef\"ugt wird.
			\item	\textbf{Das 2. und 7. Angebot:}
					\\
					Hier besteht die Verbindung etwas tiefer.
					Selbstverst\"andflich auf der einen Seite gegen\"uber zu {\Gott}.
					Und auf der anderen Seite gegen\"uber deinem Ehepartner,
					oder auch gegen\"uber einem anderen verheirateten Mitmenschen.
					In unserer modernen Zeit,
					wo man nicht immer sofort oder fr\"uh heiratet,
					k\"onnte man den Ehebruch auch noch auf jegliche Liebesbeziehung ausweiten,
					im Sinne von \q{Fremdgehen},
					oder generell eine Liebesbeziehung sch\"adigen.
					\\
					Letztendlich geht es hier um tiefen und intimen Respekt.
					Es ist gemeint,
					wenn du auf der einen Seite {\Gottes} Namen missbrauchst,
					dass das {\Ihm} gegen\"uber sehr respektlos ist,
					und das sehr sch\"adlich f\"ur eine tiefe Beziehung zu {\Gott} ist.
					\\
					Und,
					auf der anderen Seite,
					wenn du ehebrichst,
					fremdgehst,
					oder \"ahnliches,
					dann ist das sehr respektlos gegen\"uber deinem Mitmenschen.
					Es ist ja auch gleichzeitig ein Vertrauensbruch.
					Wenn du bspw. verheiratet bist,
					und dann mit einer beziehungsfremden Person intim wirst,
					woher will dein Ehepartner wissen,
					dass du das nie wieder tun wirst?
					Oder anders herum:
					Wie w\"urdest du dich f\"uhlen?
					W\"urdest du das wollen?
					\\
					\textit{Anmerkung:
					Hier geht es nicht um das Thema \q{lockere/offene Beziehung}.
					Ich m\"ochte das ganze nicht unn\"otig verkomplizieren.}
			\item	\textbf{Das 3. und 8. Angebot:}
					\\
					Die Verbindung in der Versuchung {\Gottes},
					und im Diebstahl besteht darin,
					dass man auf der einen Seite,
					im symbolischen Sinne,
					einen Teil von {\Gottes} Allmacht \q{stehlen} will.
					\\
					Ich dir ein Extrembeispiel zeigen:
					Angenommen,
					du springst aus dem Fenster,
					und sagst dir sowas wie:
					\q{Wenn es {\Gott} gibt, wird {\Er} mich auffangen.}.
					Damit w\"urdest du also einen Teil {\Seiner} Allmacht \q{stehlen} (wollen).
					\\
					Und auf der anderen Seite,
					gegen\"uber deinem Mitmenschen,
					ist es ja klar,
					dass gemeint ist,
					dass du deinem Mitmenschen nichts unberechtigt entwendest,
					also richtiger Diebstahl,
					oder auch Dienstleistungen unberechtigt in Anspruch nimmst:
					\q{Dienst\-leistungs-Dieb\-stahl},
					was man \"ublicherweise mit \q{Betrug} bezeichnet.
			\item	\textbf{Das 4. und 9. Angebot:}
					\\
					Hier ist die Parallele,
					dass es gewisse Dinge gibt,
					die heilig sind,
					und von daher auch ehrenhaft behandelt werden sollen.
					\\
					Auf der einen Seite ist das der Sabbat,
					also der \textbf{siebte} Tag,
					da {\Gott} nach der Sch\"opfung am \textbf{siebten} Tag geruht hat.
					Daher kommt das ja auch in unserer Gesellschaft,
					dass der Sonntag,
					der \textbf{siebte} Tag der Woche,
					in den meisten Branchen ein arbeitsfreier Tag ist.
					\\
					Auf der anderen Seite wiederum gilt auch hier,
					dass das,
					was du mitteilst,
					heilig bzw. ehrenhaft sein soll.
					Deswegen sollst du die Wahrheit sprechen,
					und gerecht gegen\"uber deinem N\"achsten sein.
					Gleichzeitig,
					und darauf bin ich im \q{\hyperref[DasNeunteAngebot]{Neunten Angebot}} bereits eingegangen,
					kann ja das \q{schw\"oren} auch mit dem englischen \q{(to) swear} zusammenh\"angen.
					Und Beschw\"oren,
					Fluchen,
					oder Verfluchen
					sind alles andere als ehrenhaft und heilig.
			\item	\textbf{Das 5. und 10. Angebot:}
					\\
					Hier ist die Verbindung,
					meiner Meinung nach,
					mehr als offensichtlich.
					Deine \textit{biologischen} Eltern haben dir das Leben geschenkt,
					und in der Regel haben sie dich auch gro{\ss}gezogen und ern\"ahrt.
					\\
					Freilich kann es auch sein,
					dass du beispielsweise Adoptiveltern hast,
					oder es ganz andere Umst\"ande bei dir gibt.
					Aber zweifelsohne wurdest du von zwei Menschen gezeugt und geboren.
					Und diese beiden haben dir dein Leben,
					deinen K\"orper geschenkt.
					\\
					Und selbst wenn,
					angenommen du w\"urdest im Extremfall wirklich deine leiblichen Eltern nicht kennen,
					hast du einen K\"orper.
					Und du hast nur dieses eine irdische Leben.
					Von daher ist es wichtig,
					dass du auch dir selbst gut tust.
					Und auch,
					wenn die Rede vom K\"orper ist,
					sind auch deine Gedanken,
					und deine Psyche gemeint,
					dein Charakter und deine Pr\"agungen,
					die sich in deinem Gehirn befinden,
					was effektiv wiederum ein Teil deines K\"orpers ist.
					\\
					Achte stets auf das,
					was von au{\ss}en kommt,
					sowohl die k\"orperliche,
					als auch die geistige und geistliche Nahrung.
					Es ist weniger wichtig,
					wie lange du lebst,
					sondern dass es dir,
					in der Zeit,
					in der du am Leben bist,
					auch so gut wie m\"oglich geht.
		\end{itemize}
	
	\newpage
	\subsection{Zusammenfassung}
		Gek\"urzt kann man die 10 \textit{(An-)}Gebote wie folgt zusammenfassen:
		\\
		\begin{enumerate}[nosep]
			\item Ich will keine anderen G\"otter neben {\Dir} haben.
			\item Ich will {\Deinen} Namen nicht missbrauchen.
			\item Ich will {\Dich} nicht versuchen.
			\item Ich will {\Dir} den Sabbat heiligen.
			\item Ich will Vater und Mutter ehren.
			\item Ich will nicht t\"oten \textit{(oder morden)}.
			\item Ich will nicht ehebrechen.
			\item Ich will nicht stehlen \textit{(oder betr\"ugen)}.
			\item Ich will nicht unehrlich reden \textit{(wider meinem N\"achsten)}.
			\item Ich will mein Leben, {\Dein} Geschenk, ehren.
		\end{enumerate}
	
	\newpage
	\section{Tugenden und (Tod-)S\"unden}
		Im folgenden will ich hier auf die Sieben Tugenden,
		und die Sieben Tods\"unden eingehen.
		Ich habe darauf geachtet,
		nach M\"oglichkeit Alternativbezeichnungen zu w\"ahlen.
		Es kann aber durchaus sein,
		dass sie offiziell anders hei{\ss}en.
	
	\subsection{Die Sieben Tugenden}
	
	\subsubsection{Die Erste Tugend}
		\textbf{Die erste Tugend ist Bescheidenheit,
		Demut,
		Devotion und Dienstwille.}
		Sei also im Allgemeinen bereit deinem N\"achsten zu dienen,
		ohne eine Gegenleistung zu erwarten.
		Ich wei{\ss},
		dass es in unserer Gesellschaft schwierig ist,
		das immer umzusetzen,
		da man ja bspw. eine Arbeitsstelle ben\"otigt,
		um sich von seinem Gehalt oder Lohn sein Leben zu finanzieren.
		Aber bspw. unter Freunden muss man nicht immer auf eine Gegenleistung hoffen.

	\subsubsection{Die Zweite Tugend}
		\textbf{Die zweite Tugend ist Hochachtung,
		Liebe,
		Mildt\"atigkeit,
		N\"achstenliebe,
		Teuerung,
		Wohlt\"atigkeit
		und Wohlwollen.}
		Sch\"atze deine Mitmenschen,
		liebe und respektiere sie.
		Jeder hat bspw. auch schwierige Zeiten,
		also steh ihnen Liebe und Wohlwollen beiseite.
	
	\subsubsection{Die Dritte Tugend}
		\textbf{Die dritte Tugend ist Abstinenz,
		Enthaltsamkeit,
		Keuschheit,
		M\"a{\ss}igung und Triebverzicht.}
		Besonders wenn du den Sabbat halten willst,
		h\"altst du dich erst recht fern von bspw. Alkohol,
		oder anderen berauschenden Substanzen.
		Aber auch au{\ss}erhalb des Sabbats,
		ist es nicht sch\"oner seinen Trieb mit seinem Partner auszuleben,
		den man liebt,
		anstelle t\"aglich wechselnde Partner zu haben.
		Ist letzteres nicht zu stressig?

	\subsubsection{Die Vierte Tugend}
		\textbf{Die vierte Tugend ist Geduld,
		Gelassenheit,
		Hoffnung,
		Langmut und Standhaftigkeit.}
		Und ja,
		genau hier darf auch ich mich an die eigene Nase fassen.
		Dort darf ich selbst an mir arbeiten,
		denn wenn etwas nicht gelingen will,
		und wieder nicht,
		und immer wieder nicht,
		verliere ich schon schnell die Geduld.
		Also,
		lass uns gemeinsam daran arbeiten!

	\subsubsection{Die F\"unfte Tugend}
		\textbf{Die f\"unfte Tugend ist Ma{\ss},
		M\"a{\ss}igkeit und M\"a{\ss}igung.}
		Von allem,
		was man haben kann oder will,
		muss es ja nicht immer zu viel sein.
		Wenn du Lust auf Schokolade hast,
		reicht nicht ein St\"uck,
		anstelle einer ganzen Tafel?
		Wenn du Lust auf Wein hast,
		reicht nicht ein Glas,
		anstelle einer ganzen Flasche?
		Wenn du deine Mahlzeit zu dir nimmst,
		reicht es nicht zu essen,
		bis du satt bist,
		anstelle alles reinzustopfen,
		hauptsache,
		du hast aufgegessen?		

	\subsubsection{Die Sechste Tugend}
		\textbf{Die sechste Tugend ist Benevolenz,
		Dankbarkeit,
		Empathie,
		Gunst,
		Offenheit,
		Solidarrit\"at,
		Sympathie und Wohlwollen.}
		Auch in schwierigen Zeiten,
		und besonders auch danach,
		darfst du dankbar sein.
		Du hast sie schlie{\ss}lich \"uberstanden.
		Du hast ein ganzes Leben voller M\"oglichkeiten geschenkt bekommen,
		wof\"ur du dankbar sein darfst.
		Und gib diese Dankbarkeit deinen Mitmenschen weiter.
		Das macht dich sympathisch,
		und verhilft euch zu gegenseitiger Offenheit und Wohlwollen.
	
	\subsubsection{Die Siebte Tugend}
		\textbf{Die siebte Tugend ist Eifer,
		Flei{\ss},
		Kampfeseifer und Zielstrebigkeit.}
		Mit \q{Kampf} ist nat\"urlich kein echter Kampf,
		oder ein Streit gemeint.
		Es ist eher im symbolischen Sinne gemeint.
		Arbeite flei{\ss}ig an einer Sache,
		und bleib dran,
		auch wenn es manchmal schwerf\"allt.
		Du darfst auch Pausen machen,
		um dich zu erholen und neue Erergie zu sch\"opfen.

	\subsection{Die Sieben Tods\"unden}
	
	\subsubsection{Die Erste Tods\"unde}
		\textbf{Die erste Tods\"unde ist Eitelkeit,
		Hochmut,
		Stolz und \"Ubermut.}
		Das hei{\ss}t nicht,
		dass du auf nichts mehr stolz sein darfst.
		Es geht vielmehr darum,
		dass manche Menschen auf die falschen Dinge \q{stolz} sind,
		f\"ur die sie nichts getan haben,
		z.B. ihre Nationalit\"at,
		und dann sogar mit etwas angeben,
		um sich damit zu schm\"ucken.
		Lass dich f\"ur eine gute Leistung loben,
		aber h\"ang es nicht an die gro{\ss}e Glocke!

	\subsubsection{Die Zweite Tods\"unde}
		\textbf{Die zweite Tods\"unde ist Geiz,
		Habgier und Habsucht.}
		Es ist schon in Ordnung,
		Geld zu haben.
		In unserer modernen Gesellschaft kommt man ohne nicht mehr aus.
		Aber sei doch mal ganz ehrlich zu dir:
		Brauchst du wirklich Millionen und Aber-Millionen auf deinem Konto?
		Oder wieviel reicht dir wirklich f\"ur einen vern\"unfitgen Lebensstandard,
		mit dem dennoch mehr als nur \"uberleben kann?
		Und wenn du etwas \"ubrig hast,
		sei bereit,
		es mit deinem N\"achsten zu teilen!
	
	\subsubsection{Die Dritte Tods\"unde}
		\textbf{Die dritte Tods\"unde ist Ausschweifung,
		Genussucht,
		Begehren,
		Unkeuschheit und Wollust.}
		Brauchst du das wirklich f\"ur dich,
		bspw. jedes Wochenende Saufen bis der Arzt kommt?
		Brauchst du jeden Tag jemand anderes im Bett?
		W\"are es nicht viel sch\"oner,
		sich in einem sicheren Hafen zu wissen?
	
	\subsubsection{Die Vierte Tods\"unde}
		\textbf{Die vierte Tods\"unde ist Ungeduld,
		J\"ahzorn,
		Rachsucht,
		Wut und Zorn.}
		Wenn dich jemand oder etwas verletzt hast,
		ist es in Ordnung die Emotionen zuzulassen.
		F\"uhle sie,
		aber lass sie nicht an deinem N\"achsten aus!
		Gib bescheid und zieh dich lieber zur\"uck.
		Man kann sich darin \"uben,
		eine gewisse emotionale Intelligenz zu entwickeln.
		Aber wenn du w\"utend bist,
		ist und war es bereits in dir,
		und deine Mitmenschen k\"onnen nichts daf\"ur!

	\subsubsection{Die F\"unfte Tods\"unde}
		\textbf{Die f\"unfte Tods\"unde ist Gefr\"a{\ss}igkeit,
		Ma{\ss}losigkeit,
		Selbstucht,
		Unm\"a{\ss}igkeit und V\"ollerei.}
		Reicht es nicht,
		zu essen bist du satt bist?
		Muss es gar ein riesiger,
		\"uberf\"ullter Teller sein?
		Muss denn tats\"achlich Masse \"uber Klasse gehen?
		Und wenn du schon etwas \"ubrig hast,
		kannst du gut und gerne davon abgeben!
	
	\subsubsection{Die Sechste Tods\"unde}
		\textbf{Die sechste Tods\"unde ist Eifersucht,
		Missgunst und Neid.}
		Dein N\"achster hat es also nicht verdient,
		auch etwas zu haben?
		Warum denn nicht?
		Bist du selbst nicht in der Lage,
		es dir selbst zu erarbeiten?
		Oder was steckt hinter deinem Neid?

	\subsubsection{Die Siebte Tods\"unde}
		\textbf{Die siebte Tods\"unde ist Faulheit,
		Feigheit,
		Ignoranz,
		Tr\"agheit und \"Uberdruss.}
		Verwechsle das nicht mit dem Pausieren!
		Wenn du ersch\"opft bist,
		von deiner Arbeit,
		dann erhol dich,
		bis du wieder Energie zum weiterarbeiten hast.
		Aber ist es nicht beispielsweise ungerecht,
		daheim herumzusitzen,
		sich vom Staat bezahlen zu lassen,
		und von anderen aushalten zu lassen?

	\newpage
	\subsection{Gegen\"uberstellung}
		Die Tugenden und die Todsünden kann man also wie folgt gegen\"uberstellen:
		\begin{enumerate}
			\item \textbf{Bescheidenheit $\Longleftrightarrow$ Eitelkeit}
			\item \textbf{Wohlwollen $\Longleftrightarrow$ Habsucht}
			\item \textbf{Keuschheit $\Longleftrightarrow$ Wollust}
			\item \textbf{Gelassenheit $\Longleftrightarrow$ Wut}
			\item \textbf{M\"a{\ss}igung $\Longleftrightarrow$ Gefr\"a{\ss}igkeit}
			\item \textbf{Dankbarkeit $\Longleftrightarrow$ Missgunst}
			\item \textbf{Flei{\ss} $\Longleftrightarrow$ Faulheit}
		\end{enumerate}

	\newpage
	\section{Komme ich in den {\Himmel}?}
	
	\subsection{Ganz allgemein gesagt}
		Ob wir in den {\Himmel} kommen,
		das liegt wohl letztenendes komplett bei {\Gott} {\Selbst}.
		Ich will mir hier nicht anma{\ss}en,
		zu behaupten,
		inwiefern wir f\"ur irgendwelche S\"unden bestraft werden.
		\\
		\\
		Ich glaube eher,
		dass wir uns m\"oglicherweise schon f\"ur unsere Taten rechtfertigen d\"urfen.
		Ich sage nicht,
		dass wir tats\"achlich \q{bestraft} werden,
		und ich sage auch nur \q{m\"oglicherweise}.
		Ich hoffe schon darauf,
		dass {\Gott} g\"utig und gn\"adig ist,
		und uns unsere S\"unden vergibt.
		\\
		\\
		Und es geht auch nicht darum,
		ob du,
		lieber Leser,
		speziell jetzt an {\Jesus} glaubst,
		oder nicht.
		Aber wir selbst k\"onnen uns von unseren S\"unden nicht befreien,
		denn was wir getan haben,
		haben wir getan.
		Das l\"asst sich nicht einfach ungeschehen machen.
		
	\subsection{Wir sind alle S\"under}
		Ich m\"ochte hier nochmal auf {\Gottes} \textit{(An-)}Gebote,
		und auf die sieben Tugenden und Tods\"uden eingehen.
		Sei gerne 'mal ganz ehrlich zu dir selbst.
		\\
		\begin{itemize}[nosep]
			\item	Hast du,
					im weitesten Sinne,
					nicht schonmal einen anderen \q{Gott},
					auch im symbolischen Sinne,
					angebetet oder ihn gelobt?
					Dazu geh\"ort auch das \q{verg\"ottern} von Personen und Dingen.
					\\
			\item	Hast du dir nicht schonmal vorgestellt,
					wie {\Gott} aussehen k\"onnte?
					Das nennt man auch,
					sich ein G\"otzen- bzw. Gottesbild machen.
					\\
			\item	Hast du nicht schonmal eine dieser typischen,
					umgangssprachlichen Phrasen,
					wie \qq{Oh (mein) G...} verwendet?
					Und zwar au{\ss}erhalb eines Gebetes,
					ohne {\Gott} zu meinen?
					Nur,
					weil man das \q{halt so sagt}?
					Man kann ja auch schlie{\ss}lich sowas wie
					\qq{Ach, du meine G\"ute.}
					sagen.
					\\
			\item	Hast du nicht schonmal vor Verzweiflung,
					oder aus anderen Gr\"unden,
					sowas \"ahnliches wie
					\qq{Wie kann {\Gott} solches Leid zulassen?}
					gedacht oder sogar gesagt?
					\\
			\item	Sind dir Feiertage wirklich heilig,
					oder \q{feierst} du sie nur wegen dem Kommerz?
					Hauptsache an Ostern ein sch\"ones Osternest gesucht,
					und hauptsache an Weihnachten ein sch\"ones Geschenk bekommen?
					\\
			\item	Warst du schonmal respektlos gegen\"uber deinen Eltern,
					entweder direkt oder indirekt?
					Hast du auf deine Eltern geschimpft?
					K\"onnte es nicht sein,
					dass sie dennoch ihr Bestes geben,
					und es auch das ein oder andere Mal an dir lag,
					oder an ganz anderen Umst\"anden,
					auf die deine Eltern keinen Einfluss haben?
					\\
			\item	K\"ummerst du dich genug um deine Beziehungen?
					Und damit meine ich nicht nur bspw. die Beziehung zum (Ehe-)Partner.
					Ich meine jegliche zwischenmenschliche Beziehung:
					Deine Eltern,
					deine Kinder (wenn du welche hast),
					deine Freunde,
					deine Kollegen,
					und und und.
					\\
			\item	Hast du schon get\"otet oder gar gemordet?
					Bist du eventuell Soldat oder \"ahnliches?
					Wenn ja,
					wer bist du,
					dass du dir herausnimmst,
					\"uber Leben oder eher Ableben deines \q{Feindes} zu entscheiden?
					\\
			\item	Hast du schonmal jemanden beleidigt oder gekr\"ankt?
					Hast du bspw. im Stra{\ss}enverkehr jemandem \qq{Idiot} hinterhergerufen,
					weil er dir die Vorfahrt genommen hat?
					\\
			\item	Hast du schon ein Insekt,
					z.B. eine Fliege,
					oder anderes kleines Getier get\"otet,
					nur weil es dir l\"astig war?
					\\
			\item	Bist du schonmal fremdgegangen?
					Wenn nicht,
					hast du zumindest schonmal einem anderen Mann,
					oder einer anderen Frau hinterhergeschaut,
					weil er oder sie dir gefallen hat?
					\\
			\item	Und jetzt 'mal ganz konkret:
					Wie stehst du zu Pornographie,
					besonders wenn du dich in einer Beziehung befindest?
					Was w\"urdest du von deinem Partner halten,
					wenn er solche Inhalte konsumieren w\"urde?
					\\
			\item	Hast du jemandem schonmal etwas weggenommen?
					Auch hier an dich, falls du Vater,
					Mutter oder Lehrer bist:
					Hast du deinem Sohn,
					deiner Tochter,
					oder deinem Sch\"uler,
					schonmal vor\"ubergehend etwas wegenommen,
					weil er (oder sie) \q{unartig} war?
					Auch Kinder haben das Recht auf Eigentum.
					Wer also bist du,
					ihm oder ihr etwas wegzunehmen?
					\\
			\item	Warst du schonmal neidisch,
					weil jemand etwas hatte,
					was du auch gerne gehabt h\"attest?
					Warum g\"onnst du ihm das nicht?
					\\
			\item	Hast du schon gelogen?
					Auch wenn du dich f\"ur einen (realtiv) ehrlichen Menschen h\"altst,
					was sagst du,
					wenn dich jemand fragt:
					\qq{Wie geht es dir?}
					Sagst du einfach nur \qq{Gut.} aus H\"oflichkeit,
					um dein Gegen\"uber nicht zu belasten,
					und in Wirklichkeit geht es dir total schlecht?
					\\
			\item	Schimpfst du,
					wnn du die Geduld verlierst?
					Oder verwendest du generell viele Schimpfw\"orter,
					egal in welchem Zusammenhang?
					\\
			\item	Brauchst du immer das neueste,
					technische Ger\"at,
					sei es ein Smartphone,
					eine Smartwatch,
					oder was auch immer,
					nur um deinem N\"achsten zu imponieren?
					Brauchst du das wirklich?
					\\
			\item	Ist bei dir alles nur ein ewiges Geben-und-Nehmen?
					Oder kannst du auch etwas tun,
					ohne gleich eine Gegenleistung zu verlangen?
					\\
			\item	Respektierst du deine Mitmenschen?
					Oder f\"allst du dir oft vorschnell Vorurteile?
					Du musst nicht jeden sympathisch finden,
					aber jeder Mensch hat seine eigene Lebensgeschichte.
					Urteile \"uber niemanden,
					in dessen \q{Schuhe} du nicht mindestens einen Tag lang gelaufen bist!
					\\
			\item	Wie oft hast du gerne 'mal das ein oder andere Glas Bier oder Wein zuviel getrunken?
					Warum hast du das n\"otig?
					Damit es dir am n\"achsten Tag besch... \textit{(sehr schlecht)} geht?
					\\
			\item	Brauchst du jeden Tag jemanden anderes im Bett?
					\\
			\item	Bist du sehr geduldig,
					oder rastest du leicht aus?
					\\
			\item	Isst du beim Essen immer ganz auf,
					weil man es als Kind beigebracht hat,
					auch wenn du schon satt bist?
					Oder kannst du dich m\"a{\ss}igen,
					und von vornherein weniger auf deinen Teller tun?
					\\
			\item	Bist du dankbar f\"ur deinen K\"orper?
					Akzeptierst du ihn grunds\"atzlich?
					Oder schadest du ihm bspw. mit Rauchen oder anderem?
					\\
			\item	Bist du insgesamt dankbar f\"ur dein Leben,
					oder hast du an allem etwas auszusetzen?
					Denk daran,
					was du alles geschafft und \"uberstanden hast,
					um bis hierhin zu kommen!
					\\
			\item	H\"altst du bis zum Ende durch?
					F\"uhrst du deine Projekte bis zum Ende,
					oder gibtst du mittendrin,
					auf halber Strecke auf?
					\\
			\item	Bist du stolz darauf,
					einer bestimmten Nation anzugeh\"ren,
					ein Mann oder eine Frau zu sein,
					ein Kind,
					ein Erwachsener,
					oder sonstiges,
					wof\"ur du nichts kannst?
					Es ist in Ordnung auf eigene Leistungen stolz zu sein.
					Aber was hast du daf\"ur getan,
					um z.B. als Mann geboren zu sein?
					\\
			\item	Beh\"altst du alles f\"ur dich und gibst nichts ab?
					Brauchst du alles f\"ur dich allein?
					Insbesondere beim Thema Geld?
					Hast du nicht vielleicht den ein oder anderen \q{Groschen} \"ubrig,
					um deinen Mitmenschen zu helfen?
					\\
			\item	Bist du st\"andig w\"utend auf deine Mitmenschen,
					ohne Grenzen?
					Denkst du dir stets,
					dass du ihnen nicht vergeben kannst,
					egal,
					was vorgefallen ist?
					\\
		\end{itemize}
		Mir w\"urde sicher noch einiges mehr einfallen,
		aber das w\"urde den Rahmen sprengen.
		Und ja,
		ich darf mich auch an die eigene Nase fassen!
		
	\subsection{Was kann ich tun?}
		Wie bereits erw\"ahnt,
		was du getan hast,
		hast du getan.
		Du kannst nichts tun,
		um es ungeschehen zu machen.
		Sowohl deine guten Taten,
		als auch deine schlechten.
		Deine Erfolge,
		und deine \q{Fehler}.
		\\
		\\
		Du kannst von dir aus nichts tun,
		um {\Gott} zu \q{gefallen}.
		Du kannst nur {\Sein} Geschenk annehmen,
		{\Seine} Gnade.
		Wenn du dich f\"ur ein Leben mit {\Gott} entscheidest,
		und {\Jesus} als deinen {\Erloeser} annimmst,
		dann nimmst du {\Gottes} Geschenk an.
		Lies gerne dazu folgende Bibelstellen durch:
		\href{https://www.die-bibel.de/bibeln/online-bibeln/lesen/LU17/JHN.14/Johannes-14}{Johannes 14, 6},
		\href{https://www.die-bibel.de/bibeln/online-bibeln/lesen/LU17/ACT.4/Apostelgeschichte-4}{Apostelgeschichte 4, 12},
		\href{https://www.die-bibel.de/bibeln/online-bibeln/lesen/LU17/ROM.3/Römer-3}{R\"omer 3,23-24},
		\href{https://www.die-bibel.de/bibeln/online-bibeln/lesen/LU17/GAL.2/Galater-2}{Galater 2, 16} und \href{https://www.die-bibel.de/bibeln/online-bibeln/lesen/LU17/EPH.2/Epheser-2}{Epheser 2, 8-9}.
		\\
		\\
		\textit{Und jetzt kommt etwas GANZ wichtiges:}
		Du \textbf{darfst} dennoch {\Gottes} \textit{(An-)}Gebote halten,
		aber achte darauf,
		welches Motiv dahinter steckt!
		Wenn du sie nur einh\"altst,
		in der Hoffnung,
		dass du in den {\Himmel} kommst,
		hast du das falsch verstanden.
		\\
		\\
		Doch der Fehler liegt nicht bei dir.
		Wir leben (leider) in einer Welt des st\"andigen Gebens und Nehmens.
		Einfaches Beispiel:
		Du gehst jeden Tag in die Arbeit,
		und am Monatsende bekommst du dein Geld daf\"ur.
		Und dann bezahlst du deine Miete,
		damit du weiterhin in deiner Wohnung leben darfst.
		Du bezahlst Strom und Gas,
		damit du Licht hast,
		und im Winter nicht frierst.
		Und du kaufst Lebensmittel,
		damit du nicht verhungerst.
		Also immer Leistung und Gegenleistung.
		Unsere Gesellschaft funktioniert halt nunmal so.
		\\
		\\
		Und das \"ubertragen wir f\"alschlicherweise auf {\Gott}.
		Aber {\Gott} \textbf{schenkt} dir {\Seine} Gnade und G\"ute,
		weil {\Er} dich liebt.
		Und zwar so wie du bist.
		Du musst nichts,
		absolut nichts daf\"ur tun.
		\\
		\\
		Aber wir alle denken
		- und ich bin da keine Ausnahme
		- wir sind es aus irgendeinem Grund nicht wert,
		von {\Gott} geliebt zu werden,
		weil wir ja dauernd s\"undigen.
		Hiermal eine kleine (Not-)L\"uge,
		da mal jemanden aus Wut beschimpft.
		Und dann halten wir uns f\"ur ungeliebt.
		\\
		\\
		Aber es gibt eine gute Nachricht:
		{\Jesus} ist f\"ur dich,
		und deine S\"unden am Kreuz gestorben.
		Das einzige,
		was du \q{tun} brauchst,
		ist,
		{\Ihn} in dein Leben zu lassen,
		{\Ihm} dein Herz zu \"offnen.
		Denn {\Er} liebt dich bedingungslos.
		Und genau so solltest du auch handeln.
		\\
		\\
		Wenn du {\Gottes} Gebote einhalten willst,
		nicht \q{damit} du in den {\Himmel} kommst.
		Nein,
		einfach aus Dankbarkeit,
		weil {\Gott},
		weil {\Jesus} dieses riesige Opfer gebracht hat,
		das niemand je wieder gutmachen k\"onnte.
		\\
		\\
		Halte nicht seine Gebote ein,
		und erwarte oder verlange dann in den {\Himmel} zu kommen!
		Achte auch auf folgendes:
		Nicht,
		\q{wenn} du {\Gott} liebst, h\"altst du seine Gebote,
		sondern \q{weil} du {\Ihn} liebst.
		\\
		\\
		Du darfst dich aber auch nicht darauf ausruhen,
		indem du dir sinngem\"a{\ss} so etwas sagst wie:
		\q{Naja,
		{\Jesus} ist ja sowieso auch f\"ur mich gestorben,
		jetzt kann ich s\"undigen ohne Ende.}
		Das w\"are ebenfalls egoistisch,
		in Bezug auf die Beziehung zu {\Gott}.
		\\
		\\
		Es mag vielleicht schwierig sein,
		das alles zu unterscheiden.
		Aber achte zumindest darauf,
		wenn du bewusst {\Gottes} Gebote einhalten willst,
		warum du es tust.
		Tust du es gerne,
		aus Dankbarkeit zu {\Jesus}?
		Oder erhoffst du dir \q{ein St\"uckchen} {\Himmel}?

	\newpage
	\subsection{Zusammenfassung}
		Von daher kann ich nur sagen,
		hoffe ich einfach auf {\Gottes} Gnade,
		und auf {\Seine} G\"ute.
		Ich hoffe,
		dass wenn wir es ernst mit {\Ihm} meinen,
		und wir {\Seinen} Weg gehen,
		{\Seinen} Willen erf\"ullen,
		dann wird sich {\Gott} auch gn\"adig zeigen.
		\\
		\\
		Und mit \q{{\Seinen} Willen} meine ich nicht,
		dass wir willenlose Marionetten sind.
		Schlie{\ss}lich schenkte uns {\Gott} einen eigenen Willen,
		von {\Sich} aus,
		in {\Seiner} G\"ute.
		Das hei{\ss}t,
		wir k\"onnen selbst entscheiden,
		ob wir mit {\Ihm} leben oder nicht.
		\\
		\\
		Ich meine lediglich,
		dass wir {\Seine} Gebote befolgen,
		und ehrlich und liebevoll miteinander umgehen.

	\newpage
	\section{Verschiedene S\"unden genauer betrachtet}
		In diesem Kapitel m\"ochte ich auf ein paar S\"unden genauer eingehen.
		Manche davon macht man vielleicht bewusst,
		manche unbewusst,
		manche sind schwer zu vermeiden,
		und so weiter.
		\\
		\\
		Ich m\"ochte dir einfach nur zeigen,
		dass das Leben nicht immer Schwarz-Wei{\ss} ist.
		Es gibt oftmals nicht einfach nur \q{Gut} und \q{B\"ose}.
		Manchmal gibt es Grauzonen.
		Und ich bin mir sicher,
		das sieht {\Gott} genauso.
		\\
		\\
		Deswegen gab {\Er} uns ja den freien Willen,
		damit wir selbst entscheiden k\"onnen.
		Damit wir selbst die Erfahrungen machen,
		und daraus lernen k\"onnen.
		\\
		\\
		Auf der n\"achsten Seite geht es los ...

	\newpage
	\subsection{Vom Sabbatbruch}
		Was hei{\ss}t Sabbatbruch?
		Ganz einfach:
		Am Siebten Tage seiner Sch\"opfung hat {\Gott},
		der {\Herr},
		geruht,
		und somit lautet sein Gebot an uns Menschen,
		ebenfalls am siebten Tag der Woche zu ruhen.
		\\
		\\
		Daher kommen ja auch unsere sieben Wochentage,
		bei denen der Sonntag in vielen Branchen nicht gearbeitet wird.
		\\
		\\
		Aber auch hier gilt:
		Es gibt kein eindeutiges Schwarz-Wei{\ss}-Muster,
		es kommt auf verschiedene Dinge an.
		Deswegen gehe ich im folgenden etwas detaillierter darauf ein.
	
	\subsubsection{Was ist mit anderen Feiertagen?}
		Es gibt auch andere Tage,
		die man als Feiertage betrachten kann,
		an denen man ebenfalls ruhen sollte.
		Viele von ihnen sind haupts\"achlich religi\"os gepr\"agt,
		deswegen werde ich als Freier Christ nicht darauf eingehen.
		\\
		\\
		Zum Stand November 2023 bin ich noch nicht bibelfest genug,
		um zu sagen,
		welche Tage neben dem Sabbat \textit{(Sonntag)} noch eines Feier-,
		oder Gedenktags w\"urdig sind,
		au{\ss}er folgende:
		\begin{itemize}[noitemsep]
			\item {\Christi} Geburt
			\item das Letzte Abendmahl
			\item die Kreuzigung {\Jesu} {\Christi}
			\item die Auferstehung {\Jesu} {\Christi} (von den Toten)
			\item {\Christi} Himmelfahrt
		\end{itemize}
		Auch,
		wenn die Kirche selbst bestimmte Tage im Jahr festgesetzt hat,
		an denen wir obige Tage feiern,
		gibt es dennoch Hinweise auf konkrete Daten.
		\\
		\\
		Ich habe n\"amlich ein Video des Diplom-Theologen Markus Voss gesehen,
		in dem er {\Jesu} Geburt und {\Seinen} Tod relativ genau datiert.
		Laut diesem soll {\Jesus} etwa M\"arz/April im Jahre 4 v. Chr. geboren worden sein,
		und {\Er} ist etwa am 7. April 30 n. Chr. gestorben.
		Denn Link zum Video findest du \href{https://www.youtube.com/watch?v=Bv3Rfd1oZ9I}{hier}.

	\subsubsection{\q{In meiner Branche arbeite ich auch Sonntags.}}
		Du m\"ochtest gerne wissen,
		ob es S\"unde ist,
		zu arbeiten,
		wo du arbeitest?
		Meiner Meinung nach spielen dort wenigstens zwei Faktoren eine Rolle.
		\\
		\\
		Das erste ist die Motivation.
		Geht es dir nur,
		und ausschlie{\ss}lich darum,
		Geld zu verdienen?
		Und ich meine nicht,
		dass man sein Leben in unserer Gesellschaft irgendwie finanzieren muss,
		sondern nur deine ganz pers\"onliche Motivation.
		Wenn es dir nur ums Geld geht,
		dann ist es definitiv S\"unde,
		ganz egal,
		was du arbeitest.
		\\
		\\
		Wenn es dir aber nicht nur ums Geld geht,
		dann gibt es auch viele Berufe,
		die meiner Meinung nach v\"ollig in Ordnung sind.
		Und zwar sind dies haupts\"achlich Berufe,
		bei denen du dich um andere Menschen k\"ummerst;
		konkrete Beispiele w\"aren
		Altenpfleger,
		Krankenpfleger,
		oder Sanit\"ater.
		\\
		\\
		Denn selbst {\Jesus} hat beispielsweise einen Mann geheilt,
		der seit 38 Jahren krank war,
		und das am Sabbat,
		der ja auch bei den Juden heilig ist.
		Lies das gerne in der Bibelstelle \textit{\href{https://www.die-bibel.de/bibeln/online-bibeln/lesen/LU17/JHN.5/Johannes-5}{Johannes 5, 1-18}} nach.
		Man k\"onnte also sagen,
		wenn man seinem Mitmenschen etwas zwischenmenschlich gutes damit tut
		(eine bessere Formulierung f\"allt mir leider nicht ein),
		dann ist es keine Arbeit,
		und somit auch keine S\"unde.
		\\
		\\
		Es gibt nat\"urlich aber noch Berufe,
		die im Graubereich sind.
		Wenn du bei der Polizei arbeitest,
		sorgst du ja zumindest f\"ur Sicherheit.
		Wenn du allerdings eine Waffe tragen \q{musst},
		dann halte ich da nichts davon!
		Siehe auch \q{\hyperref[VomToeten]{Vom T\"oten}}.

	\subsubsection{Was ist mit Bereitschaftsdienst?}
		Einen Teil davon haben wir ja im vorherigen Unterkapitel gekl\"art.
		Wenn du in einem anderen Beruf arbeitest,
		bei der bspw. keine Menschenleben involviert sind,
		und du nicht nur an Werktagen arbeiten sollst,
		dann kann das gut sein,
		dass du definitiv den Sabbat brichst.
		\\
		\\
		Aber ich bin nicht hier,
		um \"uber dich zu urteilen.
		Wir alle haben von {\Gott} einen freien Willen bekommen.
		Deswegen entscheide f\"ur dich selbst,
		ob du es f\"ur richtig h\"altst,
		oder nicht.

	\subsubsection{Was gilt nun als Arbeit?}
		Theoretisch kann man auch die allt\"aglichen Haushaltsarbeiten,
		wie Geschirrsp\"ulen oder W\"aschewaschen als Arbeit betrachten.
		Das hei{\ss}t,
		diese Dinge kann man auch sicherlich auch unter der Woche erledigen.
		\\
		\\
		Letzen Endes musst du es dennoch mit deinem Gewissen vereinbaren,
		was du tust,
		und was nicht.

	\newpage
	\subsection{Vom T\"oten} \label{VomToeten}
		Hier m\"ochte ich vom tats\"achlichen,
		physischen T\"oten eingehen,
		und auch vom Verletzen,
		also nicht im \"ubertragenen Sinne,
		wie es {\Jesus} beschrieben hat
		\textit{(siehe \href{https://www.die-bibel.de/bibeln/online-bibeln/lesen/LU17/MAT.5/Matthäus-5}{Matth\"aus 5, 21-26})}.
		Das Thema kann sehr sensibel und schwierig sein,
		und auch hier gibt es kein einfaches Schwarz-Wei{\ss}-Schema,
		im Sinne von \q{T\"oten ist immer falsch}.
		Ich m\"ochte hier lediglich das Bewusstsein st\"arken,
		und meine eigene Meinung dazu \"au{\ss}ern.
	
	\subsubsection{T\"oten von Mitmenschen}
		Doch, fangen wir klein an.
		Es kann eventuell sein,
		dass du mir so etwas \"ahnliches sagen m\"ochtest,
		wie \q{Ich t\"ote nicht.
		Und ich habe noch nie get\"otet.
		Ich bin doch kein M\"order!}
		\\
		\\
		Und auf den Gro{\ss}teil der Menschen mag das bestimmt zutreffen,
		dass sie nicht einfach hinausgehen,
		und wahllos jeden umbringen,
		der ihnen entgegenkommt.
		\\
		\\
		Und auch ich bin grunds\"atzlich gegen T\"oten.
		Ich m\"ochte selbst leben d\"urfen,
		und will von daher auch nicht get\"otet werden.
		Und auch ich sage,
		dass jeder meiner Mitmenschen das Recht hat,
		egal ob juristisch,
		moralisch,
		oder wie auch immer,
		leben zu d\"urfen.
		\\
		\\
		Auch wenn man nicht jeden mag oder sympathisch findet,
		es sollte zumindest soviel N\"achstenliebe und Respekt da sein,
		dass der andere Leben darf.
		Und ich finde,
		kein Mensch hat - zun\"achst - das Recht,
		\"uber das Leben,
		oder Nicht-Leben eines anderen Menschen zu entscheiden.
		\\
		\\
		\textbf{Auch keine Authorit\"at!}
		\\
		\\
		Deswegen halte ich auch Kriege f\"ur falsch.
		Das gr\"o{\ss}ere Problem ist auch noch,
		dass diejenigen,
		die den Krieg wollen,
		selbst nicht aktiv daran teilnehmen,
		indem sie selbst an die Front gehen.
		Sprich,
		ich meine entsprechende Politiker,
		Pr\"asidenten,
		fr\"uher die K\"onige,
		und wie sie nicht alle hei{\ss}en.
		Anstatt sich selbst gegenseitig die K\"opfe einzuschlagen,
		schicken sie Soldaten in den Krieg.
		\\
		\\
		Aber auch Soldaten,
		die aktiv an der Front k\"ampfen,
		und andere Menschen t\"oten,
		sind in meinen Augen M\"order.
		Es soll ja Stimmen geben,
		die behaupten,
		Soldaten h\"atten ja gar keine Wahl,
		sie befolgen einfach nur \q{Befehle}.
		\\
		\\
		\textbf{Doch!
		Haben sie!}
		\\
		\\
		Wenn mir jemand den Befehl erteilt,
		einen anderen Menschen zu t\"oten,
		kann ich jederzeit \q{Nein!} sagen.
		Oder ich kann von vornherein sagen,
		dass ich nicht zum Milit\"ar gehe,
		oder zumindest keine Waffe in die Hand nehme.
		\\
		\\
		Ich halte auch die Todesstrafe f\"ur falsch.
		Wenn ich einen M\"order damit bestrafe,
		dass ich ihn umbringe,
		bin ich dann besser als er?
		Klar,
		man sollte einen M\"order schon zeigen,
		dass seine Taten Konsequenzen haben.
		\\
		\\
		Doch wer bin ich,
		dass ich dar\"uber bestimmen w\"urde,
		ob auch er leben oder sterben soll?
		Ich finde nur einer hat dieses Recht,
		und das ist {\Gott} allein.
		Und wenn es soweit sein sollte,
		wird {\Er} uns schon zu {\Sich} holen.
		
	\subsubsection{Ein Spezialfall}
		In diesem Abschnitt wird es etwas schwieriger.
		Denn hier habe ich selbst Schwierigkeiten,
		mir eine konkrete Meinung zu bilden.
		Denn es gibt wenige,
		aber wirklich nur wenige F\"alle,
		in denen es m\"oglicherweise sein kann,
		dass beispielsweise ein anderer Mensch \"uber Leben oder Tod eines anderen Menschen entscheiden \q{muss}.
		\\
		\\
		Ich sagte zwar im vorherigen Abschnitt,
		dass nur {\Gott} das Recht hat,
		\"uber Leben und Tod zu entscheiden.
		Das hei{\ss}t,
		es kann sein,
		dass ich mir jetzt m\"oglicherweise selbst widerspreche.
		\\
		\\
		Ein Szenario,
		was ich mir vorstellen k\"onnte,
		was in etwa in diese Richtung ginge,
		w\"are,
		wenn jemand im Krankenhaus im Koma liegt,
		und das nicht erst seit einer Woche,
		sondern vielleicht sehr viele Monate,
		vielleicht sogar schon ein paar Jahre.
		Und die Person wird nur mithilfe von Maschinen am Leben erhalten.
		Und die \"Arzte sehen kaum noch Chancen,
		dass die Person je wieder aufwacht.
		\\
		\\
		Und,
		bei allem Respekt gegen\"uber Leben und Mitmenschen,
		aber im Krankenhaus zu liegen kostet ja auch Geld.
		Und in diesem Fall w\"urden vermutlich die Angeh\"origen die Kosten \"ubernehmen,
		wenn sie nicht eine gute Versicherung haben.
		\\
		\\
		Oder es k\"onnte auch sein,
		dass die Versicherung irgendwann gar nicht mehr bezahlt,
		und die Kosten sind f\"ur die Angeh\"origen auf Dauer schwer zu tragen.
		Was ist,
		wenn die \"Arzte diese Person einfach nicht mehr aus dem Koma holen k\"onnen?
		Soll man einfach die Maschinen abstellen,
		und die Person stirbt?
		Was,
		wenn trotz geringer Chancen die Person dennoch wieder gekommen w\"are?
		\\
		\\
		Viele Fragen kann man hier stellen,
		und viele M\"oglichkeiten mit vielen Variationen kann es geben.
		Wenn ich ein paar W\"unsche frei h\"atte,
		dann w\"are sicherlich einer davon,
		dass es in meinem Leben niemals der Fall sein wird,
		dass ich vor die Wahl gestellt werde,
		ob jemand leben oder sterben soll.
		
	\subsubsection{Das Trolley-Problem}
		Ein anderes Szenario ist das sogenannte \q{\href{https://de.wikipedia.org/wiki/Trolley-Problem}{Trolley-Problem}},
		ein Gedankenexperiment,
		vereinfacht erkl\"art,
		bei dem ein Zug auf eine Gruppe von bspw. f\"unf Menschen zurast,
		welche das aber nicht mitbekommt.
		Und auf einem anderen Gleis steht auch ein Mensch,
		der das ganze ebenfalls nicht mitbekommt.
		\\
		\\
		Und durch Stellen der Weichen,
		hat man die Gelegenheit,
		dass der Zug das andere Gleis nimmt,
		und stattdessen der einzelne Mensch stirbt.
		\\
		\\
		Die Entscheidung ist dann,
		l\"asst man f\"unf Menschen sterben,
		oder einen einzigen,
		weil das dann vier weniger w\"aren?
		
	\subsubsection{Andere Formen des aktiven T\"otens}
		Doch im \q{Alltagsleben} t\"otet man ja nicht st\"andig seine Mitmenschen.
		Es gibt auch andere Arten von T\"otungen,
		die man m\"oglicherwiese gar nicht bewusst als solche wahrnimmt.
		\\
		\\
		Was tust du,
		wenn bspw. eine Fliege oder eine Spinne,
		oder \"ahnliches,
		in deiner N\"ahe ist?
		Sagen wir in deiner Wohnung?
		Sind dir Insekten und andere Krabbeltiere l\"astig,
		oder findest du sie gar eklig?
		Oder hast du vielleicht sogar Angst vor ihnen,
		z.B. vor Bienen oder Wespen,
		weil du selbst schon gestochen wurdest?
		Wie gehst du damit um?
		W\"urdest du sie eher versuchen zu t\"oten,
		um nicht selbst gestochen zu werden?
		Oder wenn die sie nur eklig oder l\"astig findest,
		w\"urdest du sie dann auch t\"oten wollen,
		einfach damit du sie loswirst?
		\\
		\\
		Aber auf der anderen Seite,
		sind das nicht auch Lebewesen,
		und damit {\Gottes} Kreaturen?
		Haben sie nicht auch das Recht auf Leben?
		\\
		\\
		Ja,
		in \href{https://www.die-bibel.de/bibeln/online-bibeln/lesen/LU17/GEN.1/1.-Mose-1}
		{1. Mose 1, 26} steht \q{Und {\Gott} sprach:
		Lasset {\Uns} Menschen machen,
		ein Bild,
		das {\Uns} gleich sei,
		die da herrschen \"uber die Fische im Meer,
		und \"uber die V\"ogel unter dem Himmel,
		und \"uber das Vieh,
		und \"uber die ganze Erde und über alles Gew\"urm,
		das auf Erden kriecht.}
		Doch \q{herrschen} hei{\ss}t nicht \q{t\"oten}.
		\\
		\\
		Ich selbst versp\"ure auch Angst oder Ekel vor manchen Insekten,
		und auch Spinnen sind mir zu wider.
		Fr\"uher h\"atte ich noch jede Spinne eiskalt get\"otet,
		oder von meiner Frau t\"oten lassen,
		nur damit sie weg ist.
		\\
		\\
		Doch mittlerweile versuche ich mich immer \"ofter selbst zu \"uberwinden,
		und sie aus der Wohnung zu bringen.
		Da ich die Spinne nicht ber\"uhren mag,
		nehme ich ein gro{\ss}es Trinkglas,
		und eine feste,
		d\"unne Unterlage,
		bspw. einen d\"unnen Karton.
		Dann st\"ulpe ich vorsichtig das Glas \"uber die Spinne,
		damit sie erstmal gefangen ist.
		Wenn sie dann auf die Bodenseite des Glases gekrabbelt ist,
		schiebe ich vorsichtig den Karton unter das Glas,
		und transportiere das ganze dann in den Hof hinaus.
		Dann stelle ich alles auf den Boden,
		lege das Glas hin,
		gehe zur Seite,
		und warte bis die Spinne herausgekrabbelt ist.
		Dann nehme ich mir das Glas und den Karton wieder,
		und kehre zur\"uck in meine Wohnung.
		So werde ich eine Spinne los,
		ohne sie zu t\"oten.
		\\
		\\
		Oder w\"urdest du get\"otet werden wollen,
		\q{nur} weil du jemandem l\"astig bist?

	\subsubsection{Passives T\"oten}
		Jetzt wird es noch viel heikler.
		Es gibt mehrere Arten von passivem T\"oten.
		Und ich spreche nicht von Mitmenschen,
		sondern von anderen Lebewesen.
		Das erste bezieht sich auf die Ern\"ahrung.
		Und ja,
		auch ich esse gerne Fleisch in jeglicher Form,
		also beispielsweise Wurstaufschnitt,
		Bratw\"urste,
		Hackfleisch,
		ab und zu Burger,
		ab und zu Steaks,
		auch gerne mal Gefl\"ugel oder Fisch,
		und was es sonsrt noch so gibt.
		Ich m\"ochte nat\"urlich kein \q{Moralapostel} sein,
		aber effektiv,
		damit ich mein Fleisch (egal welches) essen kann,
		musste daf\"ur bspw. eine Kuh,
		ein Schwein,
		ein Huhn oder ein Fisch sterben.
		Das ist leider aber auch in der ganzen Natur so,
		das eine Lebewesen \"uberlebt nur,
		indem es sich von einem anderen Lebewesen ern\"ahrt;
		im Extremfall eben,
		indem es das andere t\"otet.
		Ich m\"ochte aber auch nicht vegetarisch oder vegan leben,
		dazu schmecken mir Fleisch und Fleischprodukte zu gut.
		Ich habe bereits Alternativprodukte probiert,
		und f\"ur mich haben die oft so einen seltsamen Nachgeschmack.
		Ich mag auch geschmacklich nicht diese Milchersatzprodukte.
		Aber jeder mag eben etwas anderes.
		Es geht auch nicht darum,
		dass ich dir vorschreiben will,
		was du essen sollst,
		und was nicht.
		Ich will dir - und mir selbst auch - nur bewusst machen,
		wo auch im Alltag ein st\"andiges T\"oten stattfindet.
		
	\subsubsection{Weitere T\"otungsarten}
		Die zweite sieht wie folgt aus:
		Stell dir vor,
		du wirst pl\"otzlich krank.
		Zum Beispiel eine Erk\"altung oder \"ahnliches.
		Was passiert hier biologisch?
		Naja,
		du hast dir Krankheitserreger eingefangen,
		und zwar m\"oglicherweise Bakterien.
		Und damit du wieder gesund wirst,
		ben\"otigst du m\"oglicherweise Medikamente,
		um die Krankheitserreger zu bek\"ampfen.
		Aber selbst dein K\"orper,
		dein Immunsystem k\"ampft bereits gegen die Erreger.
		Das,
		was man als (Krankheits-)Symptome bezeichnet,
		ist die Reaktion des K\"orpers auf die Krankheitserreger.
		Effektiv will dein Immunsystem diese t\"oten.
		Und da kannst du aktiv gar nichts dagegen tun.
		Damit du nicht an der Krankheit stirbst,
		m\"ussen sozusagen diese Bakterien sterben.

	\subsubsection{Zusammenfassung}
		Ich kann nat\"urlich nicht auf alle Arten des T\"otens eingehen,
		daf\"ur gibt es eventuell zu viele,
		und das ganze Thema ist auch zu komplex.
		Ich wollte dir mit obigem nur zeigen,
		wie komplex das Thema T\"oten ist,
		wenn man einfach mal dar\"uber nachdenkt.
		Ich behaupte auch nicht von dir,
		dass du ein M\"order bist,
		und bewusst andere Lebewesen t\"otest.
		Ich kenne dich ja gar nicht.
		Ich will dir einfach nur ein paar Gedanken mit auf den Weg geben,
		und dein Bewusstsein st\"arken.
		{\Gott} sei mit dir.
	
	\subsection{Vom Ehebrechen}
		Falls du zur Zeit Single bist,
		oder eher offene Beziehungen f\"uhrst,
		und deine Beziehungspartner davon wissen,
		und damit zurecht kommen,
		brauchst du dieses Unterkapitel nicht unbedingt lesen.
		Dennoch bist du bei Interesse herzlich zum Weiterlesen eingeladen.

	\subsubsection{Bist du in einer Beziehung?}
		Bist du verheiratet,
		oder anderweitig in einer festen (Liebes-)Beziehung?
		Dann begl\"uckw\"unsche ich dich erst einmal,
		und hoffe,
		dass du dich in einer gesunden Beziehung befindest.
		Eine Ehe ist eine gro{\ss}e Verantwortung,
		und nicht etwas,
		was man wegwirft,
		wie schimmliges Brot,
		weil man ja jederzeit neues kaufen k\"onnte.
		Die Ehe sollte heilig sein,
		egal woran du glaubst,
		oder nicht glaubst,
		Und selbst,
		wenn du ehelos mit einer Person liiert bist,
		m\"ochte sich dein Partner geliebt und wertgesch\"atzt werden.
		Ich werde im folgenden der Einfachheit halber von einer Ehe ausgehen,
		aber entsprechende Ausf\"uhrungen gelten auch,
		wenn du mit deinem Partner nicht verheiratet bist.

	\subsubsection{Wie steht es um deine Beziehung?}
		Bevor ich richtig in das Thema eintauche,
		m\"ochte ich dich fragen,
		wie es um deine Beziehung steht?
		Wenn du im Allgemeinen eine gesunde,
		gl\"uckliche Beziehung f\"uhrst,
		dann kannst du den n\"achsten Abschnitt getrost \"uberspringen.

	\subsubsection{Warum bist du geblieben?}
		Gibt es in deiner Beziehung Schwierigkeiten?
		Und ich meine nicht die kleinen allt\"aglichen Dinge,
		wer z.B. absp\"ult,
		oder den M\"ull rausbringt.
		Ich spreche schon von richtig schwerwiegenden Problemen.
		Ich bin nat\"urlich kein Eheberater oder Beziehungscoach.
		Doch denk zuerst nach:
		Woher stammen die Probleme?
		Man kann nat\"urlich immer,
		wie man es heutzutage nennt,
		einen \q{toxischen} Partner haben.
		Aber betrachte bitte deine Beziehung erst genauer,
		und sei ehrlich zu dir selbst.
		Beruhen eure Probleme vielleicht sogar auf Gegenseitigkeit?
		Denn so wie eine Beziehung ein st\"andiges Geben und Nehmen ist,
		gibt es bei Schwierigkeiten auch nie nur einen Schuldigen,
		sondern es geh\"oren immer beide dazu.
		Dann sieh zu,
		was du tun kannst,
		um auf deinen Partner zuzugehen,
		damit ihr gemeinsam eure Beziehung in den Griff bekommt.
		Geht auch gerne z.B. zu einer Eheberatung.
		Nur wenn alle Stricke rei{\ss}en,
		und gar nichts mehr geht,
		kann es vielleicht f\"ur euch besser sein,
		getrennte Wege zu gehen,
		um weiteres Leid zu vermeiden.
		Aber das sollte wirklich das allerletzte Mittel eurer Beziehung sein.
		\\
		Denn eine Frage habe ich noch gar nicht beantwortet:
		Ist das Ehebruch?
		Nun, eigentlich schon!
		Das sagt ja das Wort.
		Durch das Auseinandergehen,
		bspw. durch eine Scheidung,
		wird die Beziehung \q{Ehe} auseinandergebrochen.
		Denn {\Gott} hat euch schlie{\ss}lich zusammengef\"uhrt.
		Doch ich bin nicht hier,
		um mit dem erhobenen Zeigefinger umherzugehen,
		und \"uber dich zu urteilen.
		Wie gesagt,
		wenn es das letzte Mittel ist,
		weil wirklich nichts mehr geht,
		dann ist es vielleicht besser,
		auseinander zu gehen.
		Noch dazu kommt,
		du kannst keinen Menschen zwingen,
		dich zu lieben.
		Und kein Mensch wird jemals dich zwingen k\"onnen,
		ihn zu lieben.
		Und grunds\"atzlich hat jeder das menschliche Recht,
		geliebt zu werden und gl\"ucklich zu sein.
		
	\subsubsection{Wie steht es nun mit Ehebruch?}
		Die einfachste Form,
		die auch jedem gel\"aufig sein sollte,
		ist schlicht und einfach das Fremdgehen.
		Also,
		wenn man mit einer anderen Person ins Bett geht,
		als mit dem (Ehe-)Partner.
		Und ich glaube,
		du und ich,
		wir beide sind reif und verst\"andig genug,
		um zu wissen,
		wie ich das meine.
		\\
		Was ist,
		wenn man einer anderen Frau,
		oder einem anderen Mann,
		hinterherschaut,
		und zwar auf die \q{verlangende} Art?
		Da k\"onnte sich doch der Gedanke einschleichen,
		wie toll es m\"oglicherweise w\"are,
		mit dieser Person etwas zu haben.
		Und wenn es nur f\"ur eine Nacht sein soll.
		Denn wenn ich eine andere Person begehren w\"urde,
		als meine Frau,
		warum bin ich dann \"uberhaupt mit meiner Frau zusammen,
		und nicht mit der anderen Person?
		Ist das denn wertsch\"atzend?
		Wie w\"urdest du dich f\"uhlen?
		Stell dir vor,
		du und dein Partner,
		ihr geht gerade spazieren.
		Und dann taucht da eine gut aussehende Person auf,
		geht vorbei,
		und dein Partner schaut der Person hinterher.
		Wie w\"urde sich das anf\"uhlen?
		Lass den Gedanken gern einmal kurz auf dich wirken,
		und sei dabei ganz ehrlich zu dir selbst.
		Ich finde schon,
		dass das Ehebruch ist.
		Wie gesagt,
		stell dir die Situation vor,
		und stell dir vor,
		wenn du das tun w\"urdest,
		und dann denk dar\"uber nach,
		warum du dann mit deinem Partner zusammen bist,
		wenn du jemand anderes begehrst?
		Die Bibel,
		genauer gesagt {\Jesus},
		hat dazu eine ganz klare Meinung.
		Das kannst du in \href{https://www.die-bibel.de/bibeln/online-bibeln/lesen/LU17/MAT.5/Matthäus-5}{Matth\"aus 5, 27-32} nachlesen:
		\begin{enumerate}[noitemsep,start=27]
			\item	Ihr habt geh\"ort,
					dass gesagt ist:
					\q{Du sollst nicht ehebrechen.}
			\item	Ich aber sage euch:
					Wer eine Frau ansieht,
					sie zu begehren,
					der hat schon mit ihr die Ehe gebrochen,
					in seinem Herzen.
			\item	Wenn dich aber dein rechtes Auge verf\"uhrt,
					so rei{\ss} es aus,
					und wirf's von dir.
					Es ist besser f\"ur dich,
					dass eins deiner Glieder verderbe,
					und nicht der ganze Leib in die H\"olle geworfen werde.
			\item	Wenn dich deine rechte Hand verf\"uhrt,
					so hau sie ab und wirf sie von dir.
					Es ist besser für dich,
					dass eins deiner Glieder verderbe,
					und nicht der ganze Leib in die H\"olle fahre.
			\item	Es ist auch gesagt:
					\q{Wer sich von seiner Frau scheidet,
					der soll ihr einen Scheidebrief geben.}
			\item	Ich aber sage euch:
					Wer sich von seiner Frau scheidet,
					es sei denn wegen Unzucht,
					der macht,
					dass sie die Ehe bricht;
					und wer eine Geschiedene heiratet,
					der bricht die Ehe.
		\end{enumerate}
	
	\subsubsection{Ein besonderes Erwachsenenthema}
		Siehst du dir \q{besondere Erwachsenenfilme} an?
		Und um ganz konkret zu werden,
		wie stehst du zum Thema Pornographie?
		Warum schaust du sie dir an?
		Ich wei{\ss}
		das kann nat\"urlich hier eine Grauzone sein,
		wenn ihr,
		du und dein Partner,
		das gemeinsam anschaut,
		z.B. f\"ur euer Liebesspiel.
		Und das geht mich auch nichts an.
		Aber schaust du sie dir heimlich alleine an?
		Warum?
		Ist dein Partner nicht begehrenswert genug?
		Findest du die M\"anner oder Frauen in den \q{Filmen} begehrenswerter?
		Warum gehst du dann nicht zu diesen?!
		Nein,
		ernsthaft ...
		du liebst doch deinen Partner, oder?
		Warum willst du ihm (oder ihr) weh tun,
		oder m\"oglicherweise riskieren,
		dass er (oder sie) dich eines Tages dabei erwischt,
		und das schlimme Konsequenzen f\"ur euere Beziehung hat?
						
	\subsubsection{Ist Selbstbefriedigung Ehebruch?}
		Ich hoffe,
		wir sind beide reif genug,
		um dieses Thema ernsthaft zu betrachten.
		Zun\"achst einmal:
		{\Gott} hat uns nicht umsonst einen Sexualtrieb gegeben,
		damit wir uns vermehren;
		so steht es in \href{https://www.die-bibel.de/bibeln/online-bibeln/lesen/LU17/GEN.1/1.-Mose-1}{1. Mose 1, 28}:
		\q{Und {\Gott} segnete sie,
		und sprach zu ihnen:
		Seid fruchtbar und mehret euch und f\"ullet die Erde, [...]}
		\\
		Ich werde auch nicht irgendwelche Schauerm\"archen erz\"ahlen,
		von wegen,
		es sei ungesund,
		w\"urde blind machen,
		oder zu R\"uckenmarkschwund f\"uhren.
		Wenn du nun noch dabei bist deinen K\"orper zu erkunden,
		oder du gerade Single bist,
		und \q{Druck} abbauen willst,
		dann tu' dir keinen Zwang an,
		wenn es dein Glaube zul\"asst.
		\\
		Aber ist es bei einem (Ehe-)Paar Ehebruch?
		Das ist schwierig zu sagen.
		Es handelt sich auch hier um eine Grauzone.
		Ich will dir auch keine Angst oder Sorgen machen,
		dass du vielleicht ein schlechter Mensch sein k\"onntest,
		wenn du dich an der \q{falschen} Stelle ber\"uhrst.
		Es kommt darauf an,
		warum du es tust.
		Wenn ihr euch aus irgendwelchen Gr\"unden momentan nicht sehen k\"onnt,
		weil du z.B. auf einer Gesch\"aftsreise bist,
		ist es noch in Ordnung,
		um nicht der Versuchung zu erliegen,
		mit jemand anderem die Ehe zu brechen.
		Oder vielleicht kann dein Partner aus anderen Gr\"unden zur Zeit nicht.
		Oder vieleicht baut ihr das sogar in euer Liebesspiel mit ein.
		Also,
		es gibt viele Optionen,
		bei denen man sagen kann,
		dass es kein Ehebruch ist.
		Doch es k\"onnte Ehebruch sein,
		wenn es nur noch dazu dient,
		nur deine eigene Befriedigung zu stillen,
		und du deinen Partner,
		und damit die gesamte Beziehung vernachl\"assigst.
		Es kann auch helfen,
		wenn du ganz offen mit deinem Partner dar\"uber redest.
		
	\subsubsection{Zusammenfassung}
		Es gibt m\"oglichweise noch mehr Formen des Ehebruchs,
		die mir in diesem Moment (noch) nicht bewusst sind oder einfallen.
		Eine Form kann ich dir trotzdem noch mitgeben:
		Ehebruch ist es auch,
		wenn du eine andere Beziehung zerst\"orst,
		indem du z.B. mit einem der beiden eine Aff\"are hast.
		Denn Ehebruch beschr\"ankt sich nicht nur auf die eigene Beziehung,
		sondern schlie{\ss}t alle (Liebes-)Beziehungen ein.
		Und die Liebe ist ebenso etwas Heiliges,
		und somit sind fremde Beziehungen tabu.
		\\
		Insgesamt kann man vielleicht auch nicht in jeder Situation,
		die potenziell nach Ehebruch riecht,
		eine 100\%ig klare Linie ziehen.
		Hier gilt es im Zweifel:
		H\"or in dich hinein,
		h\"or auf dein Herz,
		h\"or auf dein Gewissen,
		sei ehrlich zu dir selbst und deinen Mitmenschen,
		und bitte sie und {\Gott} um Vergebung!
					
	\subsection{Vom Fluchen und Schw\"oren}
		Um besser auf dieses Thema eingehen zu k\"onnen,
		m\"ochte ich die Bibelstelle \href{https://www.die-bibel.de/bibeln/online-bibeln/lesen/LU17/MAT.5/Matthäus-5}{Matth\"aus 5, 33-37} zitieren:
		\begin{enumerate}[noitemsep,start=33]
			\item	Ihr habt weiter geh\"ort,
					dass zu den Alten gesagt ist:
					\q{Du sollst keinen falschen Eid schw\"oren,
					und sollst dem {\Herrn} deine Eide halten.}
			\item	Ich aber sage euch,
					dass ihr \"uberhaupt nicht schw\"oren sollt,
					weder bei dem {\Himmel},
					denn er ist {\Gottes} Thron;
			\item	noch bei der Erde,
					denn sie ist der Schemel {\Seiner} F\"u{\ss}e;
					noch bei Jerusalem,
					denn sie ist die Stadt des großen K\"onigs.
			\item	Auch sollst du nicht bei deinem Haupt schw\"oren;
					denn du vermagst nicht ein einziges Haar wei{\ss} oder schwarz zu machen.
			\item	Eure Rede aber sei:
					\q{Ja,
					ja;
					nein,
					nein.
					Was dar\"uber ist,
					das ist vom B\"osen.}
		\end{enumerate}
		Lass uns das jetzt genauer betrachten!
		
	\subsubsection{(Be-)Schw\"oren}
		Ich vermute,
		mit \q{Schw\"oren} ist nicht (nur) das wortw\"ortliche Schw\"oren gemeint,
		so wie du es m\"oglicherweise aus Filmen kennst,
		wenn man vor Gericht einen Eid leisten soll.
		Leider wei{\ss} ich nicht,
		wie das bei einer echten Gerichtsverhandlung abl\"auft.
		Denn das einzige Mal,
		wo ich bei einer solchen war,
		handelte es sich um einen Schulklassenausflug,
		und wir sa{\ss}en im Publikum.
		Und an Details kann ich mich leider nicht mehr erinnern,
		denn ich war damals Teenager.
		\\
		Wenn man mehrere Sprachen,
		und damit mehre \"Ubersetzungen dieses Wortes betrachtet,
		kann man das ganze vielleicht auch so interpretieren,
		dass eine Art Magie gemeint ist,
		oder Beschw\"oren von bspw. Geistern oder D\"amonen.
		Denn das w\"urde auch erkl\"aren,
		warum {\Jesus} sagte:
		\q{[...],
		denn du vermagst nicht ein einziges Haar wei{\ss} oder schwarz zu machen}.
		Das soll also sinngem\"a{\ss} hei{\ss}en,
		dass du keine Macht \"uber das \"Ubernat\"urliche hast,
		diese auch nicht haben sollst.
		Du sollst dich auch von irgendwelchem Geistern und D\"amonen,
		insbesondere dem Teufel selbst fernhalten,
		da dich diese nur versuchen wollen.
		Und du kannst auch nicht zaubern,
		und von daher schaffst du es auch nicht,
		auf magische Art und Weise Haare umzuf\"arben.
		Und mit \q{Magie} und \q{Zauberei} meine ich nicht die aus Film und Fernsehen,
		oder irgendwelche Kartentricks.
		Ich finde die selbst manchmal ganz unterhaltsam.
		Und wenn das jemand gut kann,
		dann hat das schon etwas \q{magisches}.
		Es ist auch v\"ollig in Ordnung,
		wenn du gern Filme schaust,
		in denen bspw. Zauberer vorkommen.
		Und mit Haare f\"arben meine ich nicht mithilfe von modernen Haarf\"arbemitteln,
		sondern wirklich \q{wie durch Zauberhand}.
		
	\subsubsection{(Ver-)Fluchen}
		Gehen wir jetzt auf das Fluchen ein,
		denn das basiert thematisch auf dem Schw\"oren.
		Das Wort \q{schw\"oren} kann man ja auch,
		wenn man sich das englische Wort \q{(to) swear} betrachtet,
		auch als \q{fluchen} \textit{(englisch \q{(to) curse})},
		oder vielleicht sogar \q{verfluchen} verstehen.
		Und wenn ich das Verfluchen vorgreife,
		sind wir wieder beim gleichen Punkt,
		wie beim Schw\"oren bzw. Beschw\"oren.
		Lass dich nicht auf bereits oben erw\"ahnte Magie,
		Hexerei oder Zauberei ein,
		und verb\"unde dich nicht mit D\"amonen oder sonstwem.
		Das kann und wird schlimme Konsequenzen haben.
		Und ich spreche noch nicht einmal von der Beziehung zwischen dir und {\Gott},
		wobei das auch nicht zu vernachl\"assigen ist.
		Du hast keine \"ubernat\"urliche Macht,
		also lass auch die Finger davon!
		\\
		Doch gehen wir einen Schritt zur\"uck,
		und sehen uns nur das Fluchen an.
		Es kann auf der einen Seite so gemeint sein,
		dass du nach M\"oglichkeit keine (obsz\"onen) Schimpfworte verwenden sollst.
		Du kannst deinem \"Arger sicher auch durch weniger derbe Worte zum Ausdruck bringen.
		Aber da darf ich mich auch wieder an der eigenen Nase fassen.
		Meine Sprache ist auch nicht perfekt.
		Es geht vielmehr darum,
		dass du dabei u.a. nicht den Namen des {\Herrn} in den Mund nimmst,
		oder solche Ausdr\"ucke wie \textit{\q{Heilige Sch...}}.
		Das geht nicht nur unter die G\"urtellinie,
		sondern auch weit \"uber deinen Kopf hinaus.
		Gleichzeitig verst\"o{\ss}t du dabei gegen ein weiteres von {\Gottes} Geboten,
		dass du {\Seinen} Namen nicht missbrauchen sollst.
		
	\subsubsection{Zusammenfassung}
		Zusammengefasst kann ich sagen,
		wenn du k\"unftig wieder \q{schw\"orst},
		dann lass dein Schwur keine leeren Worte sein.
		Sieh es als ein Versprechen,
		das du einhalten wirst,
		bzw. dass du die Wahrheit sagst.
		Vielleicht hilft es dir auch,
		wenn du f\"ur vergangene Ereignisse beispielsweise statt
		\q{Ich schw\"ore, dass ich das gemacht habe.}
		lieber \q{Ich habe das (wirklich) gemacht.} sagst.
		Wenn du glaubw\"urdig bist,
		und deine Aussage glaubhaft,
		brauchst du nicht zu schw\"oren,
		oder sonst etwas!
		Und f\"ur zuk\"unftige Ereignisse kannst du anstatt \q{Ich schw\"ore, dass ich das machen werde.} lieber \q{Ich verspreche dir, dass ich es machen werde.} sagen.
		Und dann ist es nat\"urlich auch wichtig,
		dass du das einh\"altst,
		ob du nun {\Gott} oder {\Jesus} etwas versprichst,
		oder auch deinen Mitmenschen.
		
	\newpage
	\section{Gebete, Rituale und Loblieder}
		Ich m\"ochte dir in diesem Kapitel ein paar sch\"one Gebete,
		oft in der Form \q{Ich spreche zu {\Gott}},
		und Loblieder und Rituale anbieten.
	
	\subsection{Das {\Vater}-Unser}
		Wie auch bspw. bei den 10 Geboten lehne ich das {\Vater}-Unser,
		so wie es in der Bibel steht,
		auf keinen Fall ab.
		Auch hier m\"ochte ich dir eine pers\"onlichere Form zeigen,
		die auch weniger gebietend ist.
		Ich finde n\"amlich,
		dass dort zu viele Imperativformen enthalten sind.
		Anstatt \q{geheiligt werde {\Dein} Name} zu beten,
		ist es besser zu sagen \q{geheiligt ist {\Dein} Name},
		denn {\Gottes} Name \textbf{ist} heilig.
		Oder anstelle von \q{{\Dein} Wille geschehe},
		besser \q{{\Dein} Wille wird geschehen.},
		da {\Gott} existiert,
		und meiner Meinung nach,
		im Zweifel genau das geschieht,
		was er will,
		auch wenn wir es vielleicht nicht immer gleich erkennen,
		und schon gar nicht immer gleich verstehen.
	
	\subsubsection{Wenn du alleine betest}
		\begin{itemize}[nosep]
			\item	Mein {\Vater},
			\\		(der) {\Du} bist im {\Himmel},
			\\		heilig ist {\Dein} Name.
			\item	{\Dein} Reich wird kommen.
			\item	{\Dein} Wille wird geschehen,
			\\		wie im {\Himmel},
					so auf Erden.
			\item	Mein t\"aglich' Brot gibst {\Du} mir heute.
			\item	Bitte,
					vergib mir meine Schuld,
			\item	wie auch ich vergebe meinen Schuldigern.
			\item	{\Du} f\"uhrst mich nicht in Versuchung,
			\\		sondern erl\"ost mich von dem B\"osen.
			\item	Denn {\Dein} ist das Reich,
			\\		und die Kraft,
			\\		und die Herrlichkeit,
			\\		in Ewigkeit.
			\item	Amen!
		\end{itemize}
			
	\subsubsection{Wenn ihr in der Gruppe betet}
		\begin{itemize}[nosep]
			\item	Unser {\Vater},
			\\		(der) {\Du} bist im {\Himmel},
			\\		heilig ist {\Dein} Name.
			\item	{\Dein} Reich wird kommen.
			\item	{\Dein} Wille wird geschehen,
			\\		wie im {\Himmel},
					so auf Erden.
			\item	Unser t\"aglich' Brot gibst {\Du} uns heute.
			\item	Bitte,
					vergib uns unsere Schuld,
			\\		wie auch wir vergeben unseren Schuldigern.
			\item	{\Du} f\"uhrst uns nicht in Versuchung,
			\\		sondern erl\"ost uns von dem B\"osen.
			\item	Denn {\Dein} ist das Reich,
			\\		und die Kraft,
			\\		und die Herrlichkeit,
			\\		in Ewigkeit.
			\item	Amen!
		\end{itemize}

	\subsection{{\Jesus}, komm in mein Leben (\"Ubergabegebet \#1)}
		\begin{itemize}[nosep]
			\item	{\Jesus},
					ich m\"ochte,
					dass {\Du} in mein Leben kommst.
			\item	{\Jesus},
					ich \"offne {\Dir} meine \textit{(Herzens-)}T\"ur.
			\\		Ich \"offne {\Dir} mein Herz.
			\item	{\Jesus},
					ich m\"ochte,
			\\		dass mein Leben von {\Dir} gef\"uhrt wird,
			\\		dass {\Du} die Leitung \"uber mein Leben \"ubernimmst.
			\item	{\Jesus},
					ich will mit {\Dir} leben,
			\\		und ich glaube an {\Dich}.
			\item	Ich glaube an die Auferstehung;
			\\		und ich glaube daran,
			\\		dass {\Du} der Weg,
					die Wahrheit und das Leben bist.
			\item	{\Jesus},
					ich \"ubergebe {\Dir} mein Leben.
			\item	Amen!
		\end{itemize}

	\subsection{\"Ubergabegebet \#2}
		\begin{itemize}[nosep]
			\item	{\Jesus},
			\\		ich m\"ochte einfach nur verstehen,
			\\		wer {\Du} bist.
			\item	Und {\Jesus},
			\\		ich m\"ochte {\Dir} jetzt mein Leben geben.
			\item	Vielleicht verstehe ich nicht,
			\\		was es bedeutet;
			\\		und vielleicht verstehe ich nicht,
			\\		warum ich das jetzt gerade brauche.
			\item	Vielleicht verstehe ich nicht,
			\\		wer {\Du} bist,
			\\		was meine Aufgabe ist,
			\\		was mein Sinn auf dieser Welt ist.
			\item	Und {\Jesus},
			\\		vielleicht wenn ich auch auf {\Dich} sauer.
			\\		Vielleicht bin ich verletzt.
			\\		Vielleicht gibt es Dinge,
			\\		die f\"ur mich noch irgendwie im Wege stehen,
			\\		um mich {\Dir} v\"ollig hinzugeben.
			\item	Aber {\Jesus},
			\\		ich bete hiermit,
			\\		dass {\Du} all das wegnimmst,
			\\		all die Zweifel,
			\\		und alles,
			\\		was mich noch von {\Dir} trennt.
			\item	Und {\Jesus},
			\\		ich bete,
			\\		dass ich jetzt mein Leben in {\Deine} H\"ande behutsam legen kann,
			\\		und {\Du} daraus machst,
			\\		was f\"ur {\Deinen} Plan,
			\\		und f\"ur {\Dein} Reich am besten ist.
			\item	Und ich bete,
			\\		dass {\Du} mein Leben wieder in Ordnung bringst,
			\\		und mir wieder Lebensfreude gibst,
			\\		und Freude die von {\Dir} kommt,
			\\		lebendiges Wasser in mir,
			\\		das von {\Dir} kommt.
			\item	Und {\Jesus},
			\\		\textit{ich bete,}
			\\		dass {\Du} mich von Ketten befreist,
			\\		aus denen ich jetzt gerade nicht 'rauskomme,
					oder wo ich noch drin stecke.
			\item	{\Jesus},
			\\		lass mich {\Deine} Freiheit spüren,
			\\		und schenk mir deinen Frieden.
			\item	Und nimm {\Du} mein Leben in {\Deine} Hand,
			\\		und lass mich {\Dein} Kind werden;
			\\		und lass mich {\Dich} im Himmel sehen.
			\item	Amen!
		\end{itemize}

	\subsection{\"Ubergabegebet \#3}
		\begin{itemize}[nosep]
			\item	{\Herr} {\Jesus} {\Christus},
					im Glauben komme ich jetzt zu {\Dir}.
			\\		Ich danke {\Dir},
					dass {\Du} mich kennst und liebst.
			\\		Danke,
					dass {\Du} mir ewiges Leben schenken m\"ochtest.
			\item	Ich glaube,
					dass {\Du} auch f\"ur meine Schuld am Kreuz gestorben bist,
			\\		und dass {\Du} auferstanden bist und lebst.
			\item	Bisher habe ich mein Leben selbst bestimmt.
			\\		Ich habe gegen {\Dich} und gegen Menschen ges\"undigt.
			\\		Bitte vergib mir (alle) meine Schuld.
			\item	Ich gebe {\Dir} mein Leben mit Leib,
					Seele und Geist,
			\\		meine Vergangenheit,
					meine Gegenwart und meine Zukunft.
			\item	Ich will dir geh\"oren.
			\item	Komm {\Du} in mein Leben und schenke mir den {\Heiligen} {\Geist}.
			\\		Mach mich zu einem Kind {\Gottes}.
			\\		\"Ubernimm {\Du} die Herrschaft in meinem Leben.
			\\		F\"uhre und ver\"andere mich so,
					wie {\Du} mich haben willst.
			\item	Ich danke {\Dir},
					mein {\Herr} und mein {\Gott}.
			\item	Amen!
		\end{itemize}
		
	\subsection{\"Ubergabegebet \#4}
		\begin{itemize}[nosep]
			\item	{\Herr} {\Jesus},
					ich komme jetzt zu {\Dir}.
			\item	Und ich wei{\ss},
					dass ich ein S\"under bin.
			\item	Ich wei{\ss} aber auch,
					dass {\Du} f\"ur meine S\"unden gestorben bist.
			\item	Bitte, komm jetzt mit {\Deinem} kostbaren Blut,
			\\		und reinige mich von aller Schuld.
			\item	Bitte, wasch mein Herz reiner als Schnee.
			\item	{\Du} sollst von nun an mein {\Herr} sein.
			\item	Und ich bin {\Dein} Kind,
			\item	Und ich werde ewig mit {\Dir} leben.
			\item	In {\Jesu} Namen.
			\item	Amen!
		\end{itemize}
	
	\subsection{Das Glaubensbekenntnis}
		Das Glaubensbekenntnis ist hier fast 1:1 \"ubernommen.
		Es fehlt lediglich der Teil mit der Kirche,
		da ein Freier Christ nicht an eine Kirchengemeinde gebunden ist.
		\\
		\begin{itemize}[nosep]
			\item	Ich glaube an {\Gott},
			\\		den {\Vater},
					den {\Allmaechtigen},
			\\		den {\Schoepfer} des {\Himmels} und der Erde.
			\item	Ich glaube an {\Jesus} {\Christus},
			\\		{\Seinen} eingebohrenen {\Sohn},
					meinen {\Herrn},
			\\		empfangen durch den {\Heiligen} {\Geist},
			\\		geboren von der Jungfrau Maria,
			\\		gelitten unter Pontius Pilatus,
			\\		gekreuzigt,
					gestorben und begraben,
			\\		hinabgestiegen in das Reich des Todes,
			\\		am Dritten Tage auferstanden von den Toten,
			\\		aufgefahren in den {\Himmel};
			\\		{\Er} sitzt zur Rechten {\Gottes},
			\\		des {\Allmaechtigen} {\Vaters};
			\\		von dort wird {\Er} kommen,
			\\		zu richten die Lebenden und die Toten.
			\item	Ich glaube an den {\Heiligen} {\Geist},
			\\		an die Vergebung der S\"unden,
			\\		an die Auferstehung der Toten,
			\\		und das Ewige Leben.
			\item	Amen!
		\end{itemize}
		
	\subsection{Das Abendmahl}
		Das regelm\"a{\ss}ige Feiern des Abendmahls,
		auch bekannt als Eucharistiefeier,
		l\"asst sich,
		meiner Meinung nach,
		prima in den Alltag implementieren,
		da es eigentlich \q{nur kurz} ist.
		\\
		\\
		Du kannst es einmal t\"aglich machen,
		oder einmal in der Woche,
		und gerne auch in der Gruppe.
		Ich glaube,
		es gibt hier kein \q{Richtig} oder \q{Falsch}.
		Wichtig ist,
		dass es eine gewisse Regelm\"a{\ss}igkeit hat.
		Hier gebe ich dir nur eine grobe Anleitung;
		eine bessere findest du \href{https://www.youtube.com/watch?v=8iCgvvpLhvM}{hier}.
		\\
		\\
		F\"ur das Abendmahl brauchst du ein kleines,
		mundgerechtes St\"uck Brot,
		und einen kleinen Schluck (Trauben-)Saft.
		Traditionell k\"onnte man auch einen Schluck Rotwein nehmen,
		jedoch bin ich pers\"onlich kein Fan von Alkohol,
		wenn ich meine Zeit {\Gott} bzw. {\Jesus} widmen will.
		\\
		\\
		Du kannst gern in eigenen Worten sprechen,
		oder folgende traditionelle verwenden:
		\\
		\\
		\textit{\q{{\Christi} Leib f\"ur mich gegeben.}},
		\\
		\\
		und isst in Gedenken an {\Jesus} {\Christus} das St\"uck Brot.
		Danach sprichst du traditionell folgende Worte:
		\\
		\\
		\textit{\q{{\Christi} Blut fur mich vergossen.}},
		\\
		\\
		und trinkst in Gedenken an {\Jesus} {\Christus} den Saft.
		\\
		\textbf{Wichtig}:
		Es geht hierbei nicht darum,
		Hunger oder Durst zu stillen,
		sondern um das Gedenken an das Opfer,
		das {\Jesus} {\Christus} erbracht hat!
		
	\subsection{Wie ein Fest nach langer Trauer}
		\subsubsection{Infos}
			\begin{itemize}[nosep]
				\item Liederbuch: Von {\Jesus} singen 2
				\item ISBN: 9783775123099
				\item Komponist: J\"urgen Werth
			\end{itemize}
		
		\subsubsection{Text}
			\begin{itemize}
				\item	\textbf{Strophe 1:}
				\\		Wie ein Fest nach langer Trauer,
				\\		wie ein Feuer in der Nacht.
				\\		Ein off'nes Tor in einer Mauer,
				\\		f\"ur die Sonne auf gemacht.
				\\		Wie ein Brief nach langem Schweigen,
				\\		wie ein unverhoffter Gru{\ss}.
				\\		Wie ein Blatt an toten Zweigen.
				\\		Ein \q{Ich mag dich trotzdem.}-Kuss.
				\item	\textbf{Strophe 2:}
				\\		Wie ein Regen in der W\"uste,
				\\		frischer Tau auf d\"urrem Land.
				\\		Heimatkl\"ange für vermisste.
				\\		Alte Feinde Hand in Hand.
				\\		Wie ein Schl\"ussel im Gef\"angnis,
				\\		wie in Seenot \q{Land in Sicht!}.
				\\		Wie ein Weg aus der Bedr\"angnis.
				\\		Wie ein strahlendes Gesicht.
				\item	\textbf{Strophe 3:}
				\\		Wie ein Wort von toten Lippen,
				\\		wie ein Blick der Hoffnung weckt.
				\\		Wie ein Licht auf steilen Klippen,
				\\		wie ein Erdteil neu entdeckt.
				\\		Wie der Fr\"uhling,
						wie der Morgen,
				\\		Wie ein Lied,
						wie ein Gedicht.
				\\		Wie das Leben,
						wie die Liebe,
				\\		Wie {\Gott} {\Selbst},
						das wahre Licht!
				\item	\textbf{Refrain \textit{(je 2x)}:}
				\\		So ist Vers\"ohnung,
				\\		so muss der wahre Friede sein.
				\\		So ist Vers\"ohnung,
				\\		so ist vergeben und verzeih'n.
			\end{itemize}
		
	\subsection{Wundergebet}
		\begin{itemize}[nosep]
			\item	Liebender {\Vater},
					hilf mir,
					spirituelle Disziplinen zu kultivieren,
					damit ich besser empfangen kann,
					was {\Du} mir zu sagen hast,
					und bereitwilliger bin,
					das zu tun,
					was {\Du} ben\"otigst.
			\item	M\"oge ich nicht nur mit meinen N\"oten,
					meinem Lob und meiner Dankbarkeit zu {\Dir} kommen,
					sondern es mir zur Gewohnheit machen,
					Zeit mit anbetungsw\"urdigem Zuh\"oren zu verbringen.
			\item	Hilf mir,
					nicht zuzulassen,
					dass die Ablenkungen der sozialen Medien,
					und die Hektik des Lebens {\Deine} immer noch kleine Stimme \"ubert\"onen.
			\item	Hilf mir,
					mich in allen Dingen von {\Dir} leiten zu lassen,
					anstatt von meinem eigenen unvollkommenen Urteilsverm\"ogen.
			\item	Lass {\Deinen} {\Geist} heute Worte der Weisheit,
					und F\"uhrung \"uber mein Leben sprechen.
			\item	{\Du} hast einen Tr\"oster versprochen,
					der bei jedem Schritt des Weges bei mir sein wird,
					um mich zu f\"uhren und zu versorgen,
					was ich brauche.
			\item	Sei bei mir,
					{\Vater}.
			\item	Hilf mir,
					die richtigen Worte zu finden,
					die richtigen Entscheidungen zu treffen,
					und die richtigen Gelegenheiten zu w\"ahlen.
			\item	Entscheidungen zu treffen kann verwirrend sein.
			\item	{\Heiliger} {\Geist},
					erlaube mir nicht,
					Fehler zu machen.
			\item	Hilf mir zu erkennen,
					welche Richtung die richtigen T\"uren \"offnet,
					und produktive Beziehungen gedeihen l\"asst.
			\item	Amen!
		\end{itemize}

	\subsection{Gelassenheitsgebet}
		\begin{itemize}[nosep]
			\item	{\Gott},
			\\		gib mir die Gelassenheit,
			\\		die Dinge hinzunehmen,
			\\		die ich nicht \"andern kann,
			\\		Mut,
					die Dinge zu \"andern,
			\\		die ich \"andern kann,
			\\		und die Weisheit,
			\\		den Unterschied zwischen beidem zu erkennen.
			\item	Einen Tag nach dem anderen zu leben,
			\\		einen Moment nach dem anderen zu genie{\ss}en,
			\\		Beschwernis als einen Weg zum Frieden zu akzeptieren.
			\item	Diese s\"undige Welt,
			\\		wie {\Jesus} es tat,
			\\		so anzunehmen,
					wie sie ist,
			\\		nicht so, wie ich sie gern h\"atte.
			\item	Darauf zu vertrauen,
			\\		dass {\Du} alles richtig machen wirst,
			\\		wenn ich mich {\Deinem} Willen hingebe,
			\\		auf dass ich recht gl\"ucklich sein m\"oge in diesem Leben,
			\\		und \"ubergl\"ucklich mit {\Dir} auf ewig im n\"achsten.
			\item	Amen!			
		\end{itemize}

	\subsection{Weitere christliche Lieder}
		Ich m\"ochte dir auch gerne christliche Lieder teilen.
		Da ich dabei m\"oglichst keine Urheberrechte verletzen will,
		bekommst du hier eine Liste mit Titel und Interpret,
		und maximal einen Link dazu.
		Die Lieder sind teilweise auf Deutsch,
		teilweise auf Englisch.
		Mithilfe des Internets sollte es dir leicht fallen,
		den Songtext zu finden,
		und ihn ggf. zu \"ubersetzen.
		\begin{itemize}[noitemsep]
			\item	\href{https://www.youtube.com/watch?v=Oncj9JBo1xQ}{\q{Ewigkeit},
					Dero Goi}
			\item	\href{https://www.youtube.com/watch?v=PcxaUHkmnSQ}{\q{Ewigkeit},
					Outbreakband }
			\item	\href{https://www.youtube.com/watch?v=DqlpyrHB_Qk}{\q{Oceans (Where feet may fail)},
					Hillsong United}
					\\
					(auch bekannt als \q{{\Spirit} lead me (where my trust is without borders)})
		\end{itemize}

	\newpage
	\section{Mein Leben mit {\Gott}} \label{MeinLebenMitGott}
		Hierbei handelt es sich um eine Art unregelm\"a{\ss}iges \q{Tagebuch} im weitesten Sinn,
		wie ich meine Reise mit und zu {\Gott} erlebe,
		und was ich sonst noch dabei lernen darf.
	
	\subsection{Mittwoch, der 27. September 2023}
		Ich bin seit etwa Mitte 2023 auf einer Art Reise,
		bei der ich mich entschieden habe,
		{\Gott} und {\Jesus} in mein Leben zu lassen.
		Ich habe selbst noch viele Fehler,
		und obgleich der von {\Gott} gegebenen \textit{(An-)}Gebote,
		s\"undige ich noch viel zu oft.
		Wie im Vorwort erw\"ahnt,
		bin ich weit davon entfernt,
		so etwas wie der \q{perfekte Christ} zu sein.
		Viele der allt\"aglichen Gewohnheiten,
		Pr\"agungen und sonstiges haben so eine starke Sogwirkung,
		dass ich auch nicht immer an {\Gott} denke,
		nicht so oft bete,
		oder in der Bibel lese,
		wie ich gerne w\"urde.
		Und wenn ich dann \q{wieder} an {\Gott} denke,
		habe ich oft ein schlechtes Gewissen,
		weil ich {\Ihn} dann gef\"uhlt \q{vergessen} habe.
		Also kurzum:
		Ich darf noch sehr, sehr, sehr, ..., sehr viel lernen!
		
	\subsection{Freitag, der 29. September 2023}
		Heute habe ich mir ein Video angesehen,
		das mir sehr zu denken gegeben hat.
		Ich wei{\ss} nicht,
		ob es sich dabei um Gottesl\"asterung handelt.
		Trotzdem will ich mit dir teilen,
		was ich darin gesehen habe.
		Es war im Prinzip ein kurzer Trickfilm,
		in dem eine Muslimin,
		ein Atheist und ein Christ in den {\Himmel} gekommen sind.
		Da man hier nicht wirklich {\Gott} selbst gesehen hat,
		sondern lediglich eine Karikatur,
		werde ich hier die normale Gro{\ss}schreibung verwenden.
		Es ging also darum,
		von welchen Aussagen sich Gott beleidigt f\"uhlt.
		Und im Endeffekt hat er dem Atheisten seinen Frieden geschenkt,
		und ihn tats\"achlich in dem Himmel geschickt,
		weil dieser ja nie an Gott geglaubt hat,
		und ihm weder das eine,
		noch das andere nachgesagt hat.
		Und von der Muslimin und vom Christen war er entt\"auscht,
		weil sie im Prinzip in so \q{b\"ose} dargestellt haben,
		als ob er alle Menschen,
		die s\"undigen und nicht an ihn glauben,
		einfach in die H\"olle werfen w\"urde.
		Das hat ihn sehr verletzt,
		weil er sich effektiv wie ein grausames Monster gef\"uhlt hat.
		Das ist sozusagen die Kurzversion.
		Und das hat mir zu denken gegeben.
		Ich kann nat\"urlich nur spekulieren.
		Aber vielleicht ist es ja so,
		dass uns {\Gott} nirgendwo \q{hinschickt}.
		Wenn wir uns f\"ur {\Ihn} entscheiden,
		so l\"adt {\Er} uns auch nach dem Tode zu {\Sich} in den {\Himmel} ein.
		Und wenn wir uns beispielsweise f\"ur den Teufel entscheiden,
		dann kann es schon sein,
		dass wir in die H\"olle kommen.
		Aber nicht weil uns {\Gott} dort hinschickt,
		sondern weil der Teufel uns mitnimmt.
		Wie gesagt ... ich wei{\ss} es nicht.
		Das Video hat mich nur zum Nachdenken gebracht.
		Denn es ist sicher oft so,
		dass man vielleicht \"uber {\Gott} dieses und jenes sagt,
		aber es effektiv nicht wei{\ss},
		was die Wahrheit ist.
		Aber man kann ja zumindest erstmal nachdenken,
		wenn man \"uber {\Gott} etwas sagt,
		ob man selber wollen w\"urde,
		dass jemand anderes \"Ahnliches \"uber einen sagt.
		Falls du dir \"uber das Video selbst ein Urteil bilden m\"ochtest,
		hier der Link: \url{https://www.youtube.com/watch?v=ttevamkS6gw}.

	\subsection{Dienstag, der 3. Oktober 2023}
		Am vergangenen Wochenende,
		bis einschlie{\ss}lich gestern,
		war ich mit meiner Frau und meinen Eltern in Hamburg.
		Aus irgendeinem Grund ging es mir ab Sonntag im Laufe das Tages nicht so gut.
		Ich war u.a. \"uberm\"udet und gequ\"alt von Kopfschmerzen.
		Und ich vermute,
		auch mein Bewegungsmangel hat sich hier stark gezeigt,
		da ich jede Treppe als Qual empfunden habe.
		Wenigstens konnte ich morgens in meiner Bibel-App lesen,
		was auch schonmal viel Wert war.
		Zwischendrin kam mir ein Verdacht,
		woher m\"oglicherweise meine Kopfschmerzen kamen.
		Ich kann es aber nicht beweisen,
		es bleibt also bei einer Vermutung.
		Jedenfalls,
		da ich ja diesmal den Sabbat durchziehen wollte,
		habe ich sowohl u.a. auf Kaffee und potenziell zuckerhaltiges verzichtet.
		Meine Frau meinte zwar \"ofters,
		ich sollte einen Kaffee trinken,
		aber auf die Versuchung wollte ich gar nicht erst eingehen.
		Und ich habe ja 'mal geh\"ort,
		dass Zucker auch s\"uchtig machen kann.
		Und deswegen geh\"ort dies auch zu den Dingen,
		auf die ich am Sabbat verzichten will.
		Ich habe aber auch geh\"ort,
		dass bei manchen S\"uchten,
		z.B. bei Zucker,
		Kopfschmerzen eine Entzugserscheinung sein kann.
		Und wenn das stimmt,
		dann s\"undige ich au{\ss}erhalb des Sabbats ganz sch\"on viel,
		was das betrifft.
		Siehe auch: \q{\hyperref[DasZehnteAngebot]{Das Zehnte Angebot}}.
		
	\subsection{Samstag, der 7. Oktober 2023}
		Ich habe langsam das Gef\"uhl,
		es wird ernst.
		Nat\"urlich im positiven Sinne.
		Ich habe mich deswegen heute spontan dazu entschieden,
		die im Vorwort erw\"ahnte GitHub-Diskussion f\"ur dich vorzubereiten,
		und dieses Werk als monatliche Ausgabe zu releasen.
		Die erste ist die Oktoberausgabe,
		die ich vorhin ver\"offentlicht habe.
		Die Novemberausgabe sollte dann p\"unktlich zum Monatswechsel,
		bzw. zum 1. November erscheinen.
	
	\subsection{Sonntag, der 8. Oktober 2023}
		Mich hat ja jetzt leider nach dem Wochenende in Hamburg eine Erk\"altung erwischt.
		Ich m\"ochte aber dennoch mit dir heute einfach eine sch\"one Bibelstelle teilen.
		Es handelt sich um R\"omer 10,
		Verse 5 bis 11,
		mit der \"Uberschrift \q{Die Erl\"osung steht f\"ur alle bereit}
		aus der \q{Neues Leben Bibel}.
		Viel Freude beim Lesen.
		\begin{enumerate}[nosep,start=5]
			\item	Denn Mose schrieb,
					dass man alle Gebote des Gesetzes erf\"ullen muss,
					um durch das Gesetz vor {\Gott} gerecht zu werden.
			\item	Wer aber durch den Glauben vor {\Gott} bestehen will,
					dem sollt ihr sagen:
					\q{Du musst nicht in den {\Himmel} hinaufsteigen.},
					um {\Christus} zu finden und ihn herabzuholen.
			\item	Und:
					\q{Du musst nicht in die Tiefe hinabsteigen.},
					um {\Christus} wieder von den Toten heraufzuholen.
			\item	Denn in der Schrift hei{\ss}t es:
					\q{Die Botschaft ist dir ganz nahe;
					sie ist auf deinen Lippen und in deinem Herzen.}
					Es ist die Botschaft von der Erl\"osung durch den Glauben an {\Christus},
					die wir verk\"unden.
			\item	Wenn du mit deinem Mund bekennst,
					dass {\Jesus} der {\Herr} ist,
					und wenn du in deinem Herzen glaubst,
					dass {\Gott} {\Ihn} von den Toten auferweckt hat,
					wirst du gerettet werden.
			\item	Denn durch den Glauben in deinem Herzen wirst du vor {\Gott} gerecht,
					und durch das Bekenntnis deines Mundes wirst du gerettet.
			\item	So hei{\ss}t es in der Schrift:
					\q{Wer an {\Ihn} glaubt,
					wird nicht umkommen.}
		\end{enumerate}

	\subsection{Sonntag, der 22. Oktober 2023}
		Die vergangene Woche war insgesamt etwas seltsam.
		Am Anfang der Woche hatte es bereits gef\"uhlt eisige Temperaturen drau{\ss}en.
		Das habe ich auch im Laufe der Woche gemerkt,
		weil etwa am Mittwoch der Akku meines E-Scooter nur noch halb voll war,
		und ich ihn erst am Freitag davor geladen habe.
		Und normalerweise reicht mir eine Ladung etwa zwei Wochen,
		um damit zur Arbeit und wieder nach Hause zu fahren.
		Komischerweise sind die Temperaturen ab Mittwoch wieder etwas gestiegen.
		Dann habe ich auch ein paar Mal verschlafen,
		und bin deswegen ein paar Minuten zu sp\"at zur Arbeit gekommen.
		Gl\"ucklicherweise habe ich nicht so strenge Chefs.
		Aber wenn ich sp\"ater komme,
		darf ich ja auch erst sp\"ater gehen,
		weil ich ja trotzdem auf meine acht Arbeitsstunden kommen will.
		Ich habe {\Jesus} f\"ur das Wochenende gebeten,
		dass dies im Ausgleich daf\"ur einigerma{\ss}en gut verl\"auft.
		Ich habe sozusagen das Gegenteil der Redewenung \q{Man soll den Tag nicht vor dem Abend loben.} gemacht.
		Also stattdessen \q{Man soll den Tag nicht vor dem Abend verfluchen.},
		und das sogar auf die ganze Woche ausgedehnt.
		Ich hoffe,
		dass die folgende Woche etwas besser verl\"auft.

	\subsection{Freitag, der 27. Oktober 2023}
		Vergangene Nacht habe ich zu {\Gott} gebetet,
		und ein \"Ubergabegebet zu {\Jesus} gesprochen.
		Und um {\Ihm} zu zeigen,
		dass ich es ernst meine,
		habe ich auch ein paar Dinge abgeschafft.
		Ich m\"ochte da jetzt auch nicht ins Detail gehen,
		weil das einerseits sehr privat ist,
		und andererseits will ich auch nicht als eine Art \q{Prolet} dastehen.
		Ich wollte nur den aktuellen Stand der Dinge mit dir teilen,
		damit du hin und wieder etwas aus meinem Leben,
		und meiner Reise mit {\Gott} mitbekommst.
		
	\subsection{Sonntag, der 29. Oktober 2023}
		Gestern ist mir eine extrem krasse Geschichte eingefallen,
		die m\"oglichwerweise ziemlich anma{\ss}end ist,
		aber dennoch so geil w\"are,
		wenn {\Gott} zuf\"allig genau diesen Plan verfolgen w\"urde.
		Es ist nat\"urlich nur Spekulation,
		und m\"ochte auf keinen Fall blasphemisches von mir geben.
		Doch wenn es dich interessiert,
		m\"ochte ich die Geschichte dennoch gerne mit dir teilen.
		\\
		Also,
		wir wissen ja,
		dass es schon sehr lange Menschen gibt.
		Und die Menschheitsgeschichte,
		von dem Zeitpunkt an,
		wo man sagen kann,
		dass der Mensch sesshaft geworden ist,
		liegt etwa so 10.000 bis 12.000 Jahre vor {\Christus} zur\"uck.
		Grob.
		Das hei{\ss}t,
		rein menschheitsgeschichtlich,
		bis zum heutigen Tag,
		ist das,
		was wir als Mittelalter bezeichnen,
		gar nicht das echte \q{Mittelalter}.
		Aber das nur so nebenbei.
		Und es wird der Tag kommen,
		den wir oft auch als den Tag des J\"ungesten Gerichts,
		oder als Apokalypse bezeichnen,
		an dem wir uns rechtfertigen d\"urfen.
		Denn dann wird der {\Herr} wieder auf die Erde kommen.
		Und es w\"are doch zu geil,
		wenn {\Gott} uns genau in der Mitte der Menschheitsgeschichte,
		also etwa zu der Zeit,
		als {\Jesus} gelebt hat,
		mit {\Ihm} seinen eingeboren {\Sohn} geschickt hat,
		um uns all unsere S\"unden zu vergeben,
		um uns die Chance zu geben,
		uns zu {\Gott} und {\Jesus} zu bekennen.
		Und aktuell sind wir in der zweiten H\"alfte der Menschheitsgeschichte.
		Das hei{\ss}t einmal,
		wir haben als Menschheit einen begrenzten Rahmen,
		von etwa 20.000 bis 24.000 Jahren
		(vielleicht auch ein bisschen mehr).
		Das hei{\ss}t aber auch,
		dass in etwa 8.000 bis 10.000 Jahren der Tag des J\"ungesten Gerichts sein wird.
		Und damit w\"are sozusagen ein kurzer Zeitraum vor {\Jesus},
		und ein kurzer Zeitraum nach {\Jesus} das echte \q{Mittelalter},
		und davor war die echte \q{Antike},
		und danach auch die echte \q{Neuzeit}.
		Ich m\"ochte mich da nat\"urlich zeiltich nicht genau festlegen.
		Und wenn ich daran denke,
		dass die Menschen in der Zeit,
		die wir als Mittelalter bezeichnen,
		als m\"oglicherwiese unwissend und primitiv halten,
		und heutzutage bekriegen wir uns immer noch,
		da stellt sich doch die Frage:
		Sind wir,
		also die Menschheit im Gesamten,
		soviel anders als vor etwa 1.000 Jahren?
		Klar,
		die Technologien haben sich ge\"andert,
		wir leben im Digitalen Zeitalter,
		haben Maschinen,
		Computer,
		Smartphones.
		Die k\"unstliche Intelligenz entwickelt sich auch gerade.
		\\
		Wie gesagt,
		das ist nur eine Geschichte,
		die mir pl\"otzlich in den Sinn gekommen ist,
		und auch irgendwie Sinn ergibt.
		Lass dir diese Geschichte gerne mal durch den Kopf gehen.
		Es w\"are zu interessant zu wissen,
		wie du dar\"uber denkst.
		Es kann nat\"urlich auch sein,
		dass ich total daneben liege,
		und {\Gott} der Menschheit soviel Zeit schenkt,
		wie wir wollen.
		Und das hei{\ss}t nat\"urlich,
		wir sind komplett selbst daf\"ur verantwortlich,
		dass wir uns nicht ausrotten.
		Denn daf\"ur hat {\Er} uns ja den freien Willen geschenkt.

	\subsection{Freitag, der 10. November 2023}
		Ich habe vor kurzem von diesem Buch eine gedruckte Kopie bestellt,
		denn ich wollte sehen,
		wie das echte, gedruckte Exemplar aussieht.
		Und heute ist es angekommen!
		Im Allgemeinen sieht es ganz ordentlich aus,
		doch ich habe gesehen,
		dass ich strukturell noch einiges \"andern darf,
		z.B. Randeinstellungen,
		mehr Abs\"atze,
		usw.
		
	\newpage
	\section{Friede sei mit dir!}
		Zum Schluss m\"ochte ich noch einen wundersch\"onen Refrain mit dir teilen.
		Er ist aus dem Lied \q{Oceans} von der Band Hillsong United.
		\textit{Die \"Ubersetzung findest du auf der n\"achsten Seite.}
		\\
		\begin{itemize}[nosep]
			\item[]	{\Spirit} lead me,
					where my trust is without borders.
			\item[]	Let me walk upon the waters,
			\item[]	wherever {\You} would call me
			\item[]	Take me deeper than my feet could ever wander.
			\item[]	And my faith will be made stronger
			\item[]	in the presence of my {\Saviour}.
					\\
		\end{itemize}

		\begin{figure}[h]
			\centering
			\includegraphics[width=\linewidth,keepaspectratio]{"FreeChristian.jpeg"}
		\end{figure}
		
		\newpage
		Und hier die \"Ubersetung,
		falls du Englisch nicht so gut beherrschst:
		\\
		\begin{itemize}[nosep]
			\item[]	({\Heiliger}) {\Geist},
					f\"uhre mich dorthin,
					\\
					wo mein Vertrauen grenzenlos ist.
			\item[]	Lass mich \"uber die Gew\"asser gehen,
			\item[] wohin auch immer {\Du} mich rufst.
			\item[]	Nimm mich tiefer mit,
					\\
					denn meine F\"u{\ss}e je wandern k\"onnten.
			\item[]	Und mein Glaube wird st\"arker gemacht (werden)
			\item[]	in der Anwesenheit meines {\Erloesers}.
					\\
					\\
		\end{itemize}
		Dieses Lied bedeutet mir zur Zeit sehr viel.
		H\"or es dir gerne einmal an.
		Es ist wirklich wundersch\"on.
		\\
		\\
		Falls du bis hierhin gelesen hast,
		danke ich dir von ganzem Herzen,
		und w\"unsche dir f\"ur deine Zukunft alles nur Gute.

\end{document}
