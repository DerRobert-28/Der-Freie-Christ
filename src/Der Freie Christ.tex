\documentclass[12pt,a4paper]{article}

\usepackage[ngerman]{babel}
\usepackage{enumitem}
\usepackage[utf8]{inputenc}
\usepackage[T1]{fontenc}
\usepackage[left=25mm,right=25mm,top=25mm,bottom=25mm]{geometry}
\usepackage[colorlinks=true,urlcolor=blue,linkcolor=black]{hyperref}

\newcommand{\Christi}[0]{\textbf{CHRISTI}}
\newcommand{\Christus}[0]{\textbf{CHRISTUS}}
\newcommand{\Deinem}[0]{\textbf{DEINEM}}
\newcommand{\Deinen}[0]{\textbf{DEINEN}}
\newcommand{\Deiner}[0]{\textbf{DEINER}}
\newcommand{\Deines}[0]{\textbf{DEINES}}
\newcommand{\Deine}[0]{\textbf{DEINE}}
\newcommand{\Dein}[0]{\textbf{DEIN}}
\newcommand{\Deren}[0]{\textbf{DEREN}}
\newcommand{\Dich}[0]{\textbf{DICH}}
\newcommand{\Dir}[0]{\textbf{DIR}}
\newcommand{\Du}[0]{\textbf{DU}}
\newcommand{\Erloesers}[0]{\textbf{ERL\"OSERS}}
\newcommand{\Erloeser}[0]{\textbf{ERL\"OSER}}
\newcommand{\Er}[0]{\textbf{ER}}
\newcommand{\Geiste}[0]{\textbf{GEISTE}}
\newcommand{\Geist}[0]{\textbf{GEIST}}
\newcommand{\Gottes}[0]{\textbf{GOTTES}}
\newcommand{\Gott}[0]{\textbf{GOTT}}
\newcommand{\Heiligen}[0]{\textbf{HEILIGEN}}
\newcommand{\Heiliger}[0]{\textbf{HEILIGER}}
\newcommand{\Heilige}[0]{\textbf{HEILIGE}}
\newcommand{\Heilig}[0]{\textbf{HEILIG}}
\newcommand{\Herrn}[0]{\textbf{HERRN}}
\newcommand{\Herr}[0]{\textbf{HERR}}
\newcommand{\Ihm}[0]{\textbf{IHM}}
\newcommand{\Ihnen}[0]{\textbf{IHNEN}}
\newcommand{\Ihn}[0]{\textbf{IHN}}
\newcommand{\Ihr}[0]{\textbf{IHR}}
\newcommand{\Jesus}[0]{\textbf{JESUS}}
\newcommand{\Jesu}[0]{\textbf{JESU}}
\newcommand{\Messias}[0]{\textbf{MESSIAS}}
\newcommand{\Seinem}[0]{\textbf{SEINEM}}
\newcommand{\Seinen}[0]{\textbf{SEINEN}}
\newcommand{\Seiner}[0]{\textbf{SEINER}}
\newcommand{\Seines}[0]{\textbf{SEINES}}
\newcommand{\Seine}[0]{\textbf{SEINE}}
\newcommand{\Sein}[0]{\textbf{SEIN}}
\newcommand{\Sie}[0]{\textbf{SIE}}
\newcommand{\Sohnes}[0]{\textbf{SOHNES}}
\newcommand{\Sohne}[0]{\textbf{SOHNE}}
\newcommand{\Sohn}[0]{\textbf{SOHN}}

\newcommand{\q}[1]{\char"22{#1}\char"22 }

\title{\textbf{Der Freie Christ}}
\author{Robert Lang-Kirchh\"ofer}
\date{\textit{Letzte \"Anderung: 29. September 2023}}

\begin{document}
	\setlength{\parindent}{0mm}
	\maketitle
	\newpage

	\tableofcontents
	\newpage
	
	\section{Vorwort}

	\subsection{Verwendungshinweis}
		Ich werde als pers\"onliche Anrede das informelle \q{du},
		und auch das generische Maskulinum verwenden.
		Das soll einerseits eine angenehme,
		lockere Atmosph\"are zu schaffen,
		und andererseits den Lesefluss zu erleichtern.
		Selbstverst\"andlich gilt dir,
		mein lieber Leser,
		unabh\"angig von deinem tats\"achlichen Geschlecht,
		mein vollster Respekt.

	\subsection{Danksagung}
		Als n\"achstes m\"ochte ich dir,
		lieber Leser,
		meinen herzlichsten Dank aussprechen,
		dass du dich daf\"ur entschieden hast,
		hier reinzuschnuppern.
		Ich hoffe nat\"urlich,
		dass du dieses e"~Book bis zum Schluss durchlie{\ss}t,
		und seine Weiterentwicklung beobachtest.
		Ich kann nicht mit 100\%iger Sicherheit sagen,
		dass es je \q{fertig} sein wird,
		weil ich m\"oglichweise immer wieder neue Gedanken,
		oder neues Material finden werde,
		das ich hier aufnehmen werde.			
		Es handelt sich hierbei n\"amlich um ein christliches Schriftst\"uck.
		Ich will dir hiermit moralische Werte \"ubermitteln,
		insbesondere wie sie,
		nat\"urlich nach bestem Wissen und Gewissen,
		von {\Gott},
		dem {\Herrn},
		und {\Seinem} {\Sohn} {\Jesus} {\Christus} gew\"unscht sind.
		Wie du in diesem Vorwort schon erkennen kannst,
		sind Worte die sich direkt auf {\Gott},
		{\Jesus} oder auch den {\Heiligen} {\Geist} beziehen,
		in Majuskeln,
		also komplett in Gro{\ss}buchstaben,
		und zus\"atzlich in Fettschrift geschrieben.
		Wenn dir etwas am Herzen liegt,
		oder dir allgemein etwas hierzu einf\"allt,
		bist du herzlich eingeladen,
		in meiner \href{https://github.com/DerRobert-28/Der-Freie-Christ/discussions}{GitHub-Diskussion} mitzuwirken.
	
	\subsection{Ein paar Worte zu mir}
		Ich selbst wurde,
		soweit ich mich richtig erinnere,
		mit etwa ein/zwei Jahren katholisch getauft,
		bin aber Mitte August 2023 aus der Kirche ausgetreten.
		Die Gr\"unde hierf\"ur sind pers\"onlicher Art,
		und sind hier nicht von Bedeutung.
		Das hat jedoch nichts mit meinem Glauben zu tun.
		Ich selbst glaube,
		dass {\Gott} existiert,
		und dass {\Jesus} der {\Erloeser} ist.
		Das hei{\ss}t aber nicht,
		dass ich sowas wie der \q{perfekte Christ} bin,
		falls es sowas unter uns Menschen heutzutage \"uberhaupt gibt.
		Mehr zu mir kannst du im Kapitel \q{\hyperref[MeinLebenMitGott]{Mein Leben mit {\Gott}}} lesen.
	
	\section{Was macht einen \q{Freien Christen} aus?}
	
	\subsection{Keine Kirchengemeinde}
		Ein Freier Christ ist nicht an eine Kirchengemeinde gebunden.
		Das hei{\ss}t,
		man darf,
		aber man muss nicht getauft sein.
		Man kann auch aus der Kirche ausgetreten sein.
		Das spielt alles keine Rolle.
		Wichtig ist nur,
		dass man {\Gott},
		den {\Herrn},
		und {\Jesus} {\Christus},
		seinen eingeborenen Sohn,
		in sein Leben l\"asst,
		und sich zu {\Ihnen} bekennt.
	
	\subsection{Der wahre Bund}
		F\"ur mich ist der einzig wahre,
		bestehende Bund zwischen {\Gott},
		{\Jesus} {\Christus} und mir.
		Wenn ich mich zu {\Ihnen} bekenne,
		pflege ich diese Beziehung von Herzen.
		Weltliche B\"unde \textit{(Beziehungen)} sind verg\"anglich,
		dennoch ist es nicht weniger wichtig,
		auch diese herzlich zu pflegen.
	
	\subsection{Die Bibel als \q{Werkzeug}}
		Wenn es der Beziehung zwischen {\Gott},
		{\Jesus} und mir dient,
		habe ich die Freiheit,
		Bibelstellen besser, also moderner oder verst\"andlicher, auszulegen,
		und entsprechend umzuformulieren.
		Das ist jedoch \textbf{kein} Freibrief daf\"ur,
		das Wort {\Gottes} nach Gutdünken umzuschreiben,
		und damit beispielsweise {\Seinen} Willen zu beugen,
		so wie es,
		meinen Informationen und Recherchen nach,
		die Katholische Kirche in der Vergangenheit \q{gerne} gemacht hat.
		
	\section{Die Zehn Gebote}
		Die traditionellen 10 Gebote werden \"ublicherweise aus der Sicht {\Gottes} \"uberliefert,
		also in der Form \q{Du sollst (nicht) ...}.
		Im folgenden sind die 10 Gebote aus der Sicht,
		wenn man selbst zu {\Gott} sprechten w\"urde,
		und {\Ihm} die Gebote als Versprechen geben w\"urde.
		Auch sind sie etwas besser ausgearbeitet,
		da man manche Gebote bei genauerer Betrachtung auch zusammenfassen k\"onnte.
		Das bedeutet selbstverst\"andlich nicht,
		dass ich die traditionellen,
		von {\Gott} gegebenen Gebote ablehne.
		Ich m\"ochte nur eine andere Betrachtungsweise zeigen.
		Desweiteren werde ich sie \q{Angebote} nennen,
		um den guten Willen beider Seiten unterstreichen.
	
	\subsection{Das Oberste Angebot}
		Das Oberste Angebot lautet:
		Ich will {\Gott}, den {\Herrn}, von ganzem Herzen lieben und {\Ihn} ehren. Und ich will meinen N\"achsten lieben, wie auch mich selbst.
		
	\subsection{Das Erste Angebot}
		{\Du} bist der {\Herr},
		mein {\Gott},
		mein {\Erloeser}.
		Ich will keine anderen G\"otter neben {\Dir} haben,
		und sie nicht anbeten oder verehren.
		Und ich will mir kein G\"otzenbild schaffen.
		
	\subsection{Das Zweite Angebot}
		{\Du} bist der {\Herr},
		mein {\Gott}.
		Ich will {\Deinen} Namen nicht missbrauchen.
		Ich will {\Dir} nicht l\"astern.
		Und ich will mich ehrlich zu {\Dir} bekennen.
			
	\subsection{Das Dritte Angebot}
		{\Du} bist der {\Herr},
		mein {\Gott}.
		Ich will {\Dich} nicht auf die Probe stellen.
		Ich will {\Dich} nicht versuchen.
		Ich will auch in der Not zu {\Dir} stehen.
		
	\subsection{Das Vierte Angebot}
		{\Du} bist der {\Herr},
		mein {\Gott}.
		Ich will {\Dir} den Sabbat heiligen.
		Ich will am Sabbat des Fleischlichen,
		und Suchterzeugenden enthaltsam bleiben.
		
	\subsection{Das F\"unfte Angebot}
		Ich will meinen Vater und meine Mutter,
		die mir mein Leben geschenkt,
		mich gro{\ss}gezogen und ern\"ahrt haben,
		ehren.
		Und ich will \"Altere Menschen ehren.
			
	\subsection{Das Sechste Angebot}
		Ich will nicht t\"oten oder morden.
		Ich will meine Beziehungen pflegen.
		Ich will das Leben und Wohlergehen allen Lebens respektieren,
		und nach M\"oglichkeit auch sch\"utzen.
		
	\subsection{Das Siebte Angebot}
		Ich will nicht die Ehe brechen.
		Ich will nicht die Frau meines N\"achsten begehren.
		Ich will nicht den Mann meiner N\"achsten begehren.
		
	\subsection{Das Achte Angebot}
		Ich will nicht stehlen oder betr\"ugen.
		Ich will nicht rauben oder entf\"uhren.
		Ich will nicht begehren meines N\"achsten Haus.
		Ich will nicht begehren meines N\"achsten Hab und Gut.
		Ich will dem Hab und Gut meines N\"achsten keinen Schaden zuf\"ugen.
		
	\subsection{Das Neunte Angebot}
		Ich will nicht falsch Zeugnis geben wider meinem N\"achsten.
		Ich will nicht l\"ugen oder betr\"ugen.
		Ich will nicht schw\"oren.
		Ich will gegen\"uber meinem N\"achsten ehrlich und gerecht handeln.
		
	\subsection{Das Zehnte Angebot}
		Mein K\"orper ist ein Geschenk von {\Dir},
		und somit heilig.
		Ich will ihn ehren und pflegen.

	\section{Mein Leben mit {\Gott}} \label{MeinLebenMitGott}
		Hierbei handelt es sich um eine Art Tagebuch,
		wie ich meine Reise mit und zu {\Gott} erlebe,
		und was ich sonst noch dabei lernen darf.
	
	\subsection{Mittwoch, der 27. September 2023}
		Ich bin seit etwa Mitte 2023 auf einer Art Reise,
		bei der ich mich entschieden habe,
		{\Gott} und {\Jesus} in mein Leben zu lassen.
		Ich habe selbst noch viele Fehler,
		und obgleich der von {\Gott} gegebenen \textit{(An-)}Gebote,
		s\"undige ich noch viel zu oft.
		Wie im Vorwort erw\"ahnt,
		bin ich weit davon entfernt,
		so etwas wie der \q{perfekte Christ} zu sein.
		Viele der allt\"aglichen Gewohnheiten,
		Pr\"agungen und sonstiges haben so eine starke Sogwirkung,
		dass ich auch nicht immer an {\Gott} denke,
		nicht so oft bete,
		oder in der Bibel lese,
		wie ich gerne w\"urde.
		Und wenn ich dann \q{wieder} an {\Gott} denke,
		habe ich oft ein schlechtes Gewissen,
		weil ich {\Ihn} dann gef\"uhlt \q{vergessen} habe.
		Also kurzum:
		Ich darf noch sehr, sehr, sehr, ..., sehr viel lernen!

\end{document}
