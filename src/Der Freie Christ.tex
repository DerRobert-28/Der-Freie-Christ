\documentclass[12pt,a4paper]{article}

\usepackage[ngerman]{babel}
\usepackage{enumitem}
\usepackage[utf8]{inputenc}
\usepackage[T1]{fontenc}
\usepackage[left=25mm,right=25mm,top=25mm,bottom=25mm]{geometry}
\usepackage{graphicx}
\usepackage[colorlinks=true,urlcolor=blue,linkcolor=black]{hyperref}

\newcommand{\Christi}[0]{\textbf{CHRISTI}}
\newcommand{\Christus}[0]{\textbf{CHRISTUS}}
\newcommand{\Deinem}[0]{\textbf{DEINEM}}
\newcommand{\Deinen}[0]{\textbf{DEINEN}}
\newcommand{\Deiner}[0]{\textbf{DEINER}}
\newcommand{\Deines}[0]{\textbf{DEINES}}
\newcommand{\Deine}[0]{\textbf{DEINE}}
\newcommand{\Dein}[0]{\textbf{DEIN}}
\newcommand{\Deren}[0]{\textbf{DEREN}}
\newcommand{\Dich}[0]{\textbf{DICH}}
\newcommand{\Dir}[0]{\textbf{DIR}}
\newcommand{\Du}[0]{\textbf{DU}}
\newcommand{\Elohim}[0]{\textbf{ELOHIM}}
\newcommand{\Erloesers}[0]{\textbf{ERL\"OSERS}}
\newcommand{\Erloeser}[0]{\textbf{ERL\"OSER}}
\newcommand{\Er}[0]{\textbf{ER}}
\newcommand{\Geiste}[0]{\textbf{GEISTE}}
\newcommand{\Geist}[0]{\textbf{GEIST}}
\newcommand{\Gottes}[0]{\textbf{GOTTES}}
\newcommand{\Gott}[0]{\textbf{GOTT}}
\newcommand{\Heiligen}[0]{\textbf{HEILIGEN}}
\newcommand{\Heiliger}[0]{\textbf{HEILIGER}}
\newcommand{\Heilige}[0]{\textbf{HEILIGE}}
\newcommand{\Heilig}[0]{\textbf{HEILIG}}
\newcommand{\Herrn}[0]{\textbf{HERRN}}
\newcommand{\Herr}[0]{\textbf{HERR}}
\newcommand{\Ihm}[0]{\textbf{IHM}}
\newcommand{\Ihnen}[0]{\textbf{IHNEN}}
\newcommand{\Ihn}[0]{\textbf{IHN}}
\newcommand{\Ihr}[0]{\textbf{IHR}}
\newcommand{\Jahwe}[0]{\textbf{JAHWE}}
\newcommand{\Jesus}[0]{\textbf{JESUS}}
\newcommand{\Jesu}[0]{\textbf{JESU}}
\newcommand{\Messias}[0]{\textbf{MESSIAS}}
\newcommand{\Seinem}[0]{\textbf{SEINEM}}
\newcommand{\Seinen}[0]{\textbf{SEINEN}}
\newcommand{\Seiner}[0]{\textbf{SEINER}}
\newcommand{\Seines}[0]{\textbf{SEINES}}
\newcommand{\Seine}[0]{\textbf{SEINE}}
\newcommand{\Sein}[0]{\textbf{SEIN}}
\newcommand{\Sich}[0]{\textbf{SICH}}
\newcommand{\Sie}[0]{\textbf{SIE}}
\newcommand{\Sohnes}[0]{\textbf{SOHNES}}
\newcommand{\Sohne}[0]{\textbf{SOHNE}}
\newcommand{\Sohn}[0]{\textbf{SOHN}}
\newcommand{\Vater}[0]{\textbf{VATER}}
\newcommand{\Vaters}[0]{\textbf{VATERS}}

\newcommand{\q}[1]{\char"22{#1}\char"22 }

\title{\textbf{Der Freie Christ}}
\author{Robert Lang-Kirchh\"ofer}
\date{\textit{Letzte \"Anderung: 4. Oktober 2023}}

\begin{document}
	\setlength{\parindent}{0mm}
	\maketitle
	\begin{figure}[h]
		\centering
		\includegraphics[width=1\textwidth,keepaspectratio]{"FreeChristian.jpeg"}
	\end{figure}

	\newpage
	\tableofcontents
	
	\newpage
	\section{Vorwort}

	\subsection{Verwendungshinweis}
		Ich werde als pers\"onliche Anrede das informelle \q{du},
		und auch das generische Maskulinum verwenden.
		Das soll einerseits eine angenehme,
		lockere Atmosph\"are schaffen,
		und andererseits den Lesefluss erleichtern.
		Selbstverst\"andlich gilt dir,
		mein lieber Leser,
		unabh\"angig von deinem tats\"achlichen Geschlecht,
		mein vollster Respekt.
		Da ich selbst auch kein professioneller Author bin,
		ist mein Schreibstil auch nicht perfekt,
		sondern teilweise etwas lockerer.
		Solltest du die englische Version lesen,
		wirst du vielleicht auch feststellen,
		dass mein Englisch nicht perfekt ist,
		da dies nicht meine Muttersprache ist.

	\subsection{Danksagung}
		Als n\"achstes m\"ochte ich dir,
		lieber Leser,
		meinen herzlichsten Dank aussprechen,
		dass du dich daf\"ur entschieden hast,
		hier reinzuschnuppern.
		Ich hoffe nat\"urlich,
		dass du dieses e"~Book bis zum Schluss durchlie{\ss}t,
		und seine Weiterentwicklung beobachtest.
		Ich kann nicht mit 100\%iger Sicherheit sagen,
		dass es je \q{fertig} sein wird,
		weil ich m\"oglichweise immer wieder neue Gedanken,
		oder neues Material finden werde,
		das ich hier aufnehmen werde.			
		Es handelt sich hierbei n\"amlich um ein christliches Schriftst\"uck.
		Ich will dir hiermit moralische Werte \"ubermitteln,
		insbesondere wie sie,
		nat\"urlich nach bestem Wissen und Gewissen,
		von {\Gott},
		dem {\Herrn},
		und {\Seinem} {\Sohn} {\Jesus} {\Christus} gew\"unscht sind.
		Ich m\"ochte selbstverst\"andlich keine neue \q{Religion},
		oder das Christentum neu erfinden,
		sondern lediglich ein paar neue Perspektiven aufzeigen.
		Wie du in diesem Vorwort schon erkennen kannst,
		sind Worte die sich direkt auf {\Gott},
		{\Jesus} oder auch den {\Heiligen} {\Geist} beziehen,
		in Majuskeln,
		also komplett in Gro{\ss}buchstaben,
		und zus\"atzlich in Fettschrift geschrieben.
		Wenn dir etwas am Herzen liegt,
		oder dir allgemein etwas hierzu einf\"allt,
		bist du herzlich eingeladen,
		in meiner \href{https://github.com/DerRobert-28/Der-Freie-Christ/discussions}{GitHub-Diskussion} mitzuwirken.
	
	\subsection{Ein paar Worte zu mir}
		Ich selbst wurde,
		soweit ich mich richtig erinnere,
		mit etwa ein/zwei Jahren katholisch getauft,
		bin aber Ende Juli 2023 aus der Kirche ausgetreten.
		Die Gr\"unde hierf\"ur sind pers\"onlicher Art,
		und sind hier nicht von Bedeutung.
		Das hat jedoch nichts mit meinem Glauben zu tun.
		Ich selbst glaube,
		dass {\Gott} existiert,
		und dass {\Jesus} der {\Erloeser} ist.
		Das hei{\ss}t aber nicht,
		dass ich sowas wie der \q{perfekte Christ} bin,
		falls es sowas unter uns Menschen heutzutage \"uberhaupt gibt.
		Mehr zu mir kannst du im Kapitel \q{\hyperref[MeinLebenMitGott]{Mein Leben mit {\Gott}}} lesen.
	
	\newpage
	\section{Was macht einen \q{Freien Christen} aus?}
	
	\subsection{Keine Kirchengemeinde}
		Ein Freier Christ ist nicht an eine Kirchengemeinde gebunden.
		Das hei{\ss}t,
		man braucht nicht getauft sein.
		Man kann auch aus der Kirche ausgetreten sein.
		Das spielt alles keine Rolle.
		Ich halte es sogar f\"ur viel sinnvoller,
		gar nicht erst einer Kirchengemeinde anzugeh\"oren,
		insbesondere der katholischen,
		um nicht von ihr manipuliert zu werden.
		Wichtig ist nur,
		dass man {\Gott},
		den {\Herrn},
		und {\Jesus} {\Christus},
		seinen eingeborenen Sohn,
		in sein Leben l\"asst,
		und sich zu {\Ihnen} bekennt.
	
	\subsection{Der wahre Bund}
		F\"ur dich,
		als Freien Christ,
		ist der einzig wahre,
		bestehende Bund zwischen {\Gott},
		{\Jesus} {\Christus} und dir.
		Wenn du dich zu {\Ihnen} bekennst,
		pflegst du diese Beziehung von Herzen.
		Weltliche B\"unde \textit{(Beziehungen)} sind verg\"anglich,
		dennoch ist es nicht weniger wichtig,
		auch diese herzlich zu pflegen.
	
	\subsection{Die Bibel als \q{Werkzeug}}
		Wenn es der Beziehung zwischen {\Gott},
		{\Jesus} und dir dient,
		hast du,
		als Freier Christ,
		die Freiheit,
		Bibelstellen besser,
		also moderner oder verst\"andlicher,
		auszulegen,
		und entsprechend umzuformulieren.
		Das ist jedoch \textbf{kein} Freibrief daf\"ur,
		das Wort {\Gottes} nach Gutdünken umzuschreiben,
		und damit beispielsweise {\Seinen} Willen zu beugen,
		so wie es,
		meinen Informationen und Recherchen nach,
		die katholische Kirche in der Vergangenheit \q{gerne} gemacht hat.
	
	\subsection{Glaube und Wissenschaft}
		Glaube und Wissenschaft schlie{\ss}en sich,
		meiner Meinung nach,
		nicht gegenseitig aus.
		{\Gott} kann man weder beweisen,
		noch widerlegen.
		Du hast jederzeit die freie Wahl,
		ob du mit {\Gott} leben willst,
		oder nicht.
		Daf\"ur hast du als Mensch einen freien Willen bekommen.
	
	\subsection{Was wei{\ss} ich \"uber {\Gott}?}
		Nun ... was \q{wei{\ss}} man schon wirklich \"uber {\Gott}?
		Klar,
		ich kann {\Gott} und {\Jesus} durch {\Sein} Heiliges Wort,
		die Bibel,
		besser kennenlernen.
		Ich w\"urde mir jedoch niemals anma{\ss}en,
		zu behaupten,
		dass ich {\Gott} \q{kenne}.
		Schon gar nicht absolut.
		Nur {\Gott} kennt {\Sich} selbst ganz und gar.
		Und in diesem e"~Book m\"ochte ich dir einfach meine Erfahrung mitgeben.
		Und was ich \"uber {\Gott} sage,
		ist nur nach bestem Wissen und Gewissen,
		und nur zu {\Seinen} Gunsten.

	\newpage
	\section{Glaubst du (nicht) an {\Gott}?}
		Bei diesem e"~Book handelt es sich zwar um ein christliches Schriftst\"uck,
		das hei{\ss}t aber nicht,
		dass nur Christen dieses e"~Book lesen d\"urfen.
		Ganz im Gegenteil!
		Auch wenn du an etwas anderes oder auch nicht glaubst,
		bist du herzlich eingeladen,
		hier reinzust\"obern.
		Gott liebt uns alle gleich.
		Doch ... woran glaubst du eigentlich?
	
	\subsection{Ich bin Jude und glaube nicht an {\Jesus} als {\Erloeser}.}
		Dann bist du dennoch herzlich eingeladen,
		weiterzulesen.
		Wir glauben ja effektiv an den gleichen {\Gott},
		der in der Heiligen Schrift auch {\Elohim} oder {\Jahwe} genannt wird.
		Meines Wissens nach sind gro{\ss}e Teile deines heilgen Buches,
		z.B. die Torah,
		deckungsgleich mit dem Alten Testament der Bibel.
		Als Beispiel nenne ich die Entstehungsgeschichte,
		oder den Auszug aus \"Agypten.		
		Da ich ja bereits einger\"aumt habe,
		dass ich mir nicht anma{\ss}e,
		{\Gott} zu kennen,
		w\"urde ich auch nicht behaupten,
		dass du,
		nur weil du nicht an {\Jesus} glaubst,
		ein schlechterer Mensch bist,
		und deswegen in die H\"olle,
		oder sonstwohin kommst.
		Wenn du {\Jesus} nicht als deinen {\Erloeser} anerkennst,
		dann akzeptiere ich das.
		Ich bitte dich,
		es ebenfalls zu akzeptieren,
		dass f\"ur mich {\Jesus} der {\Erloeser} ist.
		Es geht mir auch gar nicht darum,
		dir {\Jesus} \q{auszuzwingen},
		sondern im Gro{\ss}en und Ganzen um moralische Werte,
		und darum,
		meine Erfahrungen mit dir zu teilen.
		
	\subsection{Ich bin Muslim und glaube an Allah.}
		Dann bist auch du herzlich eingeladen,
		weiterzulesen.
		Ich habe bereits einger\"aumt,
		dass ich mir nicht anma{\ss}e,
		{\Gott} zu kennen.
		Von daher wei{\ss} ich auch nicht,
		ob (der) {\Gott},
		an den ich glaube,
		und dein Gott,
		Allah,
		effektiv der gleiche Gott sind,
		oder zwei verschiedene,
		und die Gleichstellung vielleicht sogar Blasphemie,
		also Gottesl\"asterung,
		ist.
		Ich werde dir deinen Glauben nicht absprechen.
		Du darfst zu Allah beten,
		und ihn als deinen alleinigen Gott ansehen.
		Wenn du mich als Ungl\"aubigen betrachtest,
		nehme ich das als deine Meinung an.
		Ich akzeptiere,
		dass Allah f\"ur dich dein alleiniger Gott ist.
		Ich bitte dich,
		zu akzeptieren,
		dass {\Gott} f\"ur mich mein alleiniger Gott ist.
		Ich hoffe dennoch darauf,
		dass wir uns auch au{\ss}erhalb unseres Glaubens,
		einfach aus menschlicher Sicht,
		gegenseitig respektieren k\"onnen.
		Es geht mir hier schlie{\ss}lich gar nicht darum,
		dir einen anderen Gott \q{aufzuzwingen},
		sondern im Gro{\ss}en und Ganzen um moralische Werte,
		und darum,
		meine Erfahrungen mit dir zu teilen.
		
	\subsection{Ich geh\"ore einer anderen Religion an.}
		Ich kann leider nicht auf alle Glaubensrichtungen,
		Religionen und Sekten dieser Welt eingehen.
		Das w\"urde den Rahmen dieses e"~Books sprengen.
		Von daher verzeih mir,
		wenn ich deinen Glauben nicht explizit aufgef\"uhrt habe.
		Wenn du die beiden vorherigen Abschnitte liest,
		und du beispielsweise zu dem Schluss kommst,
		dass ich aufgrund deines Glaubens ein Ungl\"aubiger bin,
		dann nehme ich das als deine Meinung an.
		Ich akzeptiere,
		dass die Gottheit,
		oder die Gottheiten,
		und selbst wenn es der Teufel ist,
		diejenigen sind,
		die f\"ur dich anbetungsw\"urdig sind.
		Ich bitte dich ebenfalls,
		zu akzeptieren,
		dass {\Gott} f\"ur mich mein alleiniger Gott ist.
		Ich hoffe dennoch darauf,
		dass wir uns auch au{\ss}erhalb unseres Glaubens,
		einfach aus menschlicher Sicht,
		gegenseitig respektieren k\"onnen.
		Es geht mir hier,
		wie oben bereits erw\"ahnt,
		schlie{\ss}lich gar nicht darum,
		dir \q{meinen} {\Gott} aufzuzwingen,
		sondern im Gro{\ss}en und Ganzen um moralische Werte,
		und darum,
		meine Erfahrungen mit dir zu teilen.
	
	\subsection{Ich glaube an gar keinen Gott, oder bin Agnostiker.}
		Auch du bist herzlich zum Weiterlesen eingeladen.
		Denn selbst du,
		mein Freund,
		wenn du nicht (mehr) an {\Gott} glaubst,
		oder dir die Existenz von etwas g\"ottlichem gleichg\"ultig oder unbekannt ist,
		hast dennoch in irgendeiner Form moralische Werte.
		M\"ochtest du einfach so verletzt oder gar get\"otet werden?
		M\"ochtest du bestohlen oder betrogen werden?
		M\"ochtest du,
		dass dir dein Partner fremdgeht,
		wenn du nicht gerade sowas wie eine offene Beziehung f\"uhrst?
		M\"ochtest du belogen werden?
		M\"ochtest du nicht auch einen gesunden K\"orper haben,
		und eine hohe Lebensqualit\"at genie{\ss}en?
		M\"ochtest du nicht ganz tief in dir einfach nur geliebt werden,
		so wie du bist?
		Deswegen akzeptiere ich,
		wenn du in deinem Leben keinen Gott brauchst.
		Ich bitte dich dennoch,
		zu akzeptieren,
		dass {\Gott} f\"ur mich existiert und der alleinige Gott ist.
		Ich hoffe darauf,
		dass wir uns auch einfach aus menschlicher Sicht,
		gegenseitig respektieren k\"onnen.
		Es geht mir hier schlie{\ss}lich gar nicht darum,
		dir einen Gott oder einen Glauben \q{aufzuzwingen},
		sondern im Gro{\ss}en und Ganzen um moralische Werte,
		und darum,
		meine Erfahrungen mit dir zu teilen.
				
	\newpage
	\section{Die Zehn Gebote}
		Die traditionellen 10 Gebote werden \"ublicherweise aus der Sicht {\Gottes} \"uberliefert,
		also in der Form \q{Du sollst (nicht) ...}.
		Im folgenden sind die 10 Gebote aus der Sicht,
		wenn man selbst zu {\Gott} sprechten w\"urde,
		und {\Ihm} die Gebote als Versprechen geben w\"urde.
		Auch sind sie etwas besser ausgearbeitet,
		da man manche Gebote bei genauerer Betrachtung auch zusammenfassen k\"onnte.
		Das bedeutet selbstverst\"andlich nicht,
		dass ich die traditionellen,
		von {\Gott} gegebenen Gebote ablehne.
		Ich m\"ochte nur eine andere Betrachtungsweise zeigen.
		Desweiteren werde ich sie \q{Angebote} nennen,
		um den guten Willen beider Seiten unterstreichen.
	
	\subsection{Das Oberste Angebot}
		Das Oberste Angebot lautet:
		Ich will {\Gott}, den {\Herrn}, von ganzem Herzen lieben und {\Ihn} ehren. Und ich will meinen N\"achsten lieben, wie auch mich selbst.
		
	\subsubsection{Das Erste Angebot}
		{\Du} bist der {\Herr},
		mein {\Gott},
		mein {\Erloeser}.
		Ich will keine anderen G\"otter neben {\Dir} haben,
		und sie nicht anbeten oder verehren.
		Und ich will mir kein G\"otzenbild schaffen.
		
	\subsubsection{Das Zweite Angebot}
		{\Du} bist der {\Herr},
		mein {\Gott}.
		Ich will {\Deinen} Namen nicht missbrauchen.
		Ich will {\Dir} nicht l\"astern.
		Und ich will mich ehrlich zu {\Dir} bekennen.
		\\
		\textit{Kurzer Hinweis:
		Hiermit sollen auch umgangssprachliche Phrasen,
		wie beispsielsweise \q{Oh (mein) G...},
		oder \q{Um G...es Willen},
		die man schnell sagt,
		ohne aber wirklich {\Gott} selbst zu meinen,
		oder zu ihm zu beten,
		oder \"ahnliches.}
			
	\subsubsection{Das Dritte Angebot}
		{\Du} bist der {\Herr},
		mein {\Gott}.
		Ich will {\Dich} nicht auf die Probe stellen.
		Ich will {\Dich} nicht versuchen.
		Ich will auch in der Not zu {\Dir} stehen.
		\\
		\textit{Kurzer Hinweis:
		Hiermit sollen auch Situationen abgedeckt werden,
		in denen man leichtfertig solche Dinge sagt wie beispielsweise,
		wie {\Gott} dieses oder jenes Leid zulassen kann.}
		
	\subsubsection{Das Vierte Angebot}
		{\Du} bist der {\Herr},
		mein {\Gott}.
		Ich will {\Dir} den Sabbat heiligen.
		Ich will am Sabbat des Fleischlichen,
		und Suchterzeugenden enthaltsam bleiben.
		
	\subsubsection{Das F\"unfte Angebot}
		Ich will meinen Vater und meine Mutter,
		die mir mein Leben geschenkt,
		mich gro{\ss}gezogen und ern\"ahrt haben,
		ehren.
		Und ich will \"Altere Menschen ehren.
			
	\subsubsection{Das Sechste Angebot} \label{DasSechsteAngebot}
		Ich will nicht t\"oten oder morden.
		Ich will meine Beziehungen pflegen.
		Ich will das Leben und Wohlergehen allen Lebens respektieren,
		und nach M\"oglichkeit auch sch\"utzen.
		\\
		\textit{Kurzer Hinweis:
		Das T\"oten ist hier nicht nur w\"ortlich,
		also physisch gemeint,
		sondern auch symbolisch,
		indem man beispielsweise aus Zorn irgendetwas zu jemandem sagt,
		was ihn verletzt,
		und damit der Beziehung schadet.
		Lies gerne dazu die Bibelstelle \href{https://www.die-bibel.de/bibeln/online-bibeln/lesen/LU17/MAT.5/Matthäus-5}{Matthäus 5, 21-22}.}
		
	\subsubsection{Das Siebte Angebot}
		Ich will nicht die Ehe brechen.
		Ich will nicht die Frau meines N\"achsten begehren.
		Ich will nicht den Mann meiner N\"achsten begehren.
		
	\subsubsection{Das Achte Angebot}
		Ich will nicht stehlen oder betr\"ugen.
		Ich will nicht rauben oder entf\"uhren.
		Ich will nicht begehren meines N\"achsten Haus.
		Ich will nicht begehren meines N\"achsten Hab und Gut.
		Ich will dem Hab und Gut meines N\"achsten keinen Schaden zuf\"ugen.
		
	\subsubsection{Das Neunte Angebot} \label{DasNeunteAngebot}
		Ich will nicht falsch Zeugnis geben wider meinem N\"achsten.
		Ich will nicht l\"ugen oder betr\"ugen.
		Ich will nicht schw\"oren.
		Ich will gegen\"uber meinem N\"achsten ehrlich und gerecht handeln.
		\\
		\textit{Kurzer Hinweis:
		Das Wort \q{schw\"oren} ist im Englischen zweideutig.
		Da es n\"amlich mit \q{(to) swear} \"ubersetzt wird,
		kann man darunter \q{(einen Eid) schw\"oren},
		oder \q{fluchen} verstehen.
		Somit k\"onnte man es auch \"ubersetzen mit
		\q{Ich will nicht (be)schw\"oren oder (ver)fluchen.}}
		
	\subsubsection{Das Zehnte Angebot} \label{DasZehnteAngebot}
		Mein K\"orper und mein Leben sind ein Geschenk von {\Dir},
		und somit heilig.
		Ich will sie ehren und pflegen.

	\subsection{Zusammenfassung}
		Gek\"urzt kann man die 10 \textit{(An-)}Gebote wie folgt zusammenfassen:
		\\
		\begin{enumerate}[nosep]
			\item Ich will keine anderen G\"otter neben {\Dir} haben.
			\item Ich will {\Deinen} Namen nicht missbrauchen.
			\item Ich will {\Dich} nicht versuchen.
			\item Ich will {\Dir} den Sabbat heiligen.
			\item Ich will Vater und Mutter ehren.
			\item Ich will nicht t\"oten \textit{(oder morden)}.
			\item Ich will nicht ehebrechen.
			\item Ich will nicht stehlen \textit{(oder betr\"ugen)}.
			\item Ich will nicht falsch Zeugnis geben \textit{(wider meinem N\"achsten)}.
			\item Ich will mein Leben, {\Dein} Geschenk, ehren.
		\end{enumerate}
	
	\subsection{Gegen\"uberstellung}
		Im Judentum sind die 10 Gebote,
		die in der Torah sinngem\"a{\ss} \q{10 Worte, die {\Gott} gesprochen hat} hei{\ss}en,
		so \"uberliefert,
		dass man von beiden Gebotstafeln je zwei gegen\"uberstellen kann,
		und im weitesten eine Verbindung aufbauen kann.
		Beispielsweise lautet das erste Gebot dort:
		\q{Du wirst {\Gott} als {\Herrn} und Befreier aus Ägypten anerkennen.}
		Und das sechste,
		als parallele Verbindung,
		lautet:
		\q{Du wirst nicht morden.}
		Damit ist nat\"urlich nicht nur gemeint,
		dass man seinem Mitmenschen keinen k\"orperlichen Schaden zuf\"ugt,
		sondern auch keinen seelischen,
		beispielsweise durch Kr\"ankungen,
		wie bereits beim \q{\hyperref[DasSechsteAngebot]{Sechsten Angebot}} erkl\"art.
		Die Parallele besteht hier,
		dass man beim ersten Gebot {\Gott} komplett akzeptiert,
		und beim sechsten Gebot seinen Mitmenschen komplett akzeptiert.
		Es geht also in beiden F\"allen um die bedingungslose Liebe,
		einmal gegen\"uber {\Gott},
		einmal gegen\"uber seinem N\"achsten.
		Die \textbf{10 Angebote} kann man ebenso gegen\"uberstellen,
		und ich werde dir die Verbindungen zeigen:
		\\
		\begin{itemize}
			\item	\textbf{Das 1. und 6. Angebot:}
			\\		Es soll auf der einen Seite {\Gott},
					und auf der anderen Seite dein N\"achster,
					voll und ganz angenommen werden.
					Es geht also um die bedingungslose Liebe gegen\"uber {\Gott} und deinem Mitmenschen.
					Es ist zus\"atzlich gemeint,
					dass hier durch andere G\"otter,
					egal ob neben oder anstelle von {\Gott},
					der Beziehung zu {\Gott} ein Schaden zugef\"ugt wird.
					Und zus\"atzlich,
					dass z.B. durch T\"oten deinem N\"achsten ein Schaden zugef\"ugt wird.
			\item	\textbf{Das 2. und 7. Angebot:}
			\\		Hier besteht die Verbindung etwas tiefer.
					Selbstverst\"andflich auf der einen Seite gegen\"uber zu {\Gott}.
					Und auf der anderen Seite gegen\"uber deinem Ehepartner,
					oder auch gegen\"uber einem anderen verheirateten Mitmenschen.
					In unserer modernen Zeit,
					wo man nicht immer sofort oder fr\"uh heiratet,
					k\"onnte man den Ehebruch auch noch auf jegliche Liebesbeziehung ausweiten,
					im Sinne von \q{Fremdgehen},
					oder generell eine Liebesbeziehung sch\"adigen.
					Letztendlich geht es hier um tiefen und intimen Respekt.
					Es ist gemeint,
					wenn du auf der einen Seite {\Gottes} Namen missbrauchst,
					dass das {\Ihm} gegen\"uber sehr respektlos ist,
					und das sehr sch\"adlich f\"ur eine tiefe Beziehung zu {\Gott} ist.
					Und,
					auf der anderen Seite,
					wenn du ehebrichst,
					fremdgehst,
					oder \"ahnliches,
					dann ist das sehr respektlos gegen\"uber deinem Mitmenschen.
					Es ist ja auch gleichzeitig ein Vertrauensbruch.
					Wenn du bspw. verheiratet bist,
					und dann mit einer beziehungsfremden Person intim wirst,
					woher will dein Ehepartner wissen,
					dass du das nie wieder tun wirst?
					Oder anders herum:
					Wie w\"urdest du dich f\"uhlen?
					W\"urdest du das wollen?
					\textit{Anmerkung:
					Hier geht es nicht um das Thema \q{lockere/offene Beziehung}.
					Ich m\"ochte das ganze nicht unn\"otig verkomplizieren.}
			\item	\textbf{Das 3. und 8. Angebot:}
			\\		Die Verbindung in der Versuchung {\Gottes},
					und im Diebstahl besteht darin,
					dass man auf der einen Seite,
					im symbolischen Sinne,
					einen Teil von {\Gottes} Allmacht \q{stehlen} will.
					Ich dir ein Extrembeispiel zeigen:
					Angenommen,
					du springst aus dem Fenster,
					und sagst dir sowas wie:
					\q{Wenn es {\Gott} gibt, wird {\Er} mich auffangen.}.
					Damit w\"urdest du also einen Teil {\Seiner} Allmacht \q{stehlen} (wollen).
					Und auf der anderen Seite,
					gegen\"uber deinem Mitmenschen,
					ist es ja klar,
					dass gemeint ist,
					dass du deinem Mitmenschen nichts unberechtigt entwendest,
					also richtiger Diebstahl,
					oder auch Dienstleistungen unberechtigt in Anspruch nimmst:
					\q{Dienstleistungs-Diebstahl},
					was man \"ublicherweise mit \q{Betrug} bezeichnet.
			\item	\textbf{Das 4. und 9. Angebot:}
			\\		Hier ist die Parallele,
					dass es gewisse Dinge gibt,
					die heilig sind,
					und von daher auch ehrenhaft behandelt werden sollen.
					Auf der einen Seite ist das der Sabbat,
					also der \textbf{siebte} Tag,
					da {\Gott} nach der Sch\"opfung am \textbf{siebten} Tag geruht hat.
					Daher kommt das ja auch in unserer Gesellschaft,
					dass der Sonntag,
					der \textbf{siebte} Tag der Woche,
					in den meisten Branchen ein arbeitsfreier Tag ist.
					Auf der anderen Seite wiederum gilt auch hier,
					dass das,
					was du mitteilst,
					heilig bzw. ehrenhaft sein soll.
					Deswegen sollst du die Wahrheit sprechen,
					und gerecht gegen\"uber deinem N\"achsten sein.
					Gleichzeitig,
					und darauf bin ich im \q{\hyperref[DasNeunteAngebot]{Neunten Angebot}} bereits eingegangen,
					kann ja das \q{schw\"oren} auch mit dem englischen \q{(to) swear} zusammenh\"angen.
					Und Beschw\"oren,
					Fluchen,
					oder Verfluchen
					sind alles andere als ehrenhaft und heilig.
			\item	\textbf{Das 5. und 10. Angebot:}
			\\		Hier ist die Verbindung,
					meiner Meinung nach,
					mehr als offensichtlich.
					Deine \textit{biologischen} Eltern haben dir das Leben geschenkt,
					und in der Regel haben sie dich auch gro{\ss}gezogen und ern\"ahrt.
					Freilich kann es auch sein,
					dass du beispielsweise Adoptiveltern hast,
					oder es ganz andere Umst\"ande bei dir gibt.
					Aber zweifelsohne wurdest du von zwei Menschen gezeugt und geboren.
					Und diese beiden haben dir dein Leben,
					deinen K\"orper geschenkt.
					Und selbst wenn,
					angenommen du w\"urdest im Extremfall wirklich deine leiblichen Eltern nicht kennen,
					hast du einen K\"orper.
					Und du hast nur dieses eine irdische Leben.
					Von daher ist es wichtig,
					dass du auch dir selbst gut tust.
					Und auch,
					wenn die Rede vom K\"orper ist,
					sind auch deine Gedanken,
					und deine Psyche gemeint,
					dein Charakter und deine Pr\"agungen,
					die sich in deinem Gehirn befinden,
					was effektiv wiederum ein Teil deines K\"orpers ist.
					Achte stets auf das,
					was von au{\ss}en kommt,
					sowohl die k\"orperliche,
					als auch die geistige und geistliche Nahrung.
					Es ist weniger wichtig,
					wie lange du lebst,
					sondern dass es dir,
					in der Zeit,
					in der du am Leben bist,
					auch so gut wie m\"oglich geht.
		\end{itemize}
	
	\newpage
	\section{Gebete}
		Ich m\"ochte dir in diesem Kapitel ein paar sch\"one Gebete anbieten,
		oft in der Form \q{Ich spreche zu {\Gott}}.
	
	\subsection{Das {\Vater}-Unser}
		Wie auch bspw. bei den 10 Geboten lehne ich das {\Vater}-Unser,
		so wie es in der Bibel steht,
		auf keinen Fall ab.
		Auch hier m\"ochte ich dir eine pers\"onlichere Form zeigen,
		die auch weniger gebietend ist.
		Ich finde n\"amlich,
		dass dort zu viele Imperativformen enthalten sind.
		Anstatt \q{geheiligt werde {\Dein} Name} zu beten,
		ist es besser zu sagen \q{geheiligt ist {\Dein} Name},
		denn {\Gottes} Name \textbf{ist} heilig.
		Oder anstelle von \q{{\Dein} Wille geschehe},
		besser \q{{\Dein} Wille geschieht.},
		da {\Gott} existiert,
		und meiner Meinung nach,
		im Zweifel genau das geschieht,
		was er will,
		auch wenn wir es vielleicht nicht immer gleich erkennen,
		und schon gar nicht immer gleich verstehen.
	
	\subsubsection{Wenn du alleine betest}
		\begin{itemize}[nosep]
			\item[]	Mein {\Vater},
			\item[]	(der) {\Du} bist im Himmel.
			\item[]	Geheiligt ist {\Dein} Name.
			\item[]	{\Dein} Reich kommt.
			\item[]	{\Dein} Wille geschieht,
			\item[]	wie im Himmel,
					so auf Erden.
			\item[]	Mein t\"aglich' Brot gibst {\Du} mir heute.
			\item[]	Bitte,
					vergib mir meine Schuld,
			\item[]	und auch ich vergebe meinen Schuldigern.
			\item[]	{\Du} f\"uhrst mich nicht in Versuchung,
			\item[]	sondern erl\"ost mich von dem B\"osen.
			\item[]	Denn {\Dein} ist das Reich,
			\item[]	und die Kraft,
			\item[]	und die Herrlichkeit,
			\item[]	in Ewigkeit.
			\item[]	Amen.
		\end{itemize}
			
	\subsubsection{Wenn ihr in der Gruppe betet}
		\begin{itemize}[nosep]
			\item[]	Unser {\Vater},
			\item[]	(der) {\Du} bist im Himmel.
			\item[]	Geheiligt ist {\Dein} Name.
			\item[]	{\Dein} Reich kommt.
			\item[]	{\Dein} Wille geschieht,
			\item[]	wie im Himmel,
					so auf Erden.
			\item[]	Unser t\"aglich' Brot gibst {\Du} uns heute.
			\item[]	Bitte,
					vergib uns unsere Schuld,
			\item[]	und auch wir vergeben unseren Schuldigern.
			\item[]	{\Du} f\"uhrst uns nicht in Versuchung,
			\item[]	sondern erl\"ost uns von dem B\"osen.
			\item[]	Denn {\Dein} ist das Reich,
			\item[]	und die Kraft,
			\item[]	und die Herrlichkeit,
			\item[]	in Ewigkeit.
			\item[]	Amen.
		\end{itemize}

	\newpage
	\section{Mein Leben mit {\Gott}} \label{MeinLebenMitGott}
		Hierbei handelt es sich um eine Art Tagebuch,
		wie ich meine Reise mit und zu {\Gott} erlebe,
		und was ich sonst noch dabei lernen darf.
	
	\subsection{Mittwoch, der 27. September 2023}
		Ich bin seit etwa Mitte 2023 auf einer Art Reise,
		bei der ich mich entschieden habe,
		{\Gott} und {\Jesus} in mein Leben zu lassen.
		Ich habe selbst noch viele Fehler,
		und obgleich der von {\Gott} gegebenen \textit{(An-)}Gebote,
		s\"undige ich noch viel zu oft.
		Wie im Vorwort erw\"ahnt,
		bin ich weit davon entfernt,
		so etwas wie der \q{perfekte Christ} zu sein.
		Viele der allt\"aglichen Gewohnheiten,
		Pr\"agungen und sonstiges haben so eine starke Sogwirkung,
		dass ich auch nicht immer an {\Gott} denke,
		nicht so oft bete,
		oder in der Bibel lese,
		wie ich gerne w\"urde.
		Und wenn ich dann \q{wieder} an {\Gott} denke,
		habe ich oft ein schlechtes Gewissen,
		weil ich {\Ihn} dann gef\"uhlt \q{vergessen} habe.
		Also kurzum:
		Ich darf noch sehr, sehr, sehr, ..., sehr viel lernen!
		
	\subsection{Freitag, der 29. September 2023}
		Heute habe ich mir ein Video angesehen,
		das mir sehr zu denken gegeben hat.
		Ich wei{\ss} nicht,
		ob es sich dabei um Gottesl\"asterung handelt.
		Trotzdem will ich mit dir teilen,
		was ich darin gesehen habe.
		Es war im Prinzip ein kurzer Trickfilm,
		in dem eine Muslimin,
		ein Atheist und ein Christ in den Himmel gekommen sind.
		Da man hier nicht wirklich {\Gott} selbst gesehen hat,
		sondern lediglich eine Karikatur,
		werde ich hier die normale Gro{\ss}schreibung verwenden.
		Es ging also darum,
		von welchen Aussagen sich Gott beleidigt f\"uhlt.
		Und im Endeffekt hat er dem Atheisten seinen Frieden geschenkt,
		und ihn tats\"achlich in dem Himmel geschickt,
		weil dieser ja nie an Gott geglaubt hat,
		und ihm weder das eine,
		noch das andere nachgesagt hat.
		Und von der Muslimin und vom Christen war er entt\"auscht,
		weil sie im Prinzip in so \q{b\"ose} dargestellt haben,
		als ob er alle Menschen,
		die s\"undigen und nicht an ihn glauben,
		einfach in die H\"olle werfen w\"urde.
		Das hat ihn sehr verletzt,
		weil er sich effektiv wie ein grausames Monster gef\"uhlt hat.
		Das ist sozusagen die Kurzversion.
		Und das hat mir zu denken gegeben.
		Ich kann nat\"urlich nur spekulieren.
		Aber vielleicht ist es ja so,
		dass uns {\Gott} nirgendwo \q{hinschickt}.
		Wenn wir uns f\"ur {\Ihn} entscheiden,
		so l\"adt {\Er} uns auch nach dem Tode zu {\Sich} in den Himmel ein.
		Und wenn wir uns beispielsweise f\"ur den Teufel entscheiden,
		dann kann es schon sein,
		dass wir in die H\"olle kommen.
		Aber nicht weil uns {\Gott} dort hinschickt,
		sondern weil der Teufel uns mitnimmt.
		Wie gesagt ... ich wei{\ss} es nicht.
		Das Video hat mich nur zum Nachdenken gebracht.
		Denn es ist sicher oft so,
		dass man vielleicht \"uber {\Gott} dieses und jenes sagt,
		aber es effektiv nicht wei{\ss},
		was die Wahrheit ist.
		Aber man kann ja zumindest erstmal nachdenken,
		wenn man \"uber {\Gott} etwas sagt,
		ob man selber wollen w\"urde,
		dass jemand anderes \"Ahnliches \"uber einen sagt.
		Falls du dir \"uber das Video selbst ein Urteil bilden m\"ochtest,
		hier der Link: \url{https://www.youtube.com/watch?v=ttevamkS6gw}.

	\subsection{Dienstag, der 3. Oktober 2023}
		Am vergangenen Wochenende,
		bis einschlie{\ss}lich gestern,
		war ich mit meiner Frau und meinen Eltern in Hamburg.
		Aus irgendeinem Grund ging es mir ab Sonntag im Laufe das Tages nicht so gut.
		Ich war u.a. \"uberm\"udet und gequ\"alt von Kopfschmerzen.
		Und ich vermute,
		auch mein Bewegungsmangel hat sich hier stark gezeigt,
		da ich jede Treppe als Qual empfunden habe.
		Wenigstens konnte ich morgens in meiner Bibel-App lesen,
		was auch schonmal viel Wert war.
		Zwischendrin kam mir ein Verdacht,
		woher m\"oglicherweise meine Kopfschmerzen kamen.
		Ich kann es aber nicht beweisen,
		es bleibt also bei einer Vermutung.
		Jedenfalls,
		da ich ja diesmal den Sabbat durchziehen wollte,
		habe ich sowohl u.a. auf Kaffee und potenziell zuckerhaltiges verzichtet.
		Meine Frau meinte zwar \"ofters,
		ich sollte einen Kaffee trinken,
		aber auf die Versuchung wollte ich gar nicht erst eingehen.
		Und ich habe ja 'mal geh\"ort,
		dass Zucker auch s\"uchtig machen kann.
		Und deswegen geh\"ort dies auch zu den Dingen,
		auf die ich am Sabbat verzichten will.
		Ich habe aber auch geh\"ort,
		dass bei manchen S\"uchten,
		z.B. bei Zucker,
		Kopfschmerzen eine Entzugserscheinung sein kann.
		Und wenn das stimmt,
		dann s\"undige ich au{\ss}erhalb des Sabbats ganz sch\"on viel,
		was das betrifft.
		Siehe auch: \q{\hyperref[DasZehnteAngebot]{Das Zehnte Angebot}}.
		
	\newpage
	\section{Friede sei mit dir!}
	\begin{figure}[h]
		\centering
		\includegraphics[width=1\textwidth,keepaspectratio]{"FreeChristian.jpeg"}
	\end{figure}

\end{document}
