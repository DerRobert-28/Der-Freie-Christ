\documentclass[12pt,a4paper]{article}

\usepackage[english]{babel}
\usepackage{enumitem}
\usepackage[utf8]{inputenc}
\usepackage[T1]{fontenc}
\usepackage[left=25mm,right=25mm,top=25mm,bottom=25mm]{geometry}
\usepackage[colorlinks=true,urlcolor=blue,linkcolor=black]{hyperref}

\newcommand{\Christ}[0]{\textbf{CHRIST}}
\newcommand{\Christus}[0]{\textbf{CHRISTUS}}
\newcommand{\God}[0]{\textbf{GOD}}
\newcommand{\Gods}[0]{\textbf{GOD's}}
\newcommand{\He}[0]{\textbf{HE}}
\newcommand{\Him}[0]{\textbf{HIM}}
\newcommand{\Himself}[0]{\textbf{HIMSELF}}
\newcommand{\His}[0]{\textbf{HIS}}
\newcommand{\Holy}[0]{\textbf{HOLY}}
\newcommand{\Jesus}[0]{\textbf{JESUS}}
\newcommand{\Lord}[0]{\textbf{LORD}}
\newcommand{\Messiah}[0]{\textbf{MESSIAH}}
\newcommand{\Redeemer}[0]{\textbf{REDEEMER}}
\newcommand{\Savior}[0]{\textbf{SAVIOR}}
\newcommand{\Saviour}[0]{\textbf{SAVIOUR}}
\newcommand{\Son}[0]{\textbf{SON}}
\newcommand{\Spirit}[0]{\textbf{SPIRIT}}
\newcommand{\Thee}[0]{\textbf{THEE}}
\newcommand{\Their}[0]{\textbf{THEIR}}
\newcommand{\Them}[0]{\textbf{THEM}}
\newcommand{\Thine}[0]{\textbf{THINE}}
\newcommand{\Thou}[0]{\textbf{THOU}}
\newcommand{\Thy}[0]{\textbf{THY}}
\newcommand{\You}[0]{\textbf{YOU}}
\newcommand{\Your}[0]{\textbf{YOUR}}
\newcommand{\Yours}[0]{\textbf{YOURS}}

\newcommand{\q}[1]{\char"22{#1}\char"22 }

\title{\textbf{The Free Christian}}
\author{Robert Lang-Kirchh\"ofer}
\date{\textit{Last modified: October 03rd, 2023}}

\begin{document}
	\setlength{\parindent}{0mm}
	\maketitle

	\newpage
	\tableofcontents

	\newpage
	\section{Preface}

	\subsection{Using hints}
		It may happen that I will address you,
		the reader,
		directly.
		It may also happen that,
		whenever human beings in general are mentioned,
		I will use a potentially existing masculine form.
		My intension is to create a comfortable,
		casual atmosphere,
		and to ease reading flow.
		Of course,
		my dear reader,
		totally independent of your actual gender or sex,
		I am not making any assumptions about you,
		and you will have my fullest respect.
		Since I am not a professional author,
		my writing style is not perfect,
		but rather more casual.
		And because you are reading the English version of this eBook,
		you may notice that my English is not perfect either,
		because I am not a native speaker.
		
	\subsection{Thank you!}
		Next I want to state my cordial thanks to you,
		my dear reader,
		that you decided to take a look into this eBook.
		Of course,
		I hope that you also will read it until the end,
		and that you will watch its development.
		I am not a 100 percent sure that it will be \q{finished} one day,
		because possibly there will always be new thoughts,
		or new material that can be included in here.
		This here is a Christian document.
		My intention is to communicate moral values to you,
		especially as they are wished by {\God}, the {\Lord},
		and {\His} {\Son} {\Jesus} {\Christ},
		by my best knowledge and conscience.
		Of course I do not want to invent a new \q{religion},
		or to re-invent Christianity,
		but rather show some new perspectives.
		As you can tell of this preface,
		words,
		which directly relate to {\God},
		{\Jesus} or also to the {\Holy} {\Spirit},
		are written in capital and bold letters.
		If something is near to your heart,
		or you have general ideas,
		I invite you sincerely,
		to participate in my  \href{https://github.com/DerRobert-28/Der-Freie-Christ/discussions}{GitHub discussion}.
	
	\subsection{Some words about myself}
		Myself was baptized catholic,
		as far as I remember correctly,
		at the age of one or two years,
		but I left church in mid-August 2023.
		The reasons for this are of private nature,
		and are not relevant here.
		But this has nothing to do with my belief.
		I do believe that {\God} exists,
		and that {\Jesus} is the {\Messiah}.
		Anyway,
		it does not mean that I am a kind of \q{perfect Christian},
		if such a person exists at all nowadays.
		You can read more about me in chapter \q{\hyperref[MeinLebenMitGott]{My life with {\God}}}.
	
	\newpage
	\section{What makes up a \q{Free Christian}?}
	
	\subsection{No church community}
		A Free Christian is not bound to a church community.
		That means,
		you can,
		but you do not have to be baptized.
		You can also have left your church.
		That does not play a role at all.
		In my opinion,
		it is even more sensible,
		to not be a member of a church community at all,
		especially of the catholic one,
		to not be manipulated by it.
		The only important thing is,
		to let {\God},
		the {\Lord},
		and {\Jesus} {\Christ} come into your life,
		and that you confess yourself to {\Them}.
	
	\subsection{The true bond}
		For you,
		as a Free Christian,
		the only true,
		existing bond is between {\God},
		{\Jesus} {\Christ} and you.
		When you confess yourself to {\Them},
		you cherish your relationship sincerely.
		Mundane bonds \textit{(relationships)} are ephemeral,
		although it is not less important,
		to also cherish them warmly.
			
	\subsection{The Bible as a \q{tool}}
		If it serves the relationship between {\God},
		{\Jesus} and you,
		you have,
		as a Free Christian,
		the freedom to interpet biblical passages better,
		that means more modern or more understandable,
		and rephrase them accordingly.
		But this is \textbf{not a} \q{licence},
		to rewrite the Word of {\God} by free convenience,
		and then for example to bend {\His} will,
		as it was done willingly in the past by the catholic church,
		according to my information and my research.
		
	\subsection{Belief and science}
		Belief and science are,
		in my opinion,
		not mutually exclusive.
		{\God} can neither be proven,
		nor refuted.
		You always have free choice,
		whether you want to live with {\God},
		or not.
		That is,
		why you as a human being have been given free will.
	
	\newpage
	\section{The Ten Commandments}
		The traditional 10 commandments are usually handed down from the viewpoint of {\God},
		that means in the form of \q{Thou shalt (not) ...},
		or more modern \q{You shall (not) ...}.
		Below are the 10 commandments from your own viewpoint,
		as if yourself would talk to {\God},
		and make them a promise to {\Him}.
		They are also worked out a bit better,
		because on closer consideration,
		some commandments can be summarized.
		Of course,
		that does not mean that I refuse,
		or decline the traditional {\God} given commandments.
		I just want to show another perspective.
		In addition I will call them \q{offers},
		to emphasize the good will of both sides.
	
	\subsection{The Highest Offer}
		The Highest Offer is:
		I will love and praise {\God},
		the {\Lord},
		sincerely with all of my heart.
		And I will love my neighbour,
		as I also do myself.
		
	\subsection{The First Offer}
		{\Thou} art the {\Lord},
		my {\God},
		my {\Saviour}.
		I will not have other gods before or besides {\Thee},
		and I will not praise or worship them.
		And I will not create an idol or a graven image.
		
	\subsection{The Second Offer}
		{\Thou} art the {\Lord},
		my {\God}.
		I will not misuse {\Thy} name.
		I will not slander or blaspheme against {\Thee}.
		And I will confess myself honestly to {\Thee}.
		\\
		\textit{Short hint:
		This should also include colloquial phrases,
		such as \q{Oh (my) G...},
		or \q{For G...'s sake},
		that you say quickly,
		but without really meaning {\God} {\Himself},
		or to pray to {\Him},
		or something similar.}
			
	\subsection{The Third Offer}
		{\Thou} art the {\Lord},
		my {\God}.
		I will not put {\Thee} to the test.
		I will not try {\Thee}.
		Also in need I will confess myself to {\Thee}.
		\\
		\textit{Short hint:
		This is also intended to cover situations,
		in which one carelessly says such things as,
		for example,
		how {\God} can allow this or that suffering.}
		
	\subsection{The Fourth Offer}
		{\Thou} art the {\Lord},
		my {\God}.
		I will sanctify and keep the Sabbath to {\Thee}.
		On Sabbath I will stay abstinent of fleshly
		and addictive things.
		
	\subsection{The Fifth Offer}
		I will honour my father and my mother,
		who gave my life to me,
		and raised and nourished me.
		And I will honour Elder People.
			
	\subsection{The Sixth Offer} \label{TheSixthOffer}
		I will not kill or murder.
		I will cherish my relationships.
		I will respect life and the wellbeing of all living beings,
		and I will protect them as far as possible.
		\\
		\textit{Short hint:
		The killing here is not just literal,
		meaning physically,
		but also symbolic,
		for example,
		by saying something to someone out of anger,
		what hurts him,
		and thus damages the relationship.
		Feel free to read the Bible passage \href{https://www.die-bibel.de/bibeln/online-bibeln/lesen/ESV/MAT.5/Matthew-5}{Matthew 5, 21-22}.}
		
	\subsection{The Seventh Offer}
		I will not commit adultery.
		I will not covet my neighbour's woman or wife.
		I will not covet my neighbour's man or husband.
		
	\subsection{The Eighth Offer}
		I will not steal or betray.
		I will not rob or kidnap.
		I will not covet my neightbour's house.
		I will not covet my neightbour's belongings.
		I will not cause harm to my neighbour's belongings.
		
	\subsection{The Ninth Offer} \label{TheNinthOffer}
		I will not bear false witness against my neighbour.
		I will not lie or betray.
		I will not swear.
		I will act honestly and fairly against my neightbour.
		\\
		\textit{Short hint:
		The word \q{swear} here is ambiguous.
		Since it can be understood as \q{(to) swear an oath},
		or \q{(to) curse}.
		So the sentence could also be written down as
		\q{I will not swear an oath, and I will not summon or curse.}}
		
	\subsection{The Tenth Offer} \label{TheTenthOffer}
		My body is a gift, a present, of {\Thine},
		and therefore it is holy.
		I will honour and cherish it.

	\subsection{Summary}
		Shortened the 10 offers can be summarized as following:
		\\
		\begin{enumerate}[nosep]
			\item I will not have other gods before or besides {\Thee}.
			\item I will not misuse {\Thy} name.
			\item I will not put {\Thee} to the test.
			\item I will keep the Sabbath to {\Thee}.
			\item I will honour my father and my mother.
			\item I will not kill \textit{(or murder)}.
			\item I will not commit adultery.
			\item I will not steal \textit{(or betray)}.
			\item I will not bear false witness \textit{(against my neighbour)}.
			\item I will honour my life, {\Thy} gift.
		\end{enumerate}
		
	\subsection{Comparison}
		In Judaism the 10 commandments,
		which are called \q{10 words that {\God} spoke} in the Torah,
		are traditionally handed down,
		so that you can compare two of each on the commandment boards,
		and establish a connection in the broadest sense.
		For example the first commandment is:
		\q{Thou wilt recognize {\God} as {\Lord} and deliverer \textit{(liberator)} from Egypt.}
		And the sixth one as parallel connection is:
		\q{Thou wilt not kill or murder.}
		Of course,
		this means that you do neither cause physical,
		nor psychological harm to your fellow human being,
		for example due to an insult,
		as already explained in \q{\hyperref[TheSixthOffer]{The Sixth Offer}}.
		The parallel connection here is,
		to accept {\God} totally with the first commandment,
		and to accept your fellow one totally with the sixth commandment.
		So,
		in both cases,
		it is about unconditional love,
		once towards {\God},
		and once towards your fellow human being.
		The \textbf{10 Offers} are also cabable to be compared to each other,
		and I will show you the parallels:
		\\
		\begin{itemize}
			\item	\textbf{The 1st and the 6th Offer:}
			\\		On one hand {\God},
					and on the other hand your fellow human being,
					shall be accepted totally.
					So it is about unconditional love towards {\God},
					and towards your neighbour.
					Furthermore to having other gods,
					either besides or before {\God},
					you cause harm to your relationship to {\God}.
					Additionally,
					you cause harm to your fellow,
					e.g. by killing him or her.
			\item	\textbf{The 2nd and the 7th Offer:}
			\\		Here the connection is a bit deeper.
					Of course,
					on the one hand towards {\God}.
					And on the other hand towards your spouse,
					but also towards another married fellow.
					In our modern times,
					where you do not marry immediately or early,
					adultery can also be enlarged onto every romantic relationship,
					in the sense of \q{cheating},
					or generally harming a (romantic) relationship.
					Eventually it is a matter of deep and intimate respect.
					The meaning is,
					one the one hand,
					if you misuse the name of {\God},
					it is very disrespectful against {\Him},
					and this causes great harm to a deep relationship with {\God}.
					And,
					on the other hand,
					if you commit adultery,
					if you cheat,
					or some much,
					then this is very disrespectful towards your fellow human being.
					Simultaneous it is a breach of trust.
					For example,
					you are married,
					and you are getting intimate with a person outside of your relationship,
					how will your spouse know that you will never do this again?
					Or the other way around:
					How would you feel?
					Would you want this to happen to you?
					\textit{Annotation:
					This is not about so-called
					loose or open relationships.
					I do not want to make it unnecessarily complex.}
			\item	\textbf{The 3rd and the 8th Offer:}
			\\		The connection in the trial of {\God},
					and in theft is that on the one hand,
					you want to steal,
					in a symbolic way,
					a part of {\Gods} omnipotence.
					Let me show you a very extreme example:
					Let's say,
					you are jumping out of the window,
					and you are saying something like:
					\q{If {\God} exists, {\He} will catch me.}
					With this action you would \q{steal} of {\His} almightiness.
					And on the other hand,
					towards your fellow human being,
					it should be clear that you do not take away of him without justification,
					i.e. actual theft,
					but also you shall not claim services without justification:
					\q{theft of service},
					usually called \q{betrayal}.
			\item	\textbf{The 4th and the 9th Offer:}
			\\		The parallel connection here is that there are certain things,
					which are called holy,
					and thus they shall be treated honourably.
					On the one hand,
					it is Sabbath,
					the \textbf{seventh} day,
					because {\God} rested at the \textbf{seventh} day after {\His} creation.
					This is why in our society Sunday,
					the \textbf{seventh} day of the week,
					is neither a workday nor a business day in many companies or (industrial) branches.
					On the other hand,
					everything you communicate shall be holy or honourable.
					Thus you shall tell the truth,
					and you shall be fair towards your neighbour.
					Additionally,
					what I recently mentioned with \q{\hyperref[TheNinthOffer]{The Ninth Offer}},
					\q{swearing} can also refer to \q{cursing} or \q{summoning},
					and those are anything else but honourable and holy.
			\item	\textbf{The 5th and the 10th Offer:}
			\\		Here the connection is much more than obvious,
					in my opinion.
					Your \textit{biological} parents gave life to you,
					and normally they raised and nourished you.
					Of course,
					it may be the case that you have adoptive parents,
					or there are completely different circumstances.
					But,
					without any doubt,
					there are two human beings that procreated and gave birth to you.
					And both gave your life to you.
					And even when,
					in an extreme case,
					you may not know your natural parents,
					you have a body.
					And you only have this one earthly life.
					And thus it is important that you care for yourself.
					And even when I said \q{body},
					it is also about your thoughts,
					your soul,
					your mind,
					your character,
					which can be found in your brain.
					which effectively again is a part of your body.
					Always mind,
					what comes from the outside into your inside,
					All of physical,
					mental and spiritual nourishment \textit{(food)}.
					It is less important,
					how long you live.
					More important is the quality of your lifetime.
					It should be as high as possible,
					and that you feel good.
		\end{itemize}

	\newpage
	\section{My life with {\God}} \label{MeinLebenMitGott}
		This big chapter is a kind of diary,
		how I experience my journey with and to {\God},
		and all the things a may learn on it.
	
	\subsection{Wednesday, September 27th, 2023}
		I have been on a kind of journey since mid-2023,
		on which I decided to let {\God} and {\Jesus} in my life.
		I still do have a lot of flaws,
		and despite of the {\God} given commandments \textit{(offers)},
		I sin much too often.
		As mentioned in the preface,
		I am still far,
		far away of being a kind of \q{perfect Christian}.
		Many of my everyday habits,
		of my character and other stuff,
		they have such a strong pulling effect
		that often I do not think of {\God},
		and I do not pray,
		and read the Bible,
		as often as I would like to.
		And whenever {\God} comes \q{back again} into my thinking,
		I often have a guilty conscience,
		because for me it feels like I have \q{forgotten} {\Him}.
		So,
		long story short:
		I may still learn a lot and way too much!

	\subsection{Friday, September 29th, 2023}
		Today I watched a video that gave me a lot to think about.
		I do not know,
		whether it is blasphemy.
		Still,
		I want to share with you,
		what I saw in it.
		It was basically a short animation,
		in which a Muslim,
		an atheist and a Christian went to heaven.
		Since you haven't really seen {\God} {\Himself} here,
		but just a caricature,
		I will use normal capitalization here.
		So it was all about,
		which statements God feels insulted and offended by.
		And in the end he gave the atheist his peace,
		and actually sent him to heaven,
		because he never believed in God,
		and neither said one thing nor the other about him.
		And he, God, was disappointed in the Muslim and in the Christian,
		because they basically represented him as such an \q{evil} God,
		as if he would simply throw all people into hell,
		who sin and do not believe in him.
		That hurt him very much,
		because he effectively felt like a cruel monster.
		This so far is the short version.
		And that got me thinking.
		Of course,
		I can only speculate.
		But maybe it is like that,
		that {\God} is not \q{sending} us anywhere.
		If we choose {\Him},
		{\He} invites us to stay in Heaven after death.
		And if we choose, for example, the devil,
		then it truely may be,
		that we will go to hell.
		But not because {\God} sends us there,
		but because the devil takes us away.
		Like I said ... I do not know.
		This video just made me think.
		And probably,
		it may often be the case,
		that one might say this or that about {\God},
		but he or she actually does not know,
		what is the real truth.
		But what you can do at least,
		is to think about it first,
		when you say something about {\God},
		whether you would want to be told similar things about yourself.
		If you would like to make your own judgment about the video,
		here is the link: \url{https://www.youtube.com/watch?v=ttevamkS6gw}.

	\subsection{Tuesday, October 3rd, 2023}
		Last weekend,
		until inclusive yesterday,
		I have been to Hamburg with my wife and my parents.
		But for some reason,
		starting on Sunday,
		I did not feel so well during the day.
		I was,
		among other things,
		overtired,
		and tormented by headaches.
		And I assume,
		my lack of motion or exercise also became apparent here,
		because I experienced every staircase as torture.
		At least I could read in my Bible app in the morning,
		what was already worth a lot to me.
		In the meantime I had a suspicion,
		where my headache may have come from.
		But I have no proof for it,
		so it will just stay an assumption.
		Anyway,
		since I wanted to go through the Sabbath this time,
		I stayed abstinent from coffee and potentially sugar-containing foods.
		My wife often suggested that I should drink a cup of coffee,
		but I did not want to fall for this trial.
		And I also heard once that sugar can be addictive.
		And thus it belongs to the things,
		I want to be abstinent from.
		I also heard once that one withdrawal symptom,
		e.g. with a sugar \q{addiction},
		can be headache.
		And when that is the case,
		then I sin really often outside of Sabbath for that matter.
		See also: \q{\hyperref[TheTenthOffer]{The Tenth Offer}}.

\end{document}
