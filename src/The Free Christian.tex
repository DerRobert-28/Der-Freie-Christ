%\documentclass[12pt,a4paper]{article}
\documentclass[10pt,a5paper]{article}

\usepackage[english]{babel}
\usepackage{enumitem}
\usepackage[utf8]{inputenc}
\usepackage[T1]{fontenc}
\usepackage[left=10mm,right=10mm,top=20mm,bottom=20mm]{geometry}
%\usepackage[left=25mm,right=25mm,top=25mm,bottom=25mm]{geometry}
\usepackage{graphicx}
\usepackage[colorlinks=true,urlcolor=blue,linkcolor=black]{hyperref}

\newcommand{\Allmighty}[0]{\textbf{ALLMIGHTY}}
\newcommand{\Christ}[0]{\textbf{CHRIST}}
\newcommand{\Christus}[0]{\textbf{CHRISTUS}}
\newcommand{\Creator}[0]{\textbf{CREATOR}}
\newcommand{\God}[0]{\textbf{GOD}}
\newcommand{\Gods}[0]{\textbf{GOD's}}
\newcommand{\Gott}[0]{\textbf{GOTT}}
\newcommand{\Elohim}[0]{\textbf{ELOHIM}}
\newcommand{\Father}[0]{\textbf{FATHER}}
\newcommand{\He}[0]{\textbf{HE}}
\newcommand{\Heaven}[0]{\textbf{HEAVEN}}
\newcommand{\Him}[0]{\textbf{HIM}}
\newcommand{\Himself}[0]{\textbf{HIMSELF}}
\newcommand{\His}[0]{\textbf{HIS}}
\newcommand{\Holy}[0]{\textbf{HOLY}}
\newcommand{\Jesus}[0]{\textbf{JESUS}}
\newcommand{\Lord}[0]{\textbf{LORD}}
\newcommand{\Lords}[0]{\textbf{LORD's}}
\newcommand{\Messiah}[0]{\textbf{MESSIAH}}
\newcommand{\Our}[0]{\textbf{OUR}}
\newcommand{\Ours}[0]{\textbf{OURS}}
\newcommand{\Redeemer}[0]{\textbf{REDEEMER}}
\newcommand{\Savior}[0]{\textbf{SAVIOR}}
\newcommand{\Saviour}[0]{\textbf{SAVIOUR}}
\newcommand{\Selbst}[0]{\textbf{SELBST}}
\newcommand{\Son}[0]{\textbf{SON}}
\newcommand{\Spirit}[0]{\textbf{SPIRIT}}
\newcommand{\Thee}[0]{\textbf{THEE}}
\newcommand{\Their}[0]{\textbf{THEIR}}
\newcommand{\Them}[0]{\textbf{THEM}}
\newcommand{\Thine}[0]{\textbf{THINE}}
\newcommand{\Thou}[0]{\textbf{THOU}}
\newcommand{\Thy}[0]{\textbf{THY}}
\newcommand{\Us}[0]{\textbf{US}}
\newcommand{\Yahweh}[0]{\textbf{YAHWEH}}
\newcommand{\You}[0]{\textbf{YOU}}
\newcommand{\Your}[0]{\textbf{YOUR}}
\newcommand{\Yours}[0]{\textbf{YOURS}}

\newcommand{\q}[1]{\char"22{#1}\char"22 }
\newcommand{\qq}[1]{\textit{\q{#1}}}

\title{\textbf{The Free Christian}}
\author{Robert Lang-Kirchh\"ofer}
\date{\textit{Last modified: October 17th, 2023}}

\begin{document}
	\setlength{\parindent}{0mm}
	\maketitle
	\begin{figure}[h]
		\centering
		\includegraphics[width=1\textwidth,keepaspectratio]{"FreeChristian.jpeg"}
	\end{figure}

	\newpage
	\tableofcontents

	\newpage
	\section{Preface}

	\subsection{Using hints}
		It may happen that I will address you,
		the reader,
		directly.
		It may also happen that,
		whenever human beings in general are mentioned,
		I will use a potentially existing masculine form.
		My intension's to create a comfortable,
		casual atmosphere,
		and to ease reading flow.
		Of course,
		my dear reader,
		totally independent of your actual gender or sex,
		I am not making any assumptions about you,
		and you will have my fullest respect.
		Since I am not a professional author,
		my writing style is not perfect,
		but rather more casual.
		And because you are reading the English version of this eBook,
		you may notice that my English is not perfect either,
		because I am not a native speaker.
		Furthermore,
		this eBook is not necessarily written in a way,
		so that one chapter builds upon the other.
		That means you do not need to read it completely,
		or \q{from A to Z},
		but you can read as much or as little as you like to,
		and in any order you prefer.
	
	\subsection{Thank you!}
		Next I want to state my cordial thanks to you,
		my dear reader,
		that you decided to take a look into this eBook.
		Of course,
		I hope that you also will read it until the end,
		and that you will watch its development.
		I am not a 100 percent sure that it will be \q{finished} one day,
		because possibly there will always be new thoughts,
		or new material that can be included in here.
		This here is a Christian document.
		My intention is to communicate moral values to you,
		especially as they are wished by {\God}, the {\Lord},
		and {\His} {\Son} {\Jesus} {\Christ},
		by my best knowledge and conscience.
		Of course I do not want to invent a new \q{religion},
		or to re-invent Christianity,
		but rather show some new perspectives.
		As you can tell of this preface,
		words,
		which directly relate to {\God},
		{\Jesus} or also to the {\Holy} {\Spirit},
		are written in capital and bold letters.
		If something is near to your heart,
		or you have general ideas,
		I invite you sincerely,
		to participate in my  \href{https://github.com/DerRobert-28/Der-Freie-Christ/discussions}{GitHub discussion}.
	
	\subsection{Some words about myself}
		Myself was baptized catholic,
		as far as I remember correctly,
		at the age of one or two years,
		but I left church at the end of Juli 2023.
		The reasons for this are of private nature,
		and are not relevant here.
		But this has nothing to do with my belief.
		I do believe that {\God} exists,
		and that {\Jesus} is the {\Messiah}.
		Anyway,
		it does not mean that I am a kind of \q{perfect Christian},
		if such a person exists at all nowadays.
		You can read more about me in chapter \q{\hyperref[MeinLebenMitGott]{My life with {\God}}}.
	
	\newpage
	\section{What makes up a \q{Free Christian}?}
	
	\subsection{No church community}
		A Free Christian is not bound to a church community.
		That means,
		you can,
		but you do not have to be baptized.
		You can also have left your church.
		That does not play a role at all.
		In my opinion,
		it is even more sensible,
		to not be a member of a church community at all,
		especially of the catholic one,
		to not be manipulated by it.
		The only important thing is,
		to let {\God},
		the {\Lord},
		and {\Jesus} {\Christ} come into your life,
		and that you confess yourself to {\Them}.
	
	\subsection{The true bond}
		For you,
		as a Free Christian,
		the only true,
		existing bond is between {\God},
		{\Jesus} {\Christ} and you.
		When you confess yourself to {\Them},
		you cherish your relationship sincerely.
		Mundane bonds \textit{(relationships)} are ephemeral,
		although it is not less important,
		to also cherish them warmly.
			
	\subsection{The Bible as a \q{tool}}
		If it serves the relationship between {\God},
		{\Jesus} and you,
		you have,
		as a Free Christian,
		the freedom to interpet biblical passages better,
		that means more modern or more understandable,
		and rephrase them accordingly.
		But this is \textbf{not a} \q{licence},
		to rewrite the Word of {\God} by free convenience,
		and then for example to bend {\His} will,
		as it was done willingly in the past by the catholic church,
		according to my information and my research.
		
	\subsection{Belief and science}
		Belief and science are,
		in my opinion,
		not mutually exclusive.
		{\God} can neither be proven,
		nor refuted.
		You always have free choice,
		whether you want to live with {\God},
		or not.
		That is,
		why you as a human being have been given free will.

	\subsection{What do I know about {\God}?}
		Well ... what does one actually \q{know} about {\God}?
		Sure,
		I can get to know,
		and learn more about {\God} and {\Jesus} by {\His} Holy Word,
		the Bible.
		But I never would arrogate to claim to \q{know} {\God}.
		Even less absolutely.
		Only {\God} {\Himself} knows {\Himself} completely.
		And with this eBook I only want to share my experience with you.
		And what I tell about {\God},
		is only according to my best knowledge and conscience,
		and only in {\His} favour.
		
	\newpage
	\section{Do you (not) believe in {\God}?}
		Although this eBook is a Christian document,
		it does not mean that only Christians are \q{allowed} to read it.
		Quite the opposite!
		Even if you believe in something different,
		or nothing at all,
		you are invited sincerely to browse into here.
		{\God} loves us all equally.
		But ... what do you actually believe in?

	\subsection{\q{I am a Jew and do not believe in {\Jesus} as {\Messiah}.}}
		Nevertheless,
		you are invited sincerely,
		to keep on reading.
		Effectively,
		we believe in the same {\God},
		who is also called {\Elohim} or {\Yahweh} in the Holy Scriptures.
		According to my knowledge,
		big parts of your holy book,
		e.g. the Torah,
		are congruent with the Old Testament of the Bible.
		For example I will mention the Genesis,
		or the Exodus (of Egypt).
		Because I already admitted that I do not claim to know {\God},
		I also would not claim that you are a bad person,
		because you do not believe in {\Jesus},
		and for this you might not come to {\Heaven}.
		If you do not acknowledge {\Jesus} as your {\Messiah},
		I will accept this.
		I also ask you to accept that for me {\Jesus} \textbf{is} the {\Messiah}.
		It is not about forcing you onto {\Jesus},
		but to show you moral values in the whole,
		and to share my experience to you.
		
	\subsection{\q{I am a Muslim and believe in Allah.}}
		Then you are also invited sincerely,
		to continue reading.
		I have already admitted
		that I do not presume to know {\God}.
		That is why I do not know,
		whether (the) {\God} that I believe in,
		and your god,
		Allah,
		are effectively the same god,
		or two different ones,
		and therefore setting them equal would be blasphemy.
		I will not deny your belief.
		You can pray to Allah,
		and see him as your only god.
		If you consider me a heathen or a nonbeliever,
		I take that as your opinion.
		I accept that Allah is your only god for you.
		I ask you,
		to accept that {\God} is my only god for me.
		I still hope that outside of our belief,
		just from a human perspective,
		we can respect each other.
		After all,
		I do not want to impose another god on you,
		but just show you some moral values in the whole,
		and to share my experiences to you.
	
	\subsection{\q{I belong to a different religion.}}
		Unfortunately I cannot get into every orientation of belief,
		religion, cult or sect of this world.
		This would go beyond the scope of this eBook.
		So I beg your pardon,
		if I did not explicitly mention your belief.
		If you read the two paragraphs above,
		and if it may be the case that you come to the conclusion,
		according to your belief,
		I am a heathen or a nonbeliever,
		then I accept this as your opinion.
		I accept that the god(s),
		the goddess(es),
		or even if it may be the devil,
		are those,
		who are worth to be praised or worshipped.
		I also ask you,
		to accept that for me,
		{\God} is my only god.
		I still hope that outside of our belief,
		just from a human perspective,
		we can respect each other.
		After all,
		I do not want to impose \q{my} {\God} on you,
		but just show you some moral values in the whole,
		and to share my experiences to you.
		
	\subsection{\q{I do not believe in a god, or I am agnostic.}}
		You,
		too,
		are invited sincerely to go on reading.
		Although if you,
		my friend,
		do not believe in {\God} (anymore),
		or you do not care or know about the existence of something divine,
		you do have some kind of moral values.
		Would you like to be hurt or killed just like that?
		Would you like to be stolen from or betrayed?
		Would you like to be cheated by your partner,
		if you do not have something like a so-called open relationship?
		Would you be like to be lied to?
		Would you not also like to have a healthy body,
		and enjoy a high quality of living?
		Would you not just be loved as you are,
		deep inside of you?
		Thus I accept,
		if you do not have the need of a god in your life.
		Although I ask you,
		to accept that for me {\God} exists,
		and {\He} is the only {\God}.
		I hope that just from a human perspective,
		we can respect each other.
		After all,
		I do not want to impose a god or a belief on you,
		but just show you some moral values in the whole,
		and to share my experiences to you.

	\newpage
	\section{Who and how is {\God}?}
		This chapter also serves me,
		so I can do some Bible study.
		It is about,
		as {\God} is called in the Bible,
		i.e. {\His} names and general designations,
		and how {\He} manifests {\Himself}.
		Of course,
		this also includes {\Jesus} and the {\Holy} {\Spirit}.
		I will provide a list below:
		which is usually in the style \q{{\He} is ...},
		or \q{{\He} is called ...},
		and then who or how {\God} is.
		And if I have a source,
		i will put in brackets the first Bible passage,
		where that appears.
		I will use the English Standard Version as a translation.
		Have fun discovering {\God} together.

	\subsection{{\God} is ...}
		\begin{itemize}[nosep]
			\item {\He} is {\God}. \textit{(Genesis 1, 1)}
			\item {\He} is the {\Spirit} (of {\God}). \textit{(Genesis 1, 2)}
			\item {\He} is the {\Lord}. \textit{(Genesis 2, 4)}
			\item {\He} is (called) {\Jesus} {\Christ}. \textit{(Matthew 1, 1)}
		\end{itemize}

	\newpage
	\section{The Ten Commandments}
		The traditional 10 commandments are usually handed down from the viewpoint of {\God},
		that means in the form of \q{Thou shalt (not) ...},
		or more modern \q{You shall (not) ...}.
		Below are the 10 commandments from your own viewpoint,
		as if yourself would talk to {\God},
		and make them a promise to {\Him}.
		They are also worked out a bit better,
		because on closer consideration,
		some commandments can be summarized.
		Of course,
		that does not mean that I refuse,
		or decline the traditional {\God} given commandments.
		I just want to show another perspective.
		In addition I will call them \q{offers},
		to emphasize the good will of both sides.
	
	\subsection{The Highest Offer}
		The Highest Offer is:
		I will love and praise {\God},
		the {\Lord},
		sincerely with all of my heart.
		And I will love my neighbour,
		as I also do myself.
		
	\subsubsection{The First Offer}
		{\Thou} art the {\Lord},
		my {\God},
		my {\Saviour}.
		I will not have other gods before or besides {\Thee},
		and I will not praise or worship them.
		And I will not create an idol or a graven image.
		
	\subsubsection{The Second Offer}
		{\Thou} art the {\Lord},
		my {\God}.
		I will not misuse {\Thy} name.
		I will not slander or blaspheme against {\Thee}.
		And I will confess myself honestly to {\Thee}.
		\\
		\textit{Short hint:
		This should also include colloquial phrases,
		such as \q{Oh (my) G...},
		or \q{For G...'s sake},
		that you say quickly,
		but without really meaning {\God} {\Himself},
		or to pray to {\Him},
		or something similar.}
			
	\subsubsection{The Third Offer}
		{\Thou} art the {\Lord},
		my {\God}.
		I will not put {\Thee} to the test.
		I will not try {\Thee}.
		Also in need I will confess myself to {\Thee}.
		\\
		\textit{Short hint:
		This is also intended to cover situations,
		in which one carelessly says such things as,
		for example,
		how {\God} can allow this or that suffering.}
		
	\subsubsection{The Fourth Offer}
		{\Thou} art the {\Lord},
		my {\God}.
		I will sanctify and keep the Sabbath to {\Thee}.
		On Sabbath I will stay abstinent of fleshly
		and addictive things.
		
	\subsubsection{The Fifth Offer}
		I will honour my father and my mother,
		who gave my life to me,
		and raised and nourished me.
		And I will honour Elder People.
			
	\subsubsection{The Sixth Offer} \label{TheSixthOffer}
		I will not kill or murder.
		I will cherish my relationships.
		I will respect life and the wellbeing of all living beings,
		and I will protect them as far as possible.
		\\
		\textit{Short hint:
		The killing here is not just literal,
		meaning physically,
		but also symbolic,
		for example,
		by saying something to someone out of anger,
		what hurts him,
		and thus damages the relationship.
		Feel free to read the Bible passage \href{https://www.die-bibel.de/bibeln/online-bibeln/lesen/ESV/MAT.5/Matthew-5}{Matthew 5, 21-26}.}
		
	\subsubsection{The Seventh Offer}
		I will not commit adultery.
		I will not covet my neighbour's woman or wife.
		I will not covet my neighbour's man or husband.
		
	\subsubsection{The Eighth Offer}
		I will not steal or betray.
		I will not rob or kidnap.
		I will not covet my neightbour's house.
		I will not covet my neightbour's belongings.
		I will not cause harm to my neighbour's belongings.
		
	\subsubsection{The Ninth Offer} \label{TheNinthOffer}
		I will not bear false witness against my neighbour.
		I will not lie or betray.
		I will not swear.
		I will act honestly and fairly against my neightbour.
		\\
		\textit{Short hint:
		The word \q{swear} here is ambiguous.
		Since it can be understood as \q{(to) swear an oath},
		or \q{(to) curse}.
		So the sentence could also be written down as
		\q{I will not swear an oath, and I will not summon or curse.}}
		
	\subsubsection{The Tenth Offer} \label{TheTenthOffer}
		My body is a gift, a present, of {\Thine},
		and therefore it is holy.
		I will honour and cherish it.
	
	\subsection{Comparison}
		In Judaism the 10 commandments,
		which are called \q{10 words that {\God} spoke} in the Torah,
		are traditionally handed down,
		so that you can compare two of each on the commandment boards,
		and establish a connection in the broadest sense.
		For example the first commandment is:
		\q{Thou wilt recognize {\God} as {\Lord} and deliverer \textit{(liberator)} from Egypt.}
		And the sixth one as parallel connection is:
		\q{Thou wilt not kill or murder.}
		Of course,
		this means that you do neither cause physical,
		nor psychological harm to your fellow human being,
		for example due to an insult,
		as already explained in \q{\hyperref[TheSixthOffer]{The Sixth Offer}}.
		The parallel connection here is,
		to accept {\God} totally with the first commandment,
		and to accept your fellow one totally with the sixth commandment.
		So,
		in both cases,
		it is about unconditional love,
		once towards {\God},
		and once towards your fellow human being.
		The \textbf{10 Offers} are also cabable to be compared to each other,
		and I will show you the parallels:
		\\
		\begin{itemize}
			\item	\textbf{The 1st and the 6th Offer:}
			\\		On one hand {\God},
					and on the other hand your fellow human being,
					shall be accepted totally.
					So it is about unconditional love towards {\God},
					and towards your neighbour.
					Furthermore to having other gods,
					either besides or before {\God},
					you cause harm to your relationship to {\God}.
					Additionally,
					you cause harm to your fellow,
					e.g. by killing him or her.
			\item	\textbf{The 2nd and the 7th Offer:}
			\\		Here the connection is a bit deeper.
					Of course,
					on the one hand towards {\God}.
					And on the other hand towards your spouse,
					but also towards another married fellow.
					In our modern times,
					where you do not marry immediately or early,
					adultery can also be enlarged onto every romantic relationship,
					in the sense of \q{cheating},
					or generally harming a (romantic) relationship.
					Eventually it is a matter of deep and intimate respect.
					The meaning is,
					one the one hand,
					if you misuse the name of {\God},
					it is very disrespectful against {\Him},
					and this causes great harm to a deep relationship with {\God}.
					And,
					on the other hand,
					if you commit adultery,
					if you cheat,
					or some much,
					then this is very disrespectful towards your fellow human being.
					Simultaneous it is a breach of trust.
					For example,
					you are married,
					and you are getting intimate with a person outside of your relationship,
					how will your spouse know that you will never do this again?
					Or the other way around:
					How would you feel?
					Would you want this to happen to you?
					\textit{Annotation:
					This is not about so-called
					loose or open relationships.
					I do not want to make it unnecessarily complex.}
			\item	\textbf{The 3rd and the 8th Offer:}
			\\		The connection in the trial of {\God},
					and in theft is that on the one hand,
					you want to steal,
					in a symbolic way,
					a part of {\Gods} omnipotence.
					Let me show you a very extreme example:
					Let's say,
					you are jumping out of the window,
					and you are saying something like:
					\q{If {\God} exists, {\He} will catch me.}
					With this action you would \q{steal} of {\His} almightiness.
					And on the other hand,
					towards your fellow human being,
					it should be clear that you do not take away of him without justification,
					i.e. actual theft,
					but also you shall not claim services without justification:
					\q{theft of service},
					usually called \q{betrayal}.
			\item	\textbf{The 4th and the 9th Offer:}
			\\		The parallel connection here is that there are certain things,
					which are called holy,
					and thus they shall be treated honourably.
					On the one hand,
					it is Sabbath,
					the \textbf{seventh} day,
					because {\God} rested at the \textbf{seventh} day after {\His} creation.
					This is why in our society Sunday,
					the \textbf{seventh} day of the week,
					is neither a workday nor a business day in many companies or (industrial) branches.
					On the other hand,
					everything you communicate shall be holy or honourable.
					Thus you shall tell the truth,
					and you shall be fair towards your neighbour.
					Additionally,
					what I recently mentioned with \q{\hyperref[TheNinthOffer]{The Ninth Offer}},
					\q{swearing} can also refer to \q{cursing} or \q{summoning},
					and those are anything else but honourable and holy.
			\item	\textbf{The 5th and the 10th Offer:}
			\\		Here the connection is much more than obvious,
					in my opinion.
					Your \textit{biological} parents gave life to you,
					and normally they raised and nourished you.
					Of course,
					it may be the case that you have adoptive parents,
					or there are completely different circumstances.
					But,
					without any doubt,
					there are two human beings that procreated and gave birth to you.
					And both gave your life to you.
					And even when,
					in an extreme case,
					you may not know your natural parents,
					you have a body.
					And you only have this one earthly life.
					And thus it is important that you care for yourself.
					And even when I said \q{body},
					it is also about your thoughts,
					your soul,
					your mind,
					your character,
					which can be found in your brain.
					which effectively again is a part of your body.
					Always mind,
					what comes from the outside into your inside,
					All of physical,
					mental and spiritual nourishment \textit{(food)}.
					It is less important,
					how long you live.
					More important is the quality of your lifetime.
					It should be as high as possible,
					and that you feel good.
		\end{itemize}
		
	\subsection{Summary}
		Shortened the 10 offers can be summarized as following:
		\\
		\begin{enumerate}[nosep]
			\item I will not have other gods before or besides {\Thee}.
			\item I will not misuse {\Thy} name.
			\item I will not put {\Thee} to the test.
			\item I will keep the Sabbath to {\Thee}.
			\item I will honour my father and my mother.
			\item I will not kill \textit{(or murder)}.
			\item I will not commit adultery.
			\item I will not steal \textit{(or betray)}.
			\item I will not bear false witness \textit{(against my neighbour)}.
			\item I will honour my life, {\Thy} gift.
		\end{enumerate}

	\newpage
	\section{Virtues and (Capital) Sins}
		In the following,
		I want to delve into the Seven Virtues,
		and the Seven Capital Sins,
		also known as Deadly Sins.
		I have made sure to choose alternative names whenever possible.
		However,
		it may well be that they are officially called differently.
	
	\subsection{The Seven Virtues}
	
	\subsubsection{The First Virtue}
		\textbf{The first virtue is devotion,
		humility,
		modesty and will to serve.}
		So in general,
		be ready to serve your neighbour,
		without expecting anything in return.
		I know that it is difficult in our society,
		to always implement this,
		since you need a job,
		in order to finance your life with your salary.
		But between friends,
		for example,
		you do not always have to hope for something in return.

	\subsubsection{The Second Virtue}
		\textbf{The second virtue is benevolence.
		charity,
		dearness,
		love (of your neighbour) and respect.}
		Appreciate your fellow human beings,
		love and respect them.
		For example, everyone has difficult times,
		so stand by them with love and goodwill.
	
	\subsubsection{The Third Virtue}
		\textbf{The third virtue is abstinence,
		chastity,
		moderation and renunciation of instincts.}
		Especially if you want to keep the Sabbath,
		you stay away from alcohol, for example,
		or other intoxicating substances.
		But also outside the Sabbath,
		is it not nicer to live out your urges with your partner,
		the one you love,
		instead of having different partners every day?
		Is the last one not too stressful?

	\subsubsection{The Fourth Virtue}
		\textbf{The fourth virtue is constancy,
		fortitude,
		hope,
		patience and serenity.}
		And yes,
		this is exactly where I could tell myself:
		\q{Look who's talking!}
		This is,
		where I can work on myself,
		because if something does not work,
		and again,
		and again,
		I quickly lose my patience.
		So,
		let us work on it together!

	\subsubsection{The Fifth Virtue}
		\textbf{The fifth virtue is measure and moderation.}
		Of everything you can have or want,
		it does not always have to be that much.
		If you crave chocolate,
		is a piece not just enough,
		instead of a whole bar?
		If you feel like wine,
		is a glass not enough,
		instead of a whole bottle?
		When you eat your meal,
		is it not enough to eat until you are full,
		instead of cramming everything into yourself,
		main issue of having eaten all up?

	\subsubsection{The Sixth Virtue}
		\textbf{The sixth virtue is benevolence,
		empathy,
		favour,
		goodwill,
		gratitude,
		openness,
		solidarity and sympathy.}
		Even in difficult times,
		and especially afterwards,
		you may be grateful.
		After all, you have overcome them.
		You have been given a whole life full of possibilities,
		for which you may be grateful.
		And pass on this gratitude to your fellow human beings.
		It makes you likable,
		and helps you towards mutual openness and goodwill.
	
	\subsubsection{The Seventh Virtue}
		\textbf{The seventh virtue is determination,
		diligence,
		enthusiasm and zeal.}
		By \q{struggle},
		of course,
		it does not refer to a real fight or dispute.
		It is more in a symbolic sense.
		Work diligently on something,
		and stick with it,
		even when it gets tough sometimes.
		You are allowed to take breaks,
		to recover and gather new energy.

	\subsection{The Seven Capital Sins}

	\subsubsection{The First Capital Sin}
		\textbf{The first capital sin is arrogance,
		hubris,
		pride and vanity.}
		This does not mean that you cannot be proud of anything anymore.
		It is more about that some people are proud of the wrong things,
		things they have not done anything for,
		e.g. their nationality,
		and then even boast about it,
		to adorn themselves with it.
		Let yourself be praised for a good performance,
		but do not shout it from the rooftops!

	\subsubsection{The Second Capital Sin}
		\textbf{The second capital sin is avarice,
		covetousness and greed.}
		It is okay to have money.
		In our modern society,
		one cannot do without it anymore.
		But be completely honest with yourself:
		Do you really need millions and millions in your account?
		Or how much do you really need for a reasonable standard of living,
		which still allows more than just survival?
		And if you have something to spare,
		be ready to share it with your neighbour!
	
	\subsubsection{The Third Capital Sin}
		\textbf{The third capital sin is desire,
		extravagance,
		hedonism,
		lewdness and lust.}
		Do you really need that for yourself,
		for example,
		binge-drinking every weekend?
		Do you need a different person in your bed every day?
		Would it not be much nicer to feel secure in a safe harbour?
		
	\subsubsection{The Fourth Capital Sin}
		\textbf{The fourth captial sin is anger,
		impatience,
		irritation,
		vindictiveness and wrath.}
		If someone or something has hurt you,
		it is okay to allow the emotions.
		Feel them,
		but do not take them out on your neighbour!
		Let them know and withdraw instead.
		One can practice developing a certain emotional intelligence.
		But when youare angry,
		it already was and is within you,
		and your fellow human beings are not to blame for it!

	\subsubsection{The Fifth Capital Sin}
		\textbf{The fifth capital sin is excess,
		gluttony,
		immoderation,
		overeating and self-indulgence.}
		Is it not enough to eat until you are satisfied?
		Does it have to be a huge,
		overflowing plate?
		Does quantity really have to triumph over quality?
		And if you have something to spare,
		you can gladly share it!
	
	\subsubsection{The Sixth Capital Sin}
		\textbf{The sixth capital sin is coveting,
		envy and jealousy.}
		So your neighbour does not deserve to have something, too?
		Why not?
		Are you not capable of earning it for yourself?
		Or what lies behind your envy?

	\subsubsection{The Seventh Capital Sin}
		\textbf{The seventh capital sin is apathy
		cowardice,
		ignorance,
		inertia and laziness.}
		Do not confuse it with taking a break!
		If you are exhausted from your work,
		then rest until you have the energy to work again.
		But is it not,
		for example,
		unfair to sit at home,
		get paid by the state,
		and be just supported by others?
		
	\subsection{Comparison}
		The Virtues and Sins be contrasted or opposed as follows:
		\begin{enumerate}
			\item \textbf{Humility $\Longleftrightarrow$ Vanity}
			\item \textbf{Benevolence $\Longleftrightarrow$ Greed}
			\item \textbf{Chastity $\Longleftrightarrow$ Lust}
			\item \textbf{Serenity $\Longleftrightarrow$ Wrath}
			\item \textbf{Moderation $\Longleftrightarrow$ Gluttony}
			\item \textbf{Gratitude $\Longleftrightarrow$ Envy}
			\item \textbf{Diligence $\Longleftrightarrow$ Laziness}
		\end{enumerate}

	\newpage
	\section{Will I go to {\Heaven}?}
	
	\subsection{Completely in general}
		Whether we go to {\Heaven} ultimately lies completely with {\God} {\Himself}.
		I do not want to presume here to claim,
		how we might be punished for any sins.
		I rather believe that once we possibly have to justify our actions.
		I am not saying that we are actually \q{punished},
		and I am only saying \q{possibly}.
		I do hope that {\God} is kind and mercifully forgiving us our sins.
		It is not about whether you,
		dear reader,
		believe in {\Jesus} specifically right now or not.
		But we ourselves cannot free ourselves from our sins,
		because what we have done,
		we have done.
		It cannot simply be undone.

	\subsection{We are all sinners}
		I would like to revisit {\Gods} commandments \textit{(offers)},
		and the seven virtues and capital sins.
		Feel free to be honest with yourself.
		\\
		\begin{itemize}[nosep]
			\item	Have you,
					in the broadest sense,
					ever worshiped or praised another \q{God},
					even in a symbolic sense?
					This includes \q{idolizing} people and things.
			\item	Have you ever imagined what {\God} might look like?
					This is also called creating an idol or image of {\God}.
			\item	Have you never used one of those typical colloquial phrases,
					like \qq{Oh (my) G...},
					outside of a prayer,
					without referring to {\God},
					just because it is \q{what people say}?
					You could also say something like
					\qq{Oh, my goodness.}
			\item	Have you never,
					out of despair or other reasons,
					thought or even said something like
					\qq{How can {\God} allow such suffering?}
			\item	Are holidays really holy or sacred to you,
					or do you \q{celebrate} them just for the sake of commerce?
					Just make sure to find a nice Easter nest on Easter,
					and make sure to get a nice gift on Christmas?
			\item	Have you ever been disrespectful to your parents,
					either directly or indirectly?
					Have you cursed your parents?
					Could it be that they are doing their best,
					and it has sometimes been your fault,
					or entirely different circumstances,
					over which your parents have no control?
			\item	Do you take care of your relationships enough?
					And by that,
					I do not just mean the relationship with your (spouse) partner.
					I mean any interpersonal relationship:
					your parents,
					your children (if you have any),
					your friends,
					your colleagues, and so on.
			\item	Have you ever killed or even murdered?
					Are you possibly a soldier or similar?
					If so,
					who are you to take it upon yourself,
					to decide about the life or rather the death of your \q{enemy}?
			\item	Have you ever insulted or offended someone?
					For example,
					in traffic,
					have you shouted \qq{Idiot} at someone who cut you off?
			\item	Have you ever killed an insect,
					like a fly,
					or other small creature,
					just because it was annoying?
			\item	Have you ever cheated?
					If not,
					have you at least looked at another man or woman,
					because you found them attractive?
			\item	And now,
					more concretely,
					what is your stance on pornography,
					especially if you are in a relationship?
					What would you think of your partner,
					if he or she consumed such content?
			\item	Have you ever taken something away from someone?
					Here,
					too,
					if you are a father,
					mother or teacher:
					have you ever temporarily taken something away from your son,
					daughter or student,
					because he (or she) was "naughty"?
					Even children have the right to property.
					So who are you,
					to take something away from him or her?
			\item	Have you ever been jealous,
					because someone had something you would have liked?
					Why do you not wish him that?
			\item	Have you ever lied?
					Even if you consider yourself a (relatively) honest person,
					what do you say when someone asks you
					\qq{How are you?}
					Do you just say \qq{Fine.},
					out of politeness,
					to not burden your counterpart,
					when in reality you feel terrible?
			\item	Do you curse when you lose patience?
					Or do you generally use many swear words,
					no matter the context?
			\item	Do you always need the latest technical device,
					be it a smartphone,
					a smartwatch,
					or whatever,
					just to impress your neighbour?
					Do you really need that?
			\item	Is everything with you just an eternal give-and-take?
					Or can you also do something,
					without immediately expecting something in return?
			\item	Do you respect your fellow human beings?
					Or do you often jump to premature conclusions?
					You do not have to find everyone likable,
					but every person has their own life story.
					Do not judge anyone,
					unless you have walked in their \q{shoes} for at least a day!
			\item	How often have you just had one or two glasses of beer,
					or wine,
					too many?
					Why do you need that?
					So that you feel terrible the next day?
			\item	Do you need someone different in your bed every day?
			\item	Are you very patient,
					or do you easily lose your temper?
			\item	When eating,
					do you always finish everything,
					just because it was taught to you as a child,
					even if you are already full?
					Or can you moderate yourself,
					and put less on your plate from the start?
			\item	Are you grateful for your body?
					Do you fundamentally accept it?
					Or are you harming it,
					for example,
					with smoking or other vices?
			\item	Are you generally grateful for your life,
					or do you have something to complain about everything?
					Remember what you have accomplished and overcome to get this far!
			\item	Will you stick it out to the end?
					Do you complete your projects to the end,
					or do you give up halfway?
			\item	Are you proud to belong to a particular nation,
					to be a man or a woman,
					a child,
					an adult,
					or whatever you have no control over?
					It is okay to be proud of your own achievements.
					But what have you done to,
					for example,
					be born as a man?
			\item	Do you keep everything for yourself and give nothing away?
					Do you need everything for yourself alone?
					Especially when it comes to money?
					Do you not perhaps have one or two \q{pennies} left,
					to help your fellow human beings?
			\item	Are you constantly angry at your fellow human beings,
					without limits?
					Do you always think that you cannot forgive them,
					no matter what happened?
					\\
		\end{itemize}
		I could think of much more for sure,
		but that would surely go beyonod the scope.
		And yes,
		I must also take a look at myself!
		
		
	\subsection{What can I do?}
		Like I already mentioned,
		you did what you did.
		And you can do nothing to make it undone.
		Both your good actions,
		and also your bad ones.
		Both your achievements,
		and also your \q{failures}.
		You cannot do anything on your own,
		to \q{please} {\God}.
		You can only accept {\His} gift,
		{\His} grace,
		{\His} mercy.
		If you decide for a living with {\God},
		and you accept {\Jesus} as your {\Saviour},
		then you also accept {\Gods} gift.
		Feel free to read the following Bible verses:
		\href{https://www.die-bibel.de/bibeln/online-bibeln/lesen/ESV/JHN.14/John-14}{John 14, 6},
		\href{https://www.die-bibel.de/bibeln/online-bibeln/lesen/ESV/ACT.4/Acts-4}{Act 4, 12},
		\href{https://www.die-bibel.de/bibeln/online-bibeln/lesen/ESV/ROM.3/Romans-3}{Romans 3, 23-24},
		\href{https://www.die-bibel.de/bibeln/online-bibeln/lesen/ESV/GAL.2}{Galatians 2, 16} and \href{https://www.die-bibel.de/bibeln/online-bibeln/lesen/ESV/EPH.2/Ephesians-2}{Ephesians 2, 8-9}.
		\\
		\textit{And now, the following is VERY important:}
		You \textbf{may} still keep {\Gods} commandments \textit{(offers)},
		but make sure what motive is behind it!
		If you only keep them,
		hoping to get into {\Heaven},
		you have misunderstood it.
		But the mistake is not yours.
		Unfortunately,
		we live in a world of constant give and take.
		A simple example:
		You go to work every day,
		and at the end of the month,
		you get paid for it.
		Then you pay your rent so you can continue living in your apartment.
		You pay for electricity and gas,
		so you have light,
		and do not freeze in winter.
		And you buy groceries,
		so you will not starve.
		Always a service for a service in return.
		Our society just works that way.
		And we wrongly transfer that to {\God}.
		But {\God} gives you {\His} grace and goodness,
		because {\He} loves you.
		Just as you are.
		You do not have to do anything,
		absolutely nothing for it.
		But we all think
		- and I am no exception
		- that for some reason,
		we are not worth being loved by {\God},
		because we sin continuously.
		A little (emergency) lie here,
		insulting someone out of anger there.
		And then we think we are unloved.
		But there is good news:
		{\Jesus} died for you and your sins on the cross.
		The only thing you \q{need} to do,
		is let {\Him} into your life,
		open your heart to {\Him}.
		Because {\He} loves you unconditionally.
		And that is exactly how you should act.
		If you want to keep {\Gods} commandments,
		not \q{so that} you get into {\Heaven}.
		No,
		simply out of gratitude,
		because {\God},
		because {\Jesus} made this huge sacrifice that no one could ever make up for.
		Do not keep {\His} commandments,
		and then demand or expect to get into heaven!
		Also,
		pay attention to the following:
		Not,
		\q{if} you love {\God},
		you keep {\His} commandments,
		but \q{because} you love {\Him}.
		But you also should not rest on it,
		by saying something like:
		\q{Well,
		{\Jesus} died for me anyway,
		now I can sin endlessly.}
		That would also be selfish in relation to the relationship with {\God}.
		It might be difficult to distinguish all that.
		But at least pay attention,
		to why you consciously want to keep {Gods} commandments.
		Do you do it willingly,
		out of gratitude to {\Jesus}?
		Or are you hoping for \q{a piece of heaven}?
		
	\subsection{Summary}
		Therefore,
		I can only say that I hope for {\Gods} grace and goodness.
		I hope that if we are serious about {\Him},
		and we walk {\His} path,
		fulfill {\His} will,
		then {\God} will also show mercy.
		And by \q{{\His} will},
		I do not mean that we are submissive or weak-willed puppets.
		After all,
		{\God} gave us all our own will,
		out of {\His} goodness.
		That means,
		we can decide for ourselves,
		whether we want to live with {\Him} or not.
		It simply means that we follow {\His} commandments,
		and treat each other honestly and lovingly.

	\newpage
	\section{Examining different sins in detail}
		In this chapter,
		I want to take a closer look at some sins.
		Some of them may be committed consciously,
		some unconsciously,
		some are hard to avoid,
		and so on.
		I want to show you that life is not always black-and-white.
		There is often not just \q{good} and \q{evil}.
		Sometimes,
		there are grey areas.
		And I am sure that is,
		how {\God} sees it,
		too.
		That is,
		why {\He} gave us free will,
		so we can make our own decisions.
		So that we can have our own experiences,
		and learn from them.
	
	\subsection{About killing}
		Here,
		I would like to go into actual,
		physical killing,
		so not into metaphorical sense,
		like {\Jesus} described it
		\textit{(see \href{https://www.die-bibel.de/bibeln/online-bibeln/lesen/ESV/MAT.5/Matthew-5}{Matthew 5, 21-26})}.
		The topic can be very sensitive and difficult,
		and there is also no simple black-and-white scheme here,
		in the sense of \q{Killing is always wrong}.
		I just want to raise awareness here,
		and express my own opinion on the matter.

	\subsubsection{Killing fellow human beings}
		Well,
		let us start small.
		You might possibly want to tell me something like,
		\q{I do not kill.
		I have never killed.
		And I am not a murderer!}
		And for the majority of people,
		that is probably true.
		Most people do not just go out,
		and randomly kill anyone who crosses their path.
		And I am fundamentally against killing.
		I want to live myself,
		and therefore,
		I do not want to be killed.
		And I also say that every fellow human being has the right,
		whether legally,
		morally,
		or however you see it,
		to live.
		Even if you do not like or sympathize with everyone,
		there should be enough love and respect,
		to allow others to live.
		And I think that no one - initially - has the right,
		to decide over the life or non-life of another person.
		\\
		No authority either!
		That is why I also consider wars to be wrong.
		The bigger problem is also that those,
		who want war,
		do not actively participate by going to the front themselves.
		In other words,
		I mean corresponding politicians,
		presidents,
		formerly the kings,
		and whatever they are all called.
		Instead of bashing each other's heads in,
		they send soldiers to war.
		\\
		But also soldiers who actively fight on the front,
		and kill other people are,
		in my eyes,
		murderers.
		There are voices claiming that soldiers have no choice;
		they simply follow \q{commands} or \q{orders}.
		Yes,
		they do have a choice!
		If someone orders me to kill another person,
		I can say \q{No!} at any time.
		Or I can say from the outset that I will not go to the military,
		or at least I will not take a weapon into my hand.
		\\
		I also consider the death penalty to be wrong.
		If I punish a murderer by killing him,
		am I better than him?
		Sure,
		one should show a murderer that his actions have consequences.
		But who am I to decide,
		whether he should live or die?
		I believe only one has this right,
		and that is {\God} {\Himself}.
		And if the time comes,
		{\He} will take us to {\Himself}.
		
	\subsubsection{A special case}
		In this section,
		it gets a bit more difficult.
		Because here,
		I have difficulty forming a concrete opinion myself.
		There are few,
		but really very few cases,
		where it might be possible that,
		for example,
		another person \q{must} decide on the life or death of another person.
		In the previous section I said that only {\God} has the right,
		to decide on life and death.
		That means,
		I might contradict myself,
		now.
		A scenario I could imagine,
		going in this direction,
		would be,
		if someone is in a coma in the hospital,
		and not just for a week,
		but maybe for many months,
		perhaps even a few years.
		And the person is only kept alive with the help of machines.
		And the doctors see little chance that the person will ever wake up again.
		And,
		with all due respect for life and fellow human beings,
		lying in the hospital costs money.
		And in this case,
		the relatives would probably take over the costs,
		if they do not have good insurance.
		Or it could also be that the insurance stops paying at some point,
		and the costs are difficult for the relatives to bear in the long run.
		What if the doctors cannot bring this person out of the coma anymore?
		Should one just turn off the machines,
		and the person dies?
		What if,
		despite low chances,
		the person would still have come back?
		Many questions can be asked here,
		and there can be many possibilities with many variations.
		If I had a few wishes,
		one of them surely would be that in my life,
		it will never be the case that I am faced with this choice,
		of whether someone should live or die.
		
	\subsubsection{The trolley problem}
		Another scenario is the so-called \q{\href{https://en.wikipedia.org/wiki/Trolley_problem}{Trolley problem}},
		a thought experiment,
		simplified,
		where a train is heading towards a group of,
		for example,
		five people,
		who are unaware of it.
		And on another track,
		there is also one person,
		who is also unaware of the whole situation.
		And by adjusting the switches,
		there is the opportunity for the train to take the other track,
		and instead,
		the individual person dies.
		The decision then is,
		do you let five people die,
		or one single person,
		because that would be four less?
		
	\subsubsection{Other forms of active killing}
		But in our \q{everyday life},
		we do not constantly kill our fellow human beings.
		There are also other forms of killing that one might not consciously perceive as such.
		What do you do when,
		for example,
		a fly or a spider,
		or something similar,
		is near you?
		Let us say at your home?
		Do you find insects and other creepy-crawlies annoying,
		or do you even find them disgusting?
		Or are you perhaps afraid of them,
		e.g. bees or wasps,
		because you have been stung before?
		How do you deal with it?
		Would you try to kill them to avoid getting stung yourself?
		Or if you find them just gross or annoying,
		would you also want to kill them just to get rid of them?
		But on the other hand,
		are they not also living beings,
		and thus creatures of {\God}?
		Do they also not have the right to live?
		Yes, in \href{https://www.die-bibel.de/bibeln/online-bibeln/lesen/ESV/GEN.1/Genesis-1}{Genesis 1:26},
		it says, 
		\q{Then {\God} said:
		Let {\Us} make man in {\Our} image,
		after {\Our} likeness.
		And let them have dominion over the fish of the sea,
		and over the birds of the heavens,
		and over the livestock,
		and over all the earth,
		and over every creeping thing that creeps on the earth.}
		But \q{dominion} does not mean \q{killing}.
		I myself,
		too,
		feel fear or disgust towards some insects,
		and spiders are repulsive to me.
		In the past,
		I would have killed every spider without a second thought,
		or had my wife do it,
		just to get rid of it.
		But lately,
		I try to overcome myself more often and bring them out of my home.
		Since I do not want to touch the spider,
		I take a large drinking glass and a firm,
		thin surface,
		such as a thin piece of cardboard.
		I carefully place the glass over the spider to trap it.
		Once it has crawled onto the bottom of the glass,
		I gently slide the cardboard under the glass and transport the whole thing outside.
		I then place everything on the ground,
		put the glass down,
		step aside,
		and wait for the spider to crawl out.
		Then I take the glass and cardboard back,
		returning to my home.
		This way,
		I get rid of a spider without killing it.
		Or would you want to be killed \q{just},
		because you are bothering someone?

	\subsubsection{Passive killing}
		Now it gets even more delicate.
		There are two types of passive killing here,
		and I am not talking about fellow human beings,
		but other living beings.
		The first relates to nutrition.
		And yes,
		I also enjoy eating meat in any form,
		such as cold cuts,
		sausages,
		ground meat,
		occasional burgers,
		steaks,
		poultry,
		or fish,
		and whatever else is available.
		I do not want to be a \q{moralizer},
		but effectively,
		for me to eat my meat (whatever kind),
		an animal like a cow,
		pig,
		chicken,
		or fish had to die for it.
		Unfortunately,
		this is also the case throughout nature;
		one living being survives only by feeding on another living being,
		in extreme cases by killing it.
		However,
		I do not want to live a vegetarian or vegan lifestyle;
		I enjoy the taste of meat and meat products too much.
		I have tried alternative products,
		and they often have a strange aftertaste for me.
		I also do not like the taste of milk substitutes.
		But everyone has different preferences.
		It is not about me telling you,
		what you should or should not eat.
		I just want to make you aware - and myself as well - that constant killing also takes place in everyday life.
		
	\subsubsection{Other forms of killing}
		The second one looks like this:
		Imagine you suddenly get sick,
		like catching a cold or something similar.
		What happens here biologically?
		Well,
		you have caught pathogens,
		possibly bacteria.
		And to get healthy again,
		you may need medication to fight the pathogens.
		But even your body,
		your immune system,
		is already fighting against the pathogens.
		What is called (disease) symptoms,
		is the body's reaction to the pathogens.
		Effectively,
		your immune system wants to kill them.
		And there is nothing you can actively do about it.
		So,
		in a sense,
		these bacteria must die for you not to die from the illness.

	\subsubsection{Summary}
		Of course,
		I cannot cover all forms of killing;
		there might be too many,
		and the whole topic is too complex.
		I just wanted to show you with the above,
		how complex the subject of killing is,
		when you think about it.
		I am not claiming that you are a murderer consciously killing other beings.
		I do not even know you.
		I just want to share some thoughts with you,
		and strengthen your awareness.
		May {\God} be with you.

	\newpage
	\section{Prayers, Hymns and Alleluiahs}
		In this chapter I would like to offer you some beautiful prayers,
		often presented in the form \q{I am talking to {\God}},
		and hymns and alleluiahs.

	\subsection{The {\Lords} Prayer}
		As already mentioned with the 10 commandments,
		I do not refuse the {\Lords} Prayer,
		as it is written down in the Bible.
		Here,
		I also want to show you a more personal form,
		which is also less \q{commanding}.
		In my opinion,
		the {\Lords} Prayer contains too much imperatives.
		Instead of praying \q{hallowed be {\Thy} name},
		or \q{hallowed by {\Your} name},
		it is better to say \q{{\Thy} name is hallowed},
		or \q{{\Thy} name is holy}.
		Or instead of \q{{\Thy} will be done},
		or \q{{\Your} will be done},
		better say \q{{\Thy} will happens},
		because {\God} exists,
		and in my opinion,
		in doublt,
		exactly what {\God} wants will happen,
		also if we do not recognize immediately,
		or more we do not unserstand immediately.

	\subsubsection{If you are praying alone}
		\begin{itemize}[nosep]
			\item	My {\Father} in {\Heaven},
					\\
					{\Thy} name is hallowed \textit{(holy)}.
			\item	{\Thy} Kingdom comes.
			\item	{{\Thine} will happen},
					\\
					on Earth,
					as it is in {\Heaven}.
			\item	{\Thou} givest me my daily bread,
					today.
			\item	Please,
					forgive me my sins,
					\\
					and I also forgive those who sin against me.
			\item	{\Thou} leadest me not in temptation,
					\\
					but {\Thou} deliverst me from evil.
			\item	For {\Thine} is the Kingdom,
					\\
					and the power,
					\\
					and the glory,
					\\
					now and forever.
			\item	Amen!
	\end{itemize}
	
	\subsubsection{If you are praying in a group}
		\begin{itemize}[nosep]
			\item	Our {\Father} in {\Heaven},
					\\
					{\Thy} name is hallowed \textit{(holy)}.
			\item	{\Thy} Kingdom comes.
			\item	{{\Thine} will happen},
					\\
					on Earth,
					as it is in {\Heaven}.
			\item	{\Thou} givest us our daily bread,
					today.
			\item	Please,
					forgive us our sins,
					\\
					and we also forgive those who sin against us.
			\item	{\Thou} leadest us not in temptation,
					\\
					but {\Thou} deliverst us from evil.
			\item	For {\Thine} is the Kingdom,
					\\
					and the power,
					\\
					and the glory,
					\\
					now and forever.
			\item	Amen!
		\end{itemize}

	\subsubsection{If you want to use modern English}
		\begin{itemize}[nosep]
			\item	My {\Father} in {\Heaven},
					\\
					{\Your} name is holy.
			\item	{\Your} Kingdom comes.
			\item	{{\Yours} will happen},
					\\
					on Earth,
					as it is in {\Heaven}.
			\item	{\You} give me my daily bread,
					today.
			\item	Please,
					forgive me my sins,
					\\
					and I also forgive those who sin against me.
			\item	{\You} do not lead me in temptation,
					\\
					but {\You} deliver me from evil.
			\item	For {\Yours} is the Kingdom,
					\\
					and the power,
					\\
					and the glory,
					\\
					now and forever.
			\item	Amen!
		\end{itemize}

	\subsection{{\Jesus}, come into my life}
		\begin{itemize}[nosep]
			\item	{\Jesus},
					I would like {\You} to come into my life.
			\item	{\Jesus},
					I am opening my \textit{(heart's)} door.
					\\
					I am opening my heart to {\You}.
			\item	{\Jesus},
					I would like my life to be led by {\You},
					\\
					and that {\You} take the leadership of my life.
			\item	{\Jesus},
					I want to live with {\You},
					\\
					and I believe in {\You}.
			\item	I believe in resurrection;
					\\
					and I believe that {\You} are the way,
					\\
					the truth and life.
			\item	{\Jesus},
					I hand over my life to {\You}.
			\item	Amen!
		\end{itemize}

		\subsection{Surrender Prayer}
			\begin{itemize}[nosep]
				\item	{\Jesus},
						I just want to understand,
						who {\You} are.
				\item	And {\Jesus},
						I want to surrender my life to {\You} now.
				\item	Maybe I do not understand what it means;
						\\
						and maybe I do not understand,
						why I need it right now.
				\item	Maybe I do not understand who {\You} are,
						\\
						what my mission is,
						what my purpose is in this world.
				\item	And {\Jesus},
						maybe I am also angry with {\You}.
						\\
						Maybe I am hurt.
						\\
						Maybe there are things that somehow stand in my way,
						\\
						to fully surrender to {\You}.
				\item	But {\Jesus},
						I pray hereby that {\You} take away all that,
						\\
						all the doubts,
						and everything that still separates me from {\You}.
				\item	And {\Jesus},
						I pray that I can now gently lay my life in {\Your} hands,
						\\
						and {\You} make of it,
						what is best for {\Your} plan,
						\\
						and for {\Your} kingdom.
				\item	And I pray that {\You} restore my life,
						\\
						and give me vitality again,
						\\
						and joy that comes from {\You},
						\\
						living water within me,
						coming from {\You};
				\item	and {\Jesus},
						that {\You} free me from chains,
						\\
						from which I cannot get out right now,
						\\
						or where I'm still trapped.
				\item	{\Jesus},
						let me feel {\Your} freedom,
						\\
						and grant me {\Your} peace.
				\item	And take my life into {\Your} hands,
						\\
						and let me become {\Your} child;
						\\
						and let me see {\You} in heaven.
				\item	Amen!
			\end{itemize}

	\subsection{The Creed of Belief}
		The Creed of Belief is adopted nearly 1-by-1.
		The only missing part is the church-part,
		because a Free Christian is not bound to a chruch community.
		\\
		\begin{itemize}[nosep]
			\item	I believe in {\God},
					\\
					the {\Father},
					the {\Allmighty},
					\\
					the {\Creator} of {\Heaven} and Earth.
			\item	I believe in {\Jesus} {\Christ},
					\\
					{\His} only {\Son},
					my {\Lord};
					\\
					he was conceived by the {\Holy} {\Spirit},
					\\
					and born by the virgin Mary;
					\\
					{\He} suffered under Pontius Pilate;
					\\		
					{\He} was crucified, {\He} died and was buried,
					\\
					and descended to the Realm of Death;
					\\
					on the Third Day {\He} rose again,
					\\
					{\He} ascended into {\Heaven};
					\\
					{\He} is seated to the right hand of the {\Father};
					\\
					{\He} will come again,
					\\
					to judge the living and the dead.
			\item	I believe in the {\Holy} {\Spirit},
					\\
					the forgiveness of sins,
					\\
					the resurrection of the body,
					\\
					and in everlasting life.
			\item	Amen!
		\end{itemize}

	\subsection{Wie ein Fest nach langer Trauer}
		\subsubsection{Information}
		\begin{itemize}[nosep]
			\item English: Like a celebration after a long sorrow.
			\item Song book: Von {\Jesus} singen 2 (\q{Singing about {\Jesus} 2})
			\item ISBN: 9783775123099
			\item Composer: J\"urgen Werth
		\end{itemize}
		
		\subsubsection{Lyrics}
		\begin{itemize}
			\item	\textbf{Verse 1:}
			\\		Wie ein Fest nach langer Trauer,
			\\		wie ein Feuer in der Nacht.
			\\		Ein off'nes Tor in einer Mauer,
			\\		f\"ur die Sonne auf gemacht.
			\\		Wie ein Brief nach langem Schweigen,
			\\		wie ein unverhoffter Gru{\ss}.
			\\		Wie ein Blatt an toten Zweigen.
			\\		Ein \q{Ich mag dich trotzdem.}-Kuss.
			\item	\textbf{Verse 2:}
			\\		Wie ein Regen in der W\"uste,
			\\		frischer Tau auf d\"urrem Land.
			\\		Heimatkl\"ange für vermisste.
			\\		Alte Feinde Hand in Hand.
			\\		Wie ein Schl\"ussel im Gef\"angnis,
			\\		wie in Seenot \q{Land in Sicht!}.
			\\		Wie ein Weg aus der Bedr\"angnis.
			\\		Wie ein strahlendes Gesicht.
			\item	\textbf{Verse 3:}
			\\		Wie ein Wort von toten Lippen,
			\\		wie ein Blick der Hoffnung weckt.
			\\		Wie ein Licht auf steilen Klippen,
			\\		wie ein Erdteil neu entdeckt.
			\\		Wie der Fr\"uhling,
					wie der Morgen,
			\\		Wie ein Lied,
					wie ein Gedicht.
			\\		Wie das Leben,
					wie die Liebe,
			\\		Wie {\Gott} {\Selbst},
					das wahre Licht!
			\item	\textbf{Chorus \textit{(2x each)}:}
			\\		So ist Vers\"ohnung,
			\\		so muss der wahre Friede sein.
			\\		So ist Vers\"ohnung,
			\\		so ist vergeben und verzeih'n.
		\end{itemize}
		
		\subsubsection{Translation}
		\begin{itemize}
			\item	\textbf{Verse 1:}
			\\		Like a celebration after long sorrow,
			\\		like a fire in the night.
			\\		An open gate in a wall,
			\\		opened for the sun.
			\\		Like a letter after long silence,
			\\		like an unexpected greeting.
			\\		Like a leaf on dead branches.
			\\		An \q{I like you anyway.} kiss.
			\item	\textbf{Verse 2:}
			\\		Like rain in the desert,
			\\		fresh dew on dry land.
			\\		Homeland sounds for the missed.
			\\		Old enemies hand in hand.
			\\		Like a key in prison,
			\\		like \q{Land in sight!} in distress.
			\\		Like a way out of difficulties.
			\\		Like a bright face.
			\item	\textbf{Verse 3:}
			\\		Like a word from the lips of dead,
			\\		like a look that awakens hope.
			\\		Like light on steep cliffs,
			\\		like a continent newly discovered.
			\\		Like spring,
					like the morning,
			\\		Like a song,
					like a poem.
			\\		Like life,
					like love.
			\\		Like {\God} {\Himself},
					the true light!
			\item	\textbf{Chorus \textit{(2x each)}:}
			\\		This is reconciliation,
			\\		this is how true peace must be.
			\\		This is reconciliation,
			\\		this is forgiving and condoning.
		\end{itemize}

	\newpage
	\section{My life with {\God}} \label{MeinLebenMitGott}
		This chapter is a kind of irregular \q{diary} in the broadest sense,
		how I experience my journey with and to {\God},
		and all the things a may learn on it.
	
	\subsection{Wednesday, September 27th, 2023}
		I have been on a kind of journey since mid-2023,
		on which I decided to let {\God} and {\Jesus} in my life.
		I still do have a lot of flaws,
		and despite of the {\God} given commandments \textit{(offers)},
		I sin much too often.
		As mentioned in the preface,
		I am still far,
		far away of being a kind of \q{perfect Christian}.
		Many of my everyday habits,
		of my character and other stuff,
		they have such a strong pulling effect
		that often I do not think of {\God},
		and I do not pray,
		and read the Bible,
		as often as I would like to.
		And whenever {\God} comes \q{back again} into my thinking,
		I often have a guilty conscience,
		because for me it feels like I have \q{forgotten} {\Him}.
		So,
		long story short:
		I may still learn a lot and way too much!

	\subsection{Friday, September 29th, 2023}
		Today I watched a video that gave me a lot to think about.
		I do not know,
		whether it is blasphemy.
		Still,
		I want to share with you,
		what I saw in it.
		It was basically a short animation,
		in which a Muslim,
		an atheist and a Christian went to {\Heaven}.
		Since you haven't really seen {\God} {\Himself} here,
		but just a caricature,
		I will use normal capitalization here.
		So it was all about,
		which statements God feels insulted and offended by.
		And in the end he gave the atheist his peace,
		and actually sent him to heaven,
		because he never believed in God,
		and neither said one thing nor the other about him.
		And he, God, was disappointed in the Muslim and in the Christian,
		because they basically represented him as such an \q{evil} God,
		as if he would simply throw all people into hell,
		who sin and do not believe in him.
		That hurt him very much,
		because he effectively felt like a cruel monster.
		This so far is the short version.
		And that got me thinking.
		Of course,
		I can only speculate.
		But maybe it is like that,
		that {\God} is not \q{sending} us anywhere.
		If we choose {\Him},
		{\He} invites us to stay in Heaven after death.
		And if we choose, for example, the devil,
		then it truely may be,
		that we will go to hell.
		But not because {\God} sends us there,
		but because the devil takes us away.
		Like I said ... I do not know.
		This video just made me think.
		And probably,
		it may often be the case,
		that one might say this or that about {\God},
		but he or she actually does not know,
		what is the real truth.
		But what you can do at least,
		is to think about it first,
		when you say something about {\God},
		whether you would want to be told similar things about yourself.
		If you would like to make your own judgment about the video,
		here is the link: \url{https://www.youtube.com/watch?v=ttevamkS6gw}.

	\subsection{Tuesday, October 3rd, 2023}
		Last weekend,
		until inclusive yesterday,
		I have been to Hamburg with my wife and my parents.
		But for some reason,
		starting on Sunday,
		I did not feel so well during the day.
		I was,
		among other things,
		overtired,
		and tormented by headaches.
		And I assume,
		my lack of motion or exercise also became apparent here,
		because I experienced every staircase as torture.
		At least I could read in my Bible app in the morning,
		what was already worth a lot to me.
		In the meantime I had a suspicion,
		where my headache may have come from.
		But I have no proof for it,
		so it will just stay an assumption.
		Anyway,
		since I wanted to go through the Sabbath this time,
		I stayed abstinent from coffee and potentially sugar-containing foods.
		My wife often suggested that I should drink a cup of coffee,
		but I did not want to fall for this trial.
		And I also heard once that sugar can be addictive.
		And thus it belongs to the things,
		I want to be abstinent from.
		I also heard once that one withdrawal symptom,
		e.g. with a sugar \q{addiction},
		can be headache.
		And when that is the case,
		then I sin really often outside of Sabbath for that matter.
		See also: \q{\hyperref[TheTenthOffer]{The Tenth Offer}}.

	\subsection{Saturday, October 7th, 2023}
		I am slowly getting the feeling that it is getting serious.
		Of course,
		in a positive sense.
		Because of this,
		I spontaneously decided today to prepare the GitHub discussion
		(mentioned in the preface)
		for you,
		and release this work as a monthly edition.
		The first one is the October edition,
		which I published earlier.
		The November edition should then be released promptly at the turn of the month,
		planned on November 1st.
				
	\subsection{Sunday, October 8th, 2023}
		Unfortunately,
		I caught a cold after the weekend in Hamburg.
		Nevertheless,
		I still want to share a beautiful Bible passage with you today.
		It is Romans 10,
		verses 5 to 11,
		with the heading
		\q{Salvation is available to all}
		from the German \q{Neues Leben Bibel} \textit{(New Life Bible)},
		translated into English.
		Enjoy reading.
		\begin{enumerate}[nosep,start=5]
			\item	For Moses writes that the law’s way of making a person right with {\God} requires obedience to all of its commands.
			\item	But faith’s way of getting right with {\God} says,
					do not say in your heart,
					\q{Who will go up to heaven?},
					to bring {\Christ} down to earth.
			\item	And do not say,
					\q{Who will go down to the place of the dead?},
					to bring {\Christ} back to life again.
			\item	In fact,
					it says,
					\q{The message is very close at hand;
					it is on your lips and in your heart.}
					And that message is the very message about faith that we preach.
			\item	If you openly declare that {\Jesus} is the {\Lord},
					and believe in your heart that {\God} raised {\Him} from the dead,
					you will be saved.
			\item	For it is by believing in your heart that you are made right with {\God},
					and it is by openly declaring your faith that you are saved.
			\item	As the scriptures tell us,
					\q{Anyone who trusts in {\Him} will never be disgraced.}
		\end{enumerate}

	\newpage
	\section{Peace be with you!}
		At the end,
		I would like to share a beautiful chorus with you.
		It is from the song \q{Oceans} by the band Hillsong United.
		Since the text is,
		to the best of my knowledge,
		copyrighted still in October 2023,
		I would like to rely on the right to quote,
		and reproduce the text unchanged:
		\\	
		\begin{itemize}[nosep]
			\item[]	{\Spirit} lead me,
			where my trust is without borders.
			\item[] Let me walk upon the waters,
			\item[] wherever {\You} would call me
			\item[]	Take me deeper than my feet could ever wander.
			\item[]	And my faith will be made stronger
			\item[]	in the presence of my {\Saviour}.
			\\
		\end{itemize}
		\begin{figure}[h]
			\centering
			\includegraphics[width=0.75\textwidth,keepaspectratio]{"FreeChristian.jpeg"}
		\end{figure}
	
\end{document}
