\documentclass[12pt,a4paper]{article}

\usepackage[english]{babel}
\usepackage{enumitem}
\usepackage[utf8]{inputenc}
\usepackage[T1]{fontenc}
\usepackage[left=25mm,right=25mm,top=25mm,bottom=25mm]{geometry}
\usepackage[colorlinks=true,urlcolor=blue,linkcolor=black]{hyperref}

\newcommand{\Christ}[0]{\textbf{CHRIST}}
\newcommand{\Christus}[0]{\textbf{CHRISTUS}}
\newcommand{\God}[0]{\textbf{GOD}}
\newcommand{\He}[0]{\textbf{HE}}
\newcommand{\Him}[0]{\textbf{HIM}}
\newcommand{\Himself}[0]{\textbf{HIMSELF}}
\newcommand{\His}[0]{\textbf{HIS}}
\newcommand{\Holy}[0]{\textbf{HOLY}}
\newcommand{\Jesus}[0]{\textbf{JESUS}}
\newcommand{\Lord}[0]{\textbf{LORD}}
\newcommand{\Messiah}[0]{\textbf{MESSIAH}}
\newcommand{\Redeemer}[0]{\textbf{REDEEMER}}
\newcommand{\Savior}[0]{\textbf{SAVIOR}}
\newcommand{\Saviour}[0]{\textbf{SAVIOUR}}
\newcommand{\Son}[0]{\textbf{SON}}
\newcommand{\Spirit}[0]{\textbf{SPIRIT}}
\newcommand{\Thee}[0]{\textbf{THEE}}
\newcommand{\Their}[0]{\textbf{THEIR}}
\newcommand{\Them}[0]{\textbf{THEM}}
\newcommand{\Thine}[0]{\textbf{THINE}}
\newcommand{\Thou}[0]{\textbf{THOU}}
\newcommand{\Thy}[0]{\textbf{THY}}
\newcommand{\You}[0]{\textbf{YOU}}
\newcommand{\Your}[0]{\textbf{YOUR}}
\newcommand{\Yours}[0]{\textbf{YOURS}}

\newcommand{\q}[1]{\char"22{#1}\char"22 }

\title{\textbf{The Free Christian}}
\author{Robert Lang-Kirchh\"ofer}
\date{\textit{Last modified: September 29th, 2023}}

\begin{document}
	\setlength{\parindent}{0mm}
	\maketitle
	\newpage

	\tableofcontents
	\newpage
	
	\section{Preface}

	\subsection{Using hints}
		It may happen that I will address you,
		the reader,
		directly.
		It may also happen that,
		whenever human beings in general are mentioned,
		I will use a potentially existing masculine form.
		My intension is to create a comfortable,
		casual atmosphere,
		and to ease reading flow.
		Of course,
		my dear reader,
		totally independent of your actual gender or sex,
		I am not making any assumptions about you,
		and you will have my fullest respect.
		
	\subsection{Thank you!}
		Next I want to state my cordial thanks to you,
		my dear reader,
		that you decided to take a look into this eBook.
		Of course,
		I hope that you also will read it until the end,
		and that you will watch its development.
		I am not a 100 percent sure that it will be \q{finished} one day,
		because possibly there will always be new thoughts,
		or new material that can be included in here.
		This here is a Christian document.
		My intention is to communicate moral values to you,
		especially as they are wished by {\God}, the {\Lord},
		and {\His} {\Son} {\Jesus} {\Christ},
		by my best knowledge and conscience.
		Of course I do not want to invent a new \q{religion},
		or to re-invent Christianity,
		but rather show some new perspectives.
		As you can tell of this preface,
		words,
		which directly relate to {\God},
		{\Jesus} or also to the {\Holy} {\Spirit},
		are written in capital and bold letters.
		If something is near to your heart,
		or you have general ideas,
		I invite you sincerely,
		to participate in my  \href{https://github.com/DerRobert-28/Der-Freie-Christ/discussions}{GitHub discussion}.
	
	\subsection{Some words about myself}
		Myself was baptiszed catholic,
		as far as I remember correctly,
		at the age of one or two years,
		but I left church in mid-August 2023.
		The reasons for this are of private nature,
		and are not relevant here.
		But this has nothing to do with my belief.
		I do believe that {\God} exists,
		and that {\Jesus} is the {\Messiah}.
		Anyway,
		it does not mean that I am a kind of \q{perfect Christian},
		if such a person exists at all nowadays.
		You can read more about me in chapter \q{\hyperref[MeinLebenMitGott]{My life with {\God}}}.
	
	\section{What makes up a \q{Free Christian}?}
	
	\subsection{No church community}
		A Free Christian is not bound to a church community.
		That means,
		you can,
		but you do not have to be baptized.
		You can also have left your church.
		That does not play a role at all.
		The only important thing is,
		to let {\God},
		the {\Lord},
		and {\Jesus} {\Christ} come into your life,
		and that you confess yourself to {\Them}.
	
	\subsection{The true bond}
		For me,
		the only true,
		existing bond is between {\God},
		{\Jesus} {\Christ} and me.
		When I confess myself to {\Them},
		I cherish my relationship sincerely.
		Mundane bonds \textit{(relationships)} are ephemeral,
		although it is not less important,
		to also cherish them warmly.
	
	\subsection{The Bible as a \q{tool}}
		If it serves the relationship between {\God},
		{\Jesus} and me,
		I have the freedom to interpet biblical passages better,
		that means more modern or more understandable,
		and rephrase them accordingly.
		But this is \textbf{not a} \q{licence},
		to rewrite the Word of {\God} by free convenience,
		and then for example to bend {\His} will,
		as it was done willingly in the past by the catholic church,
		according to my information and my research.
		
	\subsection{Belief and science}
		Faith and science are,
		in my opinion,
		not mutually exclusive.
		{\God} can neither be proven,
		nor refuted.
		You always have free choice,
		whether you want to live with {\God},
		or not.
		That is,
		why you as a human being have been given free will.
		
	\section{The Ten Commandments}
		The traditional 10 commandments are usually handed down from the viewpoint of {\God},
		that means in the form of \q{Thou shalt (not) ...},
		or more modern \q{You shall (not) ...}.
		Below are the 10 commandments from your own viewpoint,
		as if yourself would talk to {\God},
		and make them a promise to {\Him}.
		They are also worked out a bit better,
		because on closer consideration,
		some commandments can be summarized.
		Of course,
		that does not mean that I refuse,
		or decline the traditional {\God} given commandments.
		I just want to show another perspective.
		In addition I will call them \q{offers},
		to emphasize the good will of both sides.
	
	\subsection{The Highest Offer}
		The Highest Offer is:
		I will love and praise {\God},
		the {\Lord},
		sincerely with all of my heart.
		And I will love my neighbour,
		as I also do myself.
		
	\subsection{The First Offer}
		{\Thou} art the {\Lord},
		my {\God},
		my {\Saviour}.
		I will not have other gods before or besides {\Thee},
		and I will not praise or worship them.
		And I will not create an idol or a graven image.
		
	\subsection{The Second Offer}
		{\Thou} art the {\Lord},
		my {\God}.
		I will not misuse {\Thy} name.
		I will not slander or blaspheme against {\Thee}.
		And I will confess myself honestly to {\Thee}.
		\\
		\textit{Short hint:
		This should also include colloquial phrases,
		such as \q{Oh (my) G...},
		or \q{For G...'s sake},
		that you say quickly,
		but without really meaning {\God} {\Himself},
		or to pray to {\Him},
		or something similar.}
			
	\subsection{The Third Offer}
		{\Thou} art the {\Lord},
		my {\God}.
		I will not put {\Thee} to the test.
		I will not try {\Thee}.
		Also in need I will confess myself to {\Thee}.
		\\
		\textit{Short hint:
		This is also intended to cover situations,
		in which one carelessly says such things as,
		for example,
		how {\God} can allow this or that suffering.}
		
	\subsection{The Fourth Offer}
		{\Thou} art the {\Lord},
		my {\God}.
		I will sanctify and keep the Sabbath to {\Thee}.
		On Sabbath I will stay abstinent of fleshly
		and addictive things.
		
	\subsection{The Fifth Offer}
		I will praise my father and my mother,
		who gave my life to me,
		and raised and nourished me.
		And I will praise Elder People.
			
	\subsection{The Sixth Offer}
		I will not kill or murder.
		I will cherish my relationships.
		I will respect life and the wellbeing of all living beings,
		and I will protect them as far as possible.
		\\
		\textit{Short hint:
		The killing here is not just literal,
		meaning physically,
		but also symbolic,
		for example,
		by saying something to someone out of anger,
		what hurts him,
		and thus damages the relationship.
		Feel free to read the Bible passage \href{https://www.die-bibel.de/bibeln/online-bibeln/lesen/ESV/MAT.5/Matthew-5}{Matthew 5, 21-22}.}
		
	\subsection{The Seventh Offer}
		I will not commit adultery.
		I will not covet my neighbour's woman or wife.
		I will not covet my neighbour's man or husband.
		
	\subsection{The Eighth Offer}
		I will not steal or betray.
		I will not rob or kidnap.
		I will not covet my neightbour's house.
		I will not covet my neightbour's belongings.
		I will not cause harm to my neighbour's belongings.
		
	\subsection{The Ninth Offer}
		I will not bear false witness against my neighbour.
		I will not lie or betray.
		I will not swear.
		I will act honestly and fairly against my neightbour.
		
	\subsection{The Tenth Offer}
		My body is a present of {\Thine},
		and therefore it is holy.
		I will honour and cherish it.

	\section{My life with {\God}} \label{MeinLebenMitGott}
		This big chapter is a kind of diary,
		how I experience my journey with and to {\God},
		and all the things a may learn on it.
	
	\subsection{Wednesday, September 27th, 2023}
		I have been on a kind of journey since mid-2023,
		on which I decided to let {\God} and {\Jesus} in my life.
		I still do have a lot of flaws,
		and despite of the {\God} given commandments \textit{(offers)},
		I sin much too often.
		As mentioned in the preface,
		I am still far,
		far away of being a kind of \q{perfect Christian}.
		Many of my everyday habits,
		of my character and other stuff,
		they have such a strong pulling effect
		that often I do not think of {\God},
		and I do not pray,
		and read the Bible,
		as often as I would like to.
		And whenever {\God} comes \q{back again} into my thinking,
		I often have a guilty conscience,
		because for me it feels like I have \q{forgotten} {\Him}.
		So,
		long story short:
		I may still learn a lot and way too much!

	\subsection{Friday, September 29th, 2023}
		Today I watched a video that gave me a lot to think about.
		I do not know,
		whether it is blasphemy.
		Still,
		I want to share with you,
		what I saw in it.
		It was basically a short cartoon,
		in which a Muslim woman,
		an atheist and a Christian went to heaven.
		Since you haven't really seen {\God} {\Himself} here,
		but just a caricature,
		I will use normal capitalization here.
		So it was all about,
		which statements God feels insulted and offended by.
		And in the end he gave the atheist his peace,
		and actually sent him to heaven,
		because he never believed in God,
		and neither said one thing nor the other about him.
		And he, God, was disappointed in the Muslim and in the Christian,
		because they basically represented him as such an \q{evil} God,
		as if he would simply throw all people into hell,
		who sin and do not believe in him.
		That hurt him very much,
		because he effectively felt like a cruel monster.
		This so far is the short version.
		And that got me thinking.
		Of course,
		I can only speculate.
		But maybe it is like that,
		that {\God} is not \q{sending} us anywhere.
		If we choose {\Him},
		{\He} invites us to stay in Heaven after death.
		And if we choose, for example, the devil,
		then it truely may be,
		that we will go to hell.
		But not because {\God} sends us there,
		but because the devil takes us away.
		Like I said... I do not know.
		This video just made me think.
		Because, certainly, it may often be the case,
		that one might say this or that about {\God},
		but he or she actually does not know,
		what is the real truth.
		But what you can do at least,
		is to think about it first,
		when you say something about {\God},
		whether you would want to be told similar things about yourself.
		If you would like to make your own judgment about the video,
		here is the link: \url{https://www.youtube.com/watch?v=ttevamkS6gw}.

\end{document}
