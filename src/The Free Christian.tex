\documentclass[12pt,a4paper]{article}

\usepackage[english]{babel}
\usepackage{enumitem}
\usepackage[utf8]{inputenc}
\usepackage[T1]{fontenc}
\usepackage[left=25mm,right=25mm,top=25mm,bottom=25mm]{geometry}
\usepackage[colorlinks=true,urlcolor=blue,linkcolor=black]{hyperref}

\newcommand{\Christ}[0]{\textbf{CHRIST}}
\newcommand{\Christus}[0]{\textbf{CHRISTUS}}
\newcommand{\God}[0]{\textbf{GOD}}
\newcommand{\He}[0]{\textbf{HE}}
\newcommand{\Him}[0]{\textbf{HIM}}
\newcommand{\His}[0]{\textbf{HIS}}
\newcommand{\Holy}[0]{\textbf{HOLY}}
\newcommand{\Jesus}[0]{\textbf{JESUS}}
\newcommand{\Lord}[0]{\textbf{LORD}}
\newcommand{\Messiah}[0]{\textbf{MESSIAH}}
\newcommand{\Redeemer}[0]{\textbf{REDEEMER}}
\newcommand{\Savior}[0]{\textbf{SAVIOR}}
\newcommand{\Saviour}[0]{\textbf{SAVIOUR}}
\newcommand{\Son}[0]{\textbf{SON}}
\newcommand{\Spirit}[0]{\textbf{SPIRIT}}
\newcommand{\Thee}[0]{\textbf{THEE}}
\newcommand{\Their}[0]{\textbf{THEIR}}
\newcommand{\Them}[0]{\textbf{THEM}}
\newcommand{\Thine}[0]{\textbf{THINE}}
\newcommand{\Thou}[0]{\textbf{THOU}}
\newcommand{\Thy}[0]{\textbf{THY}}
\newcommand{\You}[0]{\textbf{YOU}}
\newcommand{\Your}[0]{\textbf{YOUR}}
\newcommand{\Yours}[0]{\textbf{YOURS}}

\newcommand{\q}[1]{\char"22{#1}\char"22 }

\title{\textbf{The Free Christian}}
\author{Robert Lang-Kirchh\"ofer}
\date{\textit{Last modified: September 2023}}

\begin{document}
	\setlength{\parindent}{0mm}
	\maketitle
	\newpage

	\tableofcontents
	\newpage
	
	\section{Preface}

	\subsection{Using hints}
		It may happen that I will address you,
		the reader,
		directly.
		It may also happen that,
		whenever human beings in general are mentioned,
		I will use a potentially existing masculine form.
		My intension is to create a comfortable,
		casual atmosphere,
		and to ease reading flow.
		Of course,
		my dear reader,
		totally independent of your actual gender or sex,
		I am not making any assumptions about you,
		and you will have my fullest respect.
		
	\subsection{Thank you!}
		Next I want to state my cordial thanks to you,
		my dear reader,
		that you decided to take a look into this eBook.
		Of course,
		I hope that you also will read it until the end,
		and that you will watch its development.
		I am not a 100 percent sure that it will be finished one day,
		because possibly there will always be new thoughts,
		or new material that can be included in here.
		This here is a Christian document.
		My intention is to communicate moral values to you,
		especially as they are wished by {\God}, the {\Lord},
		and {\His} {\Son} {\Jesus} {\Christ},
		by best knowledge and conscience.
		As you can tell of this preface,
		words,
		which directly relate to {\God},
		{\Jesus} or also to the {\Holy} {\Spirit},
		are written in capital and bold letters.
		If something is near to your heart,
		or you have general ideas,
		I invite you sincerely,
		to participate in my  \href{https://github.com/DerRobert-28/Der-Freie-Christ/discussions}{GitHub discussion}.
	
	\subsection{Some words about myself}
		Myself was baptiszed catholic,
		as far as I remember correctly,
		at the age of one or two,
		bit I left church in mid-August 2023.
		The reasons for this are of private nature,
		and are not relevant here.
		But it does not have soething to do with my belief.
		I myself believe that {\God} exists,
		and that {\Jesus} is the {\Messiah}.
		But although,
		it does not mean that I am a kind of \q{perfect Christian},
		if such a person exits at all nowadays.
		You can read more about me in chapter \q{\hyperref[MeinLebenMitGott]{My life with {\God}}}.
	
	\section{What makes up a \q{Free Christian}?}
	
	\subsection{No church community}
		A Free Christian is not bound to a church community.
		That means,
		you can,
		but you do not have to be baptized.
		You can also have left your church.
		That does not play a role at all.
		The only important thing is,
		to let {\God},
		the {\Lord},
		and {\Jesus} {\Christ} come into your life,
		and that you confess yourself to {\Them}.
	
	\subsection{The True Bond}
		The only true,
		existing bond is between {\God},
		{\Jesus} {\Christ} and me.
		When you confess yourself to {\Them},
		you cherish your relationship sincerely.
		Mundane bonds \textit{(relationships)} are ephemeral,
		although it is not less important,
		to also cherish them warmly.
	
	\subsection{The Bible as a \q{tool}}
		If it serves the relationship between {\God},
		{\Jesus} and me,
		I have the freedom to interpet biblical passages better,
		that means more modern or more understandable,
		and rephrase them accordingly.
		But this is \textbf{not a} \q{licence},
		to rewrite the Word of {\God} by free convenience,
		and then for example to bend {\His} will,
		as it was done willingly in the past by the catholic church,
		according to my information and my research.
		
	\section{The Ten Commandments}
		The traditional 10 commandments are usually handed down from the viewpoint of {\God},
		that means in the form \q{Thou shalt (not) ...}.
		Below are the 10 commandments from your own viewpoint,
		as if yourself would talk to {\God},
		and make them a promise to {\Him}.
		They are also worked out a bit better,
		because on closer consideration,
		some commandments can be summarized.
		Of course,
		that does not mean that I refuse,
		or decline the traditional {\God} given commandments.
		I just want to show another perspective.
		In addition I will call them \q{Offers},
		to emphasize the good will of both sides.
	
	\subsection{The Highest Offer is ...}
		I will love and praise {\God},
		the {\Lord},
		sincerely with all of my heart.
		And I will love my neighbour,
		as I also do myself.
		
	\subsection{The First Offer}
		{\Thou} art the {\Lord},
		my {\God},
		my {\Saviour}.
		I will not have other gods before or besides {\Thee},
		and I will not praise or worship them.
		And I will not create an idol or a graven image.
		
	\subsection{The Second Offer}
		{\Thou} art the {\Lord},
		my {\God}.
		I will not misuse {\Thy} name.
		I will not slander or blaspheme against {\Thee}.
		And I will confess myself honestly to {\Thee}.
			
	\subsection{The Third Offer}
		{\Thou} art the {\Lord},
		my {\God}.
		I will not put {\Thee} to the test.
		I will not try {\Thee}.
		I also will confess myself to {\Thee} in need.
		
	\subsection{The Fourth Offer}
		{\Thou} art the {\Lord},
		my {\God}.
		I will sanctify and keep the Sabbath to {\Thee}.
		On Sabbath I will stay abstinent of fleshly
		and addictive things.
		
	\subsection{The Fifth Offer}
		I will praise my farther and my mother.
		And I will praise Elder People.
			
	\subsection{The Sixth Offer}
		I will not kill or murder.
		I will cherish my relationships.
		I will respect life and the wellbeing of all living beings,
		and I will protect them as far as possible.
		
	\subsection{The Seventh Offer}
		I will not commit adultery.
		I will not covet my neighbour's woman or wife.
		I will not covet my neighbour's man or husband.
		
	\subsection{The Eighth Offer}
		I will not steal or betray.
		I will not rob or kidnap.
		I will not covet my neightbour's house.
		I will not covet my neightbour's belongings.
		I will not cause harm to my neighbour's belongings.
		
	\subsection{The Ninth Offer}
		I will not bear false witness against my neighbour.
		I will not lie or betray.
		I will not swear.
		I will act honestly and fairly against my neightbour.
		
	\subsection{The Tenth Offer}
		My body is a present of {\Thine},
		and therefore it is holy.
		I will honour and cherish it.

	\section{My life with {\God}} \label{MeinLebenMitGott}
		This big chapter is a kind of diary,
		how I experience my journey with and to {\God},
		and all the things a may learn on it.
	
	\subsection{Wednesday, September 27th, 2023}
		I have been on a kind of journey since mid-2023,
		on which I decided to let {\God} and {\Jesus} in my life.
		I myself do still have a lot of flaws,
		and despite of the {\God} given commandments \textit{(offers)},
		I sin much too often.
		As mentioned in the preface,
		I am still far far away of being a kind of \q{perfect Christian}.
		Many of my everyday habits,
		of my character and other stuff,
		they have such a strong pulling effect
		that often I do not think of {\God},
		and I do not pray,
		and read the Bible,
		as often as I would like to.
		And whenever {\God} come \q{back again} into my thinking,
		I often have a guilty conscience,
		because for me it feels like I have \q{forgotten} {\Him}.
		So long story short:
		I may still learn a lot and way too much!

\end{document}
