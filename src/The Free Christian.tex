\documentclass[12pt,a4paper]{article}

\usepackage[english]{babel}
\usepackage{enumitem}
\usepackage[utf8]{inputenc}
\usepackage[T1]{fontenc}
\usepackage[left=25mm,right=25mm,top=25mm,bottom=25mm]{geometry}
\usepackage[colorlinks=true,urlcolor=blue,linkcolor=black]{hyperref}

\newcommand{\Christ}[0]{\textbf{CHRIST}}
\newcommand{\Christus}[0]{\textbf{CHRISTUS}}
\newcommand{\God}[0]{\textbf{GOD}}
\newcommand{\He}[0]{\textbf{HE}}
\newcommand{\Him}[0]{\textbf{HIM}}
\newcommand{\His}[0]{\textbf{HIS}}
\newcommand{\Holy}[0]{\textbf{HOLY}}
\newcommand{\Jesus}[0]{\textbf{JESUS}}
\newcommand{\Lord}[0]{\textbf{LORD}}
\newcommand{\Messiah}[0]{\textbf{MESSIAH}}
\newcommand{\Redeemer}[0]{\textbf{REDEEMER}}
\newcommand{\Savior}[0]{\textbf{SAVIOR}}
\newcommand{\Saviour}[0]{\textbf{SAVIOUR}}
\newcommand{\Son}[0]{\textbf{SON}}
\newcommand{\Spirit}[0]{\textbf{SPIRIT}}
\newcommand{\Thee}[0]{\textbf{THEE}}
\newcommand{\Their}[0]{\textbf{THEIR}}
\newcommand{\Them}[0]{\textbf{THEM}}
\newcommand{\Thou}[0]{\textbf{THOU}}
\newcommand{\Thy}[0]{\textbf{THY}}
\newcommand{\You}[0]{\textbf{YOU}}
\newcommand{\Your}[0]{\textbf{YOUR}}

\newcommand{\q}[1]{\char"22{#1}\char"22 }

\title{\textbf{The Free Christian}}
\author{Robert Lang-Kirchh\"ofer}
\date{\textit{Last modified: September 2023}}

\begin{document}
	\setlength{\parindent}{0mm}
	\maketitle
	\newpage

	\tableofcontents
	\newpage
	
	\section{Preface}

	\subsection{Using hints}
		It may happen that I will address you,
		the reader,
		directly.
		It may also happen that,
		whenever human beings in general are mentioned,
		I will use a potentially existing masculine form.
		My intension is to create a comfortable,
		casual atmosphere,
		and to ease reading flow.
		Of course,
		my dear reader,
		totally independent of your actual gender or sex,
		I am not making any assumptions about you,
		and you will have my fullest respect.
		
	\subsection{Thank you!}
		Next I want to state my cordial thanks to you,
		my dear reader,
		that you decided to take a look into this eBook.
		Of course,
		I hope that you also will read it until the end,
		and that you will watch its development.
		I am not a 100 percent sure that it will be finished one day,
		because possibly there will always be new thoughts,
		or new material that can be included in here.
		This here is a Christian document.
		My intention is to communicate moral values to you,
		especially as they are wished by {\God}, the {\Lord},
		and {\His} {\Son} {\Jesus} {\Christ},
		by best knowledge and conscience.
		As you can tell of this preface,
		words,
		which directly relate to {\God},
		{\Jesus} or also to the {\Holy} {\Spirit},
		are written in capital and bold letters.
		If something is near to your heart,
		or you have general ideas,
		I invite you sincerely,
		to participate in my  \href{https://github.com/DerRobert-28/Der-Freie-Christ/discussions}{GitHub discussion}.
	
	\subsection{Some words about myself}
		Myself was baptiszed catholic,
		as far as I remember correctly,
		at the age of one or two,
		bit I left church in mid-August 2023.
		The reasons for this are of private nature,
		and are not relevant here.
		But it does not have soething to do with my belief.
		I myself believe that {\God} exists,
		and that {\Jesus} is the {\Messiah}.
		But although,
		it does not mean that I am a kind of \q{perfect Christian},
		if such a person exits at all nowadays.
		You can read more about me in chapter \q{\hyperref[MeinLebenMitGott]{My life with {\God}}}.

	
	\section{What makes up a \q{Free Christian}?}
	
	\subsection{No church community}
		A Free Christian is not bound to a church community.
		That means,
		you can,
		but you do not have to be baptized.
		You can also have left your church.
		That does not play a role at all.
		The only important thing is,
		to let {\God},
		the {\Lord},
		and {\Jesus} {\Christ} come into your life,
		and that you confess yourself to {\Them}.
	
	\subsection{The True Bond}
		The only true,
		existing bond is between {\God},
		{\Jesus} {\Christ} and me.
		When you confess yourself to {\Them},
		you cherish your relationship sincerely.
		Mundane bonds \textit{(relationships)} are ephemeral,
		although it is not less important,
		to also cherish them warmly.
	
%	\subsection{Die Bibel als \q{Werkzeug}}
%		Wenn es der Beziehung zwischen {\Gott},
%		{\Jesus} und mir dient,
%		habe ich die Freiheit,
%		Bibelstellen besser, also moderner oder verst\"andlicher, auszulegen,
%		und entsprechend umzuformulieren.
%		Das ist jedoch \textbf{kein} Freibrief daf\"ur,
%		das Wort {\Gottes} nach Gutdünken umzuschreiben,
%		und damit beispielsweise {\Seinen} Willen zu beugen,
%		so wie es,
%		meinen Informationen und Recherchen nach,
%		die Katholische Kirche in der Vergangenheit gerne gemacht hat.
%		
%	\section{Die 10 Gebote}
%		Die traditionellen 10 Gebote werden \"ublicherweise aus der Sicht {\Gottes} \"uberliefert,
%		also in der Form \q{Du sollst (nicht) ...}.
%		Im folgenden sind die 10 Gebote aus der Sicht,
%		wenn man selbst zu {\Gott} sprechten w\"urde,
%		und ihm die Gebote als Versprechen geben w\"urde.
%		Auch sind sie etwas besser ausgearbeitet,
%		da man manche Gebote bei genauerer Betrachtung auch zusammenfassen k\"onnte.
%	
%	\subsection{Das oberste Gebot lautet ...}
%		Ich will {\Gott}, den {\Herrn}, von ganzem Herzen lieben und {\Ihn} ehren. Und ich will meinen N\"achsten lieben, wie auch mich selbst.
%		
%	\subsection{Das erste Gebot}
%		{\Du} bist der {\Herr},
%		mein {\Gott},
%		mein {\Erloeser}.
%		Ich will keine anderen G\"otter neben {\Dir} haben,
%		und sie nicht anbeten oder verehren.
%		Und ich will mir kein G\"otzenbild schaffen.
%		
%	\subsection{Das zweite Gebot}
%		{\Du} bist der {\Herr},
%		mein {\Gott}.
%		Ich will {\Deinen} Namen nicht missbrauchen.
%		Ich will {\Dir} nicht l\"astern.
%		Und ich will mich ehrlich zu {\Dir} bekennen.
%			
%	\subsection{Das dritte Gebot}
%		{\Du} bist der {\Herr},
%		mein {\Gott}.
%		Ich will {\Dich} nicht auf die Probe stellen.
%		Ich will {\Dich} nicht versuchen.
%		Ich will auch in der Not zu {\Dir} stehen.
%		
%	\subsection{Das vierte Gebot}
%		{\Du} bist der {\Herr},
%		mein {\Gott}.
%		Ich will {\Dir} den Sabbat heiligen.
%		Ich will am Sabbat des Fleischlichen,
%		und Suchterzeugenden enthaltsam bleiben.
		
	\subsection{Das f\"unfte Gebot}
		Ich will meinen Vater und meine Mutter ehren.
		Und ich will \"Altere Menschen ehren.
			
	\subsection{Das sechste Gebot}
		Ich will nicht t\"oten.
		Ich will meine Beziehungen pflegen.
		Ich will das Leben und Wohlergehen allen Lebens respektieren,
		und nach M\"oglichkeit auch sch\"utzen.
		
	\subsection{Das siebte Gebot}
		Ich will nicht die Ehe brechen.
		Ich will nicht die Frau meines N\"achsten begehren.
		Ich will nicht den Mann meiner N\"achsten begehren.
		
	\subsection{Das achte Gebot}
		Ich will nicht rauben oder stehlen.
		Ich will nicht betr\"ugen oder entf\"uhren.
		Ich will nicht begehren meines N\"achsten Haus.
		Ich will nicht begehren meines N\"achsten Hab und Gut.
		Ich will dem Hab und Gut meines N\"achsten keinen Schaden zuf\"ugen.
		
	\subsection{Das neunte Gebot}
		Ich will nicht falsch Zeugnis geben wider meinem N\"achsten.
		Ich will nicht l\"ugen oder betr\"ugen.
		Ich will nicht schw\"oren.
		Ich will gegen\"uber meinem N\"achsten ehrlich und gerecht handeln.
		
%	\subsection{Das zehnte Gebot}
%		Mein K\"orper ist ein Geschenk von {\Dir},
%		und somit heilig.
%		Ich will ihn ehren und pflegen.
%
	\section{My life with {\God}} \label{MeinLebenMitGott}
		This big chapter is a kind of diary,
		how I experience my journey with and to {\God},
		and all the things a may learn on it.
%	
%	\subsection{Mittwoch, der 27. September 2023}
%		Ich bin seit etwa Mitte 2023 auf einer Art Reise,
%		bei der ich mich entschieden habe,
%		{\Gott} und {\Jesus} in mein Leben zu lassen.
%		Ich habe selbst noch viele Fehler,
%		und obgleich der von {\Gott} gegebenen Gebote,
%		s\"undige ich noch viel zu oft.
%		Wie im Vorwort erw\"ahnt,
%		bin ich weit davon entfernt,
%		so etwas wie der \q{perfekte Christ} zu sein.
%		Viele der allt\"aglichen Gewohnheiten,
%		Pr\"agungen und sonstiges haben so eine starke Sogwirkung,
%		dass ich auch nicht immer an {\Gott} denke,
%		nicht so oft bete,
%		oder in der Bibel lese,
%		wie ich gerne w\"urde.
%		Und wenn ich dann \q{wieder} an {\Gott} denke,
%		habe ich oft ein schlechtes Gewissen,
%		weil ich {\Ihn} dann gef\"uhlt \q{vergessen} habe.
%		Also kurzum:
%		Ich darf noch sehr, sehr, sehr, ..., sehr viel lernen!

\end{document}
