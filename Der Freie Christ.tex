\documentclass[12pt,a4paper]{article}

\usepackage[ngerman]{babel}
%\usepackage{enumitem}
\usepackage[utf8]{inputenc}
\usepackage[T1]{fontenc}
\usepackage[left=25mm,right=25mm,top=25mm,bottom=25mm]{geometry}
\usepackage[colorlinks=true,urlcolor=blue,linkcolor=black]{hyperref}

\newcommand{\Christi}[0]{\textbf{CHRISTI}}
\newcommand{\Christus}[0]{\textbf{CHRISTUS}}
\newcommand{\Dein}[0]{\textbf{DEIN}}
\newcommand{\Deine}[0]{\textbf{DEINE}}
\newcommand{\Deinem}[0]{\textbf{DEINEM}}
\newcommand{\Deinen}[0]{\textbf{DEINEN}}
\newcommand{\Deiner}[0]{\textbf{DEINER}}
\newcommand{\Deines}[0]{\textbf{DEINES}}
\newcommand{\Dich}[0]{\textbf{DICH}}
\newcommand{\Dir}[0]{\textbf{DIR}}
\newcommand{\Du}[0]{\textbf{DU}}
\newcommand{\Er}[0]{\textbf{ER}}
\newcommand{\Erloeser}[0]{\textbf{ERL\"OSER}}
\newcommand{\Erloesers}[0]{\textbf{ERL\"OSERS}}
\newcommand{\Ihm}[0]{\textbf{IHM}}
\newcommand{\Ihn}[0]{\textbf{IHN}}
\newcommand{\Geist}[0]{\textbf{GEIST}}
\newcommand{\Geiste}[0]{\textbf{GEISTE}}
\newcommand{\Gott}[0]{\textbf{GOTT}}
\newcommand{\Gottes}[0]{\textbf{GOTTES}}
\newcommand{\Herr}[0]{\textbf{HERR}}
\newcommand{\Herrn}[0]{\textbf{HERRN}}
\newcommand{\Jesu}[0]{\textbf{JESU}}
\newcommand{\Jesus}[0]{\textbf{JESUS}}
\newcommand{\Heilig}[0]{\textbf{HEILIG}}
\newcommand{\Heilige}[0]{\textbf{HEILIGE}}
\newcommand{\Heiligen}[0]{\textbf{HEILIGEN}}
\newcommand{\Heiliger}[0]{\textbf{HEILIGER}}

\newcommand{\q}[1]{\char"22{#1}\char"22 }

\title{\textbf{Der Freie Christ}}
\author{Robert Lang-Kirchh\"ofer}
\date{}

\begin{document}
	\setcounter{section}{-1}
	\setlength{\parindent}{0mm}
	\maketitle

	\newpage
	\tableofcontents

	\newpage
	\section{Vorwort}
	Es handelt sich hierbei um ein christliches Schriftst\"uck.
	Ich will hiermit moralische Werte \"ubermitteln,
	insbesondere wie sie,
	nat\"urlich nach bestem Wissen und Gewissen,
	von {\Gott},
	meinem {\Herrn},
	und dem {\Herrn} {\Jesus} {\Christus} gew\"unscht sind.
	Wie man in diesem Vorwort schon erkennen kann,
	sind Worte die sich direkt auf {\Gott},
	{\Jesus} oder auch den {\Heiligen} {\Geist} beziehen,
	in Majuskeln,
	also komplett in Gro{\ss}buchstaben,
	und zus\"atzlich in Fettschrift geschrieben.
	
	\section{Was macht einen Freien Christen aus?}
	
	
	\section{Die 10 Gebote}
	Die traditionellen 10 Gebote werden \"ublicherweise aus der Sicht {\Gottes} \"uberliefert,
	also in der Form \q{Du sollst (nicht) ...}.
	Im folgenden sind die 10 Gebote aus der Sicht,
	wenn man selbst zu {\Gott} sprechten w\"urde,
	und ihm die Gebote als Versprechen geben w\"urde.
	Auch sind sie etwas besser ausgearbeitet,
	da man manche Gebote bei genauerer Betrachtung auch zusammenfassen k\"onnte.
	
	\subsection{Das erste Gebot}
	{\Du} bist der {\Herr},
	mein {\Gott},
	mein {\Erloeser}.
	Ich will keine anderen G\"otter neben {\Dir} haben,
	und sie nicht anbeten oder verehren.
	Und ich will mir kein G\"otzenbild schaffen.
	
	\subsection{Das zweite Gebot}
	{\Du} bist der {\Herr},
	mein {\Gott}.
	Ich will {\Deinen} Namen nicht missbrauchen.
	Ich will {\Dir} nicht l\"astern.
	Und ich will mich ehrlich zu {\Dir} bekennen.
		
	\subsection{Das dritte Gebot}
	{\Du} bist der {\Herr},
	mein {\Gott}.
	Ich will {\Dich} nicht auf die Probe stellen.
	Ich will {\Dich} nicht versuchen.
	Ich will auch in der Not zu {\Dir} stehen.
	
	\subsection{Das vierte Gebot}
	{\Du} bist der {\Herr},
	mein {\Gott}.
	Ich will {\Dir} den Sabbat heiligen.
	Ich will am Sabbat des Fleischlichen,
	und Suchterzeugenden enthaltsam bleiben.
	
	\subsection{Das f\"unfte Gebot}
	Ich will meinen Vater und meine Mutter ehren.
	Und ich will \"Altere Menschen ehren.
		
	\subsection{Das sechste Gebot}
	Ich will nicht t\"oten.
	Ich will meine Beziehungen pflegen.
	Ich will das Leben und Wohlergehen allen Lebens respektieren,
	und nach M\"oglichkeit auch sch\"utzen.
	
	\subsection{Das siebte Gebot}
	Ich will nicht die Ehe brechen.
	Ich will nicht die Frau meines N\"achsten begehren.
	Ich will nicht den Mann meiner N\"achsten begehren.
	
	\subsection{Das achte Gebot}
	Ich will nicht rauben oder stehlen.
	Ich will nicht betr\"ugen oder entf\"uhren.
	Ich will nicht begehren meines N\"achsten Haus.
	Ich will nicht begehren meines N\"achsten Hab und Gut.
	Ich will dem Hab und Gut meines N\"achsten keinen Schaden zuf\"ugen.
	
	\subsection{Das neunte Gebot}
	Ich will nicht falsch Zeugnis geben wider meinem N\"achsten.
	Ich will nicht l\"ugen oder betr\"ugen.
	Ich will nicht schw\"oren.
	Ich will gegen\"uber meinem N\"achsten ehrlich und gerecht handeln.
	
	\subsection{Das zehnte Gebot}
	Mein K\"orper ist ein Geschenk von {\Dir},
	und somit heilig.
	Ich will ihn ehren und pflegen.
	
\end{document}
